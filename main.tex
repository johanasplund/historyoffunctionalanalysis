% Things to do:
% - the introduction is not "chapter 1", it is not numbered
% - fix chapter-section references, as
% - make the table of contents and the chapter/section fonts the same as Dieudonné's
% - the asterisk of a footnote is usually before a dot/comma, not after
% - add space before and after arguments of operators like cos/sin as in the original book
% - in sections 2.3, 4.1 there are multiple footnotes with the same (*) mark on the same page
% - the integrals in text mode are not exactly the same as those in math mode, while Diuedonné wrote them with the same symbol at the same height
% - substitute all \frac{\partial #1}{\partial #2} with \pder
% - substitute all divergences with \divergence

\documentclass[12pt,a4paper,leqno,twoside]{extbook}

\usepackage{adjustbox}
\usepackage{amsmath}
\usepackage[english]{babel}
\usepackage{bookmark}
\usepackage[autostyle=true]{csquotes} % otherwise babel complains
\usepackage[shortlabels]{enumitem}
\usepackage{etoolbox}
\usepackage[default]{fontsetup} % New Computer Modern book weight
\usepackage[bottom,hang,splitrule]{footmisc}
\usepackage{footnotebackref}
\usepackage[hmargin=2.5cm,vmargin=2cm,twoside=false]{geometry}
\usepackage{indentfirst}
\usepackage{mathtools}
\usepackage[protrusion=true,expansion=true]{microtype}
\usepackage[a-1a]{pdfx}
\usepackage{realscripts} % for OpenType super/subscripts
\usepackage[tiny,compact,explicit]{titlesec}
\usepackage{todonotes}

\usepackage[backend=biber,maxnames=2,sorting=custom]{biblatex}
\addbibresource{bibliography.bib}
\DeclareFieldFormat{pages}{p.~#1} % Dieudonné used "p." instead of "pp." even for multiple pages
\DeclareSortingScheme{custom}{\sort{\field{entrykey}}}
\DefineBibliographyStrings{english}{
    andothers = {\mkbibemph{et\addabbrvspace al\adddot}} % "et al." in italics
}
\renewcommand*{\mkbibnamefamily}[1]{\textsc{#1}} % surnames in small caps
\renewcommand*{\newunitpunct}{\addcomma\space} % comma instead of dot between authors and title

\usepackage{chngcntr}
\counterwithout{equation}{chapter}

\usepackage{hyperref}
\hypersetup{colorlinks,allcolors=blue}

\usepackage{fancyhdr}
\pagestyle{fancy}
\fancyhf{} % clear all header and footer fields
\fancyhead[EL]{\thepage} % odd pages, left header
\fancyhead[OR]{\thepage} % even pages, right header
\titleformat{\chapter}[display]{\centering\scshape}{\bfseries\chaptertitlename~\thechapter}{0pt}{#1}
\titleformat{\section}[block]{\centering\bfseries}{\S\arabic{section}. #1.}{0em}{}
\renewcommand{\chaptermark}[1]{\markboth{#1}{#1}} % a hack needed for fancyhdr to display just the chapter name
\renewcommand{\headrulewidth}{0pt} % remove all header rules

\usepackage{perpage}
\MakePerPage{footnote}
\renewcommand*{\thefootnote}{\tiny{($\ast$)}}

\usepackage{tocloft}
\makeatletter
    \renewcommand{\cftchappresnum}{Chapter~} % prefix before number
    \renewcommand{\cftchapaftersnum}{:~} % text after number
    \setlength{\cftchapnumwidth}{8em} % wrapped lines align with the chapter title
\makeatother
\renewcommand{\cftchapfont}{\bfseries\scshape}
\renewcommand{\cftchapleader}{\cftdotfill{\cftdotsep}}

\renewcommand\footnotemargin{\parindent} % footnote text aligned with text margin
\addtolength{\footnotesep}{1mm} % add vertical space between footnotes

\renewcommand{\arraystretch}{1.2}
\renewcommand{\thechapter}{\Roman{chapter}}
\renewcommand{\thesection}{§\arabic{section}}

\newcommand{\mytodo}[1]{\todo[inline]{TODO: #1}}
\newcommand{\supth}{\texorpdfstring{\textsuperscript{th}}{th}}
\newcommand{\sub}[1]{\texorpdfstring{\textsubscript{#1}}{#1}}

\DeclarePairedDelimiter{\norm}{\lVert}{\rVert}
\newcommand{\B}{\mathcal{B}}
\newcommand{\Cont}{\mathcal{C}}
\newcommand{\C}{\mathbb{C}}
\newcommand{\R}{\mathbb{R}}
\newcommand{\Sph}{\mathbb{S}}
\newcommand{\diam}{\operatorname{diam}}
\newcommand{\grad}{\operatorname{grad}}
\newcommand{\vol}{\operatorname{vol}}
\newcommand{\Ker}{\operatorname{Ker}}
\newcommand{\pder}[2]{\frac{\partial #1}{\partial #2}}
\newcommand{\divergence}[1]{\left(\pder{#1}{x}\right)^2 + \left(\pder{#1}{y}\right)^2 + \left(\pder{#1}{z}\right)^2}

\allowdisplaybreaks
\frenchspacing
\raggedbottom % prevent huge vertical gaps between paragraphs
\sloppy % do not overfill

\title{History of Functional Analysis}
\author{Jean Dieudonné}
\date{1981}

\begin{document}

\maketitle

\tableofcontents

\fancyhead[C]{\textsc{\leftmark}} % header for the introduction (it's different from the headers of the chapters)

\chapter*{Introduction}
\markboth{Introduction}{Introduction} % otherwise it shows "CONTENTS" and the names of the each chapter becomes the name of its previous chapter
\addcontentsline{toc}{chapter}{\protect\numberline{}Introduction}

One may give different definitions of ``Functional Analysis''. Its name name might suggest that it contains all parts of mathematics which deal with functions, but that would practically mean \emph{all} mathematical Analysis. We shall adopt a narrower definition: for us, it will be the study of topological vector spaces and of mappings \(u: \Omega \rightarrow F\) from a part of \(\Omega\) of a topological vector space \(E\) into a topological vector space \(F\), these mappings being assumed to satisfy various algebraic and topological conditions. A moment of reflection shows that this already covers a large part of modern Analysis, in particular the theory of partial differential equations.

Functional analysis thus appears as a rather complex blend of Algebra and Topology, and it should therefore surprise no one that the development of these two branches of mathematics had a strong influence on its own evolutions. As a matter of fact, it is almost impossible to dissociate the early history of General Topology (and even of the set-theoretic language) from the beginnings of Functional Analysis, since the sets and spaces which (after the subsets of \(\R^n\)) attracted most attention consisted of \emph{functions}.

With regard to Algebra, as the most frequently studied mappings between topological vector spaces are \emph{linear}, it is quite natural that linear Algebra should have greatly influenced Functional Analysis. In fact, at the end on the XIX\supth{} century, the old idea that infinitesimal Calculus was derived from the algebraic ``Calculus of differences'' by a ``limit process'' began to acquire a more precise and more influential form when Volterra applied a similar idea to an integral equation
\begin{equation}
    \label{eq:1}
    \int_a^y \varphi(x) H(x,y) dx = f(y)
\end{equation}
for an unknown function \(\varphi\), the functions \(f\) and \(H\) being continuous in \([a,b]\) and \([a,b] \times [a,b]\) respectively, with \(f(a) = 0\). He divides \([a,b]\) into \(n\) subintervals by the points \(y_k = a + k \frac{b-a}{n} \; (1 \leq k \leq n)\), replaces \(y\) in \eqref{eq:1} by these \(n\) values, and the integral by the corresponding Riemann sums, which gives him a system of \(n\) linear equations
\begin{equation}
    \label{eq:2}
    \begin{dcases}
        h_{11} z_1               \hfill = b_1 \\
        h_{21} z_1 + h_{22} z_2  \hfill = b_2 \\
        \dotfill \\
        h_{n1} z_1 + h_{n2} z_2 + \ldots + h_{nn} z_n = b_n
    \end{dcases}
\end{equation}
with \(j_{jk} = H(y_j, y_k), z_k = \varphi(y_k)\) and \(b_k = f(y_k)\); the integral equation \eqref{eq:1} was thus considered as obtained from systems \eqref{eq:2} by a limit process when the number of unknowns became infinite.

Unfortunately, linear Algebra, as it was understood in the XIX\supth{} century (and even much later) did not readily lend itself to affording a good guidance to such generalizations. Its own evolution had been very slow and painful, stretching over 130 years, and in a succession of stages which, to our eyes, is exactly the \emph{reverse} of the \emph{logical} sequence of notions, namely

\mytodo{put diagram at p. 3.}

In spite of the unsuccessful efforts of Grassmann and Peano, the intrinsic aspects and the geometric point of view in linear Algebra remained in the background until 1900; one would readily speak with Cayley (1843) of vectors and linear subspaces, but they were invariably considered as parts of some \(\R^n\); in other words, everything in a vector space was always referred to a \emph{fixed basis}, and linear maps were only handled through their \emph{matrices} corresponding to these bases. The various ``reduction'' theorems were known in 1880, but only through complicated computations of determinants, and without any geometric interpretation. Furthermore, Frobenius, who had been the most influential mathematician in building up a synthesis of the linear Algebra of his time, had unfortunately taken a step backward (even with respect to Cayley) by electing to work systematically with bilinear forms \(\sum_{p,q} a_{pq} x_p y_q\) instead of working with matrices \((a_{pq})\). Finally, before 1930 nobody had a correct conception of duality between finite dimensional \emph{vector spaces}; even in van der Waerden's book (1931), such a vector space and its dual are still \emph{identified}.

All this was to weigh heavily on the evolution of linear Functional Analysis; in particular it followed (over a shorted span of years) the same unfortunate succession of stages through which linear Algebra had to go; and it is only after it was realized that the current conception of vectors as ``\(n\)-tuples'' could not possibly be extended to infinite dimensional function spaces, that this conception was finally abandoned and that genuinely geometrical notions won the day.

The diagram at the end of this Introduction tries to depict graphically in some detail the successive stages of the history of Functional Analysis, by mentioning the actions and reactions of the various parts of mathematics which took part in it. If one were to reduce this complicated history to a few key words, I think the emphasis should fall on the evolution of two concepts: \emph{spectral theory} and \emph{duality}. Both of course stem from the very concrete problems encountered in the solution of linear equations (or systems of linear equations), where the unknowns are \emph{functions}. The basic concepts of spectral theory: eigenvalues, eigenfunctions and expansions in series of such functions were already known at the beginning of the XIX\supth{} century, in the theory of Fourier series; they would form the model on which all further advances were patterned. But it took 60 years of strenuous efforts to extend the theory from the Sturm--Liouville problem in ordinary differential equations to the partial differential equation of the vibrating membrane. It was gradually realized that the heart of the matter lay, not in the differential (or partial differential) equations themselves, but in \emph{integral equations} associated to them; at first they were not explicitly written down, so that one can only speak of ``crypto-integral'' equations, to designate the use of methods resting on evaluations of integrals, and which only later emerged as standard methods in the theory of integral equations.

The remarkable feature of this history is that, after such a slow incubation period, so to speak, spectral theory, in the span of a few years, reached complete maturity, giving birth in the process to the concept of linear duality, which began at last to be understood by analysts, before becoming later familiar to all mathematicians by a kind of backlash effect. What is interesting in this rapid advance is that it was accomplished in a series of what one may call discrete \emph{jumps}, in each of which the decisive step was to ignore the special features of the problem under consideration, and to make it accessible by inserting it into a more general context.

The first of these ``discontinuities'' occurred in 1896--1900, when Le Roux, Volterra and Fredholm, instead of working on the \emph{special} integral equations studied by their predecessors (Abel, Liouville, Beer--Neumann), elected to use \emph{minimal} assumptions on the kernels, and in so doing discovered that the theory was far simpler that it was generally thought.

The second step was taken by Hilbert in his 1906 papers, subordinating the too special theory of symmetric integral equations to the much more general concept of infinite ``bounded'' quadratic forms, which turned out to provide the frame needed for all subsequenct progress in ordinary and partial differential equations.

The contemporary discovery of the Lebesgue integral, and the geometric and topological concepts introduced by Fréchet in Analysis immediately led Hilbert's successors to translate his results into the language of what we now call Hilbert space, linking the euclidean geometry to integration theory, and making possible the discussion of the most general systems of linear equations in such a space.

This in turn led F. Riesz in 1910--1913 to introduce \(L^p\) and \(l^p\) spaces for any exponent \(p\) such that \(1 < p < +\infty\), and to discover the natural duality between the \emph{different} spaces \(L^p\) and \(L^q\) with \(\frac{1}{p} + \frac{1}{q} = 1\), in sharp distinction from the muddleheaded ideas on the matter, which the accidental self-duality of Hilbert space had failed to dispel.

But although F. Riesz, in the treatment of systems of linear equations in \(l^p\) spaces, was the first to obtain a condition which later was seen to consist in a particular application of the Hahn--Banach theorem, he failed to visualize that condition as amounting to an extension property of a continuous linear form defined on a subspace. This fourth ``jump'' was only accomplished by Helly in 1921, again by generalizing the theory of systems of linear equations from the special \(l^p\) spaces to \emph{any} normed subspace of \(\C^{\mathbb{N}}\). After that, only two more steps were needed to reach the present status of the theory, with the passage to general normed spaces (together with the use of transfinite induction) by Hahn and Banach and a little later the extension of duality theory to locally convex spaces during the period 1935--1945.

This process of successive generalizations may thus have reached a point of diminishing returns around the middle of the century. Inasmuch as we are able to judge from events probably too recent to allow a proper perspective, the theory of topological vector spaces, after 1950, has stabilized as one of the standard tools of modern mathematics, together with linear and multilinear Algebra, General Topology and measure theory. The advances which have been achieved during the last 30 years mainly consist in new imaginative ways to use the fundamental tools of Functional Analysis, either in theories where they had noot been applied before, such as differential geometry and differential topology (K-theory, theory of the Atiyah--Singer index, foliations), or in the construction of more powerful methods to handle functional equations (distributions, Sobolev spaces, pseudo-differential operators and their generalizations)·

This volume grew out of a series of lectures which I gave in Rio de Janeiro in 1979, at the invitation of Prof. Jorge Alberto Barroso of the Universidade Federal de Rio de Janeiro, to whom go my most heartfelt thanks. I am also very grateful to him for the pains he took in supervising the preparation of the manuscript for publication.

\clearpage

% headers for the other chapters
\fancyhead[EC]{\textsc{chapter} \thechapter}
\fancyhead[OC]{\textsc{\leftmark}}

\chapter{Linear differential equations and the Sturm--Liouville problem}
\label{ch:1}
\setcounter{equation}{0}

\section[Differential equations and partial differential equations in the XVIII\supth{} century]{Differential equations and partial differential equations\\in the XVIII\supth{} century}
\label{sec:1.1}

Until around 1750, the notion of \emph{function} of one variable was a very hazy one. The domain where it was defined was seldom described with precision; it was tacitly assumed that around each point \(x_0\), the function was equal to a power series in \(x - x_0\) and its derivatives were obtained by taking the derivatives of each term of the series. To solve a differential equation of order \(n\)
\begin{equation}
    \label{eq:1.1}
    y^{(n)} = F(x, y, y', y'', \ldots, y^{(n-1)})
\end{equation}
one would therefore substitute in \eqref{eq:1.1} for \(y\) and its derivatives a power series \(\sum\limits_{k=0}^\infty c_k (x - x_0)^k\) and its derivatives, and identify the series on both sides, which would determine each  \(c_k\) for \(k \leq n\) as a function of \(c_0, c_1, \ldots, c_{k-1}\); the solution thus depended on \(n\) arbitrary parameters \(c_0, c_1, \ldots, c_{n-1}\). The very few cases in which it was possible to write explicitly the solution by means of primitives of known functions (such as the linear equation \(y' = a(x)y + b(x)\) of order 1) were already known at the end of the XVII\supth{} century.

After 1760 began the first general study of linear equations of arbitrary order
\begin{equation}
    \label{eq:1.2}
    L(y) \equiv y^{(n)} + a_1(x) y^{(n-1)} + \ldots + a_n(x) y = b(x).
\end{equation}
D'Alembert observed that the knowledge of a particular solution of the equation and of all solutions of the homogeneous equation \(L(y) = 0\) yields by addition all solutions of \eqref{eq:1.2}. A little later, Lagrange \cite[vol.~I, p.~474]{135} showed that the general solution of \(L(y) = 0\) may be written \(\sum\limits_{k=1}^n c_k y_k\) where the \(c_k\) are arbitrary constants, and the \(y_k (1 \leq k \leq n)\) particular solutions (which he tacitly assumed to be linearly independent). Then, by his famous method of  ``variation of constants'' \cite[vol.~IV, p.~159]{135}, he showed how to obtain also the solutions of \eqref{eq:1.2} when the \(y_k\) were known: the solution is written in the form \(y = \sum\limits_{k=1}^n z_k y_k\), where the \(z_k\) are unknown functions, subject to \(n-1\) linear relations
\begin{equation}
    \label{eq:1.3}
    \sum_{k=1}^n z'_k y_k^{(\nu)} = 0 \quad (0 \leq \nu \leq n-2)
\end{equation}
These conditions imply that \(y^{(\nu)} = \sum\limits_{k=1}^n z_k y_k^{(\nu)}\) for \(0 \leq \nu \leq n-1\); replacing \(y\) by \(\sum\limits_{k=1}^n z_k y_k\) in \eqref{eq:1.2} and using the fact that the \(y_k\) satisfy \(L(y_k) = 0\), one obtains for the \(z'_k\) another linear equation
\begin{equation}
    \label{eq:1.4}
    \sum_{k=1}^n z'_k y_k^{(n-1)}  = b(x)
\end{equation}
from which, by the Cramer formulas, one can compute the \(z_k' (1 \leq k \leq n)\) and the problem is thus reduced to computing their primitives,

Lagrange also introduced \cite[vol.~I, p.~471]{135} the notion of \emph{adjoint} of a linear differential operator \(L\), which was to acquire great importance later: he showed that there exists a linear differential operator \(M\) satisfying an identity
\begin{equation}
    \label{eq:1.5}
    z L(y) - y M(z) = \frac{d}{dx}\left(B(y, z)\right)
\end{equation}
where \(B\) is bilinear in \((y, y', \ldots, y^{(n-1)})\) and \((z, z', \ldots, z^{(n-1)})\), constituting a generalization of the classical ``integration by parts''; he deduced from that formula that if a solution \(z\) of \(M(z) = 0\) was known, solutions of \(L(y) = 0\) could be obtained by solving an equation \(B(y,z) = \text{Const}\), of order \(n-1\).

Partial differential equations were not considered until the middle of the XVIII\supth{} century, in connection with problems of Mechanics or Physics and then they were of order 2 at least (see \ref{sec:1.2}). The study of partial differential equations of first order was only begun by Fuler and Lagrange after 1770. Euler was able to solve a few particular equations, and then Lagrange found general methods which enabled his followers, Charpit and Monge, to reduce the solution of a general equation of first order
\begin{equation}
    \label{eq:1.6}
    F\left(x,y,z, \frac{\partial z}{\partial x}, \frac{\partial z}{\partial y}\right) = 0
\end{equation}
to the solution of a system of ordinary differential equations, an idea which was developed later by Cauchy in his concept of ``characteristic curves''.

\section{Fourier expansions}
\label{sec:1.2}

In 1747, d'Alembert gave the first mathematical treatment of the general problem of the small vibrations of a string of length \(a\), fixed at each extremity; the string moves in a plane where the axis \(0x\) is along the position of the string at rest, the segment \(0 \leq x \leq a\); if \(y = u(x,t)\) is the equation of the string at time \(t\), d'Alembert shows that, if \(u(x,t)\) remains small, it satisfies the equation
\begin{equation}
    \label{eq:1.7}
    \frac{\partial^2 u}{\partial t^2} = c^2 \frac{\partial^2 u}{\partial x^2}
\end{equation}
where \(c\) is a known function of \(x\) alone, and is constant if the density of the string is constant. When \(c\) is constant, taking \(X = x-ct\) and \(Y = x+ct\) as new variables reduces the equation to \(\frac{\partial^2 u}{\partial X \partial Y} = 0\), and d'Alembert concluded that the solution of \eqref{eq:1.7} is given by
\begin{equation}
    \label{eq:1.8}
    u(x,t) = f(x - ct) + g(x + ct)
\end{equation}
where \(f\) and \(g\) are ``arbitrary'' functions. A year later, Euler interpreted this result as meaning that (for \(c=1\)) \(u(x,t)\) was known once the two functions of \(x\),
\begin{equation}
    \label{eq:1.9}
    u(x,0) = \varphi(x), \quad \frac{\partial u}{\partial t}(x,0) = \psi(x)
\end{equation}
were prescribed, the value of \(u(x,t)\) being explicitly given by
\begin{equation}
    \label{eq:1.10}
    u(x,t) = \frac{1}{2} \left(\varphi(x-t) + \varphi(x+t)\right) + \frac{1}{2} \int_{-t}^t \psi(x - \xi) d\xi
\end{equation}
(Euler only gives a geometric construction equivalent to this formula). Now it was well known experimentally that \(\varphi(x)\) could be quite different from an analytic function, for ins tance it could have no derivative at some points, and this led Euler to introduce, in addition to what he called ``continuous'' functions (i.e. analytic functions in our sense) more general ones which he baptized ``mechanical'' without giving their precise definition (from the context they seem to be piecewise twice differentiable functions in our terminology).

On the other hand, already in 1715, B. Taylor, by a direct argument which did not use equation \eqref{eq:1.7}, had concluded that (when \(c\) is constant) for any integer $n \geq 1$, the function
\begin{equation}
    \label{eq:1.11}
    u_n(x,t) = \sin\frac{n\pi x}{a} \cos\frac{n\pi ct}{a}
\end{equation}
represented vibrations of the string, namely for \(n = 1\) the ``fundamental'' tone, and for \(n = 2,3, \ldots\), its ``harmonics''. As it was well known that the sound emitted by a vibrating string was in general a mixture of several ``harmonics'', Daniel Bernoulli, in 1750, proposed that the general solution \eqref{eq:1.10} could also be written as a series
\begin{equation}
    \label{eq:1.12}
    u(x,t) = \sum_{n=1}^\infty a_n \sin\frac{n\pi x}{a} \cos\frac{n\pi c}{a}\left(t - b_n\right)
\end{equation}
for suitable values of the \(a_n\) and \(b_n\). However, in 1753, Euler observed that this would imply that an arbitrary ``mechanical'' function defined in an interval \(-a \leq x \leq a\) could be written as a series
\begin{equation}
    \label{eq:1.13}
    \frac{a_0}{2} + a_1 \cos\frac{\pi x}{a} b_1 \sin\frac{\pi x}{a} + a_2 \cos\frac{2\pi x}{a} + b_2 \sin\frac{2\pi x}{a} + \ldots
\end{equation}
and he believed that such a series of analytic functions could only represent an analytic function. His opinion was shared (with some variations) by almost all other mathematicians of his time, and no progress was made on this question until the beginning of Fourier's work on the theory of heat (see \cite[(2), t.~XI\sub{2}, pp.~273--300]{065}). Having to solve equations such as
\begin{equation}
    \label{eq:1.14}
    \frac{\partial^2 u}{\partial x^2} + \frac{\partial^2 u}{\partial y^2} = 0
\end{equation}
\begin{equation}
    \label{eq:1.15}
    \frac{\partial^2 u}{\partial x^2} - \frac{\partial^2 u}{\partial y} = 0
\end{equation}
for various boundary conditions, he systematically looks for solutions of the form \(u(x,y) = v(x)w(y)\) and, following D. Bernoulli, wants to obtain the most general solution as series whose terms are these particular ones. In so doing, he is brought back to the problem of expressing a function \(f\) as a series \eqref{eq:1.13}, but this time he adds to D. Bernoulli's argument the formulas giving actually the values of the coefficients \(a_n, b_n\)
\begin{equation}
    \label{eq:1.16}
    a_n = \frac{1}{\pi} \int_{-\pi}^\pi f(x)\cos{nx} dx, \quad b_n = \frac{1}{\pi} \int_{-\pi}^\pi f(x)\sin{nx} dx
\end{equation}
(when \(a = \pi\)) which as a matter of fact had already been obtained by Clairaut and Euler, without realizing their interest. Using these formulas Fourier was able to show on many examples of non analytic functions that the corresponding Fourier series converged to \(\frac{1}{2} \left(f(x_+) + f(x_-)\right)\), and expressed his conviction that this was true for ``arbitrary'' functions, although his attempts and those of Cauchy to prove that result were unsuccessful and the first proof for a piecewise monotonic and piecewise continuous function was only given by Dirichlet in 1829. One should also mention in that connection that in 1799, Parseval had given the formula
\begin{equation}
    \label{eq:1.17}
    \frac{a_0^2}{2} + \sum_{n=1}^\infty \left(a_n^2 + b_n^2\right) = \frac{1}{\pi} \int_{-\pi}^\pi \left(f(t)\right)^2 dt
\end{equation}
by a purely formal computation, without any proof of convergence.

These results gave the impetus to the vast theory of \emph{trigonometric series}, which was to be one of the main concerns of most analysts in the XIX\supth{} century, centered around the criteria of convergence of such series and the relations between its sum and its coefficients. The evolution of that theory was closely linked to a gradual precision and deepening of the notions of set of real numbers, of function and of integral. But before 1920 there was not much contact between that theory and the development of Functional Analysis as we understand it.

On the contrary, other results of Fourier in his \emph{Theory of heat} triggered the birth of \emph{spectral theory}. For instance \cite[vol.~I, p.~304]{067} he shows that the ``cooling off'' problem for a solid sphere of radius \(r\), when one assumes spherical symmetry for the problem, is governed by the partial differential equation
\begin{equation}
    \label{eq:1.18}
    \frac{\partial u}{\partial t} = k \left(\frac{\partial^2 u}{\partial x^2} + \frac{2}{x} \frac{\partial u}{\partial x}\right)
\end{equation}
with the ``boundary conditions'' that \(u(x,t)\) must remain finite when \(x\) tends to 0, and satisfy the relation
\begin{equation}
    \label{eq:1.19}
    \frac{\partial h}{\partial u} + hu = 0 \quad \text{for \(x=r\) and all \(t\)},
\end{equation}
where \(h\) and \(k\) are constants. Using his favorite method of ``separation of wvariables'', Fourier obtains solutions
\begin{equation}
    \label{eq:1.20}
    u(x,t) = \frac{1}{x} \exp{\left(-k\lambda^2 t\right)}\sin{\lambda x}
\end{equation}
provided the parameter \(\lambda\) is a solution of the transcendental equation
\begin{equation}
    \label{eq:1.21}
    \frac{\lambda r}{\operatorname{tg}{\lambda r}} = 1 - hr.
\end{equation}
He easily proves that the equation has an infinity of real roots \(\lambda_n\) tending to \(+\infty\). To obtain a solution of \eqref{eq:1.18} with boundary condition \eqref{eq:1.19} and such that \(u(x,0)\) is a given function \(f(x)\), he proceeds as before, writing \(xf(x)\) as a series \(\sum\limits_{n=1}^\infty c_n \sin{\lambda_n x}\); he shows that one has again the ``orthogonality'' relations (of course he does not use that word)
\begin{equation}
    \label{eq:1.22}
    \int_0^r \sin \lambda_n x \sin \lambda_m x dx = 0 \quad \text{for \(m \neq n\)}
\end{equation}
and from them deduces the relations
\begin{equation}
    \label{eq:1.23}
    % I use \Bigg here because they have to be the same size as those of eq. 1.34, where for some reason I don't need \Bigg for the parentheses
    c_n = \Bigg(\int_0^r xf(x)\sin{\lambda_n x} dx\Bigg) \Bigg/ \Bigg(\int_0^r \sin^2{\lambda_n x} dx\Bigg)
\end{equation}
without of course any rigorous justification, nor any proof of the fact that the series converges to \(xf(x)\).

\section{The Sturm--Liouville theory}
\label{sec:1.3}

The results of Fourier on the theory of heat were continued and expanded by Poisson. Their work led Ch. Sturm in 1836 and J. Liouville one year later to build a general theory which would include all cases considered by Fourier and Poisson, without assuming the possibility of explicit integration. They consider a second order differential equation
\begin{equation}
    \label{eq:1.24}
    y'' = q(x)y + \lambda y = 0
\end{equation}
where \(q\) is a real valued continuous function in a compact interval \([a,b]\) of \(\R\), and \(\lambda\) a complex parameter. The first problem is to consider boundary conditions of the form
\begin{equation}
    \label{eq:1.25}
    y(a)\cos\alpha - y'(a)\sin\alpha = 0, \quad y(b)\cos\beta - y'(b)\sin\beta = 0
\end{equation}
where \(alpha\) and \(\beta\) are two positive constants, and to determine for what values of \(\lambda\) the problem has a non trivial solution (an ``eigenfunction'' for the ``eigenvalue'' \(\lambda\) in our present day language).

A first remark, which had already essentially been made by Poisson, is that if \(\lambda, \mu\) are two different eigenvalues, and \(u, v\) two corresponding ``eigenfunctions'', then from the relations
\begin{equation*}
    u'' - qu + \lambda u = 0, \quad v'' - qv + \mu v = 0,
\end{equation*}
one deduces
\begin{equation*}
    u'' v - v'' u + (\lambda - \mu) uv = 0
\end{equation*}
and as \(\int\limits_a^b \left(u'' v - v'' u\right)dx = (u'v - v'u)\rvert_a^b = 0\) because of \eqref{eq:1.25}, one obtains
\begin{equation}
    \label{eq:1.26}
    (\lambda - \mu) \int_a^b u(x)v(x)dx = 0.
\end{equation}
A first consequence of this relation is that eigenvalues are necessarily \emph{real} numbers, Indeed, if \(\lambda\) was not real, then \(\overline\lambda\) would also be an eigenvalue with eigenfunction \(\overline{u}\), and substituting \(\overline\lambda\) and \(\overline{u}\) for \(\mu\) and \(v\) in \eqref{eq:1.26}, one obtains \(\int\limits_a^b |u(x)|^2 dx = 0\), contrary to assumption.

The main contribution of Sturm was the proof that there are infinitely many eigenvalues \(\lambda_1 < \lambda_2 < \ldots < \lambda_n \ldots\), tending to \(+\infty\). In his study of vibrating strings, d'Alembert had al ready considered an equation of the form \(y'' - \lambda\varphi(x)y = 0\) where \(\varphi\) is not constant, and had tried to prove that there is a single value of \(\lambda\) for which there is a solution in \([a,b]\) vanishing at \(a\) and \(b\) and nowhere else; his idea was to study the corresponding Riccati equation for \(y'/y\) when \(\lambda\) varies \cite[(2), vol.~XI\sub{2}, p.~311]{065}. Sturm elects a similar approach: he considers a solution \(u(x,\lambda)\) of \eqref{eq:1.24} satisfying the \emph{first} condition \eqref{eq:1.25}, and fixed for instance by the condition \(u(a,\lambda) = 1\) (or \(u'(a,\lambda) =1\) if \(\alpha = 0\)), and he studies the \emph{variation} of \(u(x,\lambda)\) as a function of \(\lambda\); the \(\lambda_n\) are therefore the solutions of the equation \(u(b,\lambda)\cos\beta - u' (b,\lambda)\sin\beta = 0\). He is thus led to \emph{compare} solutions of two equations
\begin{equation}
    \label{eq:1.27}
    y'' + q_1(x)y = 0, \quad y'' + q_2(x)y = 0
\end{equation}
when \(q_1(x) < q_2(x)\), and discovers many remarkable such ``comparison theorems'', of which we will only quote the one which leads to the existence of the eigenvalues. Sturm's paper is rather long-winded and not very clear (\cite{209}, \mytodo{add reference:, [S, p.~259--268]}) and there is a much simpler formulation of his result: an equation \(y'' + q(x)y = 0\) is written as a system of two first order equations by the usual introduction of two functions \(y_1 = y, y_2 = y'\), which gives \(y_1' = y_2, y_2' = -q(x)y_1\), and then one takes as new unknowns two functions \(r,\theta\) such that \(y_1 = r \sin\theta, y_2 = r \cos\theta\), which leads to the system
\begin{equation}
    \label{eq:1.28}
    r' = (1-q(x))r \sin\theta \cos\theta
\end{equation}
\begin{equation}
    \label{eq:1.29}
    \theta' = \cos^2\theta + q(x)\sin^2\theta
\end{equation}
where the second equation now is of the first order only\footnote{This device seems to have first been introduced by H. Prüfer \cite{180}.}. The comparison theorem which is needed is then the following one: consider solutions \(\varphi_0, \varphi_1\) in \([a,b]\) of the two equations
\begin{equation}
    \label{eq:1.30}
    \theta' = \cos^2\theta + q_1(x)\sin^2\theta, \quad \theta' = \cos^2\theta + q_2(x)\sin^2\theta
\end{equation}
and suppose that \(q_1(x) < q_2(x)\) in \([a,b]\). Then, if for a number \(\alpha \in ]a,b[\) one has \(\varphi_1(\alpha) \leq \varphi_2(\alpha)\), one also has \(\varphi_1(x) < \varphi_2(x)\) for \(\alpha < x < b\). The proof is very simple and consists in computing the derivative of the function \(w(x) = \varphi_2(X) - \varphi_1(x)\) and showing that there is a continuous function \(f\) in \([a,b]\) such that \(w'(x) - f(x)w(x) \geq 0\), which implies that \(w\) cannot change sign,

If now we apply the preceding change of variable to \eqref{eq:1.24}, we get the equation
\begin{equation}
    \label{eq:1.31}
    \theta' = \cos^2\theta + (\lambda - q(x))\sin^2\theta
\end{equation}
and we consider the solution \(\omega(x,\lambda)\) such that \(\omega(a,\lambda) = \alpha\); the eigenvalues \(\lambda\) are the solutions of the equations
\begin{equation}
    \label{eq:1.32}
    \omega(b, \lambda) = \beta + n\pi \quad \text{for \(n \in \mathbb{Z}\)}
\end{equation}
Sturm's comparison theorem then shows that for each \(x \in ]a,b]\) the function \(\lambda \mapsto \omega(t, \lambda)\) is \emph{strictly increasing}, and in addition, from \eqref{eq:1.31} it follows that if \(\omega(x,\lambda) = k\pi\) for an integer \(k\), then \(\frac{\partial \omega}{\partial x}(x, \lambda) = 1\). From these facts it is easy to show that each equation \eqref{eq:1.32} has one and only one solution \(\lambda_n\) for each \(n \geq 1\) and no solution for \(n \leq 0\); in addition, the corresponding eigenfunction \(u_n\) may be shown to have exactly \(n\) zeroes in the interval \(]a,b[\) \cite[p.~435--441]{052}.

Building on these results of Sturm, Liouville then proceeds to give a general formulation to the expansions of Fourier and Poisson. From relation \eqref{eq:1.26} where \(\lambda\) and \(\mu\) are replaced by \(\lambda_n\) and \(\lambda_m\) it follows that
\begin{equation}
    \label{eq:1.33}
    \int_a^b u_m(x)u_n(x)dx = 0 \quad \text{for \(m \neq n\)}.
\end{equation}
To each function \(f\), defined and continuous in \([a,b]\), Liouville associates its ``generalized Fourier coefficients''
\begin{equation}
    \label{eq:1.34}
    c_n = \left(\int_a^b f(x)u_n(x) dx\right) \Bigg/ \left(\int_a^b u_n^2(x)dx\right)
\end{equation}
and considers the ``generalized Fourier series'' \(\sum\limits_{n=1}^\infty c_n u_n(x)\). In order to study its convergence, he needs more information on the behavior of \(\lambda_n\) and \(u_n\) when \(n\) tends to \(+\infty\). He observes that, if \(\lambda = \rho^2 > 0\), any solution of \eqref{eq:1.24} satisfies a relation of the form
\begin{equation}
    \label{eq:1.35}
    y(x) = A \cos{\rho x} + B \sin{\rho x} + \frac{1}{\rho} \int_a^x q(t)y(t)\sin{\rho(x-t)}dt
\end{equation}
(which can be deduced from Lagrange's ``variation of constants'' method, by writing \eqref{eq:1.24} as \(y + \rho^2 y = q(x)y\), although this is not the way Liouville proves \eqref{eq:1.35}). Applying this to \(y = u_n\), so that \(\rho\) is replaced by \(\lambda_n^\frac{1}{2}\), he gives a sketchy proof that \(\rho_n = \frac{(n-1)\pi}{b-a} + O(1/n)\) and (if \(\cos\alpha \neq 0\)) \(u_n(x) = \sqrt\frac{2}{b-a} \cos{\rho_n x} + O(1/n)\) (when \(u_n\) is normalized by the condition \(\int\limits_a^b u_n^2(x)dx = 1\)). This allows him to prove that the series \(\sum\limits_{n=1}^\infty c_n u_n(x)\) converges, provided the usual Fourier series of \(f\) converges. He still has to show that, if \(f\) is continuous, the function \(F(x) = \sum\limits_{n=1}^\infty c_n u_n(x)\) is equal to \(f(x)\); he assumes (without proof) that \(F\) is continuous and that \(c_n = \int\limits_a^b F(x)u_n(x)dx\), and is reduced to proving that the relations \(\int\limits_a^b (F(x)-f(x))u_n(x)dx = 0\) for all \(n\) imply \(F = f\) (first appearance of the property of ``completeness'' of an orthonormal system); but this he can only do under the additional assumption that \(F-f\) has only a finite number of zeroes in \([a,b]\). The complete proof of the relation \(f(x) = \sum\limits_{n=1}^\infty c_n u_n(x)\) was only given (for \(f\) piecewise \(C^2\)) at the end of the XIX\supth{} century, as well as the relation \(\sum\limits_{n=1}^\infty c_n^2 = \int\limits_a^b f^2(x)dx\); Liouville had only proved the corresponding inequality \(c_1^2 + \ldots  + c_N^2 \leq \int\limits_a^b f^2(x)dx\) for all \(N\) (named after Bessel, who had proved it for the trigonometric system) (\cite{151} \mytodo{add reference: [S, p.268--281]}).

These remarkable results were to form the pattern of \emph{spectral theory}, the main efforts of analysts in that direction being directed to a generalization of the Sturm--Liouville theory to some types of partial differential equations; but in the first half of the XIX\supth{} century, the theory of these equations was far less advanced than the theory of ordinary differential equations, and it is only after 1880 that progress became possible (see Chapter \ref{ch:3}).

\chapter{The ``crypto-integral'' equations}
\label{ch:2}
\setcounter{equation}{0}

\section{The method of successive approximations}
\label{sec:2.1}

The study of celestial mechanics during the XVIII\supth{} century by the method of perturbations consisted, for the theory of the movements of planets, to first neglect their mutual attraction, which gave for each planet a Keplerian orbit around the sun, and then to find the deviations of the actual orbits from the Keplerian ones by taking into account the attraction of other planets; due to the fact that the masses of the planets are much smaller than the mass of the sun, these deviatims were expected to be small. Translated into mathematical terms, this amounted, in the simplest cases, to find good approximatims for the solutions of a system of differential equations
\begin{equation}
    \label{eq:2.1}
    y'_i = \varepsilon f_{1i}(x, y_1, \ldots, y_n) + \varepsilon^2 f_{2i}(x, y_1, \ldots, y_n) + \ldots \quad (1 \leq i \leq n)
\end{equation}
where the parameter \(\varepsilon\) on the right-hand sides is ``small''. The general conception of function in XVIII\supth{} century mathematics naturally led to try to express the \(y_i\) as a power series in \(\varepsilon\)
\begin{equation}
    \label{eq:2.2}
    y_i = a_i + \varepsilon y_{1i} + \varepsilon^2 y_{2i} + \ldots \quad (1 \leq i \leq n),
\end{equation}
to substitute these expressions in and identify the coefficients of the successive powers of \(\varepsilon\) on both sides. This led to a succession of equations
\mytodo{the line of dots at the end is not visible}
\begin{equation*}
\begin{aligned}
    y'_{1i} &= f_{1i}(x, a_1, \ldots, a_n) \\
    y'_{2i} &= F_{2i}(x, y_{11}, \ldots, y_{1n}) \\
    y'_{3i} &= F_{3i}(x, y_{11}, \ldots, y_{1n}, y_{21}, \ldots, y_{2n}) \\
    \dotfill
\end{aligned}
\end{equation*}
all of which had right-hand sides which were known functions, hence were reduced to mere ``quadratures''. No attempt was made to justify mathematically those procedures; the goal of these computations was merely to obtain a satisfactory agreement with observations.

It is well-known that Cauchy was the first mathematician who proved existence theorems for \emph{general} types of differential equations, for which no explicit solution is available. His strategy was to consider the various methods introduced earlier for the purpose of numerical computations, and to show that, under certain conditions, these methods actually gave \emph{convergent} approximation processes having a solution as limit. In particular, in a paper published in 1835 in Prag (\cite{040}, (2), vol.~XI, p.~399--465), he takes up the method outlined above, not for an ordinary differential equation, but for a linear partial differential equation of first order (which was known to be equivalent to a system of ordinary differential equations)
\begin{equation}
    \label{eq:2.3}
    \frac{\partial U}{\partial t} = \sum_{i=1}^p A_i(t, x_1, \ldots, x_p) \frac{\partial U}{\partial x_i};
\end{equation}
the problem is to find a solution which for \(t=0\) reduces to a given function \(u(x_1, \ldots,x_n)\), and Cauchy transforms \eqref{eq:2.3} into the equivalent ``integro-differential'' equation by considering \(x_1, \ldots, x_p\) as parameters:
\begin{equation}
    \label{eq:2.4}
    U(t, x_1, \ldots, x_p) = u(x_1, \ldots, x_p) + \int_0^t \left( \sum_{i=1}^p A_i(s, x_1, \ldots, x_p) \right) \frac{\partial U}{\partial x_i} ds
\end{equation}
which he solves by successive approximations, starting with \(U_0 = u\), and defining
\begin{equation*}
    U_n(t, x_1, \ldots, x_p) = u(x_1, \ldots, x_p) + \int_0^t \left( \sum_{i=1} A_i(s, x_1, \ldots, x_p) \right) \frac{\partial U_{n-1}}{\partial x_i} ds
\end{equation*}
by induction; but he is only able to prove convergence towards a solution when the \(A_i\) are analytic functions.

In his 1837 papers on the Sturm--Liouville problem Liouville independently applied a similar method to the linear differential equation \(y'' = f(x)y\), for which he wants to find a solution in \([a,b]\) satisfying the boundary condition \(y'(a) - hy(a) = 0\). He starts from the function \(y_0(x) = 1+h(x-a)\) satisfying that condition, and considers the series
\begin{equation}
    \label{eq:2.5}
    y = y_0 + y_1 + \ldots + y_n + \ldots
\end{equation}
where the \(y_n\) are determined for \(n > 0\) by the recursive equations
\begin{equation*}
    y_{n+1}(x) = \int_a^x dt \int_a^t f(s)y_n(s)ds.
\end{equation*}
It must be remembered that at that time the concept of uniform convergence had not yet been formulated, and no justifi cation had been given for asserting the continuity of a con vergent series of continuous functions, or differentiating or integrating such a series termwise, Liouville proves very easily that there is a constant \(C\) such that
\begin{equation*}
    \left| y_n(x) \right| \leq C^n(x-a)^{2n} / (2n)!
\end{equation*}
from which he concludes that the series \eqref{eq:2.5} giving \(y(x)\) converges for every \(x\); but he tacitly takes for granted that \(y\) is a \(C^2\) function and a solution of his problem.

In addition, Liouville makes the interesting remark that the function \(y\) can also be defined by the relation
\begin{equation}
    \label{eq:2.6}
    y = y_0 + \int_a^x dt \int_a^t f(s) y(s) ds
\end{equation}
(which he could also have writtem \(y = y_0 + \int_a^x (x-t)f(t)y(t)dt\)), thus giving what is probably the first example of what will be called later a ``Volterra integral equation of the second kind'' (see chap. \ref{ch:4}); if one writes \(z_n = y_0 + y_1 + \ldots + y_n\), Liouville observes that the \(z_n\) are given by \(z_0 = y_0\), and the recursive equations
\begin{equation}
    \label{eq:2.7}
    z_{n+1}(x) = y_0 + \int_a^x dt \int_a^t f(s) z_n(s) ds
\end{equation}
which is the standard process of ``successive approximations'' for these equations (\cite{151} \mytodo{add reference: [S, p. 268--281]}).

We have already seen that a little later in his papers of 1837, Liouville gives another ``integral equation'' equivalent to an equation \(y' = f(x)y\) (chap. \ref{ch:1}, \ref{sec:1.3}, equation \eqref{eq:1.35}). This exemplifies a general idea: if a linear differential operator \(P\) is such that the equation \(P \cdot u = f\) can be solved by a formula \(u = y_0 + G \cdot f\), where \(G\) is a linear operator, then the equation \(P \cdot u + Q \cdot u = 0\), where \(Q\) is an operator, is equivalent to \(u - G \cdot (Q \cdot u) = y_0\); in the case of Liouville, \(P \cdot u = u'' + \rho^2 u\) and \(Q \cdot u = -qu\), and \(G\) is an integral operator (cf. chap. \ref{ch:9}, \ref{sec:9.5}).

The simplest application of this idea is to the proof of Cauchy's existence and uniqueness theorem for an ordinary differential equation \(y' = f(x,y)\), which, with the initial con dition \(y(x_0) = y_0\), is equivalent to \(y = y_0 + \int_{x_0}^x f(t,y)dt\). In this general form it is given by E. Picard in his 1890 paper on successive approximations \cite[vol.~II, p.~197--200]{172}, where it comes as an afterthought, the bulk of the paper being concerned with applications of the method to partial differential equations. However, in these applications, Picard is directly influenced by the fundamental earlier works of C. Neumann on the Laplace equation and of H.A. Schwarz on the equation of vibrating membranes, which are the direct forerunners of the theory of integral equations; we will describe in detail C. Neumann's results in \ref{sec:2.4} of this chapter, and H.A. Schwarz's paper in chap. \ref{ch:3}, \ref{sec:3.1}.

\section{Partial differential equations in the XIX\supth{} century}
\label{sec:2.2}

During the whole XIX\supth{} century, the theory of partial dif ferential equations (in contrast with the theory of ordinary differential equations) has remained in an embryonic stage. The only general theorem, patterned after the Cauchy theorem on local existence and uniqueness of solutions of ordinary differential equations, is the Cauchy--Kowalewska theorem: suppose we have a system of \(r\) equations in \(r\) unknown real functions \(v_1, \ldots, v_r\) of \(p+1\) real variables \(x_1, \ldots, x_p\), of type
\begin{equation}
    \label{eq:2.8}
    \begin{gathered}
    \frac{\partial v_j}{\partial x_{p+1}} = \\
    = H_j\left(x_1, \ldots, x_{p+1}, v_1, \ldots, v_{r}, \frac{\partial v_1}{\partial x_1}, \frac{\partial v_1}{\partial x_2}, \ldots, \frac{\partial v_r}{\partial x_{p-1}}, \frac{\partial v_r}{\partial x_p}\right) \hfill (1 \leq j \leq r)
    \end{gathered}
\end{equation}
where the right hand sides do not contain any derivative with respect to \(x_{p+1}\) and are supposed to be real and \emph{analytic} with respect to their \(p+1+r+rp\) variables, in a neighborhood \(V_0\) of 0 in \(\R^{p+1+r+rp}\); then there is a small neighborhood \(V\) of 0 in \(\R^{p+1}\) such that \eqref{eq:2.8} has in \(V\) a unique solution \((v_1,\ldots,v_r)\) consisting of \emph{analytic} functions in \(V\), such that \(v_j(x_1,\ldots,x_p,0) = 0\) in \(V \cap \R^p\) for \(1 \leq j \leq r\).

The tendency (inherited from the XVIII\supth{} century) to consider that the most interesting functions were analytic was still very strong during the whole XIX\supth{} century, and therefore at first the analyticity restrictions of the Cauchy--Kowalewska theorem did not worry mathematicians very much, However, as it was known that some special types of partial differential equations, such as the scalar equation of first order and some types of second order equations, had solutions under much less stringent restrictions, people began to wonder if some other method than Cauchy's ``method of majorants'' (which could only be applied to analytic functions) would not yield a generalization of the Cauchy--Kowalewska theorem, at least for \(C^\infty\) functions. The question remained unanswered until 1956, when H. Lewy gave the surprising example of a system of
two linear equations in 3 variables, with \(C^\infty\) coefficients
\begin{equation*}
  \begin{dcases}
    \frac{\partial v_1}{\partial x_1} = \frac{\partial v_2}{\partial x_2} - 2 x_2 \frac{\partial v_1}{\partial x_3} - 2x_1 \frac{\partial v_2}{\partial x_3} - f(x_3) \\
    \frac{\partial v_2}{\partial x_1} = -\frac{\partial v_1}{\partial x_2} + 2 x_1 \frac{\partial v_1}{\partial x_3} - 2x_2 \frac{\partial v_2}{\partial x_3}
  \end{dcases}
\end{equation*}
which, for a suitable choice of the real \(C^\infty\) function \(f\), has \emph{no solution} whatsoever around any point (even if one al lows solutions which are distributions).

We shall not discuss the numerous local studies of analytic systems of partial differential equations (not necessarily reducible to the form \eqref{eq:2.8}) which followed the Cauchy--Kowalewska theorem, since they had no influence on the development of Functional Analysis as we understand it.

The remainder of the theory of partial differential equations until 1890 was limited to very special scalar equations (mostly linear equations or order 2) generally derived from physical problems\footnote{See the interesting description of these problems given by Poincaré in the Introduction of his 1890 paper on the equations of mathematical physics (\cite{177}, vol.~IX, p.~28--32)}, such as the equation of vibrating strings and its generalizations to 3 and 4 variables (the ``wave equations''), the Laplace equation \(\Delta u = 0\) in 2 and 3 variables, the heat equation in 2, 3 and 4 variables. For these equations, the techniques of ``separation of variables'' or of Fourier transforms (see chapter \ref{ch:7}, \ref{sec:7.6}) gave special solutions or solutions depending on ``arbitrary'' functions. But until 1825 the determination of solutions by boundary condi tions (of which we have seen a few examples in Chapter I) was always restricted to \emph{explicitly described and particular} such conditions.

A first attempt of classification of second order equations in 2 variables had been made by Laplace \cite[vol.~IX, p.~21--28]{137} He considered ``quasi-linear'' equations, i.e. those of the form
\begin{equation}
  \label{eq:2.9}
  \begin{gathered}
    A(x,y) \frac{\partial^2 z}{\partial x^2} + B(x,y) \frac{\partial^2 z}{\partial x \partial y} + \\
    + C(x,y) \frac{\partial^2 z}{\partial y^2} + F\left( x,y,z, \frac{\partial z}{\partial x}, \frac{\partial z}{\partial y} \right) = 0
  \end{gathered}
\end{equation}
linear in the second order derivatives. As he did not have a clear idea of the distinction between real and complex vari ables, and therefore did not hesitate to give complex values to \(x\) and \(y\), he asserted that a suitable change of variables could reduce the terms of \eqref{eq:2.9} containing second order derivatives either to \(\frac{\partial^2 z}{\partial x \partial y}\) or to \(\frac{\partial^2 z}{\partial x^2}\) when \(A, B, C\) are not all identically zero! With the development of the theory of functions of one complex variable, it was soon realized that, for real variables \(x, y\), equations \eqref{eq:2.9} where the second order derivatives enter by \(\frac{\partial^2 z}{\partial x^2} + \frac{\partial^2 z}{\partial y^2}\) (called \emph{elliptic} equations) had to be sharply distinguished from those (called \emph{hyperbolic} equations) where the second order derivatives enter by \(\frac{\partial^2 z}{\partial x \partial y}\) or \(\frac{\partial^2 z}{\partial x^2} - \frac{\partial^2 z}{\partial y^2}\). The study of general boundary conditions for hyperbolic equations only begins around 1860 and will have little contact with Functional Analysis until around 1925 (see chapter \ref{ch:9}, \ref{sec:9.5}). On the contrary, the various problems connected with the Laplace equation in 2 or 3 wvariables will be one of the main concerns of analysts from 1828 onwards, and will become the impetus leading to the theory of integral equations, and thence to our modern Functional Analysis.

\section{The beginnings of potential theory}
\label{sec:2.3}

In 1748, D. Bernoulli had introduced in the theory of newtonian attraction the function \(\Omega(M) = \sum\limits_i\left(m_i \mu/r_i\right)\) for a point \(M\) of mass \(\mu\) attracted by a finite number of punctual masses \(m_i\), where \(r_i\), is the distance of \(M\) to the mass \(m_i\); and in 1773 Lagrange observed that the knowledge of that function immediately gave the components of the attraction exerted on \(M\), by taking the derivatives of \(\Omega\) with respect to the coordinates \(x, y, z\) of \(M\). When the finite number of masses is replaced by a solid \(V\) of density \(\rho\) and the point \(M\) is outside \(V\), the function \(\Omega\) becomes
\begin{equation}
    \label{eq:2.10}
    \Omega(x,y,z) = \mu \iiint_V \frac{\rho(\xi, \eta, \zeta) d\xi d\eta d\zeta}{r(x, y, z, \xi, \eta, \zeta)}
\end{equation}
with \(r(x, y, z, \xi, \eta, \zeta) = \left( (x-\xi)^2 + (y-\eta)^2 + (z-\zeta)^2 \right)^{\frac{1}{2}}\) \cite[vol.~VI, p.~349]{135}.

In 1782 and 1785, Laplace showed that outside of \(V\) the function \(\Omega\) satisfied the equation
\begin{equation}
    \label{eq:2.11}
    \Delta \Omega \equiv \frac{\partial^2 \Omega}{\partial x^2} + \frac{\partial^2 \Omega}{\partial y^2} + \frac{\partial^2 \Omega}{\partial z^2} = 0
\end{equation}
\cite[vol.~X, p.~361--363 and vol.~XI, p.~276--280]{137}, and in 1813 Poisson completed that result by showing that if \(\rho\) discon tinuous in \(V\), the integral \eqref{eq:2.10} is still meaningful inside \(V\), and \(\Omega\) satisfies the ``Poisson equation''
\begin{equation}
    \label{eq:2.12}
    \Delta\Omega + 4\pi \rho = 0
\end{equation}
(\cite{178}, \mytodo{add reference: S, p.~342--346}). His idea is to consider the value of \(\Omega\) at a point \(M\) in \(V\) as the sum of the corresponding functions \(\Omega_1, \Omega_2\) relative to a small ball \(V_1\) of center \(M\) and to the complement \(V_2\) of \(V_1\) in \(V\); one has then \(\Delta\Omega_2 = 0\), and when the radius of \(V_1\) tends to 0 Poisson shows that \(\Delta\Omega_1\) tends to \(-4\pi\rho(M)\). (In fact his argument is not rigorous when one only assumes the continuity of \(\rho\), and the existence of \(\Delta\Omega\) is only guaranteed when \(\rho\) satisfies a Hölder condition; when \(\rho\) is merely continuous, equation \eqref{eq:2.12} is valid only if the second order derivatives are taken in the sense of the theory of distributions (chap. \ref{ch:8}, \ref{sec:8.3})).

After the discovery of Coulomb's laws (1785) the Laplace equation became of central importance in electrostatics; it also was found to govern ``stationary'' phenomena in hydrodynamics and the theory of heat. Finally the so-called ``Cauchy--Riemann'' equations for real functions \(P, Q\) of \(x, y\) such that \(P + iQ\) is an analytic function of \(x + iy\), were known since the middle of the XVIII\supth{} century, and they implied that \(P\) and \(Q\) were solutions of the Laplace equation in 2 variables. Very early in the XIX\supth{} century, Gauss was well aware of this connection and of the fact that one obtained solutions of the Laplace equation in 2 variables by replacing the function \eqref{eq:2.10} by
\begin{equation}
    \label{eq:2.13}
    \Omega(x, y) = \iint_D \rho(\xi, \eta) \log\frac{1}{r(x, y, \xi, \eta)} d\xi d\eta
\end{equation}
for a bounded domain \(D\) in the plane. The development by Cauchy of the theory of holomorphic functions of a complex variable could thus be used to yield properties of harmonic functions of 2 variables, such as for instance the non existence of relative extrema for such a function in its domain of definition; it was then natural to conjecture that similar properties were also valid for harmonic functions of 3 (and later for \(n \geq 4\)) variables, although they had to be proved by other means.

The first paper dealing with \emph{general} boundary conditions for a partial differential equation was written in 1828 by George Green, a self-taught English mathematician (1793--1841); it is concerned with electrostatics and the general study in that theory of what Green for the first time calls \emph{potential functions}. By that he not only means the functions of the form \eqref{eq:2.10}, but also what will later be called \emph{simple layer potentials}, namely functions of the type
\begin{equation}
    \label{eq:2.14}
    \Omega(M) = \iint_\Sigma \frac{\rho(P)}{MP} d\sigma(P)
\end{equation}
where \(\Sigma\) is a smooth surface, \(\rho\) (the ``density'') a continuous function on \(\Sigma\) and \(d\sigma\) the element of area on \(\Sigma\); he was naturally led to such functions by the known experimental fact that on conductors the electric charges are concentrated on their surface.

Green was interested in the relations between the surface density \(\rho\) and the potential it defines. He first establishes the famous theorem which, for the operator \(\Delta\), generalizes to 3 dimensions the relation between a differential operator and its adjoint (Chapter \ref{ch:1}, formula \eqref{eq:1.5}):
\begin{equation}
    \label{eq:2.15}
    \iiint_V \left( u\Delta v - v\Delta u \right) d\omega = \iint_\Sigma \left( v\frac{\partial u}{\partial n} - u\frac{\partial v}{\partial n} \right) d\sigma
\end{equation}
where \(\Sigma\) is a smooth surface limiting a bounded volume \(V\), \(u\) and \(v\) are \(C^2\) in a neighborhood of \(\overline{V}\), \(\frac{\partial u}{\partial n}\) is the derivative of \(u\) along the exterior normal of \(\Sigma\) \footnote{Lagrange \cite[vol.~I, p.~263]{135} and Gauss \cite[vol.~V, p.~22]{82} had already obtained more particular relations of that kind between volume and surface integrals.}. He then has the original idea \footnote{It is of course the same idea which leads to the Cauchy formula giving the value of a holomorphic function inside a domain \(D\) when it is known on the boundary of \(D\). However, it is unlikely that Green knew Cauchy's papers} of considering a function \(u\) which, still \(C^2\) for all points different from a point \(M\) in \(V\), becomes infinite at \(M\) in such a way that the difference \(u(P) - (1/MP)\) is bounded when \(P\) tends to \(M\); he applies \eqref{eq:2.15} to the volume \(V\) from which a small ball of center \(M\) has been excised, and by letting the radius of the ball tend to 0, he obtains the formula
\begin{equation}
    \label{eq:2.16}
    4\pi v(M) + \iiint_V \left( u\Delta v - v\Delta u \right) d\omega = \iint_\Sigma \left( v \frac{\partial u}{\partial n} - u \frac{\partial v}{\partial n} \right) d\sigma
\end{equation}
provided of course the triple integral exists. Taking in particular \(u(P) = 1/MP\) would give for a solution \(v\) of \(\Delta v = 0\)
\begin{equation}
    \label{eq:2.17}
    4\pi v(M) = \iint_\Sigma \left( v \frac{\partial \left(\frac{1}{r}\right)}{\partial n} - \frac{1}{r} \frac{\partial v}{\partial n} \right) d\sigma \qquad (\text{with } r(P)=MP)
\end{equation}
in other words, an integral formula which would solve the Laplace equation when \(v\) \emph{and} \(\frac{\partial v}{\partial n}\) were known on \(\Sigma\). This was in agreement with what was known at the time for partial differential equations of the second order, such as the equation of vibrating strings (Chapter \ref{ch:1}, \ref{sec:1.2}). However, experi ments showed that \(v\) was entirely determined by its values on \(\Sigma\), and therefore it was not possible to take for both \(v\) and \(\frac{\partial v}{\partial n}\) on \(\Sigma\) \emph{arbitrary} continuous functions, so that the situation appeared quite different from the boundary conditions for hyperbolic equations. Furthermore, there was at least one case when an explicit formula gave \(v\) inside \(V\) by an integral extended to \(\Sigma\), mnamely the \emph{Poisson formula} for a ball \(V\) of center 0 and radius \(a\), published in 1820:
\begin{equation}
    \label{eq:2.18}
    v(M) = \frac{1}{4\pi} \iint_\Sigma \frac{a^2 - \rho^2}{ar^3} v(P) d\sigma
\end{equation}
with \(\rho = OM\). Green observed that one would have a similar formula for general domains \(V\):
\begin{equation}
    \label{eq:2.19}
    v(M) = \frac{1}{4\pi} \iint_\Sigma v(P) \frac{\partial G}{\partial n}(M, P) d\sigma
\end{equation}
by substituting in his formula \eqref{eq:2.16} for \(u\) a function \(G(M,P)\) such that: 1° in \(V \times V\), \(G\) is \(C^2\) provided \(M \neq P\) and \(\frac{\partial G}{\partial n}\) exists on \(\Sigma\); 2° \(G(M,P) - 1/MP\) remains bounded when \(P\) tends to \(M\); 3° \(G(M,P) = 0\) when \(M\) is in \(V\) and \(P\) on \(\Sigma\); 4° when \(M\) is fixed in \(V\), \(G\), as a function of \(P\), satisfies the Laplace equation in \(V\). He could not prove the existence of such a ``Green function'', but made it plausible by an appeal to experimental facts: when the surface \(\Sigma\) is connected to the ground, and an electric charge +1 is put at the point \(M\), it ``induces'' an electric charge on \(\Sigma\) such that the total potential of that charge and the punctual charge at \(M\) is 0 on \(\Sigma\); that potential should be the function \(G(M,P)\) (\cite{090}, \mytodo{add reference: [S, p.~347--358]}).

Finally, by an ingenious use of his formula \eqref{eq:2.15}, Green could prove that in \(V \times V\), one had \(G(P,M) = G(M,P)\) for \(M \neq P\).

\section{The Dirichlet principle}
\label{sec:2.4}

Gauss had very early been interested in the Laplace egquation, both in 2 variables in connection with his work on complex numbers, and in 3 variables in relation with his astronomical studies, and we have seen that in his 1813 paper on the attraction of spheroids, he had proved particular cases of the Green formula \eqref{eq:2.15}. After 1830, he devoted much of his time to the study of magnetism, both experimentally and theoretically, and thus was led to new research on potential theory, which he published in 1840 \cite[vol.~V, p.~197--242]{090}. In that paper, he quotes no other work on the subject, and it is very unlikely that he ever heard of Green (whose work was not widely known, even in England)\footnote{The fact that Gauss also uses the word ``potential'' with the same meaning may be attributed to the fact that the word (in its Latin form) was commonly used in the XVIII\supth{} century by ``natural philosophers''.}; he expands his 1813 formulas and obtains in this way some new particular cases of Green's formula \eqref{eq:2.15}, although he does not seem to have thought of formula \eqref{eq:2.16}. The closest approach to the latter is his famous ``mean value formula''
\begin{equation}
    \label{eq:2.20}
    v(0) = \frac{1}{4\pi} \iint_\Sigma v(P)d\sigma
\end{equation}
for a harmonic function \(v\) in a sphere \(\Sigma\) of center 0, for which it is quite surprising that he should not have observed that it was a special case of Poisson's formula \eqref{eq:2.18} which he cannot have failed to know.

As Green had done, Gauss was particularly interested in the behavior of simple layer potentials \eqref{eq:2.14} when \(M\) tends to a point on the surface \(\Sigma\); by a careful study, he shows that the potential \(\Omega\) is coutinuous everywhere, and that the normal derivatives at a point \(M_0\) of \(\Sigma\) exist on both sides of the surface, but have \emph{different values}, their difference being \(4\pi \rho(M_0)\); all this had been taken for granted without proof by Green.

Gauss attacked several problems related to potential theory, some of which were to become the focus of active research after 1930. One was the \emph{equilibrium problem}: find a distribution of electric charges on a closed surface \(\Sigma\) giving a potential which is \emph{constant} on \(\Sigma\); another consisted in replacing charges inside \(\Sigma\) by charges on \(\Sigma\) in such a way that the potential outside \(\Sigma\) remains the same (what would later be called a ``sweeping-out'' process), and Gauss showed that it could be solved if the equilibrium problem had a solution.

Regarding the latter, Gauss introduced a new idea which was to become quite central in potential theory: he observed that if the potential \(\Omega\) is given by \eqref{eq:2.14} with \(rho \geq 0\), and \(U\) is any continuous function on \(\Sigma\), then if \(\rho\) is chosen such that the integral \(\iint_\Sigma (\Omega - 2U)\rho d\sigma\) takes the \emph{smallest} possible value among all possible choices of \(\rho\), then \(\Omega - U\) is constant on \(\Sigma\), and he added that the existence of such a density \(\rho\) was obvious.

By adding to \(\Omega\) a suitable constant, this method of Gauss solved the problem of finding a harmonic function \(u\) in the volume \(V\), continuous in \(\overline{V} = V \cup \Sigma\), and equal on \(\Sigma\) to a given function \(U\) \footnote{If such a problem is solved, it implies the existence of the Green function: one considers the function \(u(M, P)\) harmonic in \(V\) (as a function of \(P\)) which takes the values \(-1/MP\) on \(\Sigma\); the Green function is then \(G(M, P) = u(M, P) + (1/MP)\), provided one shows that \(\frac{\partial G}{\partial n}\) exists and is continuous on \(\Sigma\).}. The same problem was considered a little later by W. Thompson (the future Lord Kelvin) in 1847 and by Dirichlet around the same time in his lectures (published long afterwards) \mytodo{add reference: [S, p.~380--387]}; it became known as the \emph{Dirichlet problem}. Their idea is similar to Gauss's: they consider the volume integral
\begin{equation}
    \label{eq:2.21}
    \iiint_V \left( \left(\frac{\partial v}{\partial x}\right)^2 + \left(\frac{\partial v}{\partial y}\right)^2 + \left(\frac{\partial v}{\partial z}\right)^2 \right) d\omega
\end{equation}
and the function \(v\) continuous in \(\overline{V}\) with continuous and bounded first derivatives in \(V\) (\(v\) taking the given values on \(\Sigma\)), for which the integral \eqref{eq:2.21} takes its smallest value; applying the standard techniques of the Calculus of variations, they easily show that such a function is indeed harmonic in \(V\).

The great success of this idea is probably due to the imaginative use Riemann almost immediately made of it, in his epoch-making papers on holomorphic functions, Riemann surfaces and abelian integrals. By considering the real and imaginary parts of such functions, he was the first to realize that the existence theorems he needed could be derived from similar existence theorems for these harmonic functions, which he thought he could prove by adapting Dirichlet's argument to similar integrals in 2 variables, called by him ``Dirichlet principle'' \cite[p.~97]{182}.

His magnificent results attracted considerable attention, but soon mathematicians realized that they rested on three properties for which W. Thompson, Dirichlet and Riemann did not give any proof at all:

1) For a given continuous function \(g\) on \(\Sigma\), there exist continuous functions \(v\) in \(\overline{V}\) whose restriction to \(\Sigma\) is \(g\) and for which the integral \eqref{eq:2.21} is meaningful.

2) If such functions exist, there is one for which the smallest value of \eqref{eq:2.21} is attained.

3) For that function \(v\), the second order derivatives \(\frac{\partial^2 v}{\partial x^2}, {\partial^2 v}{\partial y^2}, {\partial^2 v}{\partial z^2}\) exist.

However, in 1871, F. Prym presernited an example (for two variables and \(V\) the disk \(x^2 + y^2 < 1\)) where \emph{no function} \(v\) satisfying 1) existed \cite{181}\footnote{The discovery of that fact is usually attributed to Hadamard, who published a similar example in 1906 \cite[vol.~III, p.~1245--1248]{094}}. On the other hand, in 1870, Weierstrass observed that in all problems of the Calculus of variations which had been studied since the beginning of the XVIII\supth{} century, properties 2) and 3) had been taken for granted without any proof, and he gave a very simple example in which property 2) does not hold: the problem of minimizing the integral \(\int_{-1}^1 xy'^2dx\) among all \(C^1\) functions \(y\) defined in the interval [-1, 1] and satisfying the boundary conditions \(y(-1) = a, y(1) = b\), with \(a \neq b\) \mytodo{add reference: [S, p.~390--391]}. Spurred by these difficulties, Weierstrass and his pupils (P. Du Bois-Reymond, A. Kneser, S. Zaremba) undertook to put the Calculus of variations on sounder foundations and were able to rescue many classical results from the suspicion raised by such counterexamples. But the ``Dirichlet principle'' eluded their efforts, and it was only in 1899 that Hilbert, using new ideas in what was called his ``direct method'', was able to give a complete justification of the use Riemann had made of that ``principle'' \cite[vol.~III, p.~10--37]{111}.

\section{The Beer--Neumann method}
\label{sec:2.5}

We shall see in later chapters how the concepts and tools used by Hilbert and the Weierstrass school contributed to the birth of General Topology and later to the introduction of such notions as ``weak'' solutions of partial differential equations. Meanwhile, the challenge remained to prove the existence of a solution to the Dirichlet problem and similar boundary values problems for the Laplace equation, at least under conditions such as were used in Riemann's work. Between 1870 and 1890, that challenge was successfully taken up by three mathematicians: H.A. Schwarz around 1870, C. Neumann in 1877 and H. Poincaré in 1887.

We shall not discuss in detail the contributions of Schwarz and Poincaré, which did not influence directly the development of Functional Analysis. Both are based on the idea of \emph{approximation}: starting from known solutions of the Dirichlet problem for special kinds of domains, an approximation process enables one to get solutions for much more general domains. Schwarz limits himself to 2 variables; he first comnsiders domains limited by a convex polygon, for which it is possible to prove directly (by explicit construction) the existence of a conformal mapping on the unit disk, hence the existence of a solution of the Dirichlet problem (by transferring the Poisson formula from the circle to the polygon). Using the maximum principle, it is then possible to prove the existence of the solution for a \emph{convex} domain by approximating it by a sequence of inscribed convex polygons. A little later, he invented an ingenious ``alternating process'' which enabled him to show that when one can solve the Dirichlet problem for two domains in the plane, it is also possible to solve it for their union, and from that result he finally showed that the Dirichlet problem in the plane is solvable for any domain limited by piecewise analytic curves \cite[vol.~II, p.~133--210]{196}.

Poincaré's famous ``sweeping-out method'' applies to any number of dimensions. To solve the Dirichlet problem for a bounded domain \(V\) limited by a surface \(\Sigma\), he shows (using the maximum principle) that it is enough to consider the case in which the function given on \(\Sigma\) is the restriction to \(\Sigma\) of a func tion \(\Phi\) defined in a neighborhood \(W\) of \(\overline{V}\), of class \(C^2\) and such that \(\Delta\Phi \geq 0\). By Poisson's equation \eqref{eq:2.12}, \(\Phi\) is the sum of a harmonic function and a potential \(\Phi_0\) of masses \(\geq 0\). The fundamental idea is that if \(B\) is a ball contained in \(W\), it is possible to use the Poisson integral \eqref{eq:2.18} extended to the surface of \(B\) in order to replace \(\Phi_0\) by another potential which coincides with \(\Phi_0\) outside \(B\) and is smaller than \(\Phi_0\) inside \(B\); the masses inside \(B\) have been ``swept out'' on the surface of \(B\). One then takes an infinite sequence of balls \(B_n\) whose union is \(V\), and one applies the ``sweeping-out'' process repeatedly to the \(B_n\) in the order \(B_1, B_2, B_1, B_2, B_3, B_1, B_2, B_3, B_4, \ldots\) (each \(B_n\), is ``swept-out'' in finitely many times). The corresponding sequence of potent ials is decreasing, hence has a limit in \(V\); using Harnack's inequalities (consequences of the Poisson formula \eqref{eq:2.18}) and the maximum principle, Poincaré is able to show that this limit is a solution of the Dirichlet problem, provided the boundary \(\Sigma\) satisfies a ``regularity'' condition, namely, for any point \(M \in \Sigma\), there must be a small ball whose intersection with \(\overline{V}\) is reduced to \(M\) \cite[vol.~IX, p.~33--54]{177}; later, Zaremba could replace the small ball by a small cone of vertex \(M\) in that condition.

In contrast with Schwarz's and Poincaré's papers, the Beer--Neumann method was a landmark in Functional Analysis by introducing the first example of what was later to be called a ``Fredholm integral equation of the second kind''. Green's formula \eqref{eq:2.17} naturally introduced still another type of potential:
\begin{equation}
    \label{eq:2.22}
    u(M) = \iint_\Sigma \rho(P) \frac{\partial}{n} \left( \frac{1}{MP} \right) d\sigma
\end{equation}
which was harmonic outside the surface \(\Sigma\). It also occurred in the theory of magnetism, from which it got its name of \emph{double layer potential}: it was there conceived as the limit of a difference of two simple layer potentials, one with density \(\mu\) on \(\Sigma\), the other with density \(\mu\) on a surface \(\Sigma'\) parallel to \(\Sigma\) and at an ``infinitely small'' distance \(\varepsilon\); when \(\varepsilon\) tends to 0, \(\mu\) was supposed to increase to \(+\infty\) in such a way that the product \(\mu \varepsilon\) tended to \(rho\).

Such a potential had been shown to have near \(\Sigma\) a behavior quite similar to the normal derivative of a simple layer potential, studied by Gauss: when \(M\) tends to a point \(M_0\) of \(\Sigma\) along the normal to \(\Sigma\) at \(M_0\), \(u(M)\) tends to a limit on each side of \(\Sigma\), but these limits are \emph{different} in general; however, \(\frac{\partial u}{\partial n}\) is the \emph{same} on both sides.

Formula \eqref{eq:2.22} also had a nice geometric interpretation; one has \(\frac{\partial}{\partial n} \left( \frac{1}{MP} \right) = \frac{\cos \varphi}{MP^2}\) where \(\varphi\) is the angle between \(MP\) with the normal to \(\Sigma\) at \(P\), and \(\frac{\cos \varphi}{MP^2} d\sigma\) is the infinitesimal ``solid angle'' from which \(d\sigma\) is ``seen'' from the point \(M\).

Around 1860, C. Beer proposed to obtain a solution to the Dirichlet problem by formula \eqref{eq:2.22} for a suitable density \(\rho\) on \(\Sigma\). From the continuity properties of double layer potentials, it follows that if \(\Sigma\) is a smooth surface, and \(g(M)\) is the function on \(\Sigma\) to which the solution \(u(M)\) must be equal, the unknown density must satisfy the equation
\begin{equation}
    \label{eq:2.23}
    2\pi \rho(M) + \iint_\Sigma \rho(P) \frac{\partial}{\partial n}\left(\frac{1}{MP}\right) d\sigma = g(M) \quad \text{for } M \in \Sigma.
\end{equation}
He then concluded that one could compute \(\rho\) by the usual device of ``successive approximations'' (\ref{sec:2.1}) starting with \(\rho_0(M) = \frac{1}{2\pi} g(M)\) and defining recursively \(\rho_n(M)\) by
\begin{equation*}
    2\pi \rho_n(M) + \iint_\Sigma \rho_{n-1}(P) \frac{\partial}{\partial n}\left(\frac{1}{MP}\right) d\sigma = 0 \quad \text{for } n \geq 1
\end{equation*}
so that the series \(\rho(M) = \rho_0(M) + \rho_1(M) + \ldots + \rho_n(M) + \ldots\) would give the solution to \eqref{eq:2.23}; but he made no attempt to prove that the series converged.

In 1877, Carl Neumann attempted to give such a proof \cite{165}. He restricted himself to the case in which the domain \(V\) is bounded and \emph{convex}, but he allowed a non smooth boundary \(\Sigma\); equation \eqref{eq:2.23} must then be modified to
\begin{equation}
    \label{eq:2.24}
    4\pi \rho(M) = \iint_\Sigma \left( \rho(M) - \rho(P) \right) \frac{\cos \varphi}{MP^2} d\sigma + f(M)
\end{equation}
with \(f\) continuous on \(\Sigma\), and the successive approximations are given by \(4\pi\rho_0(M) = f(M)\) and, for \(n \geq 1\),
\begin{equation}
    \label{eq:2.25}
    4\pi \rho_n(M) = \iint_\Sigma \left( \rho_{n-1}(M) - \rho_{n-1}(P) \right) \frac{\cos \varphi}{MP^2} d\sigma
\end{equation}
Neumann's idea is to consider the maximum value \(L_n\) and minimum value \(l_n\) of \(\rho_n\), and to show that there is a number \(q\) such that \(0 < q < 1\) and
\begin{equation}
    \label{eq:2.26}
    L_n - l_n \leq \left( L_0 - l_0 \right) q^{n - 1}
\end{equation}
from which he majorizes \(|\rho_n(M)|\) by a multiple of \(q^n\) using \eqref{eq:2.25}, and he can conclude that the series \(\sum\limits_{n=0}^\infty \rho_n(M)\) converges to a continuous function.

To prove \eqref{eq:2.26}, Neumann divides \(\Sigma\) into two parts \(A_n, B_n\) respectively defined by the conditions
\[
    \begin{aligned}
        \frac{1}{2} \left( L_{n-1} + l_{n-1} \right) \leq \rho_{n-1}(P) \leq L_{n-1} \quad &\text{for } A_n \\
        l_{n-1} \leq \rho_{n-1}(P) < \frac{1}{2} \left( L_{n-1} + l_{n-1} \right) \quad &\text{for } B_n
    \end{aligned}
\]
and he deduces from \eqref{eq:2.25} that for all points \(M\) of \(\Sigma\)
\[
    \left( L_{n-1} - l_{n-1} \right) \left( A_n(M) + \frac{1}{2} B_n(M) \right) \leq 4\pi \rho_n(M) \leq \left( L_{n-1} - l_{n-1} \right) \left( \frac{1}{2} A_n(M) + B_n(M) \right)
\]
where \(A_n(M)\) and \(B_n(M)\) are the solid angles from which \(A_n\) and \(B_n\) are ``seen'' from \(M\). This implies
\[
    L_n - l_n \leq \left(L_{n-1} - l_{n-1}\right) q
\]
where \(q\) is the \emph{least upper bound} of the quantity
\begin{equation}
    \label{eq:2.27}
    \Lambda(M, M', A, B) = \frac{1}{4\pi} \left( \frac{1}{2} A(M) + B(M) + A(M') + \frac{1}{2} B(M') \right)
\end{equation}
when \(M\) and \(M'\) vary arbitrarily in \(\Sigma\), \(A\) is an arbitrary closed part of \(\Sigma\) and \(B\) its complement. One is thus faced with the \emph{purely geometric} problem of showing that \(q < 1\). The expression \eqref{eq:2.27} can be written
\[
    \frac{1}{4\pi} \left( A(M) + B(M) + A(M') + B(M') - \frac{1}{2} \left( A(M) + B(M') \right) \right)
\]
and also
\[
    \frac{1}{4\pi} \left( \frac{1}{2} \left( A(M) + B(M) \right) + \frac{1}{2} \left( A(M') + B(M') \right) + \frac{1}{2} \left( A(M') + B(M) \right) \right)
\]
and as one always has \(A(M) + B(M) \leq 2\pi\) (maximum value of the solid angle from which the whole of \(\Sigma\) is ``seen'' from one of its points), the problem can also be formulated in two equivalent ways:
\begin{equation}
    \label{eq:2.28}
    A(M) + B(M') \geq 4\pi r \quad \text{for an } r > 0,
\end{equation}
\begin{equation}
    \label{eq:2.29}
    A(M) + B(M') \leq 4\pi s \quad \text{for an } s > 1,
\end{equation}
for all points \(M, M'\) in \(\Sigma\) in two parts \(A, B\). The form \eqref{eq:2.28} of that condition immediately shows that there is an \emph{exceptional} type of convex set for which it cannot be satisfied, namely the case in which \(V\) is the \emph{intersection of two convex cones} (``double cone''): indeed we then have \(A(M) = B(M') = 0\) if \(A\) is the surface of one of the cones, \(B\) the surface of the other, \(M\) the vertex of \(A\) and \(M\) the vertex of \(B\). Furthermore, this particular choice of \(A, B, M\) and \(M'\) is the \emph{only one} for which \(A(M) + B(M')\) may be 0. However, when the exceptional case is excluded, Neumann concludes, from the fact that \(A(M) + B(M') > 0\) for all choices of \(A, B, M\) and \(M'\), that there is an \(r > 0\) for which \eqref{eq:2.28} is satisfied for all these choices, and does not give a proof of that assertion valid for all convex sets other than double cones. This gap in Neumann's proof seems to have remained undetected until Lebesgue drew attention to it is 1937 (\cite{138}, vol.~IV, p.~151--166). He shows in addition how one can fill in that gap by a compactness argument: there are two points \(M_0, M_0'\) in \(\Sigma\), limits of sequences \((M_k), (M'_k)\) such that for each \(k\) there is a splitting of \(\Sigma\) in two parts \(A_k, B_k\) such that \(A_k(M_k) + B_k(M'_k)\) tends to the l.u.b. \(4\pi s\) of \(A(M) + B(M')\) for all choices of \(A, B, M, M'\). On the other hand there are a point \(N\) of \(\Sigma\) and neighborhoods \(V(M_0), V(M'_0), V(N)\) of \(M_0, M'_0, N\) respectively in \(\Sigma\) such that the planes of support at all points of \(V(N)\) do not intersect \(V(M_0)\) nor \(V(M'_0)\) (it is here that the assumption that \(V\) is not a double cone is used); an elementary geometrical ar gument then gives an upper bound \(< 1\) for \(s\). Historically, such an argument would have been barely possible in the late 1870's, but I strongly doubt that C. Neumann was familiar enough with the use of the ``Bolzano--Weierstrass'' theorem (as it was called at that time) to have thought of it. He was apparently satisfied with the fact that for simple convex sets, such as ellipsoids, it was possible to compute explicity an upper bound \(< 1\) for \(s\).

C. Neumann dealt in the same way with the Dirichlet problem in the plane, with a similar gap in his proof.

For a long time, the restrictions on the surface \(\Sigma\) in all the existence proofs of the Dirichlet problem were thought to be imperfections of the methods of proof; but in 1912, Lebesgue gave an example (in 3 dimensions) of a bounded open set \(V\) (homeomorphic to a ball) such that there is a continuous function on the boundary \(\Sigma\) of \(V\), for which the Dirichlet problem has no solution (\cite{138}, vol.~IV, p.~131). This was the starting point of modern Potential theory, where, on one hand, the initial formulation of the Dirichlet problem is modified in such a way that it always has a unique ``solution'' for any bounded domain, the word ``solution'' being interpreted in some ``weak'' sense; on the other hand, the behavior of these ``weak'' solutions on the boundary of the domain is investigated under various conditions \cite{030}. The detailed history of that extensive theory is outside the scope of this book.

\chapter{The equation of vibrating membranes}
\label{ch:3}
\setcounter{equation}{0}

\section{H.A. Schwarz's 1885 paper}
\label{sec:3.1}

The same physical arguments which lead to the equation of vibrating strings (Chap. \ref{ch:1}, \ref{sec:1.2}, equation \eqref{eq:1.7}) apply to the small vibrations of a membrane which at rest is in the plane \(Oxy\), and has a constant density: if \(z = u(x, y, t)\) is the equation of its surface at time \(t\), the function \(u\) satisfies the equation
\begin{equation}
    \label{eq:3.1}
    \frac{\partial^2 u}{\partial x^2} + \frac{\partial^2 u}{\partial y^2} = \frac{\partial^2 u}{\partial t^2}
\end{equation}
(for suitable units of length and time). The usual method of ``separation of variables'' consists here in looking for solutions \(u(x, y, t) = v(x, y)w(t)\) and one finds for \(v\) the equation (also called ``Helmholtz's equation'')
\begin{equation}
    \label{eq:3.2}
    \frac{\partial^2 u}{\partial x^2} + \frac{\partial^2 u}{\partial y^2} + \lambda v = 0
\end{equation}
for a constant \(\lambda\). If in addition the membrane at rest is a bounded portion \(\Omega\) of the plane and is \emph{fixed} at its boundary \(\Sigma\) (which means that \(u(x, y, t) = 0\) for all \(t\) if \((x,y) \in \Sigma\)), \(\lambda\) must be \(> 0, w(t) = \sin \sqrt{\lambda t}\), and one has to find a solution \(v\) of \eqref{eq:3.2} which vanishes on \(\Sigma\) and is not identically 0. Contrasting with the easy solution of the corresponding problem for the vibrating string, the elucidation of that problem was going to challenge the ingenuity of mathematicians during the whole second half of the XIX\supth{} century.

Experimental evidence, as well as the explicit solution of the problem for very special domains \(\Omega\), such as a rectangle or a disk, showed that, just as in the case of the vibrating string, solutions of \eqref{eq:3.2} vanishing on \(\Sigma\) and not identically 0 could only exist when \(\lambda\) was equal to one of an infinite sequence \((\lambda_n)\) of real numbers \(> 0\) (the ``eigenvalues'' of the problem), tending to \(+\infty\).

The first attempt to prove such a result for general domains \(\Omega\) was made by H. Weber in 1869 \cite{247}, by an adaptation of the variational method used by Riemann for the Dirichlet problem. Using Green's formula (Chapter \ref{ch:2}, formula \eqref{eq:2.15}) he first shows that if \(\mu_1, \mu_2\) are two distinct eigenvalues, \(v_1, v_2\) corresponding ``eigenfunctions'', then
\begin{equation}
    \label{eq:3.3}
    \left(\mu_1 - \mu_2\right) \iint_\Omega v_1(x, y) v_2(x, y) dxdy = 0
\end{equation}
from which he deduces, as Poisson had done for ordinary differential equations (Chap. \ref{ch:1}, \ref{sec:1.3}), that the eigenvalues are necessarily real numbers. To determine the smallest eigenvalue \(\lambda_1\), he considers the Dirichlet integral
\begin{equation}
    \label{eq:3.4}
    F(v) = \iint_\Omega \left( \left(\frac{\partial v}{\partial x}\right)^2 + \left(\frac{\partial v}{\partial y}\right)^2 \right) dxdy
\end{equation}
for \(C^2\) functions \(v\) in \(\overline\Omega\), equal to 0 on \(\Sigma\) and subject to the additional constraint
\begin{equation}
    \label{eq:3.5}
    \iint_\Omega v^2 dxdy = 1.
\end{equation}

He assumes, as Riemann, that in this set \(\mathcal{F}_1\) of functions, there is one for which \(F(v)\) is equal to its greatest lower bound \(\lambda_1\) and by the usual methods of the Calculus of variations, he shows that this function \(v_1\) is a solution of \eqref{eq:3.2} for \(\lambda = \lambda_1\).

He next considers the subset \(\mathcal{F}_2\) of \(\mathcal{F}_1\) defined by the additional condition
\begin{equation}
    \label{eq:3.6}
    \iint_\Omega v(x, y)v_1(x, y)dxdy = 0,
\end{equation}
takes the function \(v_2 \in \mathcal{F}_2\) for which \(F(v_2)\) is equal to its greatest lower bound \(\lambda_2\) and shows that \(v_2\) is a solution of \eqref{eq:3.2} for \(\lambda = \lambda_2\). The induction process is then obvious, and Weber concludes that he has proved the existence of an increasing infinite sequence \(\left(\lambda_n\right)\) of positive eigenvalues to each of which there corresponds an eigenfunction \(v_n\) normalized by condition \eqref{eq:3.5}, and orthogonal to each other. But he does not try to prove that \(\lim\limits_{n \to \infty} \lambda_n = +\infty\), nor that functions in \(\mathcal{F}_1\) possess a ``Fourier expansion'' \(\sum\limits_n c_n v_n\) defined in the same manner as in the Sturm--Liouville problem (Chap. \ref{ch:1}, \ref{sec:1.3}, formula \eqref{eq:1.33}) (a result which he states however, without proof).

Weber's proofs were of course subject to the general criticisms of Weierstrass against the Calculus of variations, but no one seems to have tried to find more rigorous ones until 1885, In that year, H.A. Schwarz published a long paper on the theory of minimal surfaces, in which he had to consider a type of equation slightly more general than \eqref{eq:3.2}:
\begin{equation}
    \label{eq:3.7}
    \frac{\partial^2 v}{\partial x^2} + \frac{\partial^2 v}{\partial y^2} + \lambda^2 pv = 0
\end{equation}
where \(p\) is a continuous function in a domain \(D\), with values \(> O\); his arguments apply for any such function, but in fact he is only interested in the particular case \(p(x, y) = - 8/(1 + x^2 + y^2)^2\) \cite[vol.~I, p.~223--269]{195}.

Schwarz's paper is extremely remarkable by the originality of its methods, which do not seem to have been inspired by any previous work; it may be that the study of the Sturm--Liouville problem led him to arguments which later could be transferred almost verbatim to general integral equations with symmetric kernels (see Chap. \ref{ch:5}, \ref{sec:5.2}), but there is no hint in his paper of such an influence, and in fact he quotes nobody, not even Weber.

His starting point is not the problem of existence of eigenvalues \(\lambda^2\) for equation \eqref{eq:3.7}, but a ``Dirichlet problem'' for the equation
\begin{equation}
    \label{eq:3.8}
    \Delta w + \xi pw = 0
\end{equation}
depending on a parameter \(\xi\); he limits himself to the case where \(w\) is subject to the condition of being equal to 1 on the boundary \(\Gamma\) of \(D\). Using the time honored method of representing the solution as a power series in \(\xi\) (Chap. \ref{ch:2}, \ref{sec:2.1})
\begin{equation}
    \label{eq:3.9}
    w = w_0 + \xi w_1 + \ldots + \xi^n w_n + \ldots
\end{equation}
he takes for \(w_0\) the constant function equal to 1, and imposes on the \(w_n\) for \(n \geq 1\) to vanish on \(\Gamma\); they are then determined inductively by the equations
\begin{equation}
    \label{eq:3.10}
    \Delta w_n + pw_{n-1} = 0 \quad \text{for } n \geq 1.
\end{equation}
He assumes that the Green function \(G(M, P)\) for the domain \(D\) exists (remember that he himself had proved that existence in extensive cases (Chap. \ref{ch:2}, \ref{sec:2.4})); the properties of that function implied that for any function \(f\) continuous in \(\overline{D}\) the equation
\begin{equation}
    \label{eq:3.11}
    \Delta w + f = 0
\end{equation}
has a unique solution vanishing on \(\Gamma\), given by the formula
\begin{equation}
    \label{eq:3.12}
    w(M) = \frac{1}{2\pi} \iint_D f(P) G(M, P) d\omega \qquad (\text{with } d\omega = dxdy)
\end{equation}
Therefore his functions are given explicitly by
\begin{equation}
    \label{eq:3.13}
    w_n(M) = \frac{1}{2\pi} \iint_D p(P) w_{n-1}(P) G(M, P) d\omega.
\end{equation}

One must now investigate the convergence of \eqref{eq:3.9} for small enough values of \(|\xi|\), and it is here that Schwarz's original contributions begin. His main tool is the inequality named after him\footnote{That inequality had been discovered by Buniakowsky in 1859, but does not seem to have been noticed nor used by many mathematicians before 1885. If is of course a direct generalization of the corresponding inequality for finite sums, which goes back at least to Cauchy.}
\[
    \left( \iint_D fg d\omega \right)^2 \leq \left( \iint_D f^2 d\omega \right) \left( \iint_D g^2 d\omega \right)
\]
for any two functions \(f, g\) continuous in \(\Gamma\); this gives from \eqref{eq:3.13}
\begin{equation}
    \label{eq:3.14}
    \begin{gathered}
    4\pi^2\left(w_n(M)\right)^2 \leq \left(\iint_D p^2(P)G^2(M, P)d\omega\right) \left(\iint_D w^2_{n-1}(P)d\omega\right) \\
    \leq A \left( \iint_D w^2_{n-1}(P)d\omega \right)
    \end{gathered}
\end{equation}
where \(A\) is a constant independent of \(n\) (due to the properties of the Green function of a bounded domain). Schwarz is thus led to study the numbers
\begin{equation}
    \label{eq:3.15}
    W_{n, k} = \iint_D p w_k w_{n-k} d\omega
\end{equation}
which, using the \emph{symmetry} of the Green function, he shows are independent of \(k\), so that \(W_{n, k} = W_{n, 0}\), which he writes \(W_n\). He also proves that
\begin{equation}
    \label{eq:3.16}
    W_{n, k} = \iint_D \left(\frac{\partial w_{k+1}}{\partial x} \frac{\partial w_{n-k}}{\partial x} + \frac{\partial w_{k+1}}{\partial y} \frac{\partial w_{n-k}}{\partial y}\right) dxdy.
\end{equation}
Finally, using the Schwarz inequality, he obtains the relation
\begin{equation}
    \label{eq:3.17}
    W_n^2 \leq W_{n-1} W_{n+1},
\end{equation}
hence the sequence of numbers \(W_n/W_{n-1}\) is increasing; on the other hand, integrating \eqref{eq:3.14} gives \(W_{2n} \leq BW_{2n-2}\) for a constant \(B\) independent of \(n\), and therefore the limit of the sequence \((W_n/W_{n-1})\) is a finite number \(c > 0\). It follows then from \eqref{eq:3.14} that the series \eqref{eq:3.9} is absolutely and uniformly convergent in \(\overline{D}\) for \(|\xi| < 1/ \sqrt{c}\); the properties of the Green function enable one to show that the derivatives of \(w\) are also given by convergent series obtained by differentiating \eqref{eq:3.9} termwise, and that \(w\) then satisfies \eqref{eq:3.8} and is equal to 1 on the boundary \(\Gamma\).

But Schwarz goes one step further. He proves that when \(\xi = 1/\sqrt{c}\), the general term of the series \eqref{eq:3.9} tends uniformly to a limit \(U_1\) which is not identically 0 in \(\overline{D}\) but vanishes on the boundary and is solution of
\begin{equation}
    \label{eq:3.18}
    \Delta w + (1/c) pw = 0.
\end{equation}
He has thus proved the existence of the \emph{smallest eigenvalue} \(\lambda_1^2 = 1/c\) of the equation \eqref{eq:3.8} for functions vanishing on the boundary, and of the corresponding eigenfunction.

It should be observed here that these developments in fact are just another treatment of a ``crypto-integral'' equation (which Schwarz does not write, however). If one writes \(w = w_0 + \xi v\) and ``solves'' equation \eqref{eq:3.8} by formula \eqref{eq:3.12} (using the same idea as Liouville in 1837 to obtain his ``Volterra integral equation'' (Chap. \ref{ch:2}, \ref{sec:2.1} and Chap. \ref{ch:1}, \ref{sec:1.3}, equation \eqref{eq:1.35})), one gets for \(v\) this time a ``Fredholm integral equation''
\begin{equation}
    \label{eq:3.19}
    v(M) = g(M) + \frac{\xi}{2\pi} \iint_D p(P) G(M, P) v(P) d\omega
\end{equation}
with
\[
    g(M) = \frac{1}{2\pi} \iint_D G(M, P) p(P) d\omega.
\]
Schwarz's procedure is therefore essentially the same as C. Neumann's for the Dirichlet problem (Chap. \ref{ch:2}, \ref{sec:2.4}), at least as a starting point; the main difference is in the emphasis put by Schwarz on the dependence on the parameter \(\xi\).

To appreciate the originalityand power of Schwarz's method, it is perhaps not superfluous to show how it can be translated, almost without change, in the theory of self-adjoint compact operators in a separable Hilbert space \(E\). Suppose \(U\) is such an operator in \(E\), which in addition we suppose \emph{positive}, i.e. \((U\cdot f| f) \geq 0\) for all \(f \in E\). The spectrum of \(U\) then consists in a decreasing sequence (finite or infinite) \(\mu_1 \geq \mu_2 \geq \ldots \geq \mu_n \geq \ldots 0\), where each \(\mu_n\) is an eigenvalue counted a number of times equal to its multiplicity; 0 is always in the spectrum but \(\Ker(U)\) may be reduced to 0 or have infinite dimension; for each \(\mu_n\) there is an eigenvector \(\varphi_n\) of norm 1, such that \(E\) is the \emph{Hilbert sum} of the one-dimensional spaces \(\C \varphi_n\) and of \(\Ker(U)\). Let
\[
    w_0 = \sum_n d_n \varphi_n + w'_0
\]
with \(w'_0 \in \Ker(U)\), be the expression of a vector \(w_0 \in E\) for that decomposition. Then, for any \(m \geq 1\), we have
\[
    U^m \cdot w_0 = \sum_n \mu^m_n d_n \varphi_n
\]
and therefore the Schwarz series \eqref{eq:3.9} is equal to
\[
    w = \sum_{m=0}^\infty \xi^m U^m \cdot w_0 = \sum_n \left( \sum_{m=0}^\infty \xi^m \mu^m_n \right) d_n \varphi_n = \sum_n \frac{d_n}{1 - \xi \mu_n} \varphi_n
\]
provided \(|\xi| < \mu_1^{-1}\), and we have \(w = \xi U \cdot w + w_0 - w'_0\). For \(\xi = 1/\mu_1\),
\[
    \xi^m U^m \cdot w_0 = \sum_n \xi^m \mu^m_n d_n \varphi_n = \sum_n \left( \mu_n / \mu_1 \right)^m d_m \varphi_n
\]
tends to \(d_1 \varphi_1\) if \(\mu_1\) is a simple eigenvalue, to the sum of the \(d_n \varphi_n\) such that \(\mu_n = \mu_1\) in general. Finally, we have
\[
    W_m = \left( U^m \cdot w_0 | w_0 \right) = \left( U^{m-k} \cdot w_0 | U^k \cdot w_0 \right) = \sum_n \mu^m_n d^2_n
\]
from which it follows that if the \(d_n\) such that \(\mu_n = \mu_1\) are not all 0, the ratio \(W_m / W_{m-1}\) tends to \(\mu_1\); furthermore, one has
\begin{equation*}
    \begin{gathered}
        W_{2m} = \norm{U^m \cdot w_0}^2 = |\left( U^{m+1} \cdot w_0 | U^{m-1} \cdot w_0 \right)| \leq \norm{U^{m+1} \cdot w_0} \cdot \norm{U^{m-1} \cdot w_0} \\
        = \left( W_{2m-2} W_{2m+2} \right)^\frac{1}{2}.
    \end{gathered}
\end{equation*}
To get the inequality \(W_m \leq \left( W_{2m-2} W_{2m+2} \right)^\frac{1}{2}\) for all integers \(m\), it is enough to consider the unique compact positive operator \(V\) such that \(V^2 = U\), and apply to \(V\) the preceding argument. Of course the concept of the ``square root'' of a positive selfad joint operator was not available to Schwarz, and this is why he had to use the expression \eqref{eq:3.16} for his numbers \(W_n\).

In 1893, E. Picard published a short Comptes-Rendus Note (\cite{172}, vol.~II, p.~545--550) in which he went one step further. For any point \(M \in D\), the function \(w(M, \xi)\) given by Schwarz's series \eqref{eq:3.9} is holomorphic in the circle \(|\xi| < \xi_1 = 1/\sqrt{c}\), and Picard investigates the \emph{analytic continuation} of \(\xi \mapsto w(M, \xi)\) beyond that circle: he shows that such a continuation exists in a circle \(|\xi| < \xi_2\), of radius independent of \(M \in D\), and that it has a \emph{simple pole} with residue \(-\xi_1 U_1(M)\) at the point \(\xi_1\). He limits himself to the case in which \(p = 1\) and \(\Gamma\) is convex and smooth, and his idea is to adapt the method of C. Neumann (Chap. \ref{ch:2}, \ref{sec:2.4}) to evaluate the differences \(\left| \xi^n_1 w_n - \xi^{n-1}_1 w_{n-1} \right|\); with apparently the same gap as in Neumann's argument (the details are not given in the Note) he ``proves'' that there are constants \(C\) and \(q < 1\) independent of \(M\), such that \(\left| \xi^n_1 w_n - \xi^{n-1}_1 w_{n-1} \right| \leq Cq^n\), hence \(\left| \xi^n_1 w_n - U_1 \right| \leq C'q^n\) for another constant \(C'\), hence his result. writing
\[
    w = \frac{U}{1 - (\xi / \xi_1)} + v
\]
he looks for a power series development
\begin{equation}
    \label{eq:3.20}
    v = v_0 + \xi v_1 + \ldots \xi^n v_n + \ldots
\end{equation}
similar to \eqref{eq:3.9} but which should converge in a circle \(|\xi| < \xi_2\) with \(\xi_2 > \xi_1\). He determines the \(v_n\) by the successive approximations
\[
    \Delta v_0 - \xi_1 U_1 = 0, \qquad \Delta v_n + v_{n-1} = 0 \quad \text{for } n \geq 1
\]
with the boundary conditions: \(v_0 = 1\) on \(\Gamma\) and \(v_n = 0\) on \(\Gamma\) for \(n \geq 1\). Introducing numbers similar to the \(W_{n, k}\) of Schwarz, he is able to prove that the radius of convergence \(\xi_2\) of \eqref{eq:3.20} is finite, but he cannot show that there is an eigenfunction corresponding to \(\xi_2\) and vanishing on \(\Gamma\).

\section{The contributions of Poincaré}
\label{sec:3.2}

In 1890, H. Poincaré published in the American Journal of Mathematics a long paper developing some of his research done since 1887, which had been announced in three Comptes-Rendus Notes (\cite{177}, vol.~IX, p.15--113). The paper consists of two completely independent parts; in the first, he describes in detail his ``sweeping-out'' method for the solution of the Dirichlet problem (Chap. \ref{ch:2}, \ref{sec:2.4}). The second part is devoted to the \emph{cooling off} problem in the theory of heat, which had been treated by Fourier in some particular cases, for instance the cooling off of a sphere when the temperature is a function of the distance to the center (Chap. \ref{ch:1}, \ref{sec:1.2}). The \emph{general} cooling off problem had been presented by Fourier in the following form: given a solid body \(V\) of constant density, isotropic for the propagation of radiations, one has to find the temperature \(u(x,y,z,t)\) inside \(V\), as a function of the coordinates \(x, y, z\) and the time \(t\), when the outside temperature is 0. Fourier shows that the function \(u\) must satisfy inside \(V\) an equation (where \(a\) is constant)
\begin{equation}
    \label{eq:3.21}
    \frac{\partial u}{\partial t} = a^2 \Delta u
\end{equation}
and in addition is subject to the boundary condition on the surface \(\Sigma\) of \(V\)
\begin{equation}
    \label{eq:3.22}
    \frac{\partial u}{\partial n} + hu = 0
\end{equation}
where \(\frac{\partial u}{\partial n}\) is the normal derivative (towards the exterior) and \(h\) is a constant \(\geq 0\) (see Chap. \ref{ch:4}, \ref{sec:4.4}). The usual method of ``separation of wvariables'' led to solutions of the form \(u(x,y,z,t) = e^{-\lambda a^2 t} v(x,y,z)\), where \(v\) should be a solution of the Helmholtz equation
\begin{equation}
    \label{eq:3.23}
    \Delta v + \lambda v = 0
\end{equation}
with a \emph{different} boundary condition from the one deriving from the equation of vibrating membranes, namely
\begin{equation}
    \label{eq:3.24}
    \frac{\partial v}{\partial n} + hv = 0 \quad \text{on } \Sigma.
\end{equation}
In his 1869 paper, H. Weber had also cousidered that problem, but he had only described his variational method to obtain eigenvalues and eigenfunctions for the particular case \(h=0\). Poincaré apparently was unaware of Weber's paper and never mentioned it in his own work; what he does in 1890 is first to repeat Weber's arguments for the general boundary condition \eqref{eq:3.24}, replacing the Dirichlet integral by the function
\begin{equation}
    \label{eq:3.25}
    F(v) = h\iint_\Sigma v^2 d\sigma + \iiint_V \left(\divergence{v}\right)d\omega.
\end{equation}
Having thus obtained an increasing infinite sequence \(\left(\lambda_n\right)\) of eigenvalues and the corresponding sequence \(\left(v_n\right)\) of eigenfunctions, Poincaré is of course aware of the non rigorous character of his ``proof''; however, having for the time being no better arguments at his disposal, he takes for granted the existence of \(\lambda_n\) and \(v_n\) and proceeds to study them in more detail, and in the first place to prove that the sequence \(\left(\lambda_n\right)\) tends to \(+\infty\), a question which Weber had not been able to answer. In his attack on that problem, it is quite remarkable to see Poincaré introducing a whole batch of \emph{completely new ideas}. In the first place, he considers the eigenvalues as functions \(\lambda_n(h, V)\) of the constant \(h\) and the domain \(V\) and begins to study the way in which they \emph{depend} on \(h\) and \(V\), a trend of thought which will later blossom in the work of H. Weyl and R. Courant, and even now has not entirely lost its interest. Poincaré first shows that, for \(V\) fixed, \(\lambda_n(h, V)\) is increasing with \(h\), by an application of Green's formula to the eigenfunctions \(v_n(h, V), v_n(h', V)\) corresponding to two values of \(h\); as he wants to prove that \(\lambda_n(h, V)\) tends to \(+\infty\), he can assume that \(h = 0\), which implies that \(\lambda_1(0, V) = 0\) and \(v_1(0, V)\) is a constant.

The second idea is to \emph{decompose} \(V\) into a union of smaller solids \(V_1, V_2, \ldots, V_p\); the variational definition of \(\lambda_n\) enables him to prove that if \(p \leq n-1, \lambda_n(0, V)\) is \emph{at least equal} to the smallest of the numbers \(\lambda_2(0, V_1), \ldots, \lambda_2(0, V_p)\). Poincaré is thus led to \emph{minorize} \(\lambda_2(0, V)\) by a number depending only on the geometry of \(V\); by definition (since \(h = 0\)), this means finding a lower bound of the expression
\begin{equation}
    \label{eq:3.26}
    \frac{\iiint_V \left(\divergence{v}\right) d\omega}{\iiint_V v^2 d\omega}
\end{equation}
where \(v\) is a \(C^2\) function in \(\overline{V}\), subject to the condition
\begin{equation}
    \label{eq:3.27}
    \iiint_V v d\omega = 0.
\end{equation}
He assumes \(V\) is \emph{convex}; using polar coordinates and the standard methods of the Calculus of variations, he obtains as lower bound
\begin{equation}
    \label{eq:3.28}
    C \cdot \vol(V)/(\diam(V))^5
\end{equation}
where \(C\) is an absolute constant; one should here stress the fact that this Poincaré inequality is the first example of what we now call \emph{``a priori'' inequalities} (cf. Chap. \ref{ch:9}, \ref{sec:9.4}). Returning to the minoration of \(\lambda_n(0, V)\), he takes \(p = n-1\), assumes that \(V\) can be decomposed in \(n-1\) solids \(V_j\) which are convex and have a diameter tending to 0 with \(1/n\) and such that the ratio of their volume to the fifth power of their diameter tends to \(+\infty\) with \(n\); this gives him his conclusion.

Poincaré's next step is to investigate how the knowledge of the \(\lambda_n\) and \(v_n\) gives the solution of the cooling off problem, when the temperature \(u(x,y,z,O)\) is a known function \(f(x,y,z)\) in \(V\) at time \(t = O\). Fourier's method consists in writing
\begin{equation}
    \label{eq:3.29}
    u(x,y,z,t) = \sum_{n=1}^\infty c_n \exp\left( -\lambda_n a^2 t \right) v_n(x,y,z)
\end{equation}
which gives for the unknown coefficients \(c_n\) the condition that \(f = \sum\limits_{n=1}^\infty c_n v_n\), hence, from the orthogonality relations, \(c_n = \iiint_V fv_n d\omega\). But Poincaré, no more than Weber, is not at that time able to prove that this Fourier expansion converges to the function \(f\) in \(V\). However, taking his cue from Tchebychef's results in approximation theory, he shows (by a clever use of Green's formula) that the integral
\[
    S_n = \iiint_V \left( u - \sum_{k=1}^n c_k \exp\left( -\lambda_k a^2 t \right) v_k \right)^2 d\omega
\]
satisfies an inequality \(S_n \leq C \cdot \exp\left(-\lambda_{n+1} a^2 t\right)\) where \(C\) is independent of \(n\) and \(t\); in other words, for \(t > O\), he proves the convergence of the series in \eqref{eq:3.29} in what we now call the topology of Hilbert space\footnote{The method of least squares of Legendre--Gauss had led Tchebychef to define a ``best approximation'' to a function \(F\), by a linear combination \(\sum\limits_{j=1}^N a_j \psi_j\) of given functions \(\psi_j (1 \leq j \leq N)\), by the condition that
\[
    \sum_{k=1}^n \rho(x_k)\left( F(x_k) - \sum_{j=1}^N a_j \psi_j(x_k) \right)^2
\]
be minimum, for given points \(x_k (1 \leq k \leq n)\) and given ``weight'' \(\rho\). Gram, in 1883, generalized the problem by considering instead of a finite sum, an integral
\begin{equation}
    \tag{+}
    \label{eq:3.+}
    \int_a^b \rho(x) \left( F(x) - \sum_{j=1}^N a_j \psi_j(x) \right)^2 dx
\end{equation}
and he solved the problem in an original way, by applying to the \(\psi_j\) the ``orthogonalization process'' usually attributed to E. Schmidt \cite{089}. He was thus reduced to the case in which the \(\psi_j\) form an orthonormal system (for the measure \(\rho dx\)), where he showed that the \(a_j\) giving the best approximation \(a\) are the ``Fourier coefficients'' \(\int_a^b \rho(x)F(x)\psi_j(x)dx\). He went on to consider an \emph{infinite} orthonormal system \((\psi_n)\) and investigated under which conditions the minimum value \(\mu_n\) of the integral \eqref{eq:3.+} tends to 0 when \(n\) increases to \(+\infty\); he was able to see that this was linked to the ``completeness'' of the system \((\psi_n)\), i.e. the fact that no function other than the constant 0 is orthogonal to all \(\psi_n\). It is unlikely that Poincaré had any knowledge of Gram's paper.}.

The final section of Poincaré's paper (if we except a kind of postscript which we will discuss later in Chapter \ref{ch:4}, \ref{sec:4.4}) is devoted to the general study of the eigenfunctions \(v_n\) (their existence being admitted). In general, if \(v\) satis fies \eqref{eq:3.23} and \eqref{eq:3.24}, use of Green's formula shows that there is a formula similar to Green's expression of the potential (Chap. \ref{ch:2}, \ref{sec:2.3}, formula \eqref{eq:2.17})
\begin{equation}
    \label{eq:3.30}
    -4\pi v(M) = \iint_\Sigma v\left( \pder{T}{n} + hT \right)d\sigma
\end{equation}
where \(T\) (replacing the function \(1/r\)) is now \(\exp(i/\sqrt{\lambda} r)/r\). Using that formula, he is able to show, after a rather long discussion (patterned on the study of double layer potentials but more difficult), that \(v\) is continuous in \(\overline{V}\), and to obtain bounds for its derivatives in \(V\).

The second paper devoted by Poincaré to the equation of vibrating membranes (\cite{177}, vol.~IX, p.~123--196) is even more original. It is likely that in 1890, he was not aware of Schwarz's paper of 1885. The publication of Picard's note in 1893 immediately attracted his attention, and in a few months he had seen that by combining Schwarz's method and his ``a priori'' inequality of 1890, he could go beyond Picard and prove the analytic continuation of the function \(\xi \mapsto w(M, \xi)\) as a \emph{meromorphic function in the whole complex plane}, obtaining at the same time the existence of the long sought eigenvalues and eigenfunctions for the Helmholtz equation (with the same boundary condition as Schwarz).

Poincaré starts with a simplification and an impvrovement of his inequality for the expression \eqref{eq:3.26}; using Schwarz's inequality, he is able to replace his lower bound \eqref{eq:3.28} by \(C/(\diam(V))^2\) for a convex solid \(V\). He then only assumes that for a general solid \(V\) it is possible to decompose it in convex solids having arbitrary small diameters, and uses this idea of decomposition to prove the following crucial lemma: given \(p\) arbitrary \(C^2\) functions \(F_1, F_2, \ldots, F_p\) in \(\overline{V}\), it is possible to choose \(p\) numbers \(\alpha_1, \ldots, \alpha_p\) in such a way that, for \(v = \alpha_1 F_1 + \ldots \alpha_p F_p\), one has \(\iiint_V v d\omega = 0\) and the ratio \eqref{eq:3.26} is \emph{at least} \(L_p\), where \(L_p\) is a number which only depends on \(V\) and \(p\) (and not on the \(F_j\)) and \emph{tends to} \(+\infty\) with \(p\). This is simply done by decomposing \(V\) in the union of \(p-1\) convex subsets \(V_j\), and choosing the coefficients \(\alpha_j\) by the \(p-1\) conditions \(\iiint_{V_j} v d\omega = 0 \; (1 \leq j \leq p - 1)\).

Poincaré, as Picard, limits himself to the case in which the function \(p\) in equation \eqref{eq:3.8} is the constant 1, but considers a problem which slightly generalizes Schwarz's, namely he looks for a function \(v\) solution of
\begin{equation}
    \label{eq:3.31}
    \Delta v + \xi v + f = 0
\end{equation}
and vanishing on the boundary \(\Sigma\), with \(f\) an \emph{arbitrary} \(C^\infty\) function (if in Schwarz's equation \eqref{eq:3.8} with \(p = 1\), one writes \(w = w_0 + \xi v\), the equation for \(v\) is \eqref{eq:3.31} with \(f=w_0\)); he will make a very clever use of this arbitrariness. He starts by observing that Schwarz's method works just as well for arbitrary \(f\) as for \(f = 1\), and proves the existence of the solution of \eqref{eq:3.31} vanishing on \(\Sigma\) for \emph{small enough} \(|\xi|\); he writes it
\[
    v = [f, \xi] = v_0 + \xi v_1 + \ldots + \xi^n v_n + \ldots
\]
with \(\Delta v_0 + f = 0, \Delta v_n + v_{n-1}=0\) for \(n \geq 1\), the \(v_n\) all vanishing on \(\Sigma\).

For any given integer \(p\), he introduces \(p\) arbitrary coefficients \(\alpha_1, \ldots, \alpha_p\) and forms the function (defined at least for small \(\xi\))
\begin{equation}
    \label{eq:3.32}
    w = [\alpha_1 f + \alpha_2 v_0 + \ldots + \alpha_p v_{p-2}, \xi] = w_0 + w_1 \xi + \ldots + w_n \xi^n + \ldots
\end{equation}
Next, applying his lemma for the evaluation of the Schwarz integrals \(W_n\) \emph{corresponding to} \(w\), he is able to show that, for a suitable \emph{choice} of the \(\alpha_j\) the series \eqref{eq:3.32} \emph{converges for} \(|\xi| \leq L_p\) (uniformly in \(\overline{V}\)). But if one writes \(u_j = \left[v_{j-2}, \xi\right]\), one has
\begin{equation}
    \label{eq:3.33}
    \begin{dcases}
        \alpha_1 v + \alpha_2 u_2 + \ldots \alpha_p u_p = w \\
        v - u_2 \xi = v_0 \\
        u_2 - u_3 \xi = v_1 \\
        \dotfill \\
        u_{p-1} - u_p \xi = v_{p-2}
    \end{dcases}
\end{equation}
a linear system from which Cramer's formulas give
\begin{equation}
    \label{eq:3.34}
    v = P / D
\end{equation}
with
\[
    D =
    \left|
    \begin{matrix}
        \alpha_1 & \alpha_2 & \alpha_3 & \ldots & \alpha_{p-1} & \alpha_p \\
        1 & -\xi & 0 & \ldots & 0 & 0 \\
        0 & 1 & -\xi & \ldots & 0 & 0 \\
        \hdotsfor{6} \\
        0 & 0 & 0 & & 1 & -\xi
    \end{matrix}
    \right|
    = \alpha_p - \alpha_{p-1}\xi + \alpha_{p-2}\xi^2 - \ldots + (-1)^{p-1} \alpha_1 \xi^p
\]
and
\[
    D =
    \left|
    \begin{matrix}
        w & \alpha_2 & \alpha_3 & \ldots & \alpha_{p-1} & \alpha_p \\
        v_0 & -\xi & 0 & \ldots & 0 & 0 \\
        v_1 & 1 & -\xi & \ldots & 0 & 0 \\
        \hdotsfor{6} \\
        v_{p-2} & 0 & 0 & \ldots & 1 & -\xi
    \end{matrix}
    \right|
\]
which shows that \(P\), as \(w\), is equal to a series
\begin{equation}
    \label{eq:3.35}
    P = P_0 + P_1 \xi + \ldots + P_n \xi^n + \ldots
\end{equation}
where the \(P_n\) are \(C^\infty\) functions in \(\overline{V}\) vanishing on \(\Sigma\), the series being uniformly convergent in \(\overline{V}\) for \(|\xi| < L_p\), and all derivatives of \(P\) (with respect to \(\xi\) or to \(x,y,z\)) being obtained by derivating termwise the series. This shows that \(\xi \mapsto v(M, \xi)\) extends to a \emph{meromorphic} function in \(|\xi|<L_p\) with a finite anumber of poles, which are roots of \(D(\xi) = 0\) and \emph{independent} of \(M\); it is easy to show, using Green's formula, that these poles are all \emph{simple}. As this is true for any \(p, (\xi \mapsto v(M, \xi)\) extends to a meromorphic function in the \emph{whole complex plane}, with simple real and positive poles independent of \(M\); furthermore, for each one of these poles \(\lambda_n\), the function \(P(M, \lambda_n)\) satisfies \(\Delta P + \lambda_n P = 0\); in other words, one has found for each \(\lambda_n\) an eigenfunction \(u_n\) corresponding to that eigenvalue. In addition, Poincaré's \emph{a priori} inequality enables him to show that \(\lambda_n \geq c \cdot n^{2/3}\), where \(c\) is a constant,

The remainder of Poincaré's 1894 paper is devoted to two questions:

A) In the last 4 sections of the paper, he takes up again the problem of Fourier expansions (when the boundary condition is \(v=0\)). Attaching to the function \(f\) its ``Fourier coefficients'' \(c_n = \iiint_V f u_n d\omega\) (where the eigenfunctions \(u_n\) have been normalized by \(\iiint_V u^2_n d\omega = 1\)), he first deduces from the relations \(\Delta u_n + \lambda_n u_n = 0\) and Schwarz's inequality, that \(|u_n| \leq A \lambda_n\) in \(\overline{V}\) (\(A\) constant), and that the \(c_n\) are uniformly bounded. From that it follows that for \(\xi\) different from the eigenvalues the unique solution of \eqref{eq:3.31} vanishing on \(\Sigma\) is given by the absolutely and uniformly convergent series
\begin{equation}
    \label{eq:3.36}
    v = -\sum_n \frac{c_n u_n \xi^2}{\lambda^2_n \left( \xi - \lambda_n \right)} + v_0 + v_1 \xi;
\end{equation}
in addition, Poincaré shows that if the series \(\sum\limits_n c_n u_n\) is absolutely convergent, its sum is equal to \(f\); he cannot prove that for ``arbitrary'' functions \(f\) (probably at least \(C^4\)), vanishing on \(\Sigma\), the series converges, but he proves absolute convergence when in addition \(\Delta f\) and \(\Delta^2 f\) also vanish on \(\Sigma\).

B) Before returning to the question of Fourier expansions, Poincaré had tried to extend his results on the existence of eigenvalues and eigenfunctions for the boundary condition \eqref{eq:3.24} of the cooling off problem. He realizes that Schwarz's method would work, and therefore also his own existence theorem, \emph{provided} one could prove the existence of a ``Green function'' for the Laplace equation with that new boundary condition, i.e. a function \(G(M, P)\) having the same properties as the usual Green function, with the exception that, for \(M \in V, P \mapsto G(M, P)\) satisfies \eqref{eq:3.24} on the boundary.

\emph{In the special case} \(h=0\), C. Neumann, in his work on the Dirichlet problem, had shown how to obtain such a ``Green function'' (also named ``Neumann function'') when \(V\) is convex and not a double cone. He had observed that by changing the sign before the integral in the Beer--Neumann equation (Chap. \ref{ch:2}, \ref{sec:2.4}, formula \eqref{eq:2.23}), the solution of that new equation gave a density \(\rho\) such that the corresponding double layer potential, in the \emph{exterior} of \(V\) (complement of \(\overline{V}\)) is harmonic, tends to 0 at infinity and to \(-g\) on the boundary \(\Sigma\) (a solution to what is called the ``exterior Dirichlet problem''). From this result, he had shown how to obtain a solution of what is now called the \emph{Neumann problem} for the Laplace equation: find in \(V\) a harmonic function \(u\) such that \(u\) is continuous in \(\overline{V}\) and has on \(\Sigma\) a normal derivative \(\pder{n}{u}\) equal to a \emph{given} continuous function \(g\); a necessary condition for the existence of the solution (deduced from Green's formula applied to \(u\) and the constant 1) is that \(\iint_\Sigma g d\sigma = 0\). Neumann proves that this condition is sufficient (the solutions being determined up to an additive constant): he considers the \emph{simple layer} potential \(w\) defined by the density \(\frac{1}{4\pi} g\); it is continuous on \(\Sigma\) and its normal derivative jumps by \(-g\) when crossing \(\Sigma\) from the interior to the exterior. Neumann next takes the \emph{double layer} potential \(v\), solution of the \emph{exterior} Dirichlet problem which tends to \(-w\) on \(\Sigma\). Then the function \(u = v + w\) is harmonic outside \(\Sigma\), and 0 in the exterior of \(V\); as the normal derivative of \(v\) is the same on both sides of \(\Sigma\), it follows at once that \(\pder{u}{n}\) tends to \(g\) from \(\Sigma\) from the interior of \(\)V, and therefore solves the Neumann problem.

C. Neumann had not been able to solve the corresponding problem when the boundary condition is \(\pder{u}{n} + hu = g\) for a an constant \(h > 0\). Poincaré tried to solve the problem by representing \(u\) as a power series in \(h, u = u_0 + h u_1 + \ldots + h^n u_n + \ldots\), and was indeed able to obtain in that way (using Neumann's results) a series convergent for all \(h \geq O\), uniformly in \(\overline{V}\); however, for the first derivatives, his method could omnly prove uniform convergence in compact subsets of \(V\), so that it was impossible to give meaning to \(\pder{u}{n}\) on the boundary \(\Sigma\), and to show that u was indeed a solution of the problem. The most interesting result in this attempt is that Poincaré, probably for the first time in history, arrives at the idea of ``weak'' solution of a boundary problem; he shows that his function \(u\) is such that, for \emph{any} function \(v\) which is \(C^2\) in \(\overline{V}\), one has
\begin{equation}
    \label{eq:3.37}
    \iiint_V u\Delta v d\omega + \iint_\Sigma gv d\sigma = \iint_\Sigma \left( \pder{v}{n} + hv \right) u d\sigma
\end{equation}
and adds that ``physically'' this is equivalent to a genuine solution.

The last of the three long papers of Poincaré on partial differential equations was witten in 1895 (\cite{177}, vol.~IX, p.~202--272). Although it is the one which contains the smallest number of new results, it probably had a greater influence than the others. From his work both on the Dirichlet problem and on the equation of vibrating membranes, Poincaré had become convinced that there were also ``eigenvalues'' and ``eigenfunctions'' linked to the Dirichlet problem. For us this is completely obvious, for if we look for a solution of \(\Delta u = 0\) taking given values on the boundary \(\Sigma\) of \(V\), we extend the function \(g\) given on \(\Sigma\) to a \(C^2\) function \(h\) in \(\overline{V}\) (when this is possible); replacing \(u\) by \(v = u-h\), we have to find a solution of \(\Delta v + f = O\), with \(f = \Delta h\), which vanishes on \(\Sigma\), and this is just the special case of Schwarz's problem for the equation \eqref{eq:3.31} with \(\xi = 0\).

At that time, however, nobody had yet thought of this simple argument\footnote{It is explicitly mentioned in 1909 by E.E. Levi \cite[vol.~II, p.~302--313]{145}; the first statement and proof of the existence of a continuous function in the whole space \(\R^3\) extending a given function defined and continuous in a closed subset (\emph{i.e.} what we now call the Tietze--Urysohn theorem) is due to Lebesgue in 1907 \cite[vol.~IV, p.~99--100]{138}.}, and Poincaré's reasoning is quite different and much more circuitous. He observes that one can formulate both the interior and exterior Dirichlet problems as special cases of the problem which consists in finding a double layer poten tial \(W\) (for a density on \(\Sigma\)) such that, for \(s \in \Sigma\),
\begin{equation}
    \label{eq:3.38}
    W(s^-) - W(s^+) - \lambda \left( W(s^-) + W(s^+) \right) = 2\Phi(s)
\end{equation}
where \(W(s^-)\) is the limit of \(W\) at \(s\) along the interior normal, \(W(s^+)\) its limit along the exterior normal, \(\lambda\) is a complex parameter and \(\Phi\) a given function on \(\Sigma\); the values \(\lambda = 1\) and \(\lambda = -1\) correspond respectively to the interior and the exterior Dirichlet problem. To this general problem Poincaré associates a new variational problem: for any simple layer potential \(\Psi\) defined by a density on \(\Sigma\), he considers the ratio \(J/J'\), where \(J\) is the Dirichlet integral \(\iiint (\grad \Psi)^2 d\omega\) extended over \(V\), and \(J'\) the integral of the same function, extended to the exterior of \(V\). The usual non rigorous arguments lead him to conjecture: 1° the existence of an increasing sequence \(0 = \lambda_0 \leq \lambda_1 \leq \ldots \leq \lambda_n \leq \ldots\) of eigenvalues, and: 2° for each \(\lambda_i\), the existence of a simple layer potential \(\Phi_i\), such that, on \(\Sigma\),
\begin{equation}
    \label{eq:3.39}
    \pder{\Phi_i}{n}(s^-) + \lambda_i \pder{\Phi_i}{n}(s^+) = 0;
\end{equation}
in addition, for \(i \neq j, \iiint_V \grad(\Phi_i) \cdot \grad(\Phi_j) d\omega = 0\). Normalizing the \(\Phi_i\) by \(\iiint_V (\grad \Phi_i)^2 d\omega = 1\), he assumes that there is a Fourier expansion \(\Phi = \sum\limits_i c_i \Phi_i\) of the given function \(\Phi\) on \(\Sigma\), and ``solves'' the equation \eqref{eq:3.39} by
\[
    W(s^-) = \sum_i A_i \Phi_i(s), \qquad W(s^+) = -\sum_i \lambda_i A_i \Phi_i(s)
\]
with \(A_i = 2c_i / (1 + \lambda_i - \lambda (1 - \lambda_i))\).

All this is of course presented by Poincaré as purely conjectural, and as a motivation for his detailed study (by methods inspired by those of Schwarz) of the ratio \(J/J'\), which forms the central part of his 1896 paper; but the only positive result he is able to deduce from his study is that the Beer--Neumann series (Chap. \ref{ch:2}, \ref{sec:2.4}) converges, not only for convex domains \(V\), but also for domains \(V \subset \R^3\) having the following property: when \(\R^3\) is imbedded in the 3-dimensional sphere \(\Sph_3\) by adjoining a point at infinity, \(V\) can be transformed into a ball by a homeomorphism of \(\Sph_3\) onto itself, leaving fixed the point at infinity, and which is \(C^2\) in \(\Sph_3\) as well as the inverse homeomorphism\footnote{Without the slightest justification, Poincaré claims as ``clear'' the fact that this property holds for \emph{any} bounded domain \(V\) such that the boundary \(\Sigma\) is a smooth simply connected surface (\cite{177}, vol.~IX, p.~223--224). With the tools of modern Differential Topology, it is now possible to prove that theorem. But in 1895, Poincaré was just beginning to formulate the first notions of that theory, and one wonders if he realized the difficulties which lay in the way of a rigorous proof if he had tried to write it down (when smoothness conditions on \(\Sigma\) and on the homeomorphism are dropped, the result is known to be false, a counterexample being the famous ``Alexander horned sphere'').}.

Almost immediately after the publication of Poincaré's papers, several mathematicians were able to complete and extend his results. In 1898, E. Le Roy \cite{144} proved the existence of the simple layer potentials \(\Phi_i\) conjectured by Poincaré in his 1896 paper; he replaced the ratio \(J/J'\) by \((J+J')/I\), where \(I\) is the surface integral \(\iint_\Sigma \rho^2 d\sigma, \rho\) being the density on \(\Sigma\) corresponding to the simple layer potential \(\Psi\), and adapted the methods of Schwarz and Poincaré to the correspond ing variational problem. In 1899, S. Zaremba \cite{232}, by a modification of the method of solution of Neumann's problem used by Poincaré, could complete the latter's solution of the ``cooling off'' problem, proving that the ``weak'' solution of Poincaré was a genuine one. In 1901, Zaremba and W. Stekloff, independently, finally showed that one could drop the global topological property of the domain \(V\) which Poincaré and Le Roy had used, and even weaken the ``smoothness'' conditions on \(\Sigma\); they made essential use of a paper of Liapounoff published 3 years earlier \cite{147}, in which he was able to prove the existence of the normal derivative on \(\Sigma\) of the solution of Dirichlet's problem under these less stringent conditions.

\chapter{The idea of infinite dimension}
\label{ch:4}
\setcounter{equation}{0}

\section{Linear algebra in the XIX\supth{} century}
\label{sec:4.1}

I think that in order to understand the trend of ideas which led to Functional Analysis, it is useful to summarize the evolution of linear algebra during the XIX\supth{} century. Until around 1830, it had consisted in the study of systems of linear equations in any number of variables, with real or complex coefficients, most of the times limited to the case in which the number of equations was equal to the number of variables; the Cramer formulas gave the unique solution when the deter minant of the system was not 0, but not much effort was spent on the elucidation of the other cases; the only result which was used occasionally was the fact that a system of \(m\) homogeneous equations in \(n > m\) variables always had a non trivial solution (obvious by induction on \(m\)).

Linear changes of variables
\begin{equation}
    \label{eq:4.1}
    y_j = \sum_{k=1}^n a_{jk} x_k \quad (1 \leq j \leq m)
\end{equation}
had been familiar since the XVITI\supth{} century (mostly for \(m=n \leq 3\)). It naturally led to computations done, not on numbers, but on \emph{rectangular arrays} \((a_{jk})\) of numbers, which ``represented'' these changes of variables. Beginning with Gauss, this trend was systematized in the 1850's by Sylvester and Cayley in the theory of \emph{matrices}.

Ever since the invention of cartesian coordinates (``analytic geometry'', as it came to be called in the XVIII\supth{} century), mathematicians had known how to interpret geometrically computations on systems of 2 or 3 variables, and many had envisioned the possibility of similarly interpreting computations on systems of any number \(n\) of vaviables in a ``geometry in \(n\) dimensions'', which however would be devoid of ``reality''. After 1840, mainly under the influence of Hamilton and Cayley, this geometrical language was gradually adopted by more and more mathematicians, and had become commonplace at the end of the century. But in the XIX\supth{} century, after 1822 ``geometry'' essentially meant \emph{projective} geometry, and most ``geometric'' interpretations of computations were done, not in the vector space \(\R^n\) or \(\C^n\), but in the complex projective spaces \(\mathbb{P}_n(\C)\); for instance, the relations \eqref{eq:4.1} for \(m = n\) were interpreted as defining also a \emph{projective transformation} in \(\mathbb{P}_{n-1}(\C)\) sending the point of homogeneous coordinates \((x_j)\) to the point of homogeneous coordinates \((y_j)\), and the efforts of Grassmann and Peano to introduce vector spaces in an axiomatic way were persistently ignored until 1900.

Between 1850 and 1880 are proved the main theorems of linear algebra, concerning what are called the ``reductions'' of square matrices. One of these is the problem of finding, for agiven square matrix \(U\), an invertible matrix \(P\) such that \(PUP^{-1}\) has a ``reduced'' unique canonical form, which here (for complex matrices \(U\)) means a diagonal array of Jordan matrices; this is the way Jordan himself treats the problem, improving on a previous result of Grassmann, who had proved the existence of a ``reduced'' triangular matrix \(PUP^{-1}\) for any \(U\) (using already the intrinsic notion of endomorphism instead of the notion of square matrix).

Unfortunately, another type of ``reduction'' interfered with the preceding one. To a quadratic form \(\sum\limits_{j \leq k} a_{jk} x_j x_k\) corresponds the \emph{symmetric} matrix \((a_{jk}) = U\), and it was well known since Cauchy that if \(U\) is \emph{real} it is possible to find an in vertible real matrix \(P\) (which may even be supposed to be \emph{orthogonal}) such that \(PUP^{-1}\) would be a (real) \emph{diagonal} matrix; this is equivalent to finding an orthogonal change of variables for which the quadratic form became equal to a linear combination \(\sum\limits_{j=1}^n \lambda_j y_j^2\) of squares, the \(\lambda_j\) being the elements of the diiginal matrix \(PUP^{-1}\), or equivalently the roots (with their multiplicity) of the ``characteristic equation'' \(\det{(U - \lambda I)} = 0\). Weierstrass, who was the first to find the ``Jordan normal form'' of a square complex matrix (which Jordan only discovered independently 2 years later\footnote{Jordan was not dealing with matrices having elements in \(\R\) or \(\C\), but with matrices having elements in a \emph{finite field} (\cite{123}, p. 114--126).}), presented it as a generalization of the ``reduction'' of a quadratic form, by considering a \emph{bilinear form} \(\sum\limits_{j,k} a_{jk} x_j y_k\) (with \(U = (a_{jk})\) an \emph{arbitrary} square matrix) and applying to the \(x_j\) and \(y_j\) two ``contragredient'' changes of variables, i.e. such that the bilinear form \(\sum\limits_{j,k} x_j y_k\) remains invariant; this amounts to replacing \(U\) by a matrix \(PUP^{-1}\). When, in 1878, Frobenius gave a systematic account of these results (\cite{078}, vol.~I, p.~343--405), he deliberately abandoned the language of matrices in favor of the language of bilinear forms, defining the ``product'' (\emph{Faltung}) of two bilinear forms \(A(x,y), B(x,y)\), as \(\sum\limits_{k=1}^n \frac{\partial A}{\partial y_k} \frac{\partial B}{\partial x_k}\)!\footnote{In 1896, Pincherle reinterpreted Weierstrass's results in terms of endomorphisms (\cite{173}, vol.~I, p.~358--367).}

Finally, the concept of duality in \emph{vector spaces} was completely foreign to mathematicians until 1900. Duality was well understood in the realm of \emph{projective geometry} (it had been one of the big discoveries of the early XIX\supth{} century), as a bijection of points on planes (in projective space of 3 dimensions) and later as a bijection of points on hyperplanes in any number of dimensions. But linear forms were identified with the systems of their coefficients, ``vectors'' and ``forms'' being thus both ``\(n\)-tuples'' of numbers, which one had to distinguish, according to the way they behaved under changes of variables, by the awkward concepts of ``contragredient'' and ``cogredient'' systems. This identification of a vector space and its dual was reverberated in the identification of endomorphisms with bilinear forms, mentioned above \footnote{In modevrn linear algebra, the space of endomorphisms of a finite dimensional vector space \(E\) is identified with the tensor product \(E^* \otimes E\), whereas the space of bilinear forms on \(E \times E\) is identified with \(E^* \otimes E^*\).}.

To sum up, at the end of the XIX\supth{} century, the main results of linear and multilinear Algebra had been found but were expressed through insufficiently clarified notions. They could therefore be of no help to the generalizations of linear Algebra to infinite dimensional spaces which were called forth by the development of Functional Analysis; these had to go through the same painful stages, first linear equations, then determinants, later bilinear forms, matrices, and only at the very end vector spaces and linear maps; in other words, the \emph{historical evolution}, just as for finite dimensional linear algebra, was exactly in the reverse order of what we not consider to be the \emph{logical} order!

\section{Infinite determinants}
\label{sec:4.2}

The first appearance of infinite systems of linear equations in infinitely many unknowns seems to occur in Fourier's work on the theory of heat. He has to determine an infinite sequence \((a_m)_{m \geq 1}\) of coefficients such that the relation
\begin{equation}
    \label{eq:4.2}
    1 = \sum_{m=1}^\infty a_m \cos(2m - 1)y
\end{equation}
holds for all \(y\) (\cite{067}, vol.~I, p.~149). Fourier's idea is to take derivatives of all orders of both sides of \eqref{eq:4.2} and identify them for \(y = 0\), which gives him the infinite system of linear equations for the \(a_m\)
\begin{equation}
    \label{eq:4.3}
    \begin{dcases}
        1 = \sum_{m=1}^\infty a_m \\
        0 = \sum_{m=1}^\infty (2m - 1)^2 a_m \\
        0 = \sum_{m=1}^\infty (am-1)^4 a_m \\
        \dotfill
    \end{dcases}
\end{equation}
To solve it, he considers the first \(k\) equations where he replaces the \(a_m\) for \(m > k\) by 0; he then solves that system by Cramer's formulas, which give him a system of \(k\) numbers \(a_1^{(k)}, a_2^{(k)}, \ldots, a^{(k)}_k\), and lets \(k\) tend to infinity in each expression of \(a_m^{(k)}\) for fixed \(m\). Using the formulas giving Vandermonde determinants, he obtains
\[
    a_1^{(k)} = \frac{3^2 \cdot 5^2 \ldots (2k-1)^2}{8 \cdot 24 \ldots (4k^2 - 4k)}
\]
tending to \(a_1 = 4/\pi\), and
\[
    \frac{a_{m+1}^{(k)}}{a_m^{(k)}} = \frac{2m-1}{2m+1} \cdot \frac{m+k}{m-k}
\]
which gives him \(a_m = (-1)^{m-1} 4/\pi(2m-1)\); when later in his book he proves the general formula giving the Fourier coefficients, he can of course check that these values of the \(a_m\) are correct. But he never bothered to give any justification of his procedure, where all questions of convergence are completely disregarded; that procedure could of course be repeated for any infinite system
\begin{equation}
    \label{eq:4.4}
    \sum_{k=1}^\infty a_{jk} x_k = b_j \qquad (j = 1, 2, \ldots)
\end{equation}
but nobody undertook to justify it before 1885\footnote{During that period, Fourier's method was used in two little-known papers, one by Fürstenau in 1860 on the computation of roots of an algebraic equation, and another by Kötteritzsch in 1870, for a system \eqref{eq:4.4} in which the \(a_{jk}\) are 0 for \(j > k\) (see \cite{184}, p.~8--12).}. In that year, P. Appell met such a system with \(a_{jk} = a_k^j\) for a given sequence \((a_k)\), in a question relative to elliptic functions, and used the same method as Fourier; his paper attracted Poincaré's attention, and he showed that for such a ``generalized Vandermonde system'', the procedure was justified provided the infinite product \(F(z) = \prod\limits_{k=1}^\infty \left( 1 - \frac{z}{a_k} \right)\) was convergent for all complex numbers \(z\).

The next year, he returned to the subject, in relation with a paper published in 1877 by the American astronomer and mathematician G.W. Hill on the lunar theory \cite{114}. Hill proposed a new approach which rested on the integration of a second order differential equation
\begin{equation}
    \label{eq:4.5}
    w'' + \left( \sum_{n=-\infty}^{+\infty} \theta_n e^{nit} \right) w = 0
\end{equation}
where the \(\theta_n\) are constants, and one looks for a solution of period \(2\pi\). Hill writes such a solution as a trigonometric series
\begin{equation}
    \label{eq:4.6}
    w = \sum_{n=-\infty}^{+\infty} b_n e^{i(n+c)t}
\end{equation}
and substituting in \eqref{eq:4.5}, obtains for the coefficients \(b_n\) the infinite system of equations
\begin{equation}
    \label{eq:4.7}
    \sum_{k=-\infty}^{+\infty} \theta_{n-k} b_k - (n+c)^2 b_n = 0, \qquad -\infty < n < +\infty.
\end{equation}
He probably was unaware of Fourier's procedure, but used a similar one, keeping this time the equations \eqref{eq:4.7} for \(-p \leq n \leq p\), replacing in these equations the \(b_m\) by 0 for \(m < -p\) or \(m > p\), and letting \(p\) tend to infinity in the solutions of the system thus obtained.

Poincaré considers a general system \eqref{eq:4.4}, where he supposes that \(a_{jj}=1\) for all \(j\) (one can always reduce \eqref{eq:4.4} to such a system by dividing the \(j\)-th equation by \(a_{jj}\), when \(a_{jj} \neq 0\)). His idea is to compare the determinant \(D_n = \det(a_{jk})_{1 \leq j, k \leq n}\) to the product \(P_n = \prod\limits_{j=1}^n \left( \sum_{k=1}^n |a_{jk}| \right)\). It is clear from the definition of a determinant that \(|D_n| \leq P_n\), and from the assumption on the diagonal terms, one has also \(|D_m - D_n| \leq P_m - P_n\); this inequality immediately gives Poincaré's \emph{sufficient} condition for the existence of \(D = \lim\limits_{n \to \infty} D_n\) , namely that the double sum \(\sum\limits_{j \neq k} |a_{jk}|\) be \emph{finite}. Furthermore, Poincaré shows that, when the \(k\)-th column of \(D\) is replaced by a sequence \((b_j)\) which is \emph{bounded}, there is still convergence for the new ``infinite determinant'', and that there is a unique bounded solution \(x_k\) of \eqref{eq:4.4} given by the usual Cramer formulas (with ``infinite determinants'' of course).\footnote{One must beware of the fact that the Fourier method (when no condition is imposed on the \(a_{jk}\)) may very well give convergent ``infinite determinants'', but the values given by the Cramer formulas may be such that the left hand sides of \eqref{eq:4.4} are \emph{divergent} series. An example is given by taking \(a_{jk} = 0\) if \(j > k, a_{jk} = 1\) if \(j \leq k, b_j = (-1)^j\); one finds as a ``solution'' \(x_k = 2(-1)^k\).} Finally he extends his results to doubly infinite systems
\begin{equation}
    \label{eq:4.8}
    \sum_{k=-\infty}^{+\infty} a_{jk} x_k = b_j \qquad (-\infty < j < +\infty)
\end{equation}
with the same restriction \(a_{jj} = 1\) for all \(j\); in particular he shows that Hill's method is justified for the system \eqref{eq:4.7} (\cite{177}, vol.~V, p.~95--107).

Ten years later, H. von Koch \cite{220} refined and generalized Poincaré's results. Insteadof making assumptions on the diagonal terms, he writes the coefficients \(\delta_{jk} + c_{jk}\) instead of \(a_{jk}\) (with the Kronecker delta), and uses the expression of a determinant \(\Delta_n = \det(\delta_{jk} + c_{jk})_{1 \leq j, k \leq n}\) as a sum of principal minors
\begin{equation}
    \label{eq:4.9}
    \begin{gathered}
        \Delta_n = 1 + \sum_{s=1}^n c_{ss} + \frac{1}{2!} \sum_{s_1, s_2}
        \left|
        \begin{matrix}
            c_{s_1 s_1} & c_{s_1 s_2} \\
            c_{s_2 s_1} & c_{s_2 s_2}
        \end{matrix}
        \right| + \\
        + \frac{1}{3!} \sum_{s_1, s_2, s_3}
        \left|
        \begin{matrix}
            c_{s_1 s_1} & c_{s_1 s_2} & c_{s_1 s_3} \\
            c_{s_2 s_1} & c_{s_2 s_2} & c_{s_2 s_3} \\
            c_{s_3 s_1} & c_{s_3 s_2} & c_{s_3 s_3}
        \end{matrix}
        \right|
        + \ldots
    \end{gathered}
\end{equation}
(an expression which will be the starting point of Fredholm's theorems on integral equations 4 years later (Chap. \ref{ch:5}, \ref{sec:5.1})). He is thus able to replace Poincaré's criterion for convergence by a weaker one: it is enough that the sums \(\sum\limits_j |c_{jj}|\) and \(\sum |c_{i_1 i_2} c_{i_2 i_3} \ldots c_{i_p i_1}|\) (extended to all sequences \((i_1, i_2, \ldots, i_p)\) of distinct indices) be finite. Another convergence criterion is that the sum \(\sum\limits_j |c_{jj}|\) and \(\sum_{j,k} |c_{jk}|^2\) be finite.

\section{Groping towards function spaces}
\label{sec:4.3}

% It should not be believed that set-theoretic concepts in
% mathematics were unknown before Boole (1847) or Cantor; they can be traced at least as far back as Aristotle., The use of the word "class" (or, in German, "Gebiet", "Inbegriff", "Mannigfaltigkeit", "System") to designate a set of objects
% having a common property, becomes frequent among mathemati cians since the beginning of the XIXth century., But it is
% only after Boole, in the second half of the century, that

% 80 CHAPTER TV
% using letters to denote more or less arbitrary sets, and com=- puting with these letters, will become a widespread practice, In particular "classes" of functions where very often con sidered in Analysis, even if their description lacks precision most of the time, Even more widespread was the use, since the XVIIIth century, of sequences of functions, or of functions depending on one or several real parameters (for instance in the Calculus of variations). It was of course dimly realigzed that such families of functions were "much smaller" than the "class" of all functions under consideration; the first at
% tempt to give a clearer expression to thabt feeling is probab ly due to Riemann, In his famous inaugural lecture on the foundations of geometry, after having tried to give an idea of what he means by a "finite dimensional multiplicity (i.e. manifold)" where the position of a point is determined by a fi nite set of numbers, he adds that there are "multiplicities" (Mannigfaltigkeiten) for which such a determination is not possible, but needs "an infinite sequence or a continuous mul tiplicity of numbers", and gives as an example "all the pos sible determinations of a function in a given domain" ([ 182], P.276).
% The extension of the concepts of limit and of continuity to mathematical objects other than numbers or points, such as curves, surfaces or functions, is also very old. However, the applications of that idea dealt with sequences of such ob jects, or families depending on a finite number of real para meters; again, Riemann seems to have been the first to con
% ceive that a whole "class" of functions might be given some kind of "geometrical" structure (what we now would call a topo-
% THE IDEA OF INFINITE DIMENSION 81
% logy), for when he speaks of the functions for which the Dirichlet integral (Chap. II, formula (21)) has a meaning, he says that "this set of functions constitutes a connected do main, closed in itself" ([182], p.30), and although it is not quite clear what he means by that, we may see in that state ment a first glimpse of the notion of compactness, which will emerge in the last part of the century (see below).
% The rigorous study of limits of sequences of func tions, which began around 1820, brought to light a phenomenon
% which had no counterpart for sequences of numbers or of points in R": there are several distinct ways for a sequence (f,) of functions to tend to a l1limit f, The first problem occur
% red with the distinction between simple and uniform convergen ce, which was only quite cleared up around 1850, This was followed in the last third of the XIXth century by a deeper study of these notions, chiefly due to the Italian school (Dini, Ascoli, Arzeld); the most important step taken by that school was the introduction by Ascoli in 1883 of the notion of equicontinuity. He discovered that the unpleasant phenomenon of a sequence of continuous functions (in a bounded closed in terval I), converging simply to a discontinuous function,
% would disappear if one assumed on the sequence the following additional property: for each ¢ > O, there exists a & > O such that, if |x’-x"| < 8, then |fn(x') - fn(x”)| < e for all indices n (in other words, the continuity is "uniform", not only with respect to x, but also with respect to n)
% L8],
% One of the fundamental properties of equicontinuous sequen ces is that, when in addition the fn are uniformly bounded,
% 82 CHAPTER TV
% it is possible to find a subsequence (fn ) which converges uniformly, a generalization of the "Bolzano-Weierstrass"
% theorem for sequences of numbers, which was well-known after 1880, This "compactness" property (which holds for functions defined in a closed bounded set of R') was thrust in the 1i
% melight by Hilbert, who apparently rediscovered it independent 1y in a special case (he does not quote the Italians) and used it as an essential tool in his famous 1900 paper where he in vented the "direct method" in the Calculus of variations ([111], vol. ITII, p.l10-14) and thus was able to justify Riemann's use of the "Dirichlet principle" (chap.II, 3)
% th th yvears of the XIX century., Already at the end of the XVIII
% Tt is also from the Calculus of variations that another no tion of "neighborhood" for a function emeirged during the last
% century mathematicians investigated the problem of deciding if a solution y of the Euler equation for an integral b
% {F(x,y,y’)dx actually gave a "relative extremum" for that ‘a
% integral, Legendre tried to give a solution to that problem by replacing y din the integral by vy + eu, where u = 8y is an arbitrary "variation" of class Cl; he thus obtains a function &(e¢) of the real parameter ¢ and if &”(0) > O (resp. 3" (0) < 0) +that function reaches a relative minimum (resp. maximum) for ¢ = O. This yields the condition 62 > 0) 3y2 (resp. < 0); but it was soon realized that this condition was
% not sufficient to guarantee that the integral would actually be smaller (resp. larger) than all numbers obtained by replac ing yv by v + 8y for a "small" variation 8y. Clearly this
% THE TIDEA OF INFINITE DIMENSION 83
% hinges on the question of what exactly is meant by the word "small", Ever since Lagrange, it had been taken for granted that the derivative (8y)’ = 8y’ is "small" whenever 8y it
% self is "small"; but Weierstrass and his school realized that this was an additional assumption, and this led them to dis tinguish between "strong extremum" and "weak extremum": the second corresponds to a notion of "neighborhood" of a Cl func tion y, where 2z is "close" to vy when the maximum of | z=y| is small, whereas for the first 2z is only considered as "close"TM to y 1if both the maximum of lz-y| and the ma ximum of 2! -y’ are small,
% Finally, we have noticed earlier that Gram and Poincaré were naturally confronted with the notion of "conveirgence in the
% mean square" in their study of "Fourier expansions" (chap.III, 2). We may therefore say that in the last years of the xrx B
% century, the idea of "function spaces" with wvarious "topolo=- gies" was so to speak "in the air", and ready to blossom forth
% as soon as it could be expressed in sufficiently general and simple terms.(*)
% The concept of mapping of a set of functions into R, or
% (*)It is, however, typical of the unpredictability of mathe matical developments that nobody seems to have been able to foresee, even conjecturally, the direction which was taken by Functional Analysis in the fateful years 1900-1910, This is clear in the communication made by Hadamard in the first In ternational Congress of mathematicians in 1897 ([94], vol.I, P.311-312); he was keenly interested in these "set-theoretical" ideas, and had great expectations of what was to comej; but he could think of no serious applications beyond the rehabilita tion of the "Dirichlet principle" and some vague ideas on what we now call "precompactness!",
% 84 CHAPTER IV
% into another set of functions, is also much older than the ge neral definition of a mapping of an arbitrary set into an ar bitrary set, which does not seem to have been formulated be fore Dedekind's famous "Was sind und was sollen die Zahlen", written in 1872 (although only published in 1888) ([48], vol. III, pP.335-391). Ever since the beginning of the Calculus of
% variations, mathematicians were familiar with the idea of at taching for instance to each Cl function y in an interval [a,b] a number f/ F(x,y,y’)dx depending on vy; such map pings would receivz the name of "functional" at the end of
% the XIXth century., Similarly, as soon as the concept of func th tion emerged at the end of the XVIIT century together with its use in Calculus, the concept of operator, yielding a new
% function when applied to a given function, was in evidence with the examples of the derivatives fFi=DYf or the trans lation operator f+=Yy(a)f (function x+r» f(x-a)); and from Leibniz to Pincherle (end of the XIXth century) many analysts were led to ponder on the algebraic properties of these ope rators, and their similarity with results of ordinary algebra (which was originally conceived as applying to numbers only).
% For instance, the similarity of Leibniz's formula for the iterated differential dn(uv) of a product, with the binomial
% theorem, probably gave him the idea of attempting to introduce differentials q< with negative or irrational exponents, a problem to which many mathematicians (such as Liouville, Riemann, Pincherle) later returned, and which has only final ly been put to rest with the modern theory of distributions. Other examples are the expression of Taylor's formula given

% THE IDEA OF INFINITE DIMENSION 85
% by Lagrange as a relation Y(-a) = e between operators, or the factoring of a differential polynomial p" + aan_l+ \ldots+aI1
% on the model of the factoring of an ordinary polynomial n n—l
% Z 4+ a.,z 1 +eo0et+ An*
% Such ideas, abundantly developed in the period 1790-1830,
% had much to do with the new conception of Algebra as dealing with symbols rather than wifh numbers, and later with the axiomatic and formalist conception of the whole of mathematics (see [54], chap. XIII, IIL); but they had no perceptible in fluence on Analysis, probably because they did not pay much attention to questions of continuity. It is only in the last yvears of the XIXth century that such questions appear, in a very episodic way, in papers by Pincherle, Bourlet aund Volterra
% The first two of these authors only consider one "space" E, the set of all holomorphic functions in a domain A of the
% complex plane, and they are exclusively concerned with linear
% operators in that space, In 1886, Pincherle studies operators which, to a holomorphic function( ) @, associate the function
% *
% xn——j’ A(x,y)p(y)dy, where T is a curve in A and A is r
% holomorphic, and he writes that function & ¢, but he limits
% himself to special cases, of the type of the Laplace transform (C173], vol.I, p.92-141)., He several times returned later to
% (*
% After Grassmann (1862), Pincherle seems to have been one
% of the first mathematicians to write a function with a single letter @, when all his contemporaries wrote ¢ (x). In his later papers, he repeatedly insists on the fact that a func tion should be considered as a "point" in some set,

% 86 CHAPTER IV
% such questions, but failed to obtain any substantial resuth*l In 1897, Bourlet [29], limiting himself to the case in which A is a disk |z| < r, explicitly determines the linear ope rators in E which are "continuous" (by which he means conti nuity for what we now call the topology of compact convergen ce), showing that they are integral operators of the form con sidered by Pincherle,
% We must finally mention the first attempts at "Functional Analysis" of the young Volterra in 1887 ([219], vol.I, p.
% 294-314), to which, under the influence of Hadamard, has been attributed an exaggerated historical importance, Volterra
% had in mind a generalization of analytic functions, which may be considered as a prefiguration of Hodge's theory(**); for this he needs what he calls "functions of 1lines", Although, from our point of view, his definitions are not Very'prafifl4*fiq he apparently considers the set E of Cl mappings of an
% (*)He,should however be credited with what is probably the first conception of a closed hyperplane in E as the kernel of a continuous linear form, and of closed subspaces of finite codimension as intersections of hyperplanes ([173], vol. I, P.39%). In 1897-98, he also has the idea of generalizing Lagrange's "adjoint" of an operator (chap.I, 1, formula (5)) by considering two vector subspaces S, S’ of E, and a nonde=- generate bilinear form (®,{) on SxS’; to a linear mapping of S into S’, he then associates the "adjoint" A, a linear mapping of S’ into S such that (A.9,y) = (p,4°y), and he observes the relation between the kernel of4 and the image of 4 ([173], vol.II, p.77-84).
% (x%) See A, Weil, Oeuvres Scientifiques, vol.II, Commentaires sur [1952 e], p.532 of the correct edition (or vol.III, p.450 of the first printing), Springer, Berlin-Heidelberg-New York, 1979.
% (%%xThis can be said of practically all mathematicians before 1906,

% THE IDEA OF INFINITE DIMENSION ' 87 interval I c R into R (the "lines"), and the mappings 3
% ve E #+ R, continuous for the topology of uniform convergence. For these "functions of lines" he immediately wants to gene ralize the classical notion of derivative; in a manner remi
% niscent of the Calculus of variations, he considers a "varia tion" 8y = yv(p+6) - v(®), where the increment 6 is sup posed to vanish outside of an interval [a,b], and then the quotient 8y/oc, where o0 = j/ |6(t)|dt; this should tend to a limit when b-a and the maiimum of |8] tend to O, With
% our experience of 50 years of Functional Analysis, we cannot help feeling that, without even the barest notions of general topology, these ad hoc definitions were decidedly premature, Nevertheless, they caught the fancy of Hadamard, who tried to apply similar ideas to Green's functions and encouraged his students to work in that direction (see [ 94, vol.I, p.401-L4Ok and 435-453] and[146] ), But these ideas have not, up to now, produced anything comparable to the applications of spectral theory and distribution theory, which we will describe in chap, VII and IX3; it might be worthwhile to reexamine them in the light of recent progress in the theory of infinite dimen sional manifolds, which could be their natural setting.

\section{The passage ``from finiteness to infinity''}
\label{sec:4.4}

% The urge to deal with "infinity" has been present from the very beginnings of Greek mathematics, in spite of all philo sophical preconceptions and objections, and has taken wvarious forms, The simplest and most "matural" passage "from finite-

% 88 CHAPTER IV
% ness to infinity" is the "indefinite repetition" of the arith metical operation of addition, on smaller and smaller summands, giving birth to the concept of convergent series, of which one can already find examplzs in Archimedes. Replace addition by
% multiplication, and you have the ianfinite product, born with Calculus in the XVIIth century; and still more sophisticated algebraic manipulations would lead to continued fractions and to the infinite determinants which we have discussed in 2,
% Another line of thought goes back at least to Eudoxus's "method of exhaustion", and was to lead in the first place to the concept of integral, But in the hands of the mathemati cians of the XVIIth and XVI[Ith century, this idea of decom posing an object into "infinitesimal" parts in which the phe nomenon they stndied became much easier to describe "in a first approximation", was developed into a more and more so phisticated method to discover the differential or partial differential equations which governed the phenomenon "in the large", It is in that way that the equation of vibrating strings (chap.I, 2, formula (7)) was established, either by
% considering, as D, Bernoulli, a massive string as a limit (for n tending to infinity) of a system of n massive points distributed on a massless string, or by analyzing, as d'Alem bert, the forces which are exerted on an "infinitesimal" por tion of the string by its neighbors,
% It is this second method that Fourier applied to obtain the heat equation; he takes for granted that in a system of small "molecules", a given molecule M 1receives in an "infinite simal" time dt a quantity of heat from another molecule M’ equal to the difference of temperatures of M and M’, mul-
% THE IDEA OF INFINITE DIMENSION 89
% tiplied by dt and by a coefficient depending only on the distance MM'; +the molecule M, if situated at the surface separating the system of molecules from the exteraal world, also radiates a quantity of heat equal to the difference of its temperature and of the external temperature, multiplied by dt and another coefficient depending on M, He then de rives the equation of the "cooling off" process (chap.III, 2, equation (21)) by decomposing the solid body V in "in
% finitesimal" cubes and evaluating the amount of heat received by one of them from its 6 neighbors in time dt, which he takes as proportional (with a constant coefficient) to the variation du of the temperature of that cubej the boundary condition (chap.ITI, 2, equation (22)) is similarly obtained by evaluating the amount of heat lost (by radiation) by an in finitesimal cube at the surface of V,
% At the end of his 1890 paper on the cooling off problem (chap.III, 2), Poincaré suggests another method reminiscent of D, Bernoulli's procedure, He first considers a large num-=- ber N of molecules Mi; following Fourievr's physical consi derations, and denoting by Vi(t) the temperature of Mi at
% time t, these functions satisfy the system of linear differ dv.i .
% ential equations
% (10) el Y Cik(vi-vk) + C;v, =0 (1< i < N), k#£i
% Cik(vi-vk) being the quantity of heat received from M, and Civi the quantity of heat radiated by Mi outside the system, But instead of letting the number of molecules increase to in
% finity, Poincaré first integrates the system (10) by the clas sical Euler-Lagrange method: he writes vi(t) = uie-at, and,
% 90 CHAPTER IV
% using the fact thaf the matrix (Cik) is symmetric, he re cognizes in the equation he obtains for @ the equation giving the eigenvalues of the symmetric matrix corresponding
% to the non degenerate positive quadratic form
% 2 2
% )T+ L Ciul .
% i#k i
% Let €l < 2 €eee< N be these eigenvalues; the classical
% theory of quadratic forms shows that one may write 2 2
% where the ®, are linear forms in the variables Ugyeee,Uy . 2 2 2 2 .
% such that @, +eee+ Py = U, +ees+ U, 3 1if for two such forms 1 N 1 N £o=a,u; +eonk By, & =B u; +eee+ Byuy one writes (flg)=
% = 3 @kBk’ the N forums @i are mutually orthogonal for that k
% scalar prnduct. It is then clear that 1 is the smallest value of the function of Upseeeyly
% (13) 5 2 2 =3 5
% (U yeee,u) 1 N )
% ul+u2+o . o+uN Cpl+o . 0+CpN
% where the u, are arbitrary; similarly €2 is the minimum of (13) for v, = o, the minimum for P, =9, = O as 55 relations between the u. ; and so on, This is of course the
% analgous procedure in N dimensions to the classical deter mination of the "axes" of an ellipsoid in 3-dimensional space. Poincaré's idea is that the expression (13) corresponds exact
% ly to the quotient
% m (GH*° + & T AD%au + hflvzdo

% : [f[[ v=du
% >

% THE IDEA OF INFINITE DIMENSION 91
% in his (or rather Weber's) procedure for the definition of the eigenvalues in the cooling off problem; these eigenvalues (the poles Xm of his function [f,€] (chap.III, 2)) correspond
% to the j in (12) and the eigenfunctions Um(M) = P(M,xm) to the @j, the orthogonality of the mj corresponding to the relations
% between the Ufi' Finally, he realizes that the same ideas
% apply as well to other problems and gives as an example the i
% theory of elasticity, and he suggests that a rigorbus proof of the existence of the km and the Um, which he had not been able to give, might be obtained by simply letting N tend to +» in the formula (12), He never came back to the question; but we cannot fail to see that this is exactly the program which Hilbert in 1904 followed to its successful con clusion for integral equations with symmetric kernel (chap.V, 2).
% A similar "passage from finiteness to infinity" emerged in
% the first general theory of integral equations, beginning with the papers of Le Roux in 1894 and Volterra in 1896, In addi
% tion to the particular integral equations which had been met by Liouville in the Sturm--Liouville problem (chap.I, 3, equa tion (34) and chap.II, 1, equation (6)) and by Beer and Neumann in the Dirichlet problem (chap.IT, 4, equation (23)) (not to speak of what we have called "crypto-integral" equa tions, where the equation is not written down explicitly but the method exactly amounts to solving it), other particular equations involving integrals had come up in connection with
% 92 CHAPTER IV
% problems not directly related to differential or partial dif ferential equations, The first one (chronologically) was the "inversion" problem for the "transform" introduced by Fourier in 1822 (and to which we shall return in chap, VII, 6); it
% associates to a function f in [O,+o[ +the function (14) p(t) = f(x)cos tx dx
% 0
% and the problem consisted in finding f when the transform ¢ is a given function, It was solved by Fourier's inversion (15) f(x) = %— o (x)cos tx dt
% formula ([67], vol.I, p.392)
% 0
% where, as usual with Fourier, both formulas are obtained by a purely formal calculation, A little later, one of the first published papers of Abel ([1], vol.I, p.11-27 and 97-101) was devoted to a problem of mechanics, which amounted (16) 2(Y)dy _ y (x)
% to finding a function ¢ such that
% X
% o NX-Yy
% (17) p(x) = L ‘”;-(Z_)_d”
% is a given function; he obtains the solution by the formula X
% X-y
% and extends his result to the case in which ,/x=-y 4is replac ed by (x-y)g for 0< @ < 1., In a letter to Holmboe, he
% even hinted at more general results, but nothing was found on the subject in his papers, After Abel, a few papers, giving partial generalizations of his results, were published until
% THE IDEA OF INFINITE DIMENSTION 93
% *
% 1890( ); but it was only in 1894 that Le Roux attacked the
% general problem of "inversion of a definite integral" (as it was called), i.e. finding a Cl function ¢ din an interval [a,b] satisfying an eguation
% (18) o(x)H(x,y)dx = £(vy)
% y
% (%%) ]
% where f and H are CT (in [a,b] and [a,b] x [a,b] respectively) and f(a) = O « In contrast with his pre
% decessors, Le Roux is not trying to find a "closed formula" similar to (15) and (17) for the unknown function., He assu mes that h(y) = H(y,y) does not vanish in [a,b], takes the | 2H : (19) h(m)o(v) + | 22 (x,y)e(x)ax = £ ()
% derivative of both sides of (18), obtaining
% /y'
% a
% and then applies the method of successive approximations which
% Picard had popularized a few years earlier:
% () = H w0 = - By, (ax
% for n = 1,
% proving easily- the convergence of the sequence (un) to a solution of (18) ([143], p.24k-246),
% In 1896, Volterra (who apparently was unaware of Le Roux's paper) tackles exactly the same problem by the same method,
% *
% ( )See the long historical introduction given by Volterra in his 1897 paper on integral equations ([219], vol,II,p.279-287)
% ( )As these conditions are not satisfied for Abel's equation,
% * ¥
% Le Roux's results (which for him are auxiliary properties which he needs in a study of partial differential equations) do not directly generalize those of Abel,
% ol CHAPTER IV
% in a series of 4 notes ([219], vol.II, p.216-262), He goes a little beyond Le Roux, by giving an explicit expression of the solution
% (20) p(y) = *‘H%,rj) " H(y) (:'E-o 5, (x,y))r’ (x)dx
% il 1 >
% y
% , i=
% where the Si are defined by induction:
% X
% 1 dH (21) s, (x¥) = 375 3y (%:¥)s S;(x,y) =) S (,y)s; ;(x8)dE

% for i 2 1.
% y

% In the later notes, he discussed the cases in which h(y) may
% vanish at a finite number of points, and the case in which H(x,y) = G(x,y)/(x=y)® with O< @& < 1 and G is continuous (the generalization of Abel's equation). But the most in fluential part of his notes was the following remark he made immediately after obtaining formula (20): "If one considers
% the system
% ( by = a;1%
% (22) fi ® [ ] ® [ J [ ] ® ®
% b2 = alle + a22x2
% \
% the concept of integral easily leads to look at the question of functional Analysis represented by equation (18) as a 1li
% miting case of the solution of a system similar to (22), in which the 2 and a., are the analogous of H(x,y) and H(y,y)s" Although he limited himself to that (somewhat vague)
% statement, it seems obvious that what he had in mind was re placing in (18) the variable vy by its values
% v, = a +-—(b-a) for 1< k< n

% THE TDEA OF INFINITE DIMENSION 95
% and replacing the integral by the corresponding "Riemann sum" for the subdivision of [a,b] by the points Vi obtaining the system of type (22)
% bea 2
% f(y.) = — X @(y )H(y ,y.) for 1< Jj< n, J n k=1 k k?7 3
% Finally, although he does not mention the product of matri
% ces, Volterra develops in these notes the formalism which to two "kernelsTM Hl(x,y), Hz(x,y) associates the kernel y
% H(X9y) = Hl(x9)H2(’Y)dg
% X
% (which much later he will write H = Hl*HZ). If, for sim
% plicity, we adopt this notation, he shows that, for an arbi trary continuous function So(x,y) if we define for i = 1 the "kernel" S. by S, =S S one has also Si =
% i i i=1%"0"
% = Si—j*sj-l for 1< j< i, and a majoration
% i+l M i
% lSi(X9Y)l < il lX-Yl i
% (23) F = % S
% This implies uniform convergence for the series (00]
% (24) FO-SO = SO*FO.
% and the relation
% i i o i=- l’
% He observes that one may "invert" that relation: if the F. are defined for i = 1 by F. = F xF, one has
% (25) Sy = (-1)7F, .
% - i
% i=0
% And finally, at the end of his notes, he arrives at the ge-
% 96 CHAPTER IV
% neral concept of what Hilbert will call an "integral equation
% of the second kind"
% (26) o(y) - | 5 (x,y)0(x)dx = £(y)
% for which the solution is given by
% (27) o(y) = £(y) + F_(x,y)f(x)dx
% y
% a
% as it follows immediately from (24), the "kernel" and the "resolvent kernel" playing completely symmetric parts in these formulas.,

\chapter{The crucial years and the definition of Hilbert space}
\label{ch:5}
\setcounter{equation}{0}

Between 1900 and 1910, there was a sudden crystallization of all the ideas and methods which had been slowly accumulating during the XIXth century and which we have described in the previous chapters. This was essentially due to the publication of \emph{four fundamental papers}:

Fredholm's 1900 paper on integral equations;

Lebesgue's thesis of 1902 on integrationg;

Hilbert's paper of 1906 on spectral theory;

Fréchet's thesis of 1906 on metric spaces.

\section{Fredholm's discovery}
\label{sec:5.1}

The name ``integral equation'' (\emph{Integralgleichung}) was used for the first time by P. du Bois-Reyfiond in 1888, in a paper on the Dirichlet problem \cite{061}; he has in mind equations of the Beer--Neumann type (chap. \ref{ch:2}, \ref{sec:2.4}) and considers that a general theory of such equations presents ``insuperable difficulties''; he is convinced that much progress would come out of such a theory but acknowledges that ``almost nothing is known on this question''. The later work of Poincaré, which we have discussed above (chap. \ref{ch:3}, \ref{sec:3.2}), and of his immediate followers, did nothing to dispel that impression; their results seemed linked to delicate estimates from potential theory. It therefore came as a complete surprise when, in a short Note published in 1900, Fredholm showed that the general theory of all integral equations (or ``crypto-integral'' equations) considered before him was in fact extremely simple (much simpler than anything known at the time in the theory of partial differential equations).

Ivar Fredholm (1866--1927) was a student of Mittag-Leffler in Stockholm in 1888--1890; he only published a few papers during his lifetime, mostly concerned with partial differential equations (we shall return to his thesis of 1898 in chapter \ref{ch:9}, \ref{sec:9.5}). After a visit to Paris, where he had been in contact with all the French analysts and had become familiar with the recent papers of Poincaré, he communicated in August 1899 his first results on integral equations to his former teacher; they were published in 1900 \cite[p.~61--68]{074} and completed 2 years later in a paper published in \emph{Acta Mathematica} (\cite[p.~81--106]{074} and \cite{075}).

Fredholm's 1900 note is entitled ``On a new method for the solution of Dirichlet's problem'', but it is characteristic that from the start, he brushes aside all the particular features of the Beer--Neumann equation, and (as Le Roux and Volterra had done with Abel's equation (chap. \ref{ch:4}, \ref{sec:4.4})) begins with a \emph{general} ``integral equation of the second kind'' (that name will only be given by Hilbert)
\begin{equation}
    \label{eq:5.1}
    \varphi(s) = f(s) + \lambda \int_a^b K(s, t) f(t) dt
\end{equation}
where \(K\) is supposed to be bounded and piecewise continuous in \((a, b] \times [a, b]\), and \(\varphi\) continuous in \([a, b]\), \(\lambda\) being a complex parameter. He briefly mentions the analogy with systems of linear equations and starts right away with the formulas describing his ``determinants'' (see below). But in a lecture given in 1909 \cite[p.~123--131]{074}, he acknowledges: 1° the inspiration derived from Volterra's idea of a ``passage to the limit'' from a system of linear equations to an integral equation; 2° the help he found in von Koch's work on infinite determinants (chap. \ref{ch:4}, \ref{sec:4.2}). From these indications, sparse as they are, it seems one can reconstruct his procedure, with great probability, as consisting in \emph{putting together three simple ideas}:
% I) Replacing the integral in (1) by Riemann sums, one obtains, with the notations of chap. IV, 4, the system of n 1linear equations for the f(yj)
% (2) £(yvy) + 512532 ; K(vy sy )f(vy) = ely;) (1= 3 = n).
% II) Writing the determinant of that system according to von
% Koch's formula (chap.IV, 2, formula (9))
% % +ooe
% KUKy Ky 7k,

% X n
% 1+l£:31 )X K(yk’yk) +
% Xz(b-a)2 2
% l’kZ
% K(y, »v, ) K(v, ,v, ) K(ykz’ykl) K(yk29yk2)

% and then letting n tend to +o, which gives the formula for what Fredholm calls the "determinant" of the integral equation (1)

% 100 CHAPTER V
% () a (b .
% 4 —
% !
% 5 Y - N 8
% where he has written
% K(Xl’yl) K(Xl’yz) ¢ o0 K(xl,ym)
% xp Xy eee X |K(xpy;) K(x¥,) eee K(xy,v,) (&) K(yi Yi D m [ ] [ ]

% '® ®
% >® [ ] ® ® [ ] [ ] ® :[ [ ] [ ] '

% [ ] [ ] [ ]
% K(x,»¥7) K(x3¥,) eeee K(x_,v_)
% III) Proving the uniform convergence of the series (3) in
% any compact set of the complex plane, for which it is enough . : . n/2 n
% to majorize the determinants (4) in asuitable wayj; in his 1899 letter, Fredholm had given the majoration n M, where M is the upper bound of |K|; he had apparently arrived inde pendently to this result, but was made aware that it was a special case of an inequality published by Hadamard in 1893 (C94], vol.I, p.239-243) for an arbitrary square matrix A= (a,.) of order n:
% (5) |det(4)|" < TT (£ [a, 7). i

% 1J
% 2 — 2 n n

% The next "natural'" steps are of course to apply Cramer's formulas to the system (2) and let again n tend to infinity
% in the numerators; the result is described by Fredholm in the following elegant way: a development of the determinant (4) according to the first row yields the formula
% THE CRUCIAL YEARS AND THE DEFINITION OF HILBERT SPACE 101
% K ( ) = K(s,t) K( ) -
% (6) -K(s,x;)K( TM)+ K(s,x,)K( ) ") -
% X X cee X X X X eee X
% t Xo eee X t xl x3 cee X
% 1 2 1 2
% = see -+ (-l) K(S,Xm)K( m )o
% m 1 2
% X X eee X
% t xl cee X -1
% On the other hand, Fredholm defines the "minor"

% (7) b b
% bS Xl

% A(s,t3)) = K(s,t) + A K(t )dxl teeet
% + fil /ff°°°/ K( 1 m) dxldxzooodxn ‘oo
% X1 a
% m S X eeoe X
% ° a a t Xl e e ©® Xm
% and replaces each integrand by its expression (6), which gives (8) A(Syt;X) = K(S9t)A(>¥) - )&( K(S,g)lg,t;k)dg.
% him the simple relation
% b
% (9) (s) =w(s)a(r) = A A(s,E32)0 (8 )a8
% He then introduces the function
% b
% (10) 8(s) + kjr K(s,t)e(t)at = o(s)A(r).
% o
% and derives from (8) the equation
% b
% The conclusion is then immediate: if A()) # O, +the function that one has b
% f(s) = 8(s)/A(A) 4dis a solution of (1). Furthermore, he shows (11) Eééfil = A(s,s3r)ds
% a
% and from this he deduces that if xn is a zero of order Vv

% 102 CHAPTER V
% of the entire function A (L), (s), for a suitable choice of o, cannot be divisible by a powerof x-ko greater than (X—xo)v-l if ¥(s) = (X—XO) @l(s), one then deduces, from (10), that
% k
% (12) @lun + A ( K(sflfl@l(tfiflzz 0; b
% in other words, if there is no nontrivial solution of the ho mogeneous equation (12), necessarily A(ko) # 0, hence the solution of (1) for A = xo exists and is unique, However,
% at that time, he does not yet prove that the existence of a non trivial solution of (12) implies that A(xo) = 0. But the end of the Note is startling: he considers the Beer 3
% Neumann equation for a bounded plane domain with a C~- bound ary; the kernel of that integral equation is {hen bounded and continuous, and for xo = 1 it is very easy to deduce from
% the properties of double layer potentials that the homogeneous equation (12) has no nontrivial solution. Therefore the exist ence and uniqueness of the soluation of Dirichlet's problem is proved, doing awav, with a single strooke of the pen, so to speak, with all the complications of the Neumann-Poincaré so lution!
% In his 1903 paper, Fredholm completed his results on some
% important points., He first defines more general "minors" A ( sh) = K( ) +
% S S ® & ¢ S S S ® & o S
% 1 °2 m 1 72 m
% oo S e oo S X eee X

% (13)
% +3L( k(1 m 1 ?) dxy  \ldots dx . n=1 TM' |, t. eee bt X. eee X " 1 m 1 n

% Developing this time the determinants both according to the

% THE CRUCIAL YEARS AND THE DEFINITION OF HILBERT SPACE 103 S,]_ T ) Sm ( g 82 PRRPIEPY Sm
% first row and the first column, he obtains the identities A( sA) + A ] K(sq,8)A( sA)dE =
% b
% 2 2
% (14)S, eee S S, S. eee S 2 m 1 73 m
% and b
% t, eee t t, t, eee ©
% 1 1 "2 A ( )+ h ) K(,tq)a( TN )dE =
% S e o @ S S S o o O S
% tl e oo tm a g t2 e oo tm
% 2 1
% (15)S eee S S S eee S 2 ° m 2 73 m
% t ee T t, ¢t eee t
% which in particular, for m = 1, reduce to (8) and to (16) A(S,t;)\) = K(S9t)A()¥) - A ]( K(g ,t)A (s,8 32 )dg.
% b
% la
% The use he makes of these formulas is a little more sophis ticated than in his first Note, He introduces the operator
% corresponding to the kernel K, f#*—SKf such that b
% Sf(s) = £(s) + g/ K(s,t)f(t)dt, and, for two kernels K, K’, a
% writes the composite SKSK, as SK& with
% b
% (17) K (x,t) = K(x,t) + K' (s,t) + K(s,5 )K' (E,t)dE.
% a
% Suppose now that A(A) # 0, and write
% (18) R(s,t350) = =A(s,t350)/8(})
% (the resolvent kernel in the later terminology of Hilbert). It then follows from (17), (8) and (16) that we have

% 104 CHAPTER V
% (19) S, kSg = SgSyx =
% and Fredholm has thus shown that the mecessary and sufficient
% condition for the existence and uniqueness of a solution of (1) is A()) # 0, the kernel XK and the resolvent kernel R playing completely symmetric parts as in the formulas of Vol terra (chap. IV, 4, formulas (26) and (27)).
% Next he es<amines what happens when A(A) = O, First he ge m S eee S
% de 1 m neralizes (11) to
% and from this he deduces that if A(A) = O, there is always an integer m such that A(tl "

% S
% M.2) is not identically O, l o o O m
% If m is the smallest integer having that property (which is exactly the order of A as a zero of A) he exhibits, using A(t tl tm) A(tl t tB tm)
% (14),- m solutions of the homogeneous equation st TM At TM)
% vy m Ty eee Th
% for which he shows that they are linearly independent and that
% every other solution of the homogeneous equation is a linear combination of the @j (1< j € m)e He concludes the theory (which one often calls the "Fredholm alternative") by giving necessary and sufficient conditions on ¢ for the existence of a solution of (1) when )\ is a zero of order m of A, He observes that the "transposed equation" obtained from (l) by replacing K(s,t) by K(t,s) has the same "determinant",
% THE CRUCIAL YEARS AND THE DEFINITION OF HILBERT SPACE 105
% and therefore the corresponding homogeneous equation has exactly m linearly independent solutions Yl, \ldots,Y m? the condition ¢ must satisfy are then
% (22) @(X)Yj(x)dx = 0 for 1< j< m,
% b
% a
% Finally, Fredholm shows that for any two kernels K, K', if and AK' are the corresponding "determinaunts", then for Ak the "composed" kernel K’ defined by (17 ? one has
% which justifies the name "determinant", He also points out
% that his results can be generalized when the kernel KX is not bounded any more, but such that (x-y)®*K(x,y) remains bounded, with O < @ < 13 and he mentions that the extension of his theorems to any number of variables is immediate,
% This beautiful paper may be considered as the source from which all further developments of spectral theory are derived, It made a deep and lasting impression on the mathematical world, and almost overnight the theory of integral equations became a favorite topic among analysts ([23], [175], [107]).

\section{The contributions of Hilbert}
\label{sec:5.2}

% One of the most active proponents of the new theory was David Hilbert, As soon as he heard of Fredholm's results, he started doing himself research work on these questions, made them one of the main subjects discussed in his Seminar at

% 106 CHAPTER V
% *
% thtingen( ) and supervised many dissertations on the wvarious aspects of the theory. Between 1904 and 1906, he published six papers on integral equations in the GO8ttingen Nachrichten, later brought together in a single volume entitled "Grundzlge einer allgemeinen Theorie der Integralgleichungen" [112].
% In his first paper [112, p.1=-38], Hilbert starts by doing explicitly what had only been hinted at by Volterra and Fredholm, the "passage to the limit" in the system (2), res tricting himself (as he will do in almost all his results) to the case in which the kernel K is symmetric, i.e. a real continuous function such that K(t,s) = K(s,t). He soon re alized that in that particular case he might obtain much more precise results than Fredholm, In the first place, the symmet ric matrix (K(yk,yj)) is then the matrix of the quadratic form .Zk K(yk,yj)gkgj, and Hilbert undertook to apply also his "piésage to the 1limit" to that form, He thus obtained
% the results which Poincaré had foreseen in the particular case he had considered (chap.IV, 4): the roots of the Fred
% holm determinant are then real; if they are written as a se=- quence (xn), each being counted with its multiplicity, then,
% for each n there is an eigenfunction ®,» such that b
% f, @m(t)@n(t)dt = 0 for m #£ n, Finally, if one normalizes a
% (*) It is reported (by Hellinger) that Hilbert inaugurated a session of his Seminar by announcing the development of a method which would lead to the proof of the Riemann hypothesis: the problem is to prove that a particular entire function has all its zeroes on the real line, and Hilbert hoped that this function would be expressed as the "determinant" of an inte gral equation wita symmetric kernel., However, nobody has yet been able to find such an equation,

% THE CRUCIAL YEARS AND THE DEFINITION OF HILBERT SPACE 107 b
% the © , Py the condition ( mn(t)zdt = 1, and if for each a
% continuous function x din Ea,b], one defines the "Fourier coefficientsTM" (x|@n) = (/ x(t)wn(t)dt, Hilbert proves that (24) K(s,t)x(s)y(t)dsat = £ 5 (x]o.)(v]|p,)

% -b Db
% a

% for any two continuous functions x, y, a relation which he
% rightly considers as the natural generalization of the clas sical reduction of a quadratic form to its "axes", What is particularly interesting in the way Hilbert considers this formula is that he shows that the righthand side of (24) is uniformly convergent when the functions x and y are al
% lowed to vary arbitrarily, subject only to the conditions a a
% (/ x(t)zdt < 1 and '( y(t)zdt < 1, the first prefiguration
% of what will become "the unit ball in Ililbert space'" a few vears latevr, Of course Hilbert also justifies for his inte
% gral equation the variational deiinition of the eigenvalues A, first proposed by Weber (chap.III, 1). He shows that the set of the \_ is infinite, except when K(x,y) dis a linear
% combination of a finite number of functions of type u(x)v(y). He also proves that the resolvent kernel R(s,t;u) (in the sense of Fredholm) has the eigenvalues xn-u, the correspond ing eigenfunctions being wn/(xn-u) (u distinct from the xn)
% and writes the ideuntity
% (25) R(Syt;u) - R(Sat;\)) = (U-"V)( R(S9g ;U)R(g,t;\))dg b
% for U and Vv distinct from the J . Finally, he shows that if a function f can be written in the form

% 108 CHAPTER V
% (26) £(s) f K(s,t)g(t)dt
% b
% for a continuous function g, then the corresponding "Fourier (27) f(s) = z (flee (s)
% expansion"
% is absolutely and uniformly convergent, and one has the (28) £(s)%ds = 3 (flo,)>.
% "Parseval identity"
% b
% a
% However, he could only give that proof undeir the restrictive assumption that any continuous function could be approximated (in the sense of mean square value, or, as we would now say, for the topology of Hilbert space!) by functions of the form (26).
% The proof that this last condition is superfluous was given in 1905 in the dissertation of Erhard Schmidt, one of the best students of Hilbert [191]; it contained otherwise no startling new results, but it deserves some comments, since it is the
% first attempt to do away with the Fredholm "determinants", and b b
% substitute to them a more conceptual approach(*). N 2 2 N 2
% a a n
% E. Schmidt begins by proving the Bessel identity n
% =1 = 1
% (29) | (£(s) - = (floo, (s))%as = | £(s)%as - = (£]q,)
% for an arbitrary orthonormal system (@n), from which he *
% ( )Some of the results of E. Schmidt were also obtained in dependently by W, Stekloff [204],-

% THE CRUCIAL YEARS AND THE DEFINITION OF HILBERT SPACE 109
% deduces that for any continuous functions f, g, the series z (f|mn)(g|@n) is absolutely convergent, and the convergence n
% is uniform when f is allowed to vary subject to the condi tion r’ f(s)zds < A for a fixed constant A, a
% Next he assumes the existence of the eigenvalues Xn and of the corresponding normalized eigenfunctions ¢_, and using n
% (30) ) ~l-s (/ K(s,t)” dsdt
% the Bessel inequality, he proves that
% b b
% 2
% H ln a a
% from which it follows that each xn has finite multiplicity and that |\ _| tends to +» with n if there is an infinity
% of eigenvalues,
% To prepare for the proof of the existence of the eigenvalues,
% he introduces, as Fredholm and Volterra had done, the iterated (31) K (s,t) = K .(s,8)K(E,t)dE for m > 1, with K., =K m m=1 1
% kernels
% b
% a
% and shows that, if ¢ dis an eigenfunction for Km’ it is also an eigenfunction for K if m is odd, and is sum of two eigenfunctions for K 4if m is even, This allows him to apply Schwarz's method to prove the existence of at least an eigenvalue when K dis not identically O, as we have shown in chapter ITI, 1, because what he gets in this way is an eigenvalue of K2.
% Finally, for functions f given by (26), he obtains the convergence of the Fourier expansion (27) by applying his initial lemma to the functions t+ K(s,t) and g; and from

% 110 CHAPTER V
% that he derives Hilbert's formula (24) by multiplying the formula x(s) = % (x|mn)mn(s) by K(s,t)y(t) and integrat n
% ing.
% Hilbert's interest in integral equations with symmetric kernels of course stemmed from the possibility of applying them to questions of Analysis such as the Dirichlet problem; it is to such applications that he devoted the second and third of his papers on integral equations, We shall bypass them for the time being, as well as most results in his two last papers on the subject (see chapters VII and IX), to con centrate on his fourth paper, published in 1906, a masterpiece and one of the best papers he ever wrote, By the depth and novelty of its ideas, it is a turning point in the history of Functional Analysis, and indeed deserves to be considered as the very first paper published in that discipline.
% Hilbert's new departure in that paper is clear from the beginning: he deliberately abandons the point of view of in tegral equations, to return to the older conception of the in finite systems of linear equations (chap.,IV, 2), but with a new twist, This is because he realizes that the theory of
% integral equations can be subsumed as a special case of that older theory: indeed, let (w) be a complete orthonormal system of continuous functions in [a,b], and suppose the continuous function f is a solution of (1) for A = 1;
% then, if we consider the "Fourier coefficients" b ,b kg = ( ( K(s,t)wp(s)wq(t)dsdt,
% (32) , a ’‘a
% bp = ‘/ @(s)wp(s)ds, Xp = [/ f(s)wp(s)dsb

% THE CRUCIAL YEARS AND THE DEFINITION OF HILBERT SPACE 111
% the xp (p=1,2, \ldots) satisfy the infinite system of linear equations
% (33) p 7 qfl “pa*aq = Pp (p=1,2, \ldots) .
% The new twist is that, due to the Bessel identity, one has 2 2
% (34) > kgq < 4o, % bp < 4o, z Xp < 4o
% P,q P P
% Conversely, suppose we have a solution (xp) of (33) (with conditions (34)), and observe that if kq(s) = K(s,t)wqfiodt,
% b
% a
% the functions kq are continuous, and b
% 5 kp(s)2 < K(s,t)zdt;
% P a
% the series u(s) = T kap(s) B then absolutely and uniformly convergent; hence i is continuous and one has (u|wp)=:b -X_ 3 P P
% therefore, if f = @p=-u, (flwp) = Xp and from the completeness of the system (wp) it follows that f is a solution of (1) for A = 1.
% Hilbert then embarks into completely uncharted territory: 12 He exclusively considers sequences x = (xp) (for p=1,2,¢..,) of real numbers such that ¥ xi < 4o, 22 On the contrary, with regard to the double sequence (k_ ) Pqg
% of real numbers, he abandons at first any restrictive = L qp Pq
% condition such as the first condition (34), and only retains the symmetry conditions k = k .
% 32 The center of interest is not any more the solution of (35) K(x,y) = %
% the system (33), but the "symmetric bilinear form" k X

% P,q
% P9 qu

% 112 CHAPTER V
% which he wants to "reduce" by a formula which would generalize

% (24).
% . 2 2

% Of course, even under the restrictions I Xy <+, X Yo < p
% < +o, the right hand side of (35) is usually meaningless(*); proceeding as Fourier, Poincaré and von Koch (chap.IV, 2), Hilbert considers, for each integer n, the symmetric bilinear form in 2n variables ("sections" (Abschnitte) of K )

% (36) kK_(x,y) =
% n
% BTMM k X

% 2
% p=1 qg=1
% pq p q ’

% but instead of investigating the determinants of these forms, he "reduces" each one to its "axes" and is confronted with the problem of "passing to the limit" for these "reduced" forms when n tends to +o, We postpone the detailed examination of the original method by which he was able to solve that problem, to chapter VIIT, which is devoted to the history of modern spectral theory, of which this paper of Hilbert is the starting point; we shall only discuss here the various new notions he is led to introduce in that paper,
% A) Hilbert is not yet using the geometrical language which will become prevalent among his immediate successors (cf. 3), but it is obvious that everything he does in inspired by the analogy with n-dimensional FEuclidean space, 1In particular one
% of his main tools is the generalization of orthogonal trans formations: by that he means that, to every sequence (xp)
% *
% ( )To this rather awkward formulation, Hellinger and Toeplitz (ko) pq and of their "calculus" inspired from Fro- 1£p, q<+e’
% [ 106] substituted the consideration of "infinite matrices" benius, but without associating an endomorphism to a matrix,
% THE CRUCIAL YEARS AND THE DEFINITION OF HILBERT SPACE 113 with T x < +=, he associates the sequence (x;), whevre
% 2
% (37) x! = ¥ a X p—-—l 2 e e 0
% P
% P q
% and where he imposes on the double seguence (apq) the con

% ditions
% L 25q ’ % 2p5q%pr or q # q P
% a = 1 a a = 0 for r
% 2
% > ®pq ’ > “pa"nq ° # P a = 1 a a = for n

% (38)2
% from which he immediately deduces that conversely (xp) is deduced from (x;) by appling the "inverse" orthogonal trans pa pPq ap
% formation defined by (a’ ) with a’ = a _,
% B) Hilbert restricts himself to forms (35) which he calls bounded: they are the (not necessarily symmetrio) forms such that there exists an M > O for which one has lKn(x,y)l < M
% 2 2
% p p2
% for X xp <1, Z yp < 1 and for all n; he also introduces p p
% bounded linear forms L(x) = T a x.,, with T a < 4=, so that for any x (resp. y), and any bounded bilinear form K, the linear forms K(x,¢): x+» K(x,y) and K(+,y): x> K(x,vy) are bounded, One of the things he wants to do (inspired of course by the "reduction" of bilinear forms in a finite num ber of variables) is to operate an orthogonal transformation on x and vy, substituting the expressions (37) for the X, and doing the same for the yp. Unfortunately, he follows Frobenius in his conception of the "Faltung" of bilinear forms (instead of the natural idea of "composing" transforma tions), So, for two bounded bilinear forms 4,B , he has to show that the forms An(X,°) Bn(°,Y) (Faltung of An(X,Y) and B_(x,y), these forms being defined as in (36)) are the

% 114 CHAPTER V
% forms Cn(x,y) corresponding to a bounded form ¢ (x,y) which he calls again the "Faltung" of A(x,y) and B(x,y) and writes A(x,s) B(+,vy). He ca& then express the action of an \
% orthogonal transformation on d bounded bilinear form K (x,vy)
% as a "Faltung"
% (39) -K,(X!’Y') = K(°7')O(°’X’)O(°’Y’)
% where O0(x,y) = % a_ x_¥y is the bounded bilinear form p,q PG P g
% which he associates to the orthogonal transformation (37). C) For the development of Functional Analysis, the most im=- portant concepts introduced by Hilbert were what he calls "continuity" and "complete coutinuity", which correspond to
% what will later be called the "strong" and "weak" topologies on Hilbert space. If F(x) is a complex=-valued function de fined for all sequences X = (xp) such that ¥ X < +o, P
% Hilbert says that F is continuous if F(x(n)) tends to
% p
% F(x) when T (xp-xén)) tends to O, and that F is com
% 2
% P
% pletely continuous if F(x(n)) tends to F(x) when ¥ xi < 1, 2 (x(n))2 < 1 and each coordinate xén) tends to x . He Ehows that a bounded bilinear form K(x,y) dis continuous, and that Kn(x,y) tends to K(x,y) when =n tends to +»., But he pays special attention to the completely continuous sym metric bilinear forms, and gives a separate proof that an or thogonal transformation can reduce any such form to the type 1 1 1
% where the sequence (|Xn|) is either finite or tends to +w, He realizes that this is a genuine generalization of formula

% THE CRUCIAL YEARS AND THE DEFINITION OF HILBERT SPACE 115
% (24), which is the special case in which 2 k2 < +o (cor Py,q P4
% responding to what will later be called the Hilbert-Schmidt
% operators); he also mentions another special case, the one in which K(x,x) > 0 and Y kpp < 4o (Corresponding to the po p
% sitive nuclear operators of a later date). This formula (40) enables him to go beyond Fredholm by solving a system (33) which is not derived any more {from an integral eguation, but in which the kpq are only supposed to be such that the sym metric bilinear form (35) is completely continuous. A final remark is that he repeatedly uses with great power what he calls a "principle of choice", which is equivalent to what will later be called the compactness of the unit ball for the
% weak topology, and that he extends his results to hermitian (41) L1
%  K(x,v) = z kpqxpyq
% sesquilinear forms
% P,g
% where this time the sequences (xp), (yp) and (kpq) consist of complex numbers, with k = k_ .,
% qp Pq

\section{The confluence of Geometry, Topology and Analysis}
\label{sec:5.3}

% It may seem obvious to us that the results of Hilbert are but one step removed from what we now call the theory of Hilbert spacej; but if, in fact, the birth of that theory almost immediately followed the publication of Hilbert's papers, it seems to me that it is due to the fact that this publication precisely occurred during the emergence of a new concept in mathematics, the concept of structure,

% 116 CHAPTER V
% Until the middle of the XIXth century, mathematicians had been dealing with well determined mathematical "objects'": numbers, points, curves, surfaces, volumes, functions, opera tors. But the fact that algebraic manipulations on different kinds of "objects" had a strikingly similar appearance soon attracted attention (cf., chap.IV, 3), and after 1840 it gradually became clear that the essence of these maunipulations did not lie in the nature of the objects, but in the rules to be followed in handling them, which might be the same for very different types of objects. However, a precise formulation of this idea had to wait for the adoption of the set-theoretic concepts and language; and it is only in 1895 that our defi
% nition of a group, on an arbitrary underlying set, was formu lated by Weber [225]. The trend towards the definition of algebraic structures then gained momentum, and around 1920 all fundamental notions of present-day Algebra had been defined,
% In Analysis, no similar development had yet ogcurred in 1900, The extensions of the ideas of 1limit and continuity which had been formulated always were relative to special objects such as curves, surfaces or functions, The possibi lity of defining such notions in an arbitrary set is an idea which undoubtedly was first put forward by Fréchet in 1904 [69], and developed by him in his famous thesis of 1906 [71]. The simplest and most fruitful method which he proposed for
% such definitions was the introduction of the notion of dis tance (which he called "écart") on a set E, a function d(x,y) defined for any pair (x,y) of elements of E, with values =20 and such that: 1) the relation d(x,y) = O is

% THE CRUCIAL YEARS AND THE DEFINITION OF HILBERT SPACE 117
% equivalent to x =vy; 2) d(y,x) = d(x,y); 3) d(x,z) < < d(x,y) + d(y,2z) for any three elements of E, It is ex tremely remarkable that with such simple axioms it is possible
% to extend most notions and arguments relative to neighborhoods, limits and continuity in the space Rn, which usually are in troduced in relation to euclidean distance. But the greatest merit of Fréchet lies in the emphasis he put on three notions
% which were to play a fundamental part in all later develop ments of Functional Analysis: compactness, completeness and separability., Moreover, he did not 1limit himself to deriving general theorems in an abstract setting, but more than half of his thesis is devoted to very "concrete" metric spaces (as they came to be called later) closely linked to Analysis: the
% space of continuous real functions on a compact interval of R with the topology of uniform convergence, the space RN of all sequences nk> X, with the topology of simple convergen ce, the space of holomorphic functions in the disc |z| < 1,
% with the topology of uniform convergence in compact subsets,
% and finally the space of all continuous "curves", images of [0,1] in R by continuous maps, with a "distance" which is
% a special case of what was later called the Hausdorff distan ce between two compact sets,
% Clearly Hilbert's work immediately lent itself to applica tion of these ideas, and even invited a bodily transfer of euclidean geometry in "infinite dimension", This is exactly what was done by Fréchet himself [72] and by E, Schmidt [ 192] in 1908, In E, Schmidt's paper, we find the definition of 2 2
% what we now call the (complex) space 4~ (or LC), with the

% 118 CHAPTER V
% notions of scalar product and of norm (already written | 4]),
% the definition of orthogonality, of closed sets, and of vector subspaces (called "lineares Funktionengebilde"). The most interesting feature of that paper is the proof of the existen ce of the orthogonal projection of a point on a closed vector subspace, and the purely geometric way in which Schmidt uses this result to discuss the most general system of linear (42) (x|a)) = ¢ (n=1,2, \ldots)
% equations in Hilbert space
% where the a are arbitrary vectors of LZ and the c, ar
% bitrary complex numbers,
% For each n, Schmidt considers the closed linear affine varieties F_ of LZ defined by the equatioas (x|aj) = C for 1 £ j £ n, and the orthogonal projection x(n) of' the
% origin on Fn; the necessary and sufficient condition of
% existence of a solution of the system (42) is that the in=- creasing sequence (Hx(n)”) be bounded; the sequence (x(n)) then has a weak 1limit x in LZ, which is the solution of (42) of smallest norm, Of course, each F_~must be different k=1 k=1
% from the empty set, which means that any linear relation

% g xkak = O between the vectors an must imply ;
% )\ka = O3

% one can then assume (by dropping some of the equations (42)) that the a ~are linearly independent, and in that case, (43) 1x()* A,/D
% Schmidt easily obtains the explicit expression of ”x(n)l with

% THE CRUCIAL YEARS AND THE DEFINITION OF HILBERT SPACE 119
% (aglay) (a,]a,) «ee (ala)
% (ay]ay) (aylay) «en (apla)
% (a la;) (ala) «oo (a]a)
% (44)1 Co eees . |0]
% oQIeee
% This geometric outlook was already shared in 1906-1907 by two other young mathematicians, E, Fischer and F, Riesz, in the remarkable work which led them (independently) to what is now called the Fischer-Riesz theorem, introducing a hitherto
% unsuspected link between Hilbert space and the theory of in tegration ([66], [183, vol.I, p.378-395]). The latter, from
% Cauchy to Jordan and Peano, had evolved in a manner complete ly independent from spectral theory [103], When Fredholm and EF., Schmidt had tried to enlarge the scope of their results on
% integral equations by weakening the assumptions on the kernel K(x,v), they had nothing else at their disposal beyond the ¥
% horrible and useless so~called "Riemann integral"( ), and it ( )As a function K(x,y) of two variables may be "Riemann in
% %
% tegrable" even if the partial functions x+—K(x,y) are not, Fredholm is compelled to assume that integrability both for the kernel K and all its partial functions! Although E,Schmidt wrote his dissertation in 1905, he probably had no knowledge of Lebesgue's thesis at that time,

% 120 CHAPTER V
% is likely that progess in Functional Analysis might have been appreciably slowed down if the invention of the Lebesgue in tegral had not appeared, by a happy coincidence, exactly at the beginning of Hilbert's work on integral equations., With
% the help of this marvellous new tool, Fischer and F, Riesz could define the space LZ(I) over a compact interval I € R,
% consisting of square integrable functions, when two functions are identified if they only differ in a set of measure O, Their fundameutal result is that, if to each function f € LZC% one associates the sequence (xp) of its Fourier coefficients
% with respect to a complete orthonormal system (equations (32)), this defines an isomorphism of LZ(I) onto LZ; from that it follows that LZ(I) is complete and separable. A byproduct
% was of course that the results of Fredholm and E, Schmidt
% could be applied without change to any integral equation where the kernel is only supposed to belong to LZ(IxI), since it
% is then equivalent to a system of linear equations correspond ing to a "completely continuous" bilinear form in the sense of Hilbert.,
% But the most important consequence of the Fischer-Riesz
% theorem is that it opened the way to the definition of the LP spaces and to the general theory of normed spaces, which
% will be the subject of the next chapter,

\chapter{Duality and the definition of normed spaces}
\label{ch:6}
\setcounter{equation}{0}

\section{The search for continuous linear functionals}
\label{sec:6.1}

% In chap.IV, 3, we saw that in 1897 C. Bourlet solved for the first time the problem of the determination of a linear map U: E -+ F Dbetween "function spaces" by conditions of continuity. In a short Note published in 1903 ([94], vol.I,
% p.405-408) Hadamard attacked the same problem with E = ¢([a,b]), space of real continuous functions in an in terval [a,b], F = R, and "continuity" means for him that U(f,) tends to U(f) when f_ tends to f wuniformly. He chooses a fixed function F such that for any continuous function f, one has
% (1) f(x) = 1im n f(t)F(n(t-x))dt Il =pco = a
% uniformly in x3; one has then
% (2) U(f) = lim £(t)s (t)dt
% b

% N=boo
% 'a,

% where @n(t) is the value of U at the function x+nF(n(t-x)) 2
% one may take F(x) = eTMTM , so that ¢ ~is continuous, but the choice of F is largely arbitrary (the argument is a typical case of what later will be called a "regularization" 121

% 122 CHAPTER VI
% |
% |
% process).
% In two papers published in 1904 and 1905, Fréchet gave another proof of Hadamard's theorem, and, what is more inte
% resting, began to investigate the similar problems when c¢(La,b]) is replaced by another "function space"; for ins tance E?O], he remarked that if one takes for E +the space fi([a,b]) of all bounded integrable functions in [a,b] (Con
% tinuous or not) with the topology of uniform convergence, there were other continuous linear functionals than those
% given by Hadamard's formula, for instance the mappings f e clf(xl) teee+ c f(x), where the x; are arbitrary points of [a,b] and the C ; constants. Similarly, if one takes for E the space of all €' functions in [a,b], where
% convergence means uniform convergeince for the function and its derivatives up to order r, Fréchet showed that the con
% tinuous linear functionals could then be written b
% fh»Cof(a)+clf’(a)+ \ldots+cr_lf(r-l)(a)+lim f(r)(t) 2 (t)dat . n=oa
% As soon as the study of Hilbert space began (chap.V, 3), Fréchet [72] and F. Riesz ([183], vol.I, p.386-388) indepen=-
% dently showed that continuous "linear functionals" on Hilbert space Lé (for the strong topology) could be written unique ly as x +(x]|a) for a vector a € L=,
% Finally, in 1909, F. Riesz ([183], vol.I,p.400-402) was able
% to give a better form to Hadamard's theorem by removing the arbitrariness of the sequence (@n); his idea was to use the
% Stielt jes integral, as Hilbert had done in his work on spectral theory (see chap.VII, 2): he showed that any continuous linear

% DUALITY AND THE DEFINITION OF NORMED SPACES 1273 functional : ¢([a,b]) » R could be written uniquely (3) Us £ +— f(x)da (x)
% b
% where a is a function of bounded variation in [a,b], pro vided one imposed on «a the additional conditions of being continuous on the left and such that a(a) = 0. His method
% consists in considering, for any t € [a,b], the function f, € c(fa,b]) equal to x-a for ac< x< t, and to t-a for t < x < b, and the function A: trs U(ft); he shows that this function is Lipschitzian, and takes for =-g(t) one of the "derived numbers"of A at the point t; it is then easy to show that @ is a function of bounded variation, and it is a standard procedure to modify it in such a way that it sa tislfies the additional conditions meuntioned above without changing U.
% Although the contemporaries did not realize the novelty of F, Riesz's approach, we are justified in seeing in his results (as he himself did) a radical departure from the conceptions
% of linear algebra prevalent in his time:
% 12 Whereas, even for the space LZ, it was possible, due to the Fischer-Riesz theorem, to identify the elements of the space with sequences of numbers, generalizing the dominant
% Cayley concept of linear algebra as a theory of "n-tuples', no such identification was possible for ¢([a,b]), where one had to work directly on vectors, and not on their "coordinates"
% 29 Functions of bounded variation may be discontinuous at a denumerable set of points, and therefore it was impossible to identify any more the continuous linear functionals on C([a,b])

% 124 CHAPTER VI
% to the elements of that space (again in contrast to what happened in LZ according to the Riesz-Fréchet theorem).
% These features would be still more conspicuous in the theory of LP ana P spaces, which F., Riesz began to investigate in 1910 ([183], vol.I, p.4903).

\section{The \protect{\(L^p\)} and \protect{\(l^p\)} spaces}
\label{sec:6.2}

% Once the L2 spaces had been defined, it was a natural ge neralization to define similarly the function spaces Lp(I) for any interval I € R, as the set of all complex wvalued measurable functions f defined in I and such that |f|P is integrable, for any p > O (two functions being idegtified if they are almost everywhere equal). The study of these spaces was begun by F. Riesz in a fundamental paper ([183], vol.I, p.4h1-497), second only in importance for the develop=- ment of Functional Analysis to Hilbert's 1906 paper (chap.V, 2).
% Riesz limited himself from the start to the casze p > 1, in n 1 n 1/q (&) | = /p( X |bk|q) / for = + 1 1
% - k=1 P d
% order to be able to use the H8lder and Minkowski inequalities
% n
% a,b. | s (2 |a|P) k= 1 k- k k=1 k
% (5) (% |a+b PP n D 1 n
% k=1 k=1 k
% < (3 Iakl ) + (= Ib
% which he first extended to measurable functions, showing that if fe LP(1), g¢ L9(1T) then fg is integrable and T )I T
% (6) | £(x)e(x)ax| < (f'|f(x)|pdx)l/p<f'Ig(x)lqu)l/q,

% DUALITY AND THE DEFINITION OF NORMED SPACES 125
% and that if f € LP(x), ge LP(1), then fige LP(1I) and (7) (flf(><)+g(x)lpdx)l/p < If(X)Ipcl><)l/p+((|g(x)IPdX)l/p :
% I I I
% (8) {- f(x)gd(x)dx = ¢,
% His central theme is the study of infinite systems of linear equations
% I
% where the gd belong to Lq(I) and one looks for a solution fc Lp(I); this may be considered as the generalization of the problem E, Schmidt had treated in LZ, due to the Fischer=- Riesz theorem (chap.V, 3, equations (42)). In order to adapt Schmidt's method to this problem, F. Riesz begins by extend
% ing a number of definitions and results from the theory of Hilbert space: strong convergence of a sequence (fn) of functions of LP(I) to f € LP(I) 4is defined as meaning that ( If(x)-fn(x)|pdx tends to O. For weak convergence, he
% IX X a a
% first takes as definition that ( £ (t)dt tends to ('f(t)dt
% for all numbers x € I; and although he proves a little later that this definition 4is equivalent to the fact that the inte grals (' (f(x)-fn(x))g(x)dx tend to O for all g € L9(1), T
% he essentially uses the first definition to prove the genera lization of Hilbert's "principle of choice" (i.e. the weak compactness of the unit ball in Lp(I)), which will be one of his main ingredients in the solution of (8). The other in gredient is derived from a result obtained by E. Landau in
% 1907 [136]; in 1906, Hellinger and Toeplitz had shown that if a sequence (a) is such that the series 3 a x is conver n

% 126 CHAPTER VI
% gent for all sequences (xn) in Lz, then (an) itself be longed to LZ (106]; Landau proved that, more generally, if a x_ 1is convergent for all sequences (xn) such that B8TMM PM < +o, then 7 |an|q < 4o, Approximating functions
% n
% of Lp by functions having only denumerably many values, F. Riesz deduced from Landau's result that if, for a measurable function g, the product fg idis dintegrable for all functions f € Lp, then necessarily g € L4,
% His solution of (8) then proceeds along the same lines as E. Schmidt; he starts with a finite system (8), for which, using the standard method of analysis (Lagrange multipliers) he proves the existence and uniqueness of a solution f € Lp for which '( |f(x)|pdx is minimum. The problem is then to /I
% find a necessary and sufficient condition on the Ca such that, when one picks from (8) a finite system corresponding to the indices @ in an arbitrary finite subsecv H, the cor responding minima M, of the integral f’ |f(x)|pdx taken for the "minimal" solutions, are uniformly bounded (independ ently of H); the use of the two ingredients mentioned above then leads to the existence of a solution of (8) by an argument similar to E, Schmidt's. Of course, an explicity expression of My, (similar to formula (43) of chap.V, 3) is not availa ble here, and the originality of F. Riesz lies in having found a completely different type of condition, namely the ce¢xistence
% of a number M > O such that, for any finite subset H of indices, and any family (xg) of scalars, one has the ine o}
% quality

% DUALITY AND THE DEFINITION OF NORMED SPACES 127 1/q (9) | = A.c | = Mo L |a_e (x)]%ax) . €U a o . cH a
% Fo. Riesz in particular applies this conditions to the special case in which the g, are all the functions of Lq(I); (9) is then equivalent to the continuity in 1.9 of the linear functional [ defined by L(gd) = ¢C g’, and he has thus gene ralized his previous results on LZ and C(I), proving what we would now express by the statement that the dual of LI(I) can be identified with LP(1),
% Of course the name "dual" is not yet used by F, Riesz, but P . . . .y P
% he explicitly considers, for a "bounded" linear mapping T of LY dinto itself (defined by the condition that |T(£)(x)]| ax remains bounded for all £ sch that ghlf(x)|pdx < 1), the
% I
% T
% transposed mapping T’ defined by the equation
% (10) g.T(f)(x)g(x)dx = f(x)T’' (g)(x)dx for all f ¢ Lp(I).
% I I \
% Indeed, for a function g € L3(I), this defines (up %o a null set) a unique function T'(g), which (by F. Riesz's previous results) also belongs to Lq(I); furthermore, it is easy to show that the mapping T’ of LY into itself is also linear and "bounded"., F. Riesz then used this concept to obtain a necessary and sufficient condition for the mapping T to be bijective: he showed that such a condition is the existence of a number m > O such that both inegqualities
% T T
% ( IT(£)(x)|Pax = me {ilf(x)lpdx

% (11)
% T T
% ( T’ (&) (x) ] %ax > m-f | &(x)]%ax

% 128 CHAPTER VI
% are satisfied for all f € LP(I) and a1l g ¢ L9(1). F. Riesz had thus given, for the first time, examples of what
% we now call reflexive Banach spaces not isomorphic to their dual(*). In his 1913 book on infinite systems of linear
% equations ([183], vol.II, p.835-1016 and [184]) he treated in a similar way the Lp spaces for p > 1 (defined as the set of sequences (xn) of complex numbers such that E lxnlp < 4w); in addition he stated without proof that for p # 2, nNo iso= morphism of Lp and Lp existed any more, in contradistinct ion to the Fischer-Riesz theorem (ibid. Vol.I, p.4ih-lLlL5),

\section{The birth of normed spaces and the Hahn--Banach theorem}
\label{sec:6.3}

% In 1911, F, Riesz combined his methods for the treatment of the system (8) in LP with the Hadamard-Riesz theorem on 1i near functionals in C([a,b]) in order to study the systems (12) gy (*)E(x) = ¢
% of linear equations
% b
% a
% where [a,b] idis a compact interval in R, the g, are con tinuous in [a,b], the ¢ are given scalars, and one has Q to determine a function £ of bounded variation in [a,b] satisfying the equations (12) for all o. This may be consi dered as a generalization of a problem which had first been
% ( )The dual of Ll(I) for a compact interval I € R was shown
% *
% to be isomorphic to L®(I) by H. Steinhaus [202]; he uses the ffact that in that case L2(I) c Ll(I), and therefore a contin uous linear functional on L (I) is also continuous on LZ(I).

% DUALITY AND THE DEFINITION OF NORMED SPACES 129
% proposed and solved by T. Stieltjes in 1894, the "moment prob lem": it consists in determining an increasing function  in [0,+o[ such that
% (13) | xdg(x) =c 20 for =n=0,1,2, \ldots
% (" n
% /0O
% (the left hand sides are called the "moments" of the function £, a terminology stemming from pfobab?lity theory) [ 205] ;
% the same problem was later considered when the interval [0, +[ is replaced by ] -w,+o0[ (the "Hamburger moment problem") or by a compact interval [a,b] (the "Hausdorff moment problem") (3]
% The solutions to these "moment problems" consist in giving explicit conditions on the ch involving existence (or exis tence and uniqueness) of the function & (or rather of the measure df ). The condition given by F, Riesz for the exis tence of a solution £ of the general system (12) is similar to condition (9), namely the existence of a number M > O such that, for any finite family (XQ)QGH of scalars, one has (14) S A.c.| € Mesup | = A& (x)] ; laEH o G x e & ° ’
% (he explicitly observed that the right hand side of this ine quality is the limit of the right hand side of (9) when ¢ tends to +»). His proof is similar to the proof for (8); he first restricts himself to the case of finite systems (12), obtains the existence of a "minimal" solution of such a system, and then, using a "principle of choice" (in our language, the weak compactness of the unit ball in the space of Stieltjes measures), he shows that the condition (14) is sufficient for

% 130 CHAPTER VI
% an arbitrary system (12); his procedure is more complicated than for (8), because even in the case of a finite system (12), there is no more uniqueness for the "minimal" solutions ([183L
% a aQ . o a &
% We now interpret condition (9) in the following wav: first, if ¥ Ag =0 in 19(I), then T A_c, = O; this implies that, if F is the vector subspace of Lq(I) generated by the ga, there is a well determined linear form L defined in F such that L(g@) = Cy for every @. Condition (9) then means that this linear form L 1is continuous in Fj the existence of an f € LP(I) such that L(gg) = i f(x)g@(x)dx
% !
% for all <« then means that [ can be extended to a continuous linear form defined in the whole space Lq(I); in other words, it is a special case of what we now call the Hahn--Banach the
% orem. There is a similar intecrpretation of condition (1h), replacing Lq(I) by c(la,b]).
% Such an interpretation of his results was not given by F. Riesz; the first mention of that point of view appears in a paper written in 1912 by the Austrian mathematician E.Helly (1884-1943), in which he gives a different proof of F. Riesz's results on the systems (12) [107 bis]. After an interval of 9 years (due to the first World War, in which he was a prisoner of war in Russia), Helly returned to his method in a paper of 1921 [108] which again should be considered as a landmark in
% the history of Functional Analysis, since instead of consider ing special spaces such as the P, L or ¢({a,b]), he for the first time deals with general "normed sequence spaces"
% by methods which do not depend on special features of the space, contrasting with the ones used by E. Schmidt and

% DUALITY AND THE DEFINITION OF NORMED SPACES 131
% (*) F. Riesz.
% Helly considers vector subspaces of the vector space CN of
% all sequences of complex numbers, and assumes that on such a subspace E there has been defined a norm ”x” (he does not use that name nor the notation) such that: 1) |[[x|| = 0 and the relation Hx” = 0 is egquivalent to x = 0; 2) HxxH = = |K . x” for any scalar X3 3) Hx+y” < Hx” + Hy”; this defines on E a distance d(x,y) = |[x-y|]] in the sense of Fréchet. Of course norms had been defined in the spaces Lp, L? anda c¢([a,b]); but Helly seems to be the first to have noticed the relations of that notion with the concepts of
% convexity introduced earlier by Minkowski in his "Geometry of numbers ([ 161] and [162]). He had shown that the concept N L N L d n L
% of norm on a finite dimensional space R (with the scalars limited to real values) was equivalent to the notion of "sym metric convex body", i.e. a closed, symmetric, bounded convex set in which the origin O is an interior point: such a set B can be defined by an inequality p(x) < 1, where p 1is a
% uniquely determined norm. The boundary of such a set is de=- fined by the equation p(x) = 1, and Minkowski had proved that for each point X of that boundary, there existed at least one hyperplane of support H, containing X, and such that B lies entirely on one side of H. If, for an n-tuple of real numbers u = (ul, \ldots,un) and a point x = (Xl, \ldots,XnL
% (*)It seems that during the period 1910-1920, F. Riesz always had in mind possible axiomatic generalizations of his results, although he did not publish anything in that direction ([183], vol.I, p.U52),

% 132 CHAPTER VI
% one writes (u,x) = U Xy oot U X, the egquation of H has the form (u,x) = 1 for a suitable u, and one has the ine quality (u,x) < p(x) for all x € Rn, with (u,xo) = p(xo);
% the n-tuples u being identified with the corresponding 1li near forms X+ (Uu,Xx) on Rn, Minkowski had also defined the "support function" q(u) = sup (u,x)/p(x), and shown that it was also a norm on Rn, "diag" to p and such that the hyper planes of support of B are the hyperplanes (u,x) = 1 with q(u) = 13 furtherwmore, p is the norm "dual" to q, in other words p(x) = sup (u,x)/q(u).
% u#O
% To transfer to spaces of sequences these concepts and defi nitions, Helly associates to E the subspace E’ of " con sisting of all the sequences u = (un) such that the series % u x converges for all x = (xn) in E (*), and he then ;Ln&kkns (u,x) =Y u x_ . For any u € E’, the number n
% Hu” = sup (u,x)/”x” defines a norm on E’ provided it is x#0
% not O for some eiements u # O. Excluding that case, Helly first obtains a weak generalization of Minkowski's result on the hyperplanes of support; if B is the subset of E defi ned by “x” < 1, he shows that the hyperplane H defined by
% (u,x) = 1 meets B if Hu” < 1, does not meet B for ”u” > 1, Dbut if ”u” = 1, the intersection H NN B may very well be empty: an example is given by taking E = Ll, E’= =1~ and for H the hyperplane ; (l-%)xn = 1. n=1
% The central problem in Helly's paper is the solution of a system
% ( )This is not always the dual of E as we now understand
% *
% that word,

% DUALITY AND THE DEFINITION OF NORMED SPACES 133 (15) <u)’x> = C (V=1929°°°) V
% (v) where the u belong to E’ and one looks for a solution x € E. The inequality |(u,x)| < [[ull*llx|| immediately yieilds the necessary condition similar to Riesz's conditions (9) and
% (14), namely the existence of a number M > O such that (16) | & A c. | = M v=l ¥V vV il >‘\)u)”
% n
% for any n and all choices of scalars )V; but the example
% given above (for a single equation) shows that there may well be no solution such that |x|]| = M, eveu when condition (16)
% is satisfied.
% Helly, as Schmidt and F. Riesz had done, first considers the case of a finite system (15) of N equations, where as usual the u(v) may be supposed to be linearly independent.,
% The mapping £ x»~>(<u(v),x>)lssz of E into @N is then surjective; Helly shows that on CN, Hy” = inf Hx” is a norm (for us it is the natural norm on E/f-l(O) deduced fram the norm on E); condition (16) then guarantees the existence 1’ 1
% of a solution x. of (15) such that ”x” < M for any M, > > M (if not necessarily for My = M).
% The passage from finite systems (15) to the general case is the most original idea of Helly; he splits the problem in two:
% A) Given M, > M, find a linear form L: E’ 4 ¢, such that lL(u)l < Ml. u| for all wu € E’ and such that L(u(v)) = ¢ V
% for all V.
% B) When such a linear form L has been found, find if pos=- sible an element p € E such that (p,u) = L(u) for all

% 134 CHAPTER VI
% u € E’.
% To treat problem A), Helly assumes the additional condition that B is separable as a metric space; he then proves the
% existence of a solition (a special case of the Hahn--Banach theorem) in the following way. Let (p(v)) be a sequence of elements of E’ which is dense in that space., Helly chooses an increasing sequence M < M(l) < M(Z) < oo o< M1 of
% numbers, and the main point of his proof counsists in showing that there exists a family (Yv) of complex numbers such that, for any pair of integers m = 1, n =2 1, and auny pair of fa milies (Av), (uv) of scalars, one has

% v=1 v=1 V
% n m

% [ - X xvu(v) + ¥ M p(v)
% It is then easy to show that there exists a linear form [ on E’ such that L(p(v)) = Y, for all indices v, and that
% it is a solution of problem A).
% The proof of (17) is done by induction on m, the case m = O being the assumption (16). One has then to prove the existence of a point Yms1 € € which, for any integer n > 1 and any pair of families of scalars A
% ( V)lfivSn’ (HV)lsvgm
% belongs to all disks defined in ¢ by
% n m
% | X A.C. + % ou Y o+ Yy | <
% y=1 VY V =1 V Vv m+1

% (18)
% (v), plmel)y n m < M(m+l)H X xvu(v) + I u,P
% =1 v=1

% However, a general result on convex sets in a finite dimensio nal space, proved by Helly himself, reduces that question to proving that any three of the disks (18) have a common point;
% DUALITY AND THE DEFINITION OF NORMED SPACES 135
% and this is shown by Helly to be a consequence of the result proved befTore for finite systems (15).
% Turning to problemn B), Helly discovers that it is quite pos
% sible that it has no solutionj; in our language, he gives the first example of non reflexive Banach spaces( ). That example
% *
% 00 o
% k=1 N kx—nk|; Helly
% is the space FE of all sequences (x,) such that the series ¥ %, converges, with the norm HXH = supl L X
% proves that E’ consists of all sequences (uk) such that o
% [lu” = |ul| + k1 Iuk+l-uk is finite, Hu” being the natural norm on E’3 then if one takes [ (u) = 1lim u L dis contin K=
% k’
% uous on E’ Dbut there is no p € E such that L (u) = (p,u). Starting from the work of F, Riesz and Helly, it was a na=
% tural generalization to define morms on arbitrary vector spaces over R or €, and not only on spaces of functions or on subspaces of CN. This was done independently by H, Hahn [97] and S. Banach [12], who restrict themselves to
% complete spaces,
% Banach's paper is his thesis, written in 1920: although he does not mention convexity, he is careful to develop and ex tensively use a geometric language. He is mainly interested in continuous linear operators u: E = F, where ¥ and F are arbitrary normed complete spaces, and in limits of sequen=- ces of such operators. Hahn's point of view is similar,
% *
% ( )F. Riesz had already observed that one could define on the space of functions of bounded variation continuous linear funcionals which were not of the form F+{,:f(x)d(s) for a a
% continuous function f (for instauce one can take for f an increasing discontinuous function ) ([183], vol.II, p.827).

% 136 CHAPTER VI
% although he is only concerned with linear forms; neither he nor Banach are at that moment interested in the problem of
% extension of linear forms, and we postpone a more detailed discussion of their papeirs of 1922-23 to 4., We should how ever mention that in his thesis Banach gives the "abstract" formulation of the method of successive approximations (chap. IT, 1) as a "contraction principle": if F is a mapping of a complete normed space E into itself such that HF(x)-F(y)” < k”x-y” with O< k< 1, then the sequence (Xn) defined by induction as x . = F(xn) (xo arbitrary) converges to the unique "fixed point" x, such that F(x) = x.
% It was only in 1927 that Hahn returned to Helly's paper, in the general context of complete normed spaces, and completely solved the extension problem for such spaces [98]. He proceeds by induction as Helly had done, but at the same time he great
% ly simplifies and generalizes the method by introducing, for the first time in general problems of Functional Analysis (*), transfinite induction instead of the ordinary kind, In a com plete normed space E, one has a vector subspace V and there is defined on V a (real valued) linear form f such that |f(x)]| < M||x|| for x € V; the problem is to extend f to a linear form F on E such that |F(y)| < M||ly|| for y€ E., Hahn begins by showing the existence of an ordinal v, (%) . . : -
% and of a mapping & —V, which, to very ordinal £ < y asso- g Transfinite induction had been used by analysts ever since Cantor, but the application of transfinite induction closest to Hahnt's is probably the method by which Banach, in 1923, had proved the existence on R of a "measure" defined on all subsets of R and simply additive [13].

% DUALITY AND THE DEFINITION OF NORMED SPACES 137
% ciates a vector subspace Vé of E such that VO =V, V. € V for < V has codimension 1 in V, and E is the union of the V for & < y. The problem is then g easily reduced to the case in which V has codimension 1 in
% E, and then E is generated by V and an element a ¢ Vj Hahn coasiders the l.,u.b B of the numbers f(x) - M”x-a“ for x € V, and the g.l.b, A of the numbers f(x) + MHx-a” for x € V, and, using the assumption [f(x)]| < M||x|| for x€ V, he easily shows that A < B; the extension F is then defined by F(x+Xa) = f(x) + Ac for all % € R, where ¢ dis any number such that A < c < B.
% As a particular case of his theorem, Hahn shows that for any vector a O in FE there exists a continuous linear 9 form I on L such that ||[L|] = 1 and 5(a) = |a|l; he then formally introduces the dual space E’ of E ("polare Raum" in his terminology) which is not reduced to O due to the preceding result; he writes B(u,x) instead of wu(x) for x € E, u€ E’, aund considers for any x € E, the linear form c¢(x): u+—B(u,x) on E’, for which he shows that

% Fe(x)Il =[x
% « In other words, he has defined a linear iso

% metry c¢c of E dinto its second dual E’, and he says a space E is "reguldr" if ¢ is bi jective (our reflexive spaces). It may therefore rightly be said that with this paper of Hahn, duality theory at last has come into its own.,
% Two years later, Banach, who apparently was not aware of Hahn's paper, published the same theorem with the same proof (he later acknowledged Hahn's priority); in addition, he re cognized that the argument could be generalizeds: if p idis a

% 138 CHAPTER VI
% real valued function defined in a vector space E and such that p(x+y) < p(x) + p(y) and p(irx) Ap(x) for A = O, and if f 4is a linear form defined in a vector subspace V of E and such that f(x) < p(x) din V, +then it is possible
% to extend f +to a linear form F defined in E and such that F(x) < p(x) din E. This extension was to play later an important role in the development of the theory of locally convex spaces (cf. chapter VIII).

\section{The method of the gliding hump and Baire category}
\label{sec:6.4}

% In his 1922 paper [ 97], Hahn proved the following theorem: let E Dbe a complete normed space, (un) a sequence of con tinuous linear forms on E, and suppose that for each x¢€¢ E,
% the sequence of numbers |un(x)| is bounded by a number de pending on x3; then the sequence of the norms ”un” is bounded.
% The proof is by contradiction; assuming that the sequence (”un”) is unbounded, one determines by induction a sequence (Xk) in E and a sequence (nk) of integers such that: (00)
% 12 the series £ X, converges to an element x € E; k=1
% 20 Z lu. (x.)] = 1;
% j=k+1 "k Y
% k=1
% 0 fuy ()l 2 2y ()1
% k=1 o
% Then one has for each k,
% u (x)] =2 |u (x)] - = |u, (x.)]| - T lu. (x.)] =2 k-1 | Ny | I n ok =1 "k Y j=k+1 kY which contradicts the assumption. To do this, one assumes the u have been determined for Jj < k, and one considers a ball

% DUALITY AND THE DEFINITION OF NORMED SPACES 139
% B, | x|l « 27 - inf (”unj||+l)-l
% k
% Jj<k
% in E; the assumption that (HunH) is unbounded guarantees the existence of an index n, and a point Xk;E Bk for which condition 3¢ holds; conditions 19?2 and 292 are then deduced from the choice of the radius of the ball Bk'
% This is often called the "method of the gliding hump": in the sequence of values |unk(xj)| when j varies from 1 to +o, the index j = k corresponds to a "hump" much bigger than the sum of the contributions of the other indices.
% The result can be put in a different form: if the sequence (HunH) is unbounded, there exists at least one x € E such that the sequence (|un(x)|) is unbounded,
% In this form, the first example of the method of the gliding hump is probably the way in which Lebesgue, in 1905 ([138], vol.,III, p.101] and [139, p.86-88]) constructed a continuous periodic function F(x) in [0,2n] whose Fourier series di verges at the point O, He had proved that, if one writes Sn(g) for the sum of the first n terms of the Fourier se ries of a continuous function g, it is possible to find a sequernce (gn) of continuous periodc functions of bounded variation such that |gn(t)| < 1 din [0,2r] and that the sequence of values Sn(gn)(O) tends to +», He then defines
% €y ®x Kk
% F(x) = elfl(nlx) + ezfz(nzx) +oeet ekfk(nkx) tees
% o
% are > 0 and such that Z
% where the =1, the T k=1
% are continuous periodic functions of bounded variation such that [f, (t)| < 1 in [0,2n] and that |spk(fk)(o)| 2 k/e, for an increasing sequence (pk) of integers. Finally the

% 140 CHAPTER VI
% increasing seguence (nk) of integers is chosen in such a way that n,_ K > and that, for the continuous function Ny -1Proa of bounded variation Fk(x) = elfl(nlx) +ooot ek-lfk-l(nk-lx)’ all the sums Sn(Fk)(O) are <2 in absolute value for n = n, (they converge to Fk(O)). This choice implies that, for j > %k, the sum of the first n terms of the Fourier kPk series of fj(njx) is reduced to the first term of the series,
% hence is <1 in absolute value; using these definitions it is easy to check that |S (F)(0)| = k-3 for all k. “xPx One year later, Hellinger and Toeplitz, two students of
% Hilbert, found a rather surprising complement to the defini
% tion he had given of a bounded bilinear form (chap.V, 2); instead of assuming that [k (x,y)| £ M for all n and all
% p p P
% X = (xp) and y = (yp) such that I xi < 1 and I y2 < 1,
% they showed that it was enough to assume that for each such pair (x,y), one had [Kn(x,y)| < Mx,y for all n, where
% the number Mx y might depend on x, y in an arbitrary way, ’
% Indepeundently of Lebesgue, they proved that result by a "gliding hump" method, constructing a pair (x,y) for which the sequence (|Kn(x,y)|) is unbounded if X is not a bound ed form in Hilbert's sense [106].
% During the next 20 years, many more examples of the "gliding hump" method appeared in the literature: Lebesgue used it repeatedly in a 1909 paper on "singular integrals" ([138],
% vol.ITT, p.259-351), where one looks for conditions on "kernels" K_ insuring that the integrals I, f(t)Kn(t,x)dt tend to f(x) when n tends to +», for various ;inds of function f,.
% The method was also prominent in the study of "summation pro-

% DUALITY AND THE DEFINITION OF NORMED SPACES 141
% cesses", where one "transforms" a sequence (xn) into a se=- (o0}
% quence (yn) by the formulas y_ = X a_ X and has to n no1 Dm m’ look for conditions on the a__ insuring that when (xn) has a limit, (y,) tends to the same limit ([193], vol.II, p. 389-321). Hahn's paper of 1922 [97] was written to give a general background to all these results, showing that they all were consequences of his general theorem. Independently, Banach, in his thesis, proved a theorem more general than Hahn's, the u being now continuous linear operators from a complete normed space E dinto a complete normed space F3; he showed that the assumption that the norms Hun(x)H are bound ed for each x by a number depending on x, implies that the sequence of the norms Hun“ is bounded.
% Finally, in 1927, Banach and Steinhaus (using an idea of Saks) discovered that this theorem could be proved without
% using the "gliding hump" method, by an application of a theo rem Baire had proved in 1899 [11]: he had shownthat in R",
% the intersection of a denumerable family of dense open subsets is itself dense( ); this implies that if u 1is a real func-
% . . . . . n .
% *
% tion defined and lower semi-continuous in R , and if u(x) < < 4+ for each x € Rn, then any non empty open subset U of R contains a non empty open subset V such that sup u(x) < xXc€V
% < 4o, These results and their proofs immediately generalize when Rn is replaced by an arbitrary complete metric space. If now H 1is a set of linear mappings from a complete normed space E into a complete normed space F, and if for each
% ( )For n =1, the same result had been proved two years
% *
% earlier by W, Osgood [170].

% 142 CHAPTER VI
% ucH ucH
% x € E, sup ||lu(x)|| < +o, the function p(x) = sup ||u(x)| is
% lower semi-continuous, and from the Baire theorem it follows that p 1is bounded in a neighborhood of O, which implies that sup [[u]] is finite [16].
% u€H
% 5 - Banach's book and beyond
% In 1932 S. Banach published a book [15] containing a com prehensive account of all results known at that time in the theory of normed spaces, and in particular the theorems he had published in his papers of 1923 and 1929. A large part was devoted to the concept of weak convergence and its gene ralizations, which he had begun to study in 1929; we shall postpone to chap.VIII, 1 the discussion of these questions., The most remarkable result contained in that book is another consequence of Baire's theorem, discovered by Banach, and much deeper than the Banach-Steinhaus theorem: if u is a contin uous linear mapping from a complete normed space E dinto a complete normed space F, then either u(E) is meager in F (a2 set "of first category" in the terminology of Baire), or u(E) = F. An immediate consequence is the famous closed graph theorem: 4if u dis a linear mapping from E +to F having a closed graph in EXF, then u 1is continuous. These surpris ing results have become two of the most powerful tools in all applications of Functional Analysis.
% These features, as well as many applicatiouns to classical Analysis, gave the book a great appeal, and it had on Func tional Analysis the same impact that van der Waerden's book

% DUALITY AND THE DEFINITION OF NORMED SPACES 143
% had on Algebra two years earlier. Analysts all over the world began to realize the power of the new methods and to apply them to a great variety of problems; Banach's termino logy and notations werc universally adopted, complete normed spaces became known as Banach spaces, and soon their theory was considered as a compulsory part in most curricula of graduate students. After 1935, the theory of normed spaces became part of the more general theory of locally convex spaces, which we shall discuss in chapter VIII; more recently however, there has been a reuewed surge of interest in the special propevrties of normed spaces and their "geometry'"; it is too soon, as yet, to have a clear idea of the scope of these results and of their relation to other parts of mathe matics, and we refer the interested reader to [ 4], [17], [47], [50], L116], [134], [149], [150] and [185].

\chapter{Spectral theory after 1900}
\label{ch:7}
\setcounter{equation}{0}

\section{F. Riesz's theory of compact operators}
\label{sec:7.1}

% We already mentioned that Fredholm's paper attracted many mathematicians to the theory of integral equations, and also to the theory of infinite systems of linear equations, espe cially after Hilbert had given it a new impetus. We shall not examine these papers, most of which are concerned with special problems, without much bearing on the progress of Functional Analysis, and we refer the interested reader to [ 107] (in par ticular p.1543-1552 and p.1574-1575).
% The "Fredholm alternative" corresponded, in "infinite dimen sional linear algebra" to the classical relation between kernel and image of an endomorphism of a finite dimensional vector space over (€. But for such endomorphisms, much more was known, namely the Jordan normal form which characterized them up to "similitude", and a natural question was to investigate similar properties of the Fredholm operators. However, only partial results in that direction were obtained, before F.Riesz in 1916 (in a paper written in Hungarian ([183], vol.II, p.1017-1052) and only published in German in 1918 (ibid., p.1053-1080)) gave a complete answer to that question, and
% 14k

% SPECTRAL THEORY AFTER 1900 145
% found the proper context to Fredholm's results, in what is now known as the Riesz-Fredholm theory of compact operators, F, Riesz never adopted Hilbert's method of dealing with 1li
% near equations via bilinear forms, but followed Fredholm in using instead operators. In his work on 4P spaces ([ 1837, vol.II, p 876-911 and [184]), he therefore had translated Hilbert's conception of a completely continuous bilinear form (chap.V, 2) into the notion of completely continuous operators: for him it was a linear mapping of 4F into itself which
% transformed weakly convergent sequences into strongly converg ent ones, The novelty in his 1918 paper is that he realized 'that he could give an equivalent definition without mentioning
% weak convergence, using instead the general concept of compact ness introduced by Fréchet: the condition was that the linear
% operator transformed a bounded set into a relatively compact one (for the strong topology). Now this can be defined for an arbitrary normed space instead of Lp; in his 1918 paper F. Riesz restricted himself to the space (C(I) for a compact interval I € R, but he explicitly mentioned that he merely
% considered that case as a "touchstone" for more general con ceptions (ibid.,p.1053). And indeed, after he has defined the norm on C(I), he never (except when proving that the Fred
% holm operator for continuous kernels is completely continuous in C(I)) wuses anything except the axiomatic definition of a norm (remember that this definition only appeared in print 4 years later!).,
% In my opinion, F, Riesz's 1918 paper is one of the most beautiful ever written; it is entirely geometric in language
% 146 CHAPTER VII
% and spirit, and so perfectly adapted to its goal that it has never been superseded and that Riesz's proofs can still be transcribed almost verbatim, He starts from two almost obvious remarks: 1) in a normed space E, if V is a closed vector subspace not equal to E, there is a vector x € E such that ||x|| = 1 and [x-y|| = % for all y € V; 2) a subset
% S c E cannot be relatively compact if there is in S an in finite sequence (x_) such that ”xj-xk” > % for all pairs of distinct indices, The first consequeince is the celebrated theorem characterizing finite dimensional normed spaces as the only locally compact ones: one has only to cover the ball |x|| « 1 with a finite number of balls ”x-aj” < 1/4, and then theve cannot be any point such that | z|| = 1 and | z=-y|| 2 1/2 for all points of the (necessarily closed) vector subspace V generated by the aj.
% F. Riesz then considers a completely continuous linear map ping u of E dinto itself (or, as we now say, a compact 1li near mapping), and studies the endomorphism Vv = lp=-u of E. Using the two remarks above and very simple arguments, he
% proves in succession the following properties:
% a) the kernel V-l(O) has finite dimension;
% b) the image vVv(E) is closed in Ej;
% c) +the codimension of V(E) 4in E is finite.
% The next step is to consider the iterates Vk of v, the kernel N and the image Fk of’ Vk ( ); the Nk form an k
% *

% (*)
% In finite dimensional spaces, this method to obtain the
% the Jordan normal form of an endomorphism had been developed by E. Weyr [ 228].

% SPECTRAL THEORY AFTER 1900 147
% increasing sequence of closed subspaces of finite dimension, the Fk a decreasing sequence of closed subspaces of finite codimension, F. Riesz shows, by contradiction and using re mark 2) above, that there is a smallest integer n such that
% Nk+l = Nk for k =2 n; 4it is then an easy matter to prove that Fk+l = Fk for k =2 n, and that E is the topological direct sum of Fn and Nn; the restriction off v to Fn is
% a linear homeomorphism of Fn onto itself, In particular, if N, = V-l(O) = {0}, v 1is a linear homeomorphism of E onto itself, and its inverse w = vTM1 is such that (lE—u)w = = lg, in other words w = 1E + uw has the same form as v, since uw 1s compact,
% These results enable F. Riesz to treat completely the question of eigenvalues of a compact operator. There are at most denumerably many eigenvalues )\ # 0 in ¢, and each of them is isolated in (€-{0}; their set is bounded and O belongs to its closure if it is infinite. For each Xn £ 0, E splits into a topological direct sum of two closed subspa=- ces F(xn) and N(An), which are stable by wuj N(Xn) has
% finite dimension, and there is a sEallest integer kn such that the restriction of (u-xnolE) " to N(A,) dis 0O; the restriction of u-)\nolE to F(kn) is a linear homeomorphism
% of that subspace onto itself, If E is complete, the func tion gha-(u«g°lE)-l is meromorphic in €-{0} (with values in the space &£ (E) of continuous endomorphisms of E); at the points other than the xn’ that function is holomorphic, and at each xn it has a pole of order kn. Finally, if m £ n, the subspace N(xn) is contained in F(xm). However,
% 148 CHAPTER VIT
% there is in general no global decomposition of FE into a sum of subspaces N(xn), similar to what happens for compact self-adjoint operators in Hilbert space [99]. As a matter of fact, there may be no eigenvalues at all, as for instance for
% Volterra operators,
% In the study of u—g-lE, the value ( = O is completely exceptional; u(E) is not closed in general and may have in finite codimension, and u_l(O) may have infinite dimension,
% This explains the intractability of integral eguations "of the first kind", special cases of equations u(x) =y for u compact, which had baffled the early mathematicians working on integral equations,
% Although there has been much work done on compact operators of special types, the general theory of compact operators has remained pretty much what it was after the publication of F. Riesz's 1918 paper. Among more recent results, one can mention the fact that when u is a compact operator in a com plete normed space, its transposed operator tu in the dual E’ is also a compact operator [188]. It has also been proved that, even when a compact operator u 1in FE has no eigen value, there are always closed vector subspaces V of E, differeut from E and {0}, such that u(v) < v [7].

\section{The spectral theory of Hilbert}
\label{sec:7.2}

% We now return to the most original part of Hilbert's 1906 paper (chap.V, 2), in which he discovered the entirely new phenomenon of the "continuous spectrum", In his "Theory of heat", Fourier had considered trigonometric series represent-
% SPECTRAL THEORY AFTER 1900 149
% ing functions of period 2a (chap.I, 2, formula (13)) when a tends to +». The eigenvalues ) _ = (n+%) 'rrz/a2 of the
% 2
% corresponding Sturm--Liouville problem for the equation vy’ +\Ay = = 0 with boundary conditions vy(-a) = y(a) = O divide the interval [O0,+o[ in intervals of length tending to O with 1/a, and this had led Fourier to consider that the Mimiting case" of the trigonometric expansion of a function of period 2a would be, for any function f defined on R, the repre (1) f(x) =‘% dt ( f(t)cos u(x-t)du
% sentation by an integral
% —co 0
% where the "eigenvalues" would now fill the interval [0, +o[ ([67], vol.I, p.392).
% In 1897, Wirtinger [230] developed similar ideas for Hill's (2) y' + ra(x)y = O
% equation
% 4
% where q 1is a continuous periodic function of pexriod 1. The general theory of these equations was well-known at the time: starting with a fundamental system of solutions u,, U, such

% that
% ul(o,x) = 1, ui(o,x) =0 u,(0,\) = O, us (0,1) = -1

% so that uzui - ulu% was the constant function 1, one con siders complex solutions f such that f(x+1) = pf(x) for all x € R; the constants p(A) having that property are so
% (3) 0 + 3p + 1 = O
% lutions of the equation

% 150 CHAPTER VIT
% with 3()\) = ué(l,k) - ul(l,x). The solutions of period 1 correspond to "eigenvalues" )\ such that @(k) = 2, the so lutions f{ such that |[f(x+1)]| = |£f(x)]| to values of A such that =2 < @(X) < 2, which in general constitute dis j oint intexrvals Ik of R, Wirtinger looks for solutions of period n (an arbitrary integer), which correspond to values of % such that (p(3))" = 1, and he shows that when n tends to +», these values of )} tend to "fill up" the in tervals I k. The similarity with the optical spectra of mo
% lecules leads him to speak of the "Bandesspectrum" of equa tion (2) formed by the union of the intervals T k? and he thinks there should be an integral formula similar to (1), without being able to guess what that formula could be,
% Although Hilbert does not meantion Wirtinger's paper, it is probable that he had read it (it is quoted by several of his pupils), and it may be that the name "Spectrum" which he used came from it; but it is a far cry from the vague ideas of Wirtinger to the extremely general and precise results of Hilbert, On the othexr hand, the influence of Stieltjes's big paper of 1894 on continued fractions is explicitly acknowled ged by Hilbert: Stieltjes had had to take the limit of a se quence of rational functions of a complex variable

% ) n y = i2
% M.i
% Z+X. 1 i

% where the Mi and the x, are real numbers, and he had shown that the 1limit could be writteun as a "Stieltjes transform" (o0)

% SPECTRAL THEORY AFTER 1900 151
% for a function & of bounded variation (creating for that purpose the concept of "Stielt jes integral") [205]. It is a similar problem which confronts Hilbert when he wants to pass from the classical "reduction" of the n-th "Abschnitt" (4) K (X,X) = X >y k x x
% n n
% n b1 qo1 P PG
% of his "bounded" quadratic form X (x,x), to a "reduction" of K(x,x) iditself: the classical theory shows that one has

% (5) Kn(X,X) =
% @ () (x))? (L) (x))? Ay A

% where the x(n) are real numbers such that xn) < Xgn) < See o5 xén), and the Lgn)(x) are linear forms in Xyj9Xogeeey X, such that
% (6) (Ln)(x))2 +o oot (Lr(ln)(x))2 = xi tooot xi . To each x = (xp) of LZ such that xp = 0 for p > n,
% Hilbert associates the piecewise linear convex function of (7) u(n)(x;x) = pil (Lén)(x))Z(X-Kén))+ ;
% the real wvariable
% His idea is to take (if possible) the limit of each of these functions, for the points of LZ having only a finite number of coordinates #0., Using his "principle of choice" (chap.V,
% 2) and Cantor's diagonal process, he shows that these limits exist at least for a suitable subsequence of the u(n). In ffact, he only uses these limits for the points x(pp) having
% the coordinate of index p egual to 1, all others to O, and the points x(pq) for p # g, having the coordinates of in-

% 152 CHAPTER VII
% dices p and q equal to and all others to O3 he writes the corresponding limits upq(X) for all pairs of in tegers (p,q). They are convex functions of ), hence the right derivative V;q(X) and the left derivative V;q(X) exist for all ), and are equal except for an at most denu meirable set of values of )A. Hilbert writes xl,kz, \ldots,lr, \ldots the sequence of these exceptional values of A for all pairs (p,q), and defines quadratic forms
% + + - -
% v (x;x) = Y v (X)x X v (x;x) = T v (x)x X D, q Pg P Qg D,q Pqg P g
% and for each index r,
% E_(x) = vi(x50.) = v (x350,).
% All these are bounded quadratic forms, and more precisely, their wvalues in LZ are =20 and s(x|x). For the values of A distinct from the xr’ he writes
% e(x;)) = ¥ E_(x)
% A <A
% and
% o(x;3)) = v(x3\) = e(x;))
% where Vv 1is the common value of v+ and v . With these
% notations, his final result is the "reduction" ocf the qua dratic form K(x,x): for each x € LZ, the function A—0(x3)) dis a continuous function of bounded variation, and (5) (x|x) = z B_(x) + f,dc(x;x)
% one may write
% kK(x,x) = Z-%— Er(x) + (-%—dc(x;k).
% r
% SPECTRAL THEORY AFTER 1900 153
% He says that the set of the xr is the point spectrum of K and the complement of the set where all the G(x;x) are comns tant the continuous spectrum of K .
% Of course, when K 1is a "completely continuous" form, the
% continuous spectrum is absent, but Hilbert gives examples for which there is no point spectrum. For instance (see [183], vol.II, p.986-989), if (@p) is a complete orthonormal system in a compact interval [a,b] of R and f a bounded measu rable function in [a,b], one defines a bounded quadratic (9) 2 g = ( f (We, (W)e, d(u)du
% form A(x,x) = ¥ a__x x_ by the formulas
% Pad P g
% a
% and one has the "reduction" formulas

% (10) b® b
% (x]x) = (z x 0 ()
% 2
% b
% A(x,x) = £(u)(2 x 9, ()) du
% 2
% ab
% from which it is easy to see that there is no point spectrum (unless f is constant in an interval), and if f is conti
% nuous, the continuous spectrum consists of the whole interval [m,M], where m and M are the minimum and maximum value taken by f in [a,b]. If one takes a = 0O, b =1, @p(u) =/ — sin pu, f(u) = cos u, one obtains the first
% 2
% example given by Hilbert
% If one takes a = =TT, b =1, mp(u) = (Sin pd4 + cos PM)
% 1
% Jom
% for -0 < p< 4w, and f(u) = =-f -y if W < 0, f(u) =m-u

% 154 CHAPTER VII
% ifif > 0 , one g gets a bq ==
%  Bra if p+q+ # O 0, aa . = 0 if
% 1
% . P g . -
% q = =p. Making xp = 0 for p< O, one gets the second _1 Ptd p
% example of Hilbert A(x,x) ) , and making x_ = O
% X X
% pP,q=1
% for p < O, yp = 0 for p = O, one gets still another Xy example of Hilbert A(x,y) = ¥’ p?qq (where the summation P,g
% extends to the pairs (p,q) of integers >0 and distinct); both have for continuous spectrum the interval f-fl,+fl]( ).
% *
% In his 1913 book ([183],v0ol.II,p.956-989 and [184]), F.Riesz gave an exposition of Hilbert's results based on an entirely
% *
% ( )One also should mention how Stieltjes' results on continued fractions and on the moment problem were soon recognized as
% belonging to the Hilbert-von Neumann spectral theory. Jacobi, n n-1
% in 1848, had cunsidered the special quadratic forms J(x)

% = Z,'ax —ZZb k=1 k k k=1
% k+lxkxk+l’ and he had shown that the eigen-

% (F) aoiz la' cee -
% values of that form are the roots of the denominator of the limited continued fraption ([120],vol.v1,p.318-321)
% Already in 1878, Heine [104,vol.1. p.421] had hinted at the possibility that an unlimited continued fraction of type (F) 2 . . . .
% (0] <
% would be similarly related to a Jacobi quadratic form no a x. - 2 ngo bn+lxnxn+l in an infinite system of varia
% bles. What Stieltjes had done was to study directly unlimited continued fractions of type (F), representing them as "Stieltjes transforms" and being led to his "problem of moments" by the problem of determining the Stielt jes measure
% corresponding to a given Stielt jes transform; but he had not considered the relation between the continued fraction and quadratic forms. This was done by Toeplitz in 1910 [ 214] for the case of Jacobi bounded quadratic forms, and later extended to the general case; it turned out that these forms exactly corresponded to spectra of multiplicity 1 [207].

% SPECTRAL THEORY AFTER 1900 155
% different method, and which was to remain standard until around 1950, As we have already pointed out, he replaces the
% bilinear forms of Hilbert by the much more natural continuous 2
% endomorphisms of E = LR s to such an endomorphism A is as sociated the "bounded" bilinear form (x,y)r (A<x|y) and
% convevrsely each such form can be uniquely written in that way.
% F. Riesz's central idea is to define, for such endomorphisms 4 , "functions" f(4A) which would again be continuous endo morphisms of E, for suitable functions f of a real varia
% ble, and to use such "functions" to write for a symmetric en domorphism 4 (i.e. such that (4:x|y) = (x|A.y) for all x,y in E) a canonical "spectral decomposition" correspond ing to formulas (8) of Hilbert.
% To develop these ideas, F, Riesv begins by some general re sults on the algebra £(E) of all continuous eundomorphisms of E. For the norm HA” = sup HA°xH, it is a Banach space, | x]|=1
% has a unit (the identity mapping lE) and is such that lasll < [l . However, F., Riesz, for his purpose, is led to use, not the notion of ("uniform") convergence derived from the norm of S(E), but the notion of strong convergence: he says a sequence (An) converges strongly to A if, for every x € E, the sequence (HAn°x-A°xH) tends to O,
% F. Riesz then considers the subspace fi(E) of all symmetric operators; there is in fi(E) an order relation, 4 <pB mean ing that (A-x|x) < (B-x|x) for all x € E, Suppose a-lE < <A< bl E? then, for every polynomial P(E) with real coef ficients such that P() = O in the interval [a,b], one has P(A) = 0. 1Indeed, one may assume a = O, b = 1, and then P
% 156 CHAPTER VII
% is a sum of polynomials of one of the types Q(g)z, g(Q(g))Z, (1-2)(Q(€))° or £(1-€)(a(t))?, and it is enough to prove that A(lE-A) is 203 but from the Cauchy-Schwarz inequality for the positive quadratic form (A-x|x) it follows that ||A-x||4 < (A-x|x)(A2°x|on) < HXHZHA-XHZ; this first implies that ||A]l < 1, and then HA°xH4 < (A-x|x)”A°x”2 and finally (42-x|x) < (A+x|x). From this it follows at once that if a polynomial P() with real coefficients is such that m< P(E) <M in [a,b], then one has mel_ < P(A) < M-1 E E’? and [[P()] < sup(|m],|3|).*)
% These results first imply that if a sequence (Pn) of poly=- nomials converges uniformly to a continuous function f in [a,b], then the sequence (Pn(A)) is a Cauchy sequence in the Banach space (E), and its 1limit only depends on [, hence can be written f(4); furthermore the mapping f— f(4) is a homomorphism of the algebra ([a,b]) into £(E), with values in H(E), which justifies the notation; in addition, if m< f(E) < M in [a,b], one has again m+l_ < f(A4) < E
% < M-lg.
% But F. Riesz goes furthev. If (fn) is an increasing se quence of continuous functions in Ea,b], uniformly bounded, then for any x € E, the sequence of the (fn(A)-x|x) is
% increasing and bounded, hence has a limit, from which it fol lows by linearity that the sequence of the (fn(A)°x|y) con=- verges for any pair of elements x, y din Ej; the Hellinger-
% (*) This argument is not the one used by Riesz, who deduces the result by a passage to the limit from the known result for the "Abschnitte" 'An of A.
% SPECTRAL THEORY AFTER 1900 157
% Toeplitz theorem (chap.V, 4) shows that the limit can be written (Bex]|y) where B € H(E); if g is the (simple) 1imit of the sequence (f_), one writes again B = g(4), and this enables one to define g(A) for any bounded (upper or lower) semi=-continuous function in [a,b] or any linear com
% bination of such functions, which again form an algebra and for which g~» g(A) is a homomorphism.
% F. Riesz then uses these results to obtain the spectral de composition in the following way; if eg is the function de fined in R and such that eg(u) =1 for y < , eg(u) = 0 for u = g, eg(A) = Ag is defined since eg is bounded and lower semi-continuous. For any pair of vectors x, y the
% function *e»@4g°x|y) is then a function of bounded varia tion, and for any continuous function f, one has (12) (c)xly) = | £E)alg x|v)
% 400
% -C0
% (13) £(4) = £(g)aa,
% a formula which one also writes
% +00
% and which is justified by the fact that for any ¢ > O, it is possible to divide the interval [a,b] by points gk in
% such a way that
% - 1 SfA)—Zf )(A -4 < e°1 = ( k (uk gk+l gk) B
% (with B S My S k+1)
% The spectrum( ) of A, contained in the interval [a,b], is *
% ( )To pass from the Hilbert notion of "spectrum" to the one
% *
% used by F, Riesz, one must replace the parameter )\ of Hilbert by 1/.
% 158 CHAPTER VII
% the complemeut of the set of points having a neighborhood where Eg+» A is constant,
% The operator Ag is the orthogonal projector of E onto a closed subspace Eg, which is stable under A and in which (Aex|x) < &€(x|x) for all x #£ O; 1, - A is the orthogonal projector of FE onto the subspace Eg orthogonal to Eg, and in which (Ag°x|x) > E(x|x). The point spectrum of 4 is the denumerable set of values of  where at least one of the functions gb’-(Ag-x|y) is discontinuous; it consists of all the eigenvalues of 4, but the subspace N formed by
% the corresponding eigenvectors may have infinite dimension, The subspaces E are such that E c En for  < n, Eg = {0} for & < a, Eg = E for £ > b; the intersection of the subspaces En for  < nn is reduced to E if & is not in the point spectrum and is the direct sum of the (ortho gonal) subspaces Eg and N if E dis in the point spectrum,
% If (Aex|y) = =% A XYy is the "bounded" bilinear form i,
% corresponding to the operator A4, the eigenvectors x = (xk) corresponding to an eigenvalue ) have coordinates which are
% solutions of the system of linear equations
% (14) xxi - E aikxk = 0 (i:l,Z, \ldots) .
% However, Hilbert's theory left unanswered the question of de termining "objects" which would replace the eigenvectors when a number ) in the spectrum was not in the point spectrum, In some cases this question had a curious answer; for instan ce, for the form (11), where there is no point spectrum and the spectrum is the interval [-1,1], for each value J)=cos t
% SPECTRAL THEORY AFTER 1900 159
% in that interval, the system (14) does have a solution, na mely xk(t) = sin kt for k = 1,2, \ldots, as it is easily ve rified; but for such a sequence, the scries X xk(t)2 is not k
% convergent., The existence of such "generalized eigenvectors" was only incorporated in a general theorem much later (see
% chap., IX, 2); but in the case of the form (11), one could €
% observe that if one wrote pk(g) = xk(t)dt, then the vector (pk(g)) this time belonged to E for all
% )
% k=l,2,ooo
% £ € R, and it followed from the relations (14) satisfied by the xk(t) that one could write, for any interval [xl,xzj
% of R
% x2

% (15) Edp, (8) - A
% i aik(pk(KZ) - pk(xl)) =0 (i=1,2, \ldots),

% This led Hellinger [105] to study systematically the sequen ces of functions (pk(g))k_ of bounded variation which satisfied (15) for all inter;ais and for which E pk()2 is finite for all & ; he called the dpk "eigendifferentialsTM" of the operator A. This study allowed him to attack a prob lem which naturally arose from Hilbert's spectral theory, by analogy with the finite dimensional case: can one give ne cessary and sufficient conditions for two operators 4,B with symmetric matrices to be "similar", i.e. such that there is
% an orthogonal transformation U of E onto itself such that B = UAUTM'? In the finite dimensional case, the condition is that the eigenvalues of 4 and B be the same, with the same
% multiplicity for each; the combined efforts of Hellinger and . g . 2 .
% H. Hahn [ 95] succeeded in obtaining necessary and sufficient conditions for operators in LR sy expressed in terms of

% 160 CHAPTER VII
% special systems of "eigendifferentials". We shall not give here the detail of these complicated conditions, which we shall formulate in a much simpler way using the Gelfand theory of commutative Banach algebras (5).
% One of the byproducts of F, Riesz's method is that it enabled him to give a direct definition of the whole spectrum of 4, without any reference to the decomposition (12): a point A € R is in the spectrum if and only if the operator A+1l_- A4 E does not have a continuous inverse. Finally, he remarked that
% his method could (just as Hilbert's method) be extended to self-ad joint bounded operators A4 in complex Hilbert space Li, i.e. those which satisfy the same condition (4-+x|y) = (x|4°y)
% for the hermitian scalar product in that spacej; the mapping gka-(g'lE—A)-l is then holomorphic outside of the spectrum of A. After 1913, almost all papers on spectral theory in Hilbert space dealt exclusively with complex Hilbert space.

\section{The work of Weyl and Carleman}
\label{sec:7.3}

% Hilbert's method associating to an integral equation with symmetric kernel KX(s,t) a "bounded" bilinear form K(x,vy) (chap.V, 2) worked even if K(s,t)2 was not integrable in [a,b]x[a,b], but the corresponding bilinear form might not
% be "completely continuous"; already in his lectures in 1906 Hilbert had given the example K(s,t) = (s+t)'—l for the in terval [0,1] ([227], vol.I, p.83). He also had observed in
% his Seminar that the Fredholm theorems might fail when the interval [a,b] was unbounded, and had given as example an
% SPECTRAL THEORY AFTER 1900 161
% interpretation of the Fourier inversion formula (1) (ibid., p.2): for K(s,t) = cos st in the interval [O0,+=[, 4/—E— and - J-é— are the only eigenvalues, but each of them has infinite multiplicity( ). He therefore encouraged his most
% *
% gifted student, Hermann Weyl, to elucidate such "singular"
% integral equations, and in particular to determine conditions on the kernel K(s,t) implying that the bilinear form X(x,y) would be "bounded" and therefore amenable to his spectral theory. This was the theme of Weyl's dissertation; in it and in a subsequent paper (iglg., p.2-86 and 102-153) he gave ! o
% the following sufficient conditions for [a,bl = [0,+[: 12 for each s = O, the integral f/ (K(s,t))zdt is finite; 0
% 29 there is a constant M > O such that the inequality (0 0] [e0) | ( ( K(s,t)u(s)v(t)dsdt| < M Ffor all pairs of continuous

% O 0 (00)
% O 0(o0]2 functions u, v such that y/ (u(s))zds < 1 and ( (v(s)) " ds< < 1.
% Another direction of research derived from an interpreta tion of the Sturm--Liouville theory (chap.I, 3) in terms of integral equations, Consider a second order differential (16) y' = a(x)y + iy = £(x)
% equation
% where q and f are continuous functions in a compact in terval [a,b], q having real values, f real or complex values and ) 1is a complex parameter; in addition we have
% ( )This is apparently the first appearance of a relation
% *
% between Fourier transforms and Functional Analysis (see 6).

% 162 CHAPTER VII
% two boundary conditions
% (17) v(a)cos a-y’ (a)sin @ = O, y(b)cos B-v/' (b)sin B = 0
% where @ and 8 are two positive conslants. An elementary argument shows that for f = 0, +the homogeneous eqgquation Cfifl with boundary conditions (17) has no solution if )\ < -r, where 1r 1is a number >0 depending only on g. Replacing q(x) by q(x) + r and A by A+r, one may assume that for A < O, the homogeneous equation y”’ - gq(x)y + Ay = O has no solution satisfying the conditions (17).
% Now consider first equation (16) for )\ = 0; there are two solutions wu,, u, of the equation y"- q(x)y = O such that
% y
% ul(a)cos a - ui(a)sin e = O, uz(b)cos g - ué(b)sin B = O, u; and u, being linearly independent. For each ¢t € [a,b], (18)
% define the function
% K(t,x) -uz(t)ul(x)/d for a < x < t
% K(t,x) = -u,(t)u,(x)/d for t < x< Db
% where d is the constant ul(x)ué(x) - uz(x)ui(x). The function x#+> K(t,x) is then continuous in [a,b]; in each of the semi-open intervals a < x< t, t< x< b (for V4 a< t< b) it satisfies the equation ¥y’ - gq(x)y = 0, and in addition it satisfies conditions (17); finally, the function XF*-g; K(t,x) has at the point x =t a discontinuity such
% 3
% that %; K(t,t+) - %; K(t,t-) = =1. A routine calculation then shows that in order that a function y be a solution of v'- a(x)y = f(x) satisfying conditions (17), it is necessary and sufficient that
% SPECTRAL THEORY AFTER 1900 163
% (19) y(x) = - K(t,x)f(t)dt
% and therefore the solutions of (16), satisfying (17), are exactly the solutions of the integral equation (20) y(x) - A K(t,x)y(t)dt = g(x)
% b
% (21) g(x) = - K(t,x)f(t)dt.

% where
% b

% Clearly this method was patterned after Schwarz's method for solving the equation of vibrating membranes with the help of the Green function for the Laplacian (chap.III, 1), and the function K was therefore called the Green function for the /7 operator L(y) = y" - q(x)y and the boundary conditions (17) [ 35]« As obviously K(x,t) = K(t,x), the Sturm--Liouville
% problem was thus reduced to a special case of the Fredholm=- Hilbert theory of integral equations with symmetric kernels. In his second paper (1904) on integral equations, Hilbert
% developed that method and expanded it to other boundary con ditions than (17). He also was aware that many of the
% "special functions" which had been introduced in Analysis since the XVIIIth century (hypergeometric functions, Bessel functions, Legendre polynomials, Hermite functions, etc.) sa tisfied equations of type (16) but with less restrictive con ditions: the interval [a,b] would be replaced by an unbounded interval and the function gq might have singular points at the extremities of the interval; in such a case, Hilbert proposed that the corresponding boundary condition

% 164 CHAPTER VII
% should be replaced by the condition that <y remain bounded at such an extremity, or tends to infinity not faster than some given singular function. He showed then, on various examples, that one could even in such "singular" cases, de fine a "Green function" K, with the symmetric property K(t,x) = K(x,t), and the same discontinuity for the partial derivatives for x = t; for instance, if L(y) = v’"+y, one has, for the interval J-ow,+o[, K(t,x) = = % sin|x-t]|. At
% that time, he had not yet developed his theory of "bounded" bilinear forms, so he limited himself to cases in which the Green function K was a kernel to which the Fredholm theory was applicable ([112], p.39-58). But of course he was aware that in examples such as the one above, one would fall on "singular" integral equatious, and one of his students, E., Hilb, wrote his "Habilitationschrift" in 1908 on the ap plication of Hilbert's theory of "bounded" forms to two spe cial cases of "singular" Sturm--Liouville problems [110].
% Then, in 1909-1910, H. Weyl discovered that he could apply
% the results of his dissertation to handle the most general (22) I(y) = 5 (p(x) ) - alx)y
% such problems for second order operators of the type o
% where p and q are real continuous functions in an interval I R (bounded or not) subject to the only restriction that p(x) >0 in I. In his "Habilitationschrift" ([227], vol.I,
% P.248-297) he considers the case I = [0,+x[3 his very ori (23) L(y) + Ay = O
% ginal method consists in studying the equation
% SPECTRAL THEORY AFTER 1900 165
% for non real values of ), and in fact, he sees that it is enough to consider the case )\ = 1. Let u;, U, be the two solutions of L(y) + iy = 0 in [O0,+o[ satisfying the ini
% tial conditions
% ul(O) = 1, p(O)ui(O) = 0, u,(0) = 0, p(O)ué(O) =1
% Let ¢ be auny number =20, and consider the two solutions of L(y) + iy = 0
% V. = «sin @eu, + coOS a°u2, W = COS Qeu, + sinQ °u o 1 o 1 2
% so that Vg is a solution satisfying the boundary condition (24) cos a*y(0) + sin a+p(0)y’(0) = O.
% Now take any number a > O, a number g = 0, and determine the complex number ¢y by the condition that the solution u = Vv, 4+ yw satisfies the boundary condition ¥ o & (25) cos Bey(a) + sin g-p(a)y’(a) = 0.
% Weyl shows that the uniquely determined number |4 describes a circle Ta in the upper half plane Jdz > O when B8 varies from O to 2,
% From the two solutions Vg and uu one can form a Green function Ki(t,x) in the interval [0,a] by the same formu las as (18), except that now has complex values. Now
% KTM
% a
% let a tend to +o3; Weyl shows that the circles Ta form a nested family of decreasing radius, hence have a limit Pm which may be either a circle of radius >0, or a single point, o) TM AL a
% In any case, if one lets the points U € ' tend to a limit M €T 5, u tends to a solution u and K® tends to a
% 166 CHAPTER VII
% kernel KBO(t,x) which always satisfies the two conditions of Weyl's dissertation; this enables one to apply to the cor
% responding singular integral equation Hilbert's theory of "bounded" bilinear forms., Just as for the usual Sturm-Liou L‘Lo (26) y(x) = - K (t,x)f(t)dt
% ville problem the function
% (27) L(y) + iy = f(x)
% is then a solution of
% satisfying the boundary condition (24) at the extremity O,
% and the condition "at infinity"
% (28) 1im p(t)(y(t)u’ (t) - u (t)y" (t)) = o.
% t4+o Mo Mo
% Furthermore, the solution u always belongs to LZ(O,m). o
% Weyl then shows that the problem splits in two cases:
% I) The "limit circle" case, T = being a circle of radius >0, Then all solutions of L(y) + iy = O ©belong to LZ(O,m), and for all Mo € Tw, K © is a Hilbert-Schmidt kernel; the
% conclusions of the Sturm=-Liouville theory are then again wvalid for equation (23) with the boundary condition (24) at extre mity O, and the condtion that the real part of (28) vanishes at extremity +o,
% . 2
% IT) The "limit point" case, when T is reduced to a single pointg S Then %40 is (up to a constant factor) the only solution of [L(y) + iy = O belonging to L7 (0,»), and Weyl's
% main ob jetive is to set up integral formulas which should be subgituted to the "Fourier expansions" of the classical Sturm-
% SPECTRAL THEORY AFTER 1900 167
% Liouville theory, as had been expected by Wirtinger, and obtained by Hilb in special casesj; this he is able to do by applying the results of his dissertation to the "singular" integral operator having as kernel the imaginary part of K'O, and extensively using Hellinger's "eigendifferentials", He also defines the spectrum of the differential operator L as the complement of the set of parameters ) € R for which the differential equation L(y) + Ay = g(x) has a solution be longing to LZ(O,w) and satisfying (24), for every continuous function g belonging to LZ(O,w); he studies the structure
% of that spectrum under various assumptions on the functions p and g, and in particular he gives an example where that spectrum is the whole real line., Finally, in a subsequeat paper ([227], vol.I, p.222-247), he shows how his theory may be extended when the interval I is the whole line R, and the equation belongs to the "limit point" type at both extre mities,
% Viewed from the vantage point of the later von Neumann theory (see 4) these remarkable results of Weyl constitute the first study of an unbounded hermitian operator in Hilbert space, with non zero "defects", and of its self-adjoint ex
% tensions; the singular integral operator defined by the kernel K with complex values is probably the first example of a
% o
% normal operator in Hilbert space which is not self-adjoint( ). *
% (*)The councept of a normal square matrix A with complex ele=- ments had been defined in 1877 by Frobenius by the condition that AA* = A*A s he had proved that this condition was equi valent to the existence of a unitary matrix U such that UAU-l
% is a diagonal matrix [78, vol.I, p. 391].
% 168 CHAPTER VII
% Such phenomena became even more apparent in the work of T, Carleman on singular integral equations ([37] and [ 38, p.313- 342]) beginning in 1920, He starts from a kernel wh%ch only satisfies the first of Weyl's assumptions, namely ( |K0%tfl%fl; a
% is almost everywhere finite (they are now called Carleman
% kernels, and the corresponding operators, which to f asso=- b
% ciate s i— ( K(s,t)f(t)dt, Carleman operators). He treats ! a
% the theory of these operators (for hermitian kernels) by a method similar to H. Weyl's: mnamely, he considers an increas ing sequence (An) of measurable subsets of [a,b], the union of which is equal to [a,b] up to a null set, and which are such that he kernel Kn(s,t), equal to K(s,t) for (s,t) € AXA and to O outside, has a finite integral
% b b 5
% ( (’ |Kn(s,t)| dsdt. If one now cousiders the integral a ’‘a
% equation
% (29) ®(s) - A K(s,t)p(t)at = £(s)
% b
% for non real )\, f being in L ([a,b]), one approximates 2
% it by the sequence of ordinary Hilbert-Schmidt integral equa /b |
% (30) ® (s) - A J K_(s,t)p_(t)dt = £(s) tions
% which (due to the choice of A) has a unique solution », € L ({a,b]). Carleman's original procedure is to integrate
% 2
% (30) after multiplication by @n(s), which gives him the b b b
% identity between the imaginary parts
% 11 2 1 7 1 P () G | len()%as = 3] an@r()de -5 | oy () T()as
% SPECTRAL THEORY AFTER 1900 169
% independent of the kernel, and from which he deduces, by the (32) o (s)]2ds < “'kllg |f(s)|2ds.
% Cauchy-Schwarz inequality,
% bb
% 2
% a
% Applying the usual "principle of choice" and density arguments as Hilbert and F, Riesz had done, Carleman is able (by letting n tend to +») to define a linear mapping f+~-= T, +f in A L such that ¢ = Tl.f is a solution of (29) for each f GIfi
% 2
% and a passage to the limit in (32) shows that T, is contin- A uous., He even goes as far as writting an equation equivalent (33) T°f.—_-___>i__(f+[j o f

% to
% A A=A A )

% and showing that H U)\ofH < ”f
% He then realized that the solution of (29) for non real )\ might be non unique, and he gave examples of kernels where this phenomenon happens, which he called kerneis of class 1I1; the other ones he called kernels of class I, and he showed that they may be more general thaan the continuous operators b b
% of F, Riesz or those considered by H., Weyl.
% 2
% For any functions ¢,  in L , the functions a a
% (34) S.@: S +— K(S9t)@(t)dt9 S"¢= S = K(t,S)W(t)dt
% are always defined for a hermitian Carleman kernel K, but 2 2
% the set D (resp. D’) of functions of L2 such that S*y € L (resp. '« € L7) is in general a proper subspace 2
% of L7 3 D’ consists of the complex conjugates of the func tions of D. Carleman showed that a necessary and sufficient

% 170 CHAPTER VII
% b b
% condition for K +to be "of class I" was that the relation (35) f(s) (s’ -g)(s)ds = g(s)(s+f)(s)ds
% a a
% should hold for all f,g in D.
% He next proceeded to let also n tend to +o in the Hilbert formula for Hilbert-Schmidt kernels (chap.V, 2, for mula (24)) for the kernels K, and their conjugates Kn’ and obtained for the operators Tk and Ti corresponding to K and K formulas similar to those obtained by Hilbert in his theory of "bounded" quadratic forms. For kernels "of class
% II", the study of these formulas led Carleman to single out the case in which the operator U, in (33) and the corres ponding operator Ui for T; are both unitary; he shows
% that this property is independent of the choice of the non real number )\ in one of the half planes Jd)\ > 0, JA < O,
% and that it implies that the dimensions of the spaces of so=- lutioris of @-AS-@ = O and w-lS"w = O in L2 are the
% same; finally he proved that in this case there are infinite ly many operators TK and Ti having the above property (what he calls "maximal solutions") for the same kernel K.
% All these results were quite surprising, in particular the
% existence of solutions ¢ # O for the equation @-=AS-p = O in L2 for non real ), which seemed to contradict the
% classical argument (going back to Poisson (chap.I, 3), and even to Lagrange ([135], vol.XII, p.239) in the finite dimen sional case) which, from the reality of the number (S<®]|op) for all ¢, concluded to the impossibility of a non real number )\ satisfying ¢ = AS.p for o # O, since it implied
% SPECTRAL THEORY AFTER 1900 171
% (p|p) = A(S*p|®). We shall see in the next section how this apparent contradiction was resolved in the von Neumann theory, which put the pioneering results of Carleman in their proper context.

\section{The spectral theory of von Neumann}
\label{sec:7.4}

% In the fall of 1926, the young J. von Neumann (1903-1957) arrived at G8ttingen, to take up his duties as Hilbert's assistaant. Taese were the hectic years during which quantum mechanics was developing at breakneck speed, with a new idea popping up every few weeks from all over the horizon [121]. The theoretical physicists who were developing the new theory were groping for adequate mathematical tools, trying in suc
% cession infinite matrices without any consideration of con vergence(*), differential operators, "continuous" matrices (whatever that might mean) etc. It finally dawned upon them
% that their "observables" had properties which made them look like hermitian operators in Hilbert space, and that, by an extraordinary coincidence, the "spectrum" of Hilbert (a name which he had apparently chosen from a superficial analogy) was
% to be the central conception in the explanation of the "spectra" of atoms, It was therefore natural that they should enlist Hilbert's help in trying to put some mathematical sense in
% their computations., With the assistance of Nordheim and wvon Neumann, he first tried integral operators in L2, but that
% needed the use of the Dirac "-function", a concept which for
% *
% ( )As late as 1924, most physicists did not even know that a ffinite matrix was!
% 172 CHAPTER VII
% the mathematics of that time was self-contradictory (Cf. chap. VIII, 3); von Neumann therefore resolved to try another approach,
% Ever since the discovery of the Fischer-Riesz theorem (chap. v, 3) the isomorphism of the space of sequences Lé and of the LE(Q) spaces of classes of quadratically integrable functions in some subset () of an Rn had been familiar to
% analysts, but by "Hilbert space" one always understood one of these "concrete" spaces, on which the "operators" would there fore be, either "matrices" or "integral operators" of some kind., Von Neumann was the first to conceive of an "abstract" Hilbert space, defined axiomatically as a complex vector space with a hermitian scalar product, separable and complete for the corresponding norm, so that the usual "concrete" Hilbert spaces would only be "incarnations" so to speak of that "abstract" space. Obvious as it now seems to us, this was a momeritous step and opened the way to a complete unders tanding of spectral theory of normal and hermitian operators in Hilbert space, which von Neumann proceeded to develop in 3 fundamental papers published between 1929 and 1932, and which (with the exception of the description of the spetrum, see 5) are still today, in substance, the definitive account of the subject ([221], vol.II, p.1-85, 86-143 and 242-258)(*)-

% (*)
% During the same period, M.H. Stone, independently of von
% Neumann, obtained the same results concerning self-adjoint (unbounded) operators [ 206], and later gave a didactic exposi tion of the whole theory and of its applications at that time, much cleavrer than von Neumann's papers, and which remained the reference book on the subject for many years [ 207].

% SPECTRAL THEORY AFTER 1900 173
% Abandoning any "concrete" presentation of Hilbert space, von Neumann was compelled to work intrinsically, using only notions which could directly be defined from the concepts enu merated in the axioms, to the exclusion of anything else, This led him to discover a remarkable series of entirely new ideas and methods,
% 1) Most operators used in quantum mechanics could not be defined in the whole Hilbert space, as for instance in LZ(R) multiplication of functions by a fixed function such as x~—x, or derivation of functions. One therefore had to consider, in general, linear mappings T taking their values in a Hilbert space E, Dbut only defined in a proper vector subspace dom(T) (the "domain" of 7T); the most interesting case concerned the operators T densely defined, i.e. those for which dom( T) is dense in E (as in the two examples above).
% 2) If dom (T) 4is deunse in E and T is continuous, it can immediately be extended to the whole space T, and one is brought back to the Hilbert theory. But von Neumann had the
% idea to introduce a weaker substitute for continuity, namely the fact that the graph T (T) of T be closed in EXE; one then says that T is a closed opecrator. It is obvious that for dom(T) = E, if T is continuous, then T is closed, and the converse follows from the closed graph theorem (chap.VI, 5). Of the two examples given in 1), the first is closed but
% the second is not.
% 3) This last example raises the problem of extending (if possible) a densely defined operator 7T which is not closed to a closed operator; von Neumann was able to give a beautiful
% 174 CHAPTER VII
% anwer to that problem by linking it to a generalization of the notion of adjoint operator, well kunown for bounded operators. In general, for a densely defined opcrator T and a vector y € E, the linear f{orm Xt—b(T'le), defined in dom (T ),
% is not necessarily continuousj; if it is, it can be extended uniquely by continuity to E, and then can be written X > (x|y*) for a unique vector y* € E; the set of wvectors v € E having that property is a vector subspace, and if one writes v¥ = 7.y for those vectors, T* is a linear opera tor defined in that subspace (which is therefore dom( T*)). Now it 1is easy to show that T* is always closed (even if T is not), and its graph is the subspace of E which is the orthogonal supplement to the closure of J(I'( 7)), where J is the linear automorphism (x,y)+ (y,~-x) of ExE ("rotation of a right angle"!)., This interpretation of dom( T*) gave
% to von Neumann the proof of the equivalence of the two follow ing properties: a) T can be extended to a closed operator (one says T is closable); b) 7* 4is densely defined. One can easily give examples in which dom(T *) = {0}; 4if T is clos able, the smallest closed extension of T is T**, and one has T(T*") =T(7r) and (T *%* = 7%,
% 4) The fact that closed densely defined operators are not everywhere defined raises difficulties concerning algebraic
% combinations of such operators: A4 + B is only defined in dom( 4) N dom( B), AB only in dom( B ) N B-l(dom( A)); one
% can give examples of closed densely defined operators T such that dom( T2) = {0}. However, using the decomposition of EXE din the direct sum of the closed orthogonal subspaces
% SPECTRAL THEORY AFTER 1900 175
% T(T) and J(T( T%)), von Neumann could prove that for any closed densely delined operator T, dom(T*Y’) is dense, T*T 4is closed and (T*T)* = T*T . Furthermore, 1E + T*T (closed and defined in dom (7TM T )) dis a bijection of dom(T T ) onto the whole space E, the inverse -1 B= (1, + 7TM 7)) is a bounded self-adjoint and injective opexrator, the spectrum of which is contained in the interval
% (0,1].
% 5) These results enabled von Neumann to completely elucidate the spectral theory of normal operators in E. By defianition, * * * * . dom(V" ¥) = dom(#/¥") and N "N = NN*. The most important
% they are the closed densely defined operators N such that
% normal operators are the self-adjoint operators (which von
% Neumann called "hypermaximal"), defined by the coundition NV* = N (implyiang of coursc dom(lV*) = dom(¥ )), and the unitary operators, which are bounded and such that ZV*N = lE (hence invertible and such that N-l = N*).
% Now F, Riesz's definition of the spectrum of a bounded ope rator can be generalized for any closed operator T in E, One says a complex number ( 1is a regular value for T if the operator T - glE is a bijection of the subspace dom(fP) onto the whole space E and if the inverse mapping RT(Q) (also called the resolvent of T) is a bounded operator mapping E onto dom(T ); it is enough for that to know that T - Clp is
% in jective, that its image L 1is dense in E, and the inverse mapping (1’- glE)-l of L onto dom(I’) is continuous. The complement Sp(l’) of the set of regular values of 7T in C dis by definition the spectrum of T, and the mapping
% 176 CHAPTER VITI
% C— RT(Q) of € - Sp(fl’) into £(E) is holomorphic. For a number ( € Sp(T ), there are 3 possibilities:
% 1¢ T - ClE is not injective, which means there exists an x € dom( T) such that x # O and Te<x = (x, in other words ( 1is an eigenvalue of T3 one then says ( Dbelongs to the point spectrum of T,
% 20 [ = glE is injective and its image L is dense in E, but the inverse mapping (1’-Q1E)_l is not continuous in L; then ( dis said to belong to the continuous spectrum of T.
% 3¢ T - Clyp ds injective, but its image L is not dense in E;3; one says belong to the residual spectrum of T,
% For normal operators, there is no residual spectrum; selfl adjoint operators are characterized as normal operators for which the spectrum is contained in R, and unitary operators
% are normal operators for which the spectrum is contained in the unit circle U: [C]| = 1.
% 6) Generalizing F. Riesz's presentation of the Hilbert spectral theory (2), von Neumann shows that to every self ad joint operator4 in E is naturally associated a unique decomposition of unity. He means by that a fanily A+ E()\) of orthogonal projectors in E, depending on a real parameter As and such that:
% 1o £ E(@) = E() E = E(A) for A s u;
% 22 when ) > xo tends to ko, E tends to E(XO) strongly; when )\ tends to =-o, E(\A) tends strongly to O,
% and when ) tends to +o, E(x) tends strognly to lE; 32 for any x € E, +the mapping )+ ”E(X)'X”2 increases
% SPECTRAL THEORY AFTER 1900 177
% from O to ”xH2 in R; dom(A ) is exactly the set of x € E such that the Stieltjes integral
% [ aliz0))
% +c02 2
% =0
% is finite;
% Lo for any x € dom(A) and any y € E, the function +oo e
% A== (E(A)*x|y) dis a function of bounded variation, and one has the expressions
% (36) (4-x|y) =| Aa((EM)-x|y)), (x|y) =| a((EQ)-x|¥))
% —o -
% as Stieltjes integrals.
% Conversely, for any family )M+~ E of orthogonal proj ectors satisfying 1?2 and 29, conditions 3¢ and t° define a self-ad joint operator A4 and its domain, to which the given family is its decomposition of unity. The operator A4 is bounded if and only if there is a compact interval [a,B] such that E(A) = 0 for X < & and E()) = 1, for 1\ > B. The spectrum of A is the complement of the set of points u € R such that E(A) dis constaut in a neighborhood of |, and the point spectrum is the set of points |4 such that E(u-) 4dis distinct from E(u).
% For unitary operators U, there is a similar result: to U corresponds a unique decomposition of unity MA+——= E(A) sa tisfying conditions 1¢ and 2¢ above, with Z(\A) = O for A < O and E(A\) = 1z for X > 1; condition 3¢ is then au by 1
% tomatically satisfied, and the first relation (36) is replaced (37) (Vex|y) = e*MM G ((E(V)-x]y)). o
% 178 CHAPTER VII
% There is a similar "decomposition" for all normal operators, but we shall give it in a much simpler equivalent form in 5.
% 7) The most original part of von Neumann's work on spectral
% theory is his discovery and study of hermitian operators in Hilbert space E, as distinct from self-adjoint operators.,
% A hermitian operator H is a densely defined operator such that dom(H) € dom(H¥) and that the restriction of H' to dom(¥) is equal to H, in other words
% (38) (H+x]y) = (x|B+y) for x,y in dom(H),
% and in particular (H+x|x) is a real number for all x € dom(H) This implies that F is closable, and its closure H** is
% again a hermitian operator; one may therefore restrict the study to closed hermitian operators. The new idea of von
% Neumann is to adapt to Hilbert space a device introduced in 1855 by Cayley to parametrize the orthogonal group: he had
% shown that, for an nxn skew-real symmetric matrix S , such that det(I+5) # 0, U = (I-S)(I+S)nl was an orthogonal matrix, and any orthogonal matrix U such that det(I+U) £ O
% could be written in that way. Similarly, for a closed hermi tian operator H, one has Hx”2 < HH-x+ixH2 for x € dom(H#), which implies that the closed operator H+ il 1is injective in
% dom(H), and maps dom(H) on a closed subspace F in such a way that (H+iI)-l is continuous in F, and
% Ve yr—»(H—iI)(H+iI)-l-y is an isometry of F on a closed subspace V(F) of E. Conversely, if U is an isometry of a closed subspace F of E onto another closed subspace U(F) such that the image G of F by I -U is dense in E, then I-U is a bijection of F onto G, and if, for vy € G, one
% SPECTRAL THEORY AFTER 1900 179
% writes Hey = i(I+lD(I-U)-1-y, H 1is a closed hermitian ope- . . -1 rator such that dom(H) = G and U= (H-iI)(H +4iI) = ¥V de
% fined above.
% Furthermore, if E; is the orthogonal supplement of F in E, it is exactly the subspace of dom(H*) consisting of the solutions of H¥ex = ixy similarly, the orthogonal supplement E; of V(F) in E is the subspace of the solutions of H*¥+«x = =ix in dom(g*), and dom(H*) is the direct sum of the three subspaces dom(H), E; and Efi.
% 8) This method enables von Neumann to give a description of all hermitian operators Hl which extend a given hermitian operator H. It is enough to describe the isometry Vl which is the "Cayley transform" of H]: one takes a closed sub space M of E; and an isometry W of M onto a closed
% F, = E®& M, equal to V in F and to W in M; E is the 1 Hl
% subspace N of E;; Vl is then defined in the Hilbert sum
% 4 -
% orthogonal supplement of M in EH and EH the orthogonal supplement of N in E;. Such a construction is of course only possible if dim(M) < dim(Ey). The dimensions d of
% +
% + - -
% EH and d of EH are called the defects of H3; examples may be given in which they take any integral value or are in finite.
% Self-ad joint operators are by definition hermitian opera tors for which H* = H, or equivalently those for which the defects are (0,0). It follows at once from the preceding
% remarks that the closed hermitian operators which can be ex tended to self-adjoint operators are exactly those for which both defects are equal (finite or infinite); unless they are
% 180 CHAPTER VIT
% both O, there are infinitely many such self-adjoint exten sions.,
% To give an example of a closed hermitian operator of defects (1,0), von Neumann takes an orthonormal basis (en)nzo of E, and in IE considers the closed hyperplane F orthogonal to e s hence spanned by the e, with n = 13 he denotes by U the isometry of F onto E defined by U-en = e 1 for n =13 it is easy to show that the image of F by I -U 1is dense in E, and therefore U is the Cayley transform of a 2 d
% closed hermitian operator F having the required property.
% Another (non closed) hermitian operator is given by taking E =L (I), where I 1is any interval in R, and H = i'EE ’ which is defined in the subspace of E consisting of Cl functions vanishing at both extremities of I and whose de
% rivative is square integrable (or any subspace of that sub space which is still dense in LZ(I), for instance the space of ¢~ functions with compact support in I); it may then be shown that the defects of H & are (1,1) 4if I is bounded, (1,0) 4if I is only bounded from above, (O,l) if I is bounded from below and (0,0) if I = R.
% As we already mentioned (3) the results of H. Weyl on li near second order equations with real coefficients can easily
% be interpreted in the von Neuamnn theory: the differential operator I is hermitian; the defects of L"° are (2,2) in
% the "1limit circle" case, and (1,1) in the "limit point" case. They prefigurated the general spectral theory of formally self-adjoint linear differential equations which developed around 1950 (see chap.IX, 3).
% SPECTRAL THEORY AFTER 1900 181
% Similarly, Carleman's results are interpreted in the follow ing gay: for a Carleman kernel K, if one writes k(s)2 = = f/ |K(s,t)|2 dt, in order to get a hermitian operator, one a
% should restrict the operator S defined in 3 (formula (34)) to the subspace (dense in LZ) of functions f such that the integral ( k(s)|f(s)|ds 4is finite; S is then the adjoint
% b
% of that opeiator, which explains the existence of non trivial solutions of Se+p = ip in D = dom(S), and shows that the operator UX’ suitably restricted, coincides with the "Cayley transform" which von Neumann later defined in a imore general context.
% One should finally mention that von Neumann took pains, in a special paper ([221], vol.II, p.l4l-172), to investigate how hermitian operators might be represented by infinite ma trices (to which many mathematicians, and even more physicists, were sentimentally attached); he pointed out that if one
% wanted to associate to a hermitian operator H a matrix (amn) by the usual rule a = (H-emlen) for an orthonormal basis (en) of Hilbert space, one immediately ran into difficulties, since the vectors 1¥°em should be defined, in other words one should have e ¢ dom(H) for all n; furthermore, the sums ¥ |a should all be finite. But if H 4is not maxi

% n
% o

% mal (i.e. both defects are >0), any hermitian operator which extends H obviously has the same matrix (amn); and von Neumann showed in great detail how this lack of "one-to-oneness" in the correspondence between matrices and operators led to the weirdest pathology, convincing once for all the analysts that matrices were a totally inadequate tool in spectral theory.

\section{Banach algebras}
\label{sec:7.5}

% We have seen (2) that F. Riesz probably was the first ma thematician to consider the algebra £(E) of all continuous endomorphisms of a separable Hilbert space E, with its norm and what later came to be called its strong topology. In his second paper on spectral theory ([221], vol.II, p.86-143) in which he introduced the concept of normal operator in its most general form, von Neumann began a more detailed study of S(E) and its subalgebras. He introduced the weak topology on £(E) (see chap. VIII, 1), and (inspired by the work of I. Schur on linear representations of groups) the concept of commutant M’ of a subset M of £(E), but with an additional condi tion: M’ should consist of all operators A4 such that, not only A but also A*, was permutable with all elements of M, He focused his interest on the subalgebras of £(E) (later called involutive or *—subalgebras) which, with every element A, also contained its adjoint A4%*; and he proved in that
% paper the first two non trivial results on such subalgebras: the double commutant M’ of any involutive subalgebra M of £(E) containing 1, is the weak closure of M, and any weakly closed commutative subalgebra of £(E) is generated by a single self-adjoint operator A. A little later he com
% pleted this last rcsult by showing that one could define "functions f(4)" of a self-adjoint operator 4 for all uni versally measurable bounded functions f defined in R (and not only for semi-continuous functions, as F. Riesz had done), and he proved that the weakly closed subalgebra generated by
% SPECTRAL THEORY AFTER 1900 183
% A consisted of all operators f(A) +thus defined ([221], vol.II, p.177-212).
% But for von Neuwann this was only a beginning. The period 1926-1932 had seen the blossoming forth of the theory of "hypercomplex numbers'" of Molien, E. Cartan and Wedderburn into the beautiful theory of "rings with descending chain con ditions" of E. Artin and E, Noether, followed by their appli cations to linear representations of groups and number theory by R. Brauer, H, Hasse and A, Albert. Von Neumann was very much interested by these developments, and wondered if it
% could not be possible to build up some similar theory for in volutive subalgebras of S(E), where of course "chain condi
% tions" could not be expected, but suitable topological res
% trictions would be a substitute, allowing one to obtain a reasonable classification (loc.cit.,p. 89). It would take us
% too far away from our main theme to describe in some detail the series of papers, beginning in 1935, in which, with the partial collaboration of F., Murray, he achieved a great part of this program for what we now call the von Neumann algebras,
% namely the involutive subalgebras equal to their double com mutant in £(E). By the wealth and novelty of their techniques and their results, these wonderful papers are certainly the most profound and most difficult which von Neumann ever wrote ([ 221], vol.III); they revealed a large number of completely unsuspected phenomena, the most conspicuous one being the ap pearance, in the classification of the von Neumann algebras
% with trivial center (those called factors), of five types of algebras labeled In’ r - , IT 1? IIoo and III, where type In
% 184 CHAPTER VII
% means algebras of nxn matrices, type I_ the algebra £(E) itself, but the three other types were entirely unexpected and exhibit new features, such as the attribution to the pro
% jectors contained in these algebras of a "dimension" which, for algebras of Type II, may be any real number (in [0,1] or [0,+»]) dinstead of an integer. The elucidation of the pro perties of these new algebras, begun by Murray and von Neumann, has engaged many mathematicians during the last 40 years, and it is only recently that some difficult questions, such as the
% classification of algebras of type III, have begun to be un derstood (see [57], [210] and [44]). Furthermore, since 1950 the von Neumann algebras have been an important tool in the theory of linear representations of locally compact groups (see 6); more recently they have been associated to foliations and to generalizations of the Atiyah-Singer indes (see [ 58], (101, [31], [36], C45], (113], [132], [199]).
% Surprisingly enough, the difficult theory of von Neumann algebras was developed 5 years before the elementary concepts of the theory of normed algebras had been defined! The cre ation of that theory was the work of I. Gelfand in 1941 [83]; a normed algebra A (over the complex field) is an algebra over (€ on which is defined a structure of normed space with the condition that the mapping (x,y)+—xy of AxA into A be continuous. It is then possible to choose on A a norm
% compatible with the vector space structure and the topology of A, and such that in addition [ xy| < |/x Yl e If A has a unit element e, one may suppose in addition that ”e” = 1. If A has no unit element, it is always possible to imbed A
% SPECTRAL THEORY AFTER 1900 185
% into a normed algebra A with a unit element e, such that A is the direct sum of A and Ce. For any normed space E over €, the algebra £(E) of endomorphisms of E is a normed algebra for the norm ”A” = ”st HonH, but there are x|l
% many other types of normed algebras, the most elementary one being the algebra C(I) of complex continuous functions in a compact interval I of R, with | £l = sup [|[£(¢)]. teT
% Gelfand's main idea, which proved extraordinarily fruitful,
% was to extend spectral theory to elements of normed algebras; if A dis a normed algebra with unit element e, one may apply F.Riesz's definition of the spectrum (2) to define the spectrum of an arbitrary element x € A: it is the set of complex numbers ( such that x - (e 1is not invertible in A, Gelfand recognized that to get substantial results one must assume that A 1is complete as a Banach space, what is called a Banach algebra. Then very elementary arguments show that the spectrum SpA(x) of any element x € A is a non empty compact subset, contained in the disc |g| < Hx ;3 the inver tible elements in A form an open group G containing the ball Hx-e” < 1, and the topology induced on G is compatible with the group structure; for any x € A, +the map g&»(x-ge)-l of the complement @-SpA(x) into A is holomorphic. Final
% ly, Gelfand obtained a beautiful formula for the radius of the smallest disc of center O containing SpA(x); this number,
% (39) o (x) = 1am (=)TM
% called the spectral radius of x, 1is equal to
% N -co
% Next Gelfand undertook the study of general commutative
% 186 CHAPTER VII
% Banach algebras by a very original method. Probably inspired by the theory of commutative groups (see 6), he defined a character <Y of a Banach algebra A as a homomorphism of that algebra in the field ¢ (considered as C-algebra) which is not identically O, Suppose for simplicity that A has a unit element ej; then any character Y 1is such that X(e) = 1, and is a continuous linear form on A, of norm Hx” = 13 furthermore, for each x € A, one has YX(x) € SpA(x). Gelfand then associates to A the set X(A) of all charac ters of A; the map Xba-x—l(O) is a bijection of X(A) on the set of all maximal ideals in A (which are automatically closed). Now, in 1937, Stone [ 208] had already considered
% the set of maximal ideals of a very special type of ring, a "boolean ring" B, which is commutative and such that x2=:x and 2x = O for all x € Bj; this kind of ring itself had been suggested to Stone by the set of characteristic functios mM of subsets M of an arbitrary set E, where multiplica tion is the usual one, and addition + is defined by Py ¥ eN = Oyun "~ PMAN furthermore, Stone of course was well aware that for a self-adjoint operator A4 din Hilbert space, the orthogonal projectors mM(zi) ffor universally measurable subsets M of |R form a boolean ring for the same addition. The consideration of the set of maximal ideals of a commuta tive Banach algebra was therefore not at all foreign to the
% spirit of spectral theory at that time.
% As the set X(A) 4is contained in the unit ball | x’| < 1 of the dual A’ of the Banach space A, the natural embedding of A into its second dual A” associates to each element
% SPECTRAL THEORY AFTER 1900 187
% x € A the map X+¥ (x) of X(A) into €, which is called the Gelfand transform of x and is written (Gx. It is easy to see that X(A) is compact for ihe weak topology of A’, and that Gx is a continuous function on X(A) for that to pologys; one has therefore defined a continuous homomorphism x+> Gx of the Banach algebra A into the Banach algebra c(X(A)), such that the set of values of (x is the spectrum of x, and therefore |[Gx|| = p(x) < |[x|]|. The compact space X(A) is therefore called the spectrum of the Banach algebraA
% If one starts from the Banach algebra A = C(K) of contin uous functions on a compact space K, then it is easy to see that x+=Gx is an isomorphism of A onto C(Xx(4A)), Xx(A) being identified with the space of Dirac measures on K, But in general the homomorphism x~Gx of A into c(x(4)) is neither surjective nor injective. A little later, in colla
% boration with Naimark [ 85], Gelfand began to study Banach al gebras in which there is an involution x+— x* (i.e. such that (x+y)¥* = x*+y¥, (xy)* = v*x*, (Ax)* = Ax* for any scalar A and (x*)*¥ = x) for which in addition [ x¥*x| = = ”x”z; these algebras are now called C*-algebras. The main
% result proved by Gelfand and Naimark is that, for a commuta tive C*¥~algebra A having a unit element e, the mapping X+—= Gx is an isometry of A onto C(X(A)) such that Gx* = Gx for all x € A, Furthermore, if there exists in A * : . . x_ »x. and e is dense in A, then the map XF*'X(XO) is a
% an element X such that the subalgebra of A generated by
% homeomorphism of X(A) onto SpA(xo), a compact subset of € which one therefore identifies with the spectrum of A,
% 188 CHAPTER VII
% The Gelfand-Naimark theorem paved the way for a new inter pretation of Hilbert's spectral theory. Let E Dbe a separa ble Hilbert space, N a continuous normal operator in E; then the closure A in £(E) (for the normed topology) of the algebra generated by lE’ N and NV* 4is a separable commutative C¥-algebra with unit, the mapping MN: ¥ — X(N ) being a homeomorphism of X(A) onto the spectrum Sp(N) c C.
% From the Gelfand-Naimark theorem, it follows that the mapping f Q-l(fOfl) is an isometry of the algebra C(Sp(N)) on a subalgebra of £(E), which one writes f— f( N ), obtaining
% in this way a new definition of a "continuous function of a normal operator" which had been considered by F. Riesz and von Neumann., Following the method of von Neumann, it is then easy to extend the homomorphism f+> f( N ) +to the algebra u(sp(#¥#)) of all universally measurable bounded functions in Sp(N ) .
% Finally, by adapting the arguments of von Neuamnn, Hellinger and Hahn, one arrives at the modern description of the Riesz von Neumann "decomposition of unity" and of the "multiplicity
% theory" of Hellinger-Hahn:
% 19 There is a decomposition of E into a Hilbert sum (fi-

% nite or not) (E
% ) (w being an integer or +o) of J7 1< j<w

% closed subspaces, each of which is stable by N and by py¥. 29 There is a positive measure v on the compact space Sp(V¥) ¢ ¢ with support Sp(N ), and a decreasing sequence (s.) with S, = Sp( ¥), consisting of universally mea J7 1< j<w L surable sets.
% 32 For each j such that 1 < j < w, there is an isometry

% SPECTRAL THEORY AFTER 1900 189
% 13 of the Hilbert space Ej onto the Hilbert space F. = = LZ(Sp(IV),@S,°v) such that the normal operator T;NT;7 in

% J
% J

% Fj is the "multiplication operator" which, to the class of any function uj defined and square integrable (for v) in Sj’ associates the class of the function gka-guj(g).
% 4o In this description, the measure v (considered as a measure on @) is determined up to equivalence, the sets Sj are determined up to a null set (for v). The set Mj = = Sj"sj+1 is the part of Sp(N) of multiplicity j, and (when W = +o), M_ = Sj the part of Sp(N ) of infinite J
% multiplicity. If Pj is the orthogonal projector @M.( Ny, J
% and H., = P (E,), the restrictions of N +to the k or- ik k* i thogonal subspaces Hik (1 < i< k) are equivalent; the
% subspaces Ei are not uniquely determined, but the subspaces G, = P (E) = Hi,® Hy ® \ldots®0 H_, are. The equivalence class of v and the classes of the sets Sj are the unitary in variants of N which determine it up to a unitary equivalence N+— UNv~L,
% One says that this description if a diagonalization of the normal operator N. This name is Jjustified when one considers
% the classical case in which ¥ is a normal endomorphism of a finite dimensional space E: Sp(N) is then a discrete sub set of € consisting of the eigenvalues of VW, Sj the sub set consisting of the eigenvalues of multiplicity =j, Vv the
% measure having mass +1 at each point of Sp( N), and Gj is the subspace of E which is the direct sum of the eigenspaces of N corresponding to the eigenvalues of multiplicity j. It is easy, using von Neumann's results, to extend the pre-

% 190 CHAPTER VIT
% ceding description to unbounded normal operators J : Sp(lv) is then an arbitrary closed subset of € (it may be € it self), and the Sj arbitrary universally measurable subsets of Sp( N) forming a decreasing sequence; N is not defined
% in the whole subspace Ej’ but dom(¥ ) N EJ is the subspace transformed by Tj into the subspace of Fj consisting of the wu, such that the function gh»-guj(g) is square inte grable for wv.
% Furthermore, for each universally measurable function f in Sp(#) (bounded or not), f(#¥) is a (generally unbound ed) normal operator, which one may define in the following way: dom(£f(wx¥)) N Ej is the subspace transformed by Tj into the subspace of Fj consisting of the uj such that the function gh»-f(g)uj(g) is square integrable, and lhe class of this function is the image of the class of uj by -1 T f(N)T .~ J ( ) J
% For self-adjoint operators A in E, the connection with
% the "eigendifferentials" of Hellinger is made by the follow ing remark, due to F. Riesz: for every x = (xk) € LZ, a vector (pk(g)) is defined by taking Ag-x for every £ € R.

\section{Later developments}
\label{sec:7.6}

% Since 1940, an enormous number of papers have been publish ed on Banach algebras, spectral theory and their applications. I think a fair and well organized account of all these deve lopments will have to wait till more time has elapsed and has
% SPECTRAL THEORY AFTER 1900 191
% put them in their proper perspective(*). With the exception of the theory of differential (ordinary or partial) and inte gral equations, which has a complex background in which more
% than spectral theory is involved, and which will be considered in chap. IX, we shall limit our survey to bare indications of the general trends, and to references to a few papers and books.
% A) Structure of Banach algebras,
% After Gelfand and his school had investigated the general properties of all Banach algebras, mathematicians concentrat ed their efforts on two particular classes of such algebras,
% the commutative and the involutive ones.
% For a commutative Banach algebra A, a central problem was to define "functions" of elements x € A more general than polynomials, after the pattern set by F. Riesz and von Neumann. The latter had even shown that it was possible to define functions f(Nl, \ldots,Nk) of commuting normal operators
% Nj in a Hilbert space, for all continuous functions f de fined in Ck. For a general commutative Banach algebra A,
% such a definition was only possible under some restrictions on f; it Xy3eeeyXy are any elements of A, their Jjoint spectrum is the image in Ck of the mapping
% *
% ( )Glaring examples of lack of perspective are given by the Hellinger-Toeplitz article of 1923 in the Enzyklopddie der math., Wiss. [107], which gives undue emphasis to integral equations, and by Hadamard's article on Functional Analysis of 1928 ([94], vol.I, p.435-453), which barely mentions F. Riesz and does not speak of spectral theory at all!
% 192 CHAPTER VIT
% XF+-(X(xl),x(xz), \ldots,x(xk)) where X runs through X(A) (for k = 1, it is of course the spectrum of xl); one then could prove that if B is the algebra of (germs of) functions f holomorphic in a neighbourhood (depending on f) of the joint spectrum of XpsesesXy there is a homomorphism B = A written fW—>f(xl, \ldots,xk), which uniquely extends the natural homomorphism of the algebra of polynomials on Ck into A written similarly ([222], [27]).
% On the spectrum X(A) of a commutative Banach algebra A,
% one soon was led to consider a topology different from the one induced by the weak topology of the dual A’ of the Banach space A. For Boolean rings, Stone had introduced the idea of defining on the space of maximal ideals of such a ring B a topology, in which the closed sets were defined as the sets of maximal ideals containing a giveun (arbitrary) ideal of B. As the set X(A) of characters of a commutative Banach algebra A corresponds in a one-to-one way to the set of maxifial ideals, Stone's topology can be defined in the same way on X(A); dit is in general coarser than the weak topology, and one says A 1is a regular commutative Banach algebra if these two topologies on X(A) coincide; for instance, the algebra @(K) of continuous functions on a compact space K
% is a regular algebra,
% To any closed ideal J in A, one attaches the set h(J) off all characters ¥ € X(A) which vanish on J; a natural question is to ask if the intersection of all kernels X-l(O) (maximal ideals of A) such that ¥ € h(J), which always contains J, dis actually equal to J; one then says that
% SPECTRAL THEORY AFTER 1900 193
% the ideal J admits spectral synthesis., Giving conditions for a closed ideal to admit spectral synthesis in a regular commutative Banach algebra is a problem which has been exten sively studied ([21], [58]).
% Involutive Banach algebras A (not necessarily commutative) are those equipped with an involution x+» x* such that | x*|| = [|x|]] for al11 x € A; c¥*¥-algebras (5) are involutive
% algebras, but there exist involutive Banach algebras which are not C¥-algebras (see C) below). The central concept is that of representation of an involutive Banach algebra A in a Hilbert space Ej; this means a homomorphism f: A = £(E) of algebras such that in addition f(x*) = f(x)¥. They have been the subject of a large number of investigations, lecading 1o the elucidation of the structure of several classes ol C*¥-algebras; the theory of von Neumann algebras (which are special types of C*-algebras) plays a great part in these in vestigations ([58], [36]).
% B) Algebras of continuous functions.
% Since 1960, many mathematicians have been interested in the study of subalgebras of Banach algebras C(K) of con tinuous functions on a compact space K. 1In classical Analysis, one had much studied the case in which K is the unit disk |[z| £ 1 in €; there is then in C(K) a parti cularly interesting Banach subalgebra, namely the algebra B of functions which are holomorphic in the interior |z| < 1 of the disc. It can be identified with the algebra BO of the restrictions of the functions of B to the unit circle WU: |z| = 1, and B, 1is also the closure in c(U) of the alge-
% 194 CHAPTER VII
% bra of trigonometric polynomials. It turns out that the study of BO is closely linked to the completions of the space of trigonometric polynomials in the various spaces Lp(u% where |4 1is Haar measure on U, and many beautiful proper ties of these spaces (known as the Hardy spaces Hp(u)) had been discovered., But in the light of the theory of commuta tive Banach algebras, it was found that these results could
% be much better understood if they were generalized to subal gebras of an algebra C(K) where K is any compact space, and put in relation with some kinds of measures on K (see f18], [(33], [79], [116], [140]).
% C) Harmonic Analysis.
% We have already stressed the fact (chap.I, 2) that
% Fourier series provided the starting point of spectral theory when it was realizedlthat they could be generalized to "ex pansions" in series of "orthogonal'" functions arising from boundary wvalue problems,
% It was, however, very soon observed that the "trigonometric system" (elnx) possessed very peculiar properties not ncz shared by general "orthogonal systems", and linked to the . / . . /
% functional equation el(X+x ) = e ¥et¥ . For instance, if + o0 . Ffoo :
% N==co N==cw
% f(x)  a(n)enlx, g(x) = £ b(n)eTMTM were two Fourier
% series, one had Cor the Fourier series f(x)g(x) = +
% = X C(n)enlx of their product, the very simple formula (40) c(n) = Z a(p)b(n-p).
% N=-=0o
% + o

% P==co
% + oo .

% Similarly, from the formula f(x) = £ a(n)e"'TM, one obtain N==o©
% ed

% SPECTRAL THEORY AFTER 1900 195
% (41) 2 a(n+l)e¥ = e ¥ (x)
% 400 . .
% n:—oo
% and this property was used by de Moivre and even more by P
% Laplace to solve linear difference equations X @ka(n+k)==0 k=1
% by associating to the sequence (a(n)) the Fourier series f(x) = I a(n)e"TMTM, reducing the difference equation to an n
% algebraic equation for f(x).
% Similar peculiarities were observed for the "Fourier trans form" associating to an integrable function f din R the (42) Ff(x) = eT2MIXT riyat,
% function
% 400
% Its main virtue, in the eyes of Fourier, Cauchy and Poisson, was that it reduced linear partial differential equations with
% constant coefficients to algebraic problems, due to the fact that the Fourier transform of the derivative f’ is the function x« 27ix Ff(x). Furthermore, in his researches on Probability theory, Tchebycheff had shown that if F1 F2 are
% two independent "random variables" with "probability lawsTM" @,y &, Mmeasures on R with densities 810 8o the proba bility law of F, 1 + F5 had a density given by the convolution g = 81x85 def'ined as
% 40
% (43) g(x) = g, (t)e,(x-t)dt ([211] ,vol.ITI,p.481-491);
% and Tchebycheff's student Liapounov, who started to use Fourier (44) S(gl*gz) = gl'3g2 [148].
% transforms in Probability theory, observed that
% 196 CHAPTER VII
% One should also mention the Poisson formula (also discover (45) T f(n) = T 3f(n)
% ed independently by Cauchy)
% necZz ncz
% for sufficiently regular functions f on R,
% It took over 100 years to understand these peculiarities and to connect them with the notion of group, via the concepts of character and of group algebra. Characters were first de fined for arbitrary finite commutative groups by H, Weber in 1882, as complex valued functions <Y on such a group G with values #0, such that ¥X(xy) = X(x)x(y) for all x, y in Gj
% but special cases had long before been considered by Legendre, Gauss and Dirichlet. A meaningful generalization to non com mutative finite groups was discovered in 1896 by Frobenius: instead of considering homomorphisms of G dinto the multi plicative group ¥, one should consider homomorphisms s —= U(s) of G into the general linear group GL(n,C) of invertible matrices of order n, for any integer n, This is also called a linear representation of degree n of G in the vector space E = Cn: giving such a representation is equivalent to defining an action (s,x)+» sex of G on E such that se(tex) = (st)ex, e+x = x for the neutral element e € G, and such that each mapping X1 se*x is linear (with matrix U(s)).
% Now Cayley had defined, for a finite group G, the group algebra ¢€[G] as the vector space of all formal linear com binations z %Ss with S € ¢, multiplication being de SEG
% fined by
% SPECTRAL THEORY AFTER 1900 197
% 46) (2 £ s)(z s) = z E n,st. ( scc S seg | ® (s,t)€Gxa ° ° When there is given an action (s,x)~>s*x of G on E as above, it defines naturally on E a structure of left (47) (Z SS)°X  gs(sox)
% ¢[G] -module by
% s€ G s€EG
% and the study of linear representations of G is thus equi valent to the study of left ¢€[G]-modules.
% The fundamental results of Frobenius for finite groups may then be expressed in the following way. An element ) gss s€G
% of @[G] may be identified with a mapping f: s—>&_ of G into €, so that ¢€[G] may be identified with the vector space GG of all mapping of G into €, with the multipli cation written fxg and defined by
% (48) (fxg) (s) = (Card(G))-l X f(t)g(t'ls). t€G
% Then €6[G] can be written as a direct sum Al &) A.2 D oo Ah of mutually annihilating subalgebras, where h is the number of classes of conjugate elements in Gj; each A (1sk<h) is it has a basis (mk ) of n 10 elements belonging to GG, with ij
% 2
% a matrix algebra of dimension n, over C, which means that 2
% the following properties:
% k k k
% (L49) mpq*mrs = 6qrmps for 1 < pyq,r,s < n, (50) mgq*mis =0 if k £ k’.
% /
% In addition, one has
% 198 CHAPTER VII
% (51) m?i(s) = mgj(s-l) for 1<i,j <n, s € G (52) Z miq(s)mis(s) = 0 unless p=r, q=s, k=k’
% /
% s€G
% (53) z mpq(s)mpq(s) = n, card(G) for 1< p,q < n,
% k k
% s€G
% (orthogonality relations). One has n; + ng oot nfi card(G)
% 2
% and the expression of any f € CG with respect to the basis (mij) of' that space,
% k
% (54) f(s) = i,i,k cf my i (s)
% k k
% k -1 k
% is given explicitly, due to the orthogonality relations, by (55) cij = (nk Card(G)) SEG f(s)mij(s). If once writes Mk(s) the n, xn,_ matrix (n;lmij(s)), one
% has
% -1 (56) Mk(st) = Mk(s)Mk(t) and Mk(s ) = (Mk(s))*
% In other words, s+ Mk(s) is a linear representation of G of degree n, for 1 £ k € hy it is irreducible, which means that the corresponding C€[G]-module is simple (i.e. has no non trivial submodule). Furthermore, every €[G]-module of finite dimension over (€ is a direct sum of modules each of
% which corresponds to one of the linear representations S > Mk(s); one says that every linear representation of G
% is completely reducible, and that it contains the irreducible representation s~— M (s) with multiplicity d, if in the
% direct decomposition of the corresponding module, there are d, submodules corresponding to s+— Mk(s).
% SPECTRAL THEORY AFTER 1900 199
% Now linear representations of degree n can be defined in the same way for any group G, finite or not. Already in 1901, I. Schur, in his dissertation ([193],vol.I, p.1-70),
% could determine all linear representations of the general 1i near group GL(N,C) which are such that the elements of U(s) are polynomials in the elements of the matrix s € GL(N,C); he showed that these representations are again completely re ducible and he could determine explicitly the irreducible ones (see [53]); but it is clear that for such infinite groups, all the Frobenius relations described above were meaningless., However, in 1924, I, Schur observed that the restrictions of these representations to the group of rotations G = SO(N,R) gave him irreducible representations of that compact group,
% and that these representations were continuous, and could be written s~ M (s) where Mk(s), as in (56), was a unitary
% matrixs; furthermore, he proved the relations which he rightly considered as the analogues of (52)
% (57) mgq(s)mis(s)ds = O unless p=r, q=s, k=k’
% —_—
% G
% where ds is a left and right invariant measure on G, the existence of which was substantially known since S. Lie, and which had already been used to construct invariants of SO(N,R) by Hurwitz in 1898 ([193], vol.II, p.4Lo-L9L),
% This result attracted the attention of H., Weyl; in a beau tiful series of 3 papers published the next year, by a skill ful combination of Schur's ideas with the "infinitesimal" methods by which E, Cartan in 1913 had obtained all finite dimensional representations of the complex semi-simple Lie
% 200 CHAPTER VIXI
% groups, he was able to determine explicitly all continuous
% irreducible linear representations of compact semi-simple Lie groups (including, in the case of SO(N,R), the "spinor" re presentations which had escaped I. Schur). 1In all cases, the orthogonality relations (57) still held, and every continuous linear representation of a semi-simple compact Lie group was
% shown to be completely reducible ([227],vol. II, p. 633).
% Of course, for compact Lie groups, there is an infinite system of irreducible representations M k’ and relations such as (5&) were out of the question., But for the group SO(l,R): = U (the circle group), the irreducible representations were the characters (+— gn for n€ Z, and H. Weyl realized ‘that the formula which corresponded to (54) was just the Fourier series expansion of f (when f is sufficiently re gular) [ 227, vol.III, p.34-37]. He then undertook to general
% ize this expansion to all semi-simple compact Lie groups G for such a group, the functions m?j which he had determined formeé an orthogonal system in the Hilbert space LZ(G) (for
% a left and right invariant measure); the problem was to prove that this system was complete,
% This is what H. Weyl proved in 1927, in a remarkable paper written in collaboration with his student F. Peter [227, vol. IIT, p.58-75], which can be considered as the first application of spectral theory to harmonic analysis. He saw that the no
% tion which could serve as a substitute to the group algebra ¢[G] was the space (C(G) of continuous complex-valued func tions on the compact group G, on which an algebra structure is defined by convolution, generalizing (40), (43), and (48):
% SPECTRAL THEORY AFTER 1900 201
% -1 (58) (fxg)(s) = {, f(t)g(t "s)dt
% ‘G
% where the integration is for a left and right invariant posi tive measure on G with total mass 1, H. Weyl next observed that, given a linear representation st— U(s) of G by uni (59) U(f) = f(s)afgads
% tary matrices of order n, if one wrote
% G
% one obtained a homomorphism of the algebra C(G) into (60) U(fxg) = U(f) u(e)
% End(@n), in other words
% Vv -
% (61) u(f) = (u(£))”.
% and furthermore, if one wrote f(s) = f(s-l), one had v
% The crux of his proof is to show that for an f £ O in c(G), there is at least a continuous representation s U(s) for which U(f) # O; due to the complete reducibility of se—= U(s), there is then at least one irreducible represen tation s~e~Mk(s) such that Mk(f) # 0, and this hows that f cannot be orthogonal to all functions mk.. However, if 1
% the system (m. ) was not complete, there would exist a non i
% k
% negligible function g € LZ(G) orthogonal to all the m?j, and as the subspace L of C(G) generated by the mij is a two-sided ideal (one has (fxU)(s) = v*(f)U(s)), hxg would also be orthogonal to L for any function h € ¢(G), and one has hxg€ ¢(G) and hxg £ O for suitable functions h.fij (*)This is a slight simplification of Weyl's argument, which consists in obtaining for each function of C(G) the analogueof the Fischer-Riesz exgansion by an inductive application of the Schwarz-E.Schmidt method.
% 202 CHAPTER VII
% The proof of the existence of a representation [y such that U(f) £ O is deduced by Weyl from the theory of Hilbert-Schmidt inlegral equations. He considers the function g = f*g, for which U(g) = U(f)U*(f), and it is enough to show that U(g) # 0O for some U. One has g(e) = (’ |f(t)|2 dt > O3 Weyl forms the sequence of functions g1 S 8y 8, = E%8ysece g 8% g by an adaptation of the method of H.A.Schwarz, n
% as generalized by E. Schmidt to integral equations with sym metric kernels (chap.III, 1 and chap.V, 2), he proves that the sequence of numbers Y = gn(e)/gn_l(e) is increasing and tends to a 1imit Y > O, and gn/Yn tends uniformly to a continuous function u, such that gxu = u¥g = Yu and u*u = uy Y 1is an eigenvalue of the hermitian kernel k(s,t) = g(st-l), and if P ; (1<j<sr) form an orthonormal
% basis of the corresponding eigenspace, one easily proves that = ml(s)ng¥3 Toeot mr(s)5;T€3. Furthermore, for any t € G, the function Sra-@j(st-l) again is an eigenvector of the same space, hence mj(st-l) = kgl G;;?Ejmk(s), and one shows that if U(t) = (ujk(t))’ tl* U(t) dis a linear representation of G for which U(g) # O.
% H. Weyl himself remarked that this method also proved the existence of the irreducible representations M and was k? applicable to any compact Lie group (not necessarily semi simple); a little later, when A, Haar had proved in 1933 the existence of a measure invariant by left and right translation on any compact subgroup, Weyl's arguments could at once be extended to that general case.
% The next year, Pontrjagin, in view of applications to alge-
% SPECTRAL THEORY AFTER 1900 2073
% braic topology, showed that the Peter-Weyl theory, applied to commutative metrizable compact groups, led to a remarkable generalization of the duality between finite commutative groups, which had been well-known since Weber. It was of course classical that an irreducible linear representation of a commutative group G must be of degree 1, in other words it is a character X of G. The Peter-Weyl theory therefore associated to a metrizable compact commutative group G the set é of all continuous characters of G, which of course
% is itself a denumerable group for ordinary multiplication. Now, for any x € G, the map Y+ ¥(x) is clearly a cha racter of é, and Pontrjagin showed that all characters of é are of that type. Conversely, iff D 1is any denumerable grou and D the group of all its characters, one can put on 5
% the topology of simple convergence, for which it becomes a
% metrizable compact group, and then all continuous characters N
% of D are exactly the maps Y+ X(x) for all x € D, But Pontrjagin went further and could extend this duality to some locally compact commutative groups [179], and in 1936 van Kampen showed, by different methods, that Pontrjagin's results could be generalized to all such groups G. The dual é, consisting of all continuous characters on G, is given the topology of uniform convergence on compact subsets of G,
% and it is again locally compact f{or that topologys; to each
% X€ G there corresponds the continuous character n(x): X—X (x) on é, and the duality theorem of Pontrjagin-van Kampen says that 1 is an isomorphism of topological groups X
% of G onto G [216].
% 204 CHAPTER VII
% This discovery made possible a unified treatment of the Fourier series and of the Fourier integral. In general, for (62) 3f(X) = S’ F(x)X(x)dx
% any function f € Ll(G), one could define the function on G N
% G
% as the Fourier transform of f, For G = IR, continuous cha racters could be uniquely written x +— exp(Zflixy) for a real number vy, so that G could be identified with R it self, and (62) was just the definition of the Fourier integral. For G = Z, all characters are continuous and can be written n t» gn for a uniquely determined complex number ( € Uj; for a function n+ c(n) on Z such that E lc(n)| < +», the
% .right hand side of (62) becomes the absolutely convergent Fourier series £ c(n)cTM = % C(n)enie il ¢ = eie, a
% nez ne7z
% function defined on the dual ©U of Z. Finally, for G = U, continuous characters are the functions gke-gn for a unique ly determined n € Z, and the Fourier transform of a function f € Ll(U) is the sequence n+= c(n) of its "Fourier coef ficients". In 1940, A, Weil [226] showed how most results concerning Fourier series and integrals could be generalized to all locally compact commutative groupsy; the central theorem
% was the generalization of the Parseval relation: if a func tion £ on G ©belongs to Ll(G) N LZ(G), its Fourier trans
% (63) [ £(x)]* ax = ( 3r(x)|* ax
% form belongs to Lz(a), and one has the relation G G
% for a suitable Haar measure on Gj; this relation, for the case G = R, had been proved in 1910 by M. Plancherel [ 1747,

% SPECTRAL THEORY AFTER 1900 205
% and is known as the Plancherel theorem for locally compact commutative groups.
% We have already mentioned (5) that Gelfand, when he defined characters on a commutative Banach algebra, had followed the pattern set by H. Weyl and Pontrjagin., In fact, in a Jjoint paper with D. Raikov [ 86], he immediately showed how the Pontr jagin-van Kampen-A., Weil theory could be deduced from his general results on Banach algebras in a much simpler way (the
% earlier proofs relied heavily on detailed information on the structure of locally compact commutative groups). The basic
% idea is to consider, for a locally compact commutative group G, the space Ll(G) (for a Haar measure on G), on which a
% structure of Banach algebra is defined by the convolution
% product (58); it is an involutive algebra for the involution V
% £+ f, but in general it is not a C¥*-algebra. A character of that algebra (in the sense of Gelfand) can then be unique 1y written as f+ 3f(x) for a well-determined character ¥ (in the sense of Pontrjagin), so that the spectrum X(Ll(G))
% is identified (with its topology induced by the weak topology of L”(G)) with the dual group G, and then the Fourier
% transform merely becomes a special case of the Gelfand trans form, equation (44) being the expression in that special case of the fact that the Gelfand transform is a homomorphism of algebras!
% But this absorption of harmonic Analysis by spectral theory did not stop with commutative groups. One can still define the Banach algebra Ll(G) for locally compact separable uni-
% 206 CHAPTER VII
% modular groups (i.e. those for which left invariant Haar mea sure is also right invariant, for instance compact groups or semi-simple Lie groups), and it is still an involutive algebra.
% On the other hand, one can define linear representations S — U(s) of such a group G not only when U(s) is a uni tary matrix, but more generally when U(s) is an automorphism of a complex Hilbert space E; one then speaks of unitary re presentations of G in E, and one adds to the definition the additional condition that for any x ¢ E, the mapping sr»-U(s)-x of G into the Hilbert space KE should be con tinuous. For any function f € Ll(G), it is then possible to define U(f) as in (59), more precisely, one has, for any x € E and y € E
% (64) (U(£)-x]y) = £(s)(v(s)-x|y)ds G
% where ds is (left and right) invariant Haar measure on G.
% It is then remarkable that starting from the unitary repre sentations of G din E, one obtains in this way a bijection
% of the set of these representations onto the set of all homo morphisms V of Ll(G) into the C*-algebra £(E) which sa tisfy (61) and are non-degenerate (i.e. such that the V(f)ex for x €& E and f € Ll(G) generate a dense subspace of E).
% With convenient modifications, there is still a similar re sult for all locally compact groups, and the general theory of unitary representations of locally compact groups in Hilbert space is thus in a certain sense subordinate to the theory of homomorphisms of involutive Banach algebras in al gebras of operators in Hilbert space [ 58].
% SPECTRAL THEORY AFTER 19060 207
% However, most results concerning unitary representations of locally compact groups have up to now been restricted to Lie groups, where a large number of more refined and powerful tools (Lie algebras, differential geometry, partial differ ential equations, etc.) are available. We can only mention here this beautiful and difficult theory (known as non commu tative harmonic Analysis), which has known an enormous expan sion since 1950, and in which many problems are still openj the interested reader is referred to [32], [43], [152], [ 217] and [ 223]; for a detailed history of harmonic Analysis (both commutative and non commutative) and its relations with pro bability theory, quantum mechanics and number theory, see [155].
% D) Other developments.
% One of the first results on infinite dimensional repre sentations was obtained by M.H, Stone in 1930 [206]: he show ed that any unitary representation of the additive group R
% into a separable Hilbert space E was given by the formula t - eitA, where 4 is an arbitrary (in general unbounded) self-adjoint operator in E, so that one may say that the theory of unitary representations of R is equivalent to the spectral theory of unbounded self-adjoint operators.
% If now E idis an arbitrary Banach space, and A a bounded operator in E it is clear that th»-etA is a homomorphism of R into the group of invertible elements in £(E). Un
% bounded operators A4 may be defined in E just as in Hilbert space, but for such an operator, etA usually has no mea ning for every real number t. Various questions of Analysis led E. Hille, in a series of papers beginning in 1936, to in-
% 208 CHAPTER VII
% vestigate mappings t— P into £(E), only defined for t > 0 and such that, for s > 0 and <t > O
% Such mappings are called semi-groups of operators. If one assumes that ”PtH < C for all t > O and that for every x € E, t~— Pt'x is continuous for + > O, then one may
% associate to such a semi-group an unbounded operator A4 in E, (66) Aex = 1lim t-l(Pt-lE)°x .
% defined by
% t-+0
% This has been the starting point of an extensive theory with many applications in Analysis [115].
% A large literature has been devoted to various types of ope rators in Banach spaces. A very general method consists in starting with an operator AO whose properties are well-known (for instance a normal operator in a Hilbert space) and to consider operators A = AO + P which differ from AO by a "perturbation" P which is "small" in some sense; for instan ce, the norm ”P” is supposed to be small enough, or P dis a
% compact operator; such assumptions allow in many cases to extend some properties of A, to 4 (see [124]). The nice properties of the operators lE + K, where K is a compact operator (1) have inspired the study of generali
% zations of such operators, for instance the Fredholm operators U, which are defined by the properties that U-l(O) has fi nite dimension, U(E) is closed and has finite codimension, but the dimension of U~1(0) and the codimension of U(E) are not necessarily equal [133]. Finally, there is an exten-
% SPECTRAL THEORY AFTER 1900 209
% sive theory of operators for which there is a family of pro jectors having properties similar to the projectors E(k) associated to a self-adjoint operator in von Neumann's theory (4); the difficulty is of course to find criteria implying the existence of such a family ([62], vol.III).

\chapter{Locally convex spaces and the theory of distributions}
\label{ch:8}
\setcounter{equation}{0}

\section{Weak convergence and weak topology}
\label{sec:8.1}

In his thesis, Fréchet had already noticed that convergence in a metric space could not always correspond to some classical types of ``convergence'' for functions. For instance, if \(\B(\R)\) is the vector space of \emph{all} bounded real functions on \(\R\), it is not possible to define a distance on that space such that \emph{simple} convergence in \(\B(\R)\) would be identical with convergence for that distance. This results from the fact that if \(A\) is a subset of a metric space \(E\), the closure \(\overline{A}\) of \(A\) in \(E\) is identical to the set of limits of all convergent sequences of elements of \(A\). However, if one takes in \(\B(\R)\) the set \(A = \Cont(\R)\) of bounded continuous functions, the limits of sequences of elements of \(A\) for simple convergence are the Baire functions of class 1, and it is known that there are Baire functions of class 2 which are not of class 1, so that \(\overline{A}\) (for the hypothetical distance) could not consist only of functions of class 1 \cite[p.~115]{071}.

There was thus an obvious need for a generalization of the concept of metric space, but none proved adequate for Functional Analysis until Hausdorff, in 1914, created ``General topology'' as we understand it now, based on the concept of neighborhood \cite{100}; but surprizingly enough, it took some time to become aware of that adequacy. Ever since Hilbert, ``weak convergence'' of \emph{sequences} had become a ceniral theme, first in Hilbert spaces, then with F. Riesz and Helly in some types of normed spaces (chap. \ref{ch:6}), and one would have thought that Hausdorff's concept of topology would have been tested on that notion; but ultil 1934 the only mathematician who seems to have had that idea was von Neumann: he defined weak neighborhoods of a point \(x_0\) in a Hilbert space \(E\) by a finite number of conditions \(|\left(x-x_0|a_j\right)| \leq \varepsilon\) for points \(a_j \in E\), and then went on to define similarly, in the algebra \(\L(E)\) of endomorphisms of \(E\), ``strong neighborhoods'' of an operator \(U_0\) by a finite number of conditions \(\norm{(U-U_0) \cdot x_j} \leq \varepsilon\) for \(x_j \in E\), and ``weak neighborhoods'' by a finite number of con ditions \(\norm{\left( (U-U_0) \cdot x_j | y_j \right)} \leq \varepsilon\) \cite[vol.~II, p.~94--104]{221}. But he did not try to extend these ideas to other Banach spaces.

% On the contrary, "weak convergence" was at the center of Banach's book, and the results he obtained concerning that notion can be considered as some of his deepest work. But to understand what he did, it is probably better first to state
% the final form which was taken by his 3 main theorems: I) If E 4is a Banach space, and its dual E’ is given the weak topology o(E’,E), the unit ball |[[x’]] < 1 in E’ is
% compact for that topology.
% IT) In order that a vector subspace V < E’ Dbe closed for the topology ¢(E’,E), it is necessary and sufficient that for any closed ball B’ in E’, Vv N B’ be compact for that
% topology.

% 212 CHAPTER VIII
% III) In order that E be reflexive, it is necessary and sufficient that the unit ball |[|x|| < 1 in E be compact for the weak topology G(E,E').
% In this form, the theorems were proved by N. Bourbaki in 1938 [ 25], and independently a little later by L. Alaoglu [ 5].
% Their proofs use the following ingredients:
% a) The weak topology o(E’,E) is defined by taking as neigh borhoods of xg € E’ the sets defined by a finite number of relations |(x’-xg,xj)| < ¢ for arbitrary X € E; o(E,E") is defined similarly by exchanging the roles of E and E’.
% b) The word "compact" is used in the sense of N, Bourbaki, and means what was defined as "bicompact sets" (in Hausdorff spaces) by P, Alexandroff and P. Urysohn in 1924 [6]: for them a space is "bicompact" if every open covering of the space contains a finite covering (the "Borel-Lebesgue axiom").
% c) Compact sets can be characterized equivalently by means of the notion of limit of a "net'", a notion which generalizes
% the 1limit of a sequence and was introduced in 1922 by E,.H, Moore and H.L. Smith [164] (N. Bourbaki uses the equivalent
% concept of limit of a "filter").,
% d) Any product of compact spaces is compact, a theorem proved by A, Tychonoff in 1930.
% However, none of these notions or theorems was ever men
% tioned by Banach or mathematicians of his school until 1940, although they repeatedly quote Hausdorff's book of 1927 ( );
% *
% ( )The bulk of that book [101] is devoted to metric spaces,
% *
% and general topological spaces are given a very scanty treat ment in 5 pages; it seems that Hausdorff had lost faith in his ideas of 1914!

% LOCALLY CONVEX SPACES AND THE THEORY OF DISTRIBUTIONS 213
% for them, the word "compact" is always taken in the initial sense of Fréchet, meaning a space in which there is no infi nite closed discrete set, That notion is equivalent to the notion of "bicompact" space when restricted to metrizable spaces; the version of theorem I proved by Banach [15,p.123] is therefore limited to separable Banach spaces E (because in that case the ball ||x’]] £ 1 4is metrizable for the weak topology o(E’,E). It generalizes of course the "principles of choice" used by Hilbert, F. Riesz and Helly (ohap.V and VI)
% On the other hand, Banach was able to prove a theorem equi valent to Theorem II for all Banach spaces. He starts from the study of vector subspaces of E’ which are closed for
% the topology of the norm, and observes that for such a sub space V, a sequence (x;) of points of V may have a weak limit which does not belong to V; furthermore, if Vl is
% the vector space consisting of all these weak limits, it may happen that sequences of points of V have weak limits 1
% which do not belong to V and so on [15,p.209]. Without 1? speaking of weak topology, he then introduces the weakly closed vector subspaces, under the name of "regularly closed" subspaces: he defines such a space V by the property that for any xé £ V, there is an x € E such that (x',x) = 0 for all x'€¢ Vv, but (xg,x) ; O, Conscious of the fact that
% weakly convergent sequences are inadequate tools, Banach then introduces an ad hoc notion, the "limits of bounded transfi nite sequences" in the dual E’: for any family (ué)<Y of elements of E’, contained in a ball and indexed by a segment of the ordinals, he shows (by using the Hahn--Banach theorem)

% 214 CHAPTER VIII
% that there always exist elements u’€ E’ such that, for

% every x € E,
% gy gy
% lim.inf (ué,x) < (u’,x) < lim.sup (ué,x)

% and calls any such u’ '"a limit" of the transfinite sequence (ué) (there may be infinitely many such "limits"!). He then says V is "transfinitely closed" if every bounded transfini
% te sequence of elements of V has at least a "transfinite limit" in V, and what he shows, by a very clever argument,
% is that "regularly closed" and "transfinitely closed" are equivalent notions [ 15, p.121]. It was an easy matter to replace "transfinite sequences" by "nets" or "filters" in Banach's proof to obtain the equivalent statement of Theorem
% IT.
% Finally, by considering E as naturally imbcecdded in its second dual E’ and using again "transfinite limits", Banach (%) could prove Theorem III, but only when E is separable
% It should be mentioned here that these theorems enabled
% Banach to obtain a series of interesting theorems relating the properties of a continuous linear mapping u: E = F (where E
% and F are Banach spaces), those of its transposed mapping Pu: P E’, and properties of the images u(E) and tu(F') (see [517) (**),
% (*)In 1938, Goldstine proved a result which is equivalent to the property that for any Banach space E, the intersection ENB”, where B” is the unit ball |[x”||<1 in the second dual E”, is dense in B” for the weak topology o(E’ ,E’); from this Theorem III easily follows [ 88].
% ( )Some of Banach's results had been obtained by Hausdorff in
% * %
% 1931 [102]; it is quite remarkable that he makes no mention of weak topology!

\section{Locally convex vector spaces}
\label{sec:8.2}

% Although the theory of normed spaces was in the forefront of the development of Functional Analysis after 1906, it was soon realized that they did not exhawust the possibility of applying topological concepts to that disciplinej; but the va rious notions belonging to what we now call the general theory of topological vector spaces made their appearance in a rather random way and were not the subject of a systematic treatment until 1950,
% Already some examples of such spaces are to be found in Fréchet's thesis, where the emphasis is put, not on their al gebraic properties but on the possibility of defining their topology by a distance (chap.V, 3) and on the fact that the metric spaces thus obtained are complete. The fact that ad dition and multiplication by a scalar are continuous in such spaces was only explicitly emphasized by Fréchet in 1926 [73]; the idea was picked up by Banach who in his book considered in general these spaces under the name of "spaces of type @)"
% and showed that the closed graph theorem was also valid for them, A little later, the method which Fréchet had used to define the distance on his examples of 1906 was systematized
% by S, Mazur and W, Orlicz in what they called the theory of "spaces of type (BO)" [160]: they are what we now call
% Fréchet spaces, where the topology is defined by a sequence (p,) of seminorms with the condition that x # O implies that pn(x) # O for at least one index nj; the distance can be defined by the formula

% 216 CHAPTER VIII
% l+pl(x—y) 21 l+p2(x-y) ***7 nl 1+pn(x-y) d(x,y) = +ooe
% and it is supposed that the space is complete for that distan ce. For the examples of Fréchet (the space RN of all se quences and the space of holomorphic functions in ]z| < 1)
% it can easily be shown that the topology cannot be defined by a single norm,
% Other types of spaces were not even metrizable; this was
% observed by von Neumann in 1929 for the weak topology on Hilbert space (1). But already in 1910 E.H. Moore had put forward the idea of replacing uniform convergence in R by what he called "relative uniform convergence'"; this amounts to consider neighborhoods of O defined in the following way: one considers continuous functions g in R such that g(x)>0 for all x€ R, and to each such function g, one associates a neighborhood Vg of O consisting of all functions f such that |f| < g3 Wwhen restricted to functions which are continuous and have compact support, these neighborhoods are exactly those defining what will later be called the (LF)-to pology on X(R) [163].
% After 1932, a new notion emerged, that of boundedness., It
% was already realized by Banach that on the same vector space, two norms ”X“l and ”x”2 such that the ratios ”x”l/Hx”2 and Hx“z/”xnl are bounded for x # 0, define the same topo
% logy and therefore, if one defined a bounded set in a normed space F as being contained in some ball, this was a notion independent of the particular norm chosen., However, in an ar bitrary metric space, two distances may give rise to the same

% LLOCALLY CONVEX SPACES AND THE THEORY OF DISTRIBUTIONS 217
% topology and give quite different notions of "bounded sets" when one sticks to the previous definition. But for arbitrary topological vector spaces (i.e. those for which addition and multiplication by a scalar are continuous), it turned out that it is possible to give a definition of bounded set which co incides with the previous one for normed spaces, and only de pends on the topology: a set A is bounded if, for any se quence (x_) of points of A, and any sequence (tn) of scalars tending to O, the sequence (tnxn) converges to O. An elementary argument shows that this definition is equivalent to the following one: for any neighborhood V of O, there is a scalar A > O such that }A c V, The first general result using this notion was the characterization of (Hausdorff) to
% pological vector spaces for which the topology can be defined by a norm, found by A. Kolmogoroff in 1935 [130]: they are those for which there exists a bounded neighborhood of O,
% Meanwhile, a new kind of topological vector spaces was in troduced in 1934 by Kb6the and Toeplitz [ 129]. For any vector subspace E of the space o (or CN) of all real (or complx) sequences, they consider the space E* of all sequences (un) in B (resp. N) such that I Iunxn| converges (nowadays n
% one says that E¥* is the K8the dual of E). One can then E** consider the space s, Wwhich obviously contains E; when E** _ E, Kothe and Toeplitz say E dis perfect ("Vollkommen"), and it is this kind of space which is mainly studied in their Paper, as well as in many subsequent papers of KOthe and his pupils ([128], [130]). On such a space E, they first define the weak topology o¢(E,E¥) in the same way as von Neumann (1),

% 218 CHAPTER VIITI
% neighborhoods of O Dbeing defined by a finite number of ine qualities ](x,a)| < 1 with a? € E* arbitrary. For that topology, they define bounded sets as sets AcCc E such that
% each function x> (x,a*) (a*¥ € E* arbitrary) is bounded in A, which is of course a special case of the general notion mentioned above, although they introduce it without any refe rence, But their next step is particularly interesting; as E and E¥* play symmetrical parts, one can also define the weak topology O(E*,E) and bounded sets in E¥; this enables them to define on E a new topology, the strong topology, where neighborhoods of O are defined in the following way: to each bounded set B in E*, one associates the set VB of all x € E such that |(x,y*)| < 1 for all y* ¢ B (the "polar set" of DB in a later terminology), and the Vg constitute a fundamental system of neighborhoods of O for the strong topology. One can then define bounded sets in E for that topology, and one of the chief results of K8the and Toeplitz is that bounded sets in E are the same for the weak and strong topology.
% All topological vector spaces mentioned above belonged to what we now call locally convex spaces, but the general defi nition of these spaces (under the name "convex spaces") was
% only given in 1935 by von Neumann, in view of a study of al most periodic functions [221, vol.II, p.508-527]. This co incided with a revival of interest in the properties of convex sets in topological vector spaces, which after Helly had been pretty much neglected: in Banach's book, they are only brie fly mentioned in a Note at the end of the book [15, p.246]. However, in 1933, S. Mazur gave the "geometric" version of the

% LOCALLY CONVEX SPACES AND THE THEORY OF DISTRIBUTIONS 219
% Hahn--Banach theorem, generalizing Minkowski's theory by show ing that, if K is an open convex set in a normed space E, there is a closed hyperplane of support of K through each boundary point of K [158]. A little later, M. Krein and
% D. Milman introduced the concept of extreme point for a con vex set K, 1i.e. a point of K such that there is no open line segment containing the point and contained in K; they proved the remarkable fact that there are always "enough" ex treme points for a compact convex set K, more precisely K is the smallest closed convex set containing all the extreme points [131]; a theorem which was to have many important ap plications in various domains of Functional Analysis.
% Once the locally convex spaces had been defined, the K&the Toeplitz procedure could be put in a more gcneral context: one starts with two vector spaces E, F and a bilinear form
% B on EXF, which is non degenerate, i.e. such that the re lation "B(x,y) = O for all y € F " is equivalent to x = O and "B(x,y) = O for all x € E " equivalent to v = O. One then considers on E all Hausdorff locally convex topologies for which F is the ‘dual of E; +the determination of these
% topologies was done by G. Mackey [154], who showed that one gets a fundamental system of neighborhoods of O for such a topology by taking the finite intersections of the "polar" sets of a family & of subsets of F, which consists of compact symmetric sets for the weak topology g(F,E) and form a covering of F3; 1in addition, Mackey showed that for all these topologies, the bounded sets in E are the same, We shall not try to describe in detail the very numerous

% 220 CHAPTER VIII
% papers devoted to topological vector spaces which have been published since 1950, Shortly after that date appeared the first comprehensive treatises on the subject ([26], [62], [92], [125], [128], [215]). Most researches have been devoted to the study of particular types of locally convex spaces, such as Fréchet spaces and their direct limits ([ 55], [92]), vari ous types of "vollkommen" sequence spaces in the sense of K8the-Toeplitz, which yield a rich harvest of examples and counterexamples, as well as many types of spaces consisting of functions with various properties. The mest significant recent results concern the various topologies which one can define on the tensor product E @ F of two locally convex spaces; they were studied in depth in a remarkable paper by A, Grothendieck, which deserves to be considered as realizing the greatest progress in Functional Analysis after the work of Banach [91]; this study led its author to the discovery of a new class of locally convex spaces, the nuclear spaces, which in a sense are much closer to finite dimensional spaces than even Hilbert spaces (with which they have some surprising connections) [ 59]. Most spaces occurring in the theory of distributions (3) are nuclear spaces, and nuclear Fréchet spaces have become quite important in the theory of probability. Finally one should mention a large literature on convex sets in topological vector spaces, taking its origin in a beautiful
% result of Choquet giving to the Krein-Milman theorem a quanti tative interpretation: 4if C is the convex hull of the union of {0} and a compact convex set K contained in a closed hyperplane not containing O, then every point of C 4is the

% LOCALLY CONVEX SPACES AND THE THEORY OF DISTRIBUTIONS 21
% barycenter of a positive measure carried by the extreme points of C, and this measure is unique if and only if the order relation defined by the cone ¢’, union of all the )C for A > 0, is a lattice [42]. This result has important appli cations in potential theory,

\section{The theory of distributions}
\label{sec:8.3}

% Between 1930 and 1940, several mathematicians began to in vestigate systematically the concept of "weak" solution of a linear partial differential equation, which we have seen ap
% pearing episodically (and without a name) in Poincaré's work (chap.III, 2). 1In general, let P: f+>T aaDgf be any dif o}
% ferential operator with c” coefficients, defined in an open set () C mn, and write (f,g) = (’ f(x)g(x)dx for f 1local
% ly integrable in () and g continuous with compact support in (3 then it is easy to generalize Lagrange's definition of the adjoint differential operator tP s, Which in parti (1) (P<f,g) = (f, Pe+g)
% cular satisfies
% t
% when f dis C° in Q, and g dis C° 4in 0 with compact support. If f dis a ¢® solution of Pef = 0O, we have therefore (f,tP°g) = 0 for all functions g which are C in () and have compact support. Conversely, any function f locally integrable in () and having that property is called a weak solution of the equation Peu = O, even if it is not dif ferentiable at all, and the problem which had confronted

% 222 CHAPTER VIIT
% Poincaré was to prove that a weak solution is in fact a ge L w L nuine C solution,
% But in fact the same problem, for the simplest differential d
% operator D = 9% ° had already been considered and solved in the affirmative by P, Du Bois-Reymond in 1879 [60]. Prodded
% by Weierstrass's criticism of the Calculus of wvariations (chap.II, 4), he undertook to prove that if a ¢l function y in an interval [ a,b] is an extremum for the integral I(y) = (/ F(x,y,y’)dx, where F is a ¢t function, then a
% d 3F SF (2) ax(3y7 (x57,5")) = So(x,y,y") =0 the Euler equation
% makes sense, which certainly is not obvious since nothing guarantees a priori that %%T(X,y,y') is differentiable!
% Following the classical procedure of Lagrange, one writes that for any ¢l function ( having compact support in Ja,b[, the function ¢ +» I(y+e() has an extremum for ¢ = O, which (3) (%g-g + Q' gg,)dx = O3
% is equivalent to the relation
% b
% a
% but instead of integrating by parts to eliminate (', one

% instead eliminates ( by using the fact that ((a)
% g(b) = O:

% an integration by parts enables indeed to write (3) in the (4) Q'(x)f(x)dx = 0O,
% form
% /b
% > b
% where f(x) =-g-f,l,(x,y(x),y' (x)) - -g%(t,y(t),y'(t))dt is
% LOCALLY CONVEX SPACES AND THE THEORY OF DISTRIBUTIONS 2273
% only known to be continuous in ]a,b[. The problem is to prove that f is a constant, for then it will follow that 257(x,y(x),y’(x)) is indeed differentiable and equation (2) is satisfied. However this amounts to showing that any weak : (*) Bois=-Reymond proves .
% solution of Du = O 1is a constant, which is exactly what du
% Such a result of course could not be expected for any dif ferential operator: for instance, if A and B are any lo 2 2
% cally integrable functions in R, the function (x,y)rA(x)+ 2
% + B(y) 4is a weak solution of gx;y = 0, <for one has ‘R R R R
% °0 g 8 g ( A(x)dx ( 333 dy = 0 and ( B(y)dy f/ 33y dx = 0 for any C2 function g with compact support.
% A step further would lead to defining a "generalized" ope rator P, acting on functions which were not supposcd difl Terentiable at all: for a locally integrable function f in 0, P.f would be (by definition) a locally integrable func tion h such that, for any ¢® function g in ) with (5) <h9g> =<f’ P.g>0
% compact support, one has
% t
% In a slightly different context, E. Cartan in 1922 [ 39] had observed that it was sometimes possible to define an "exterior derivative" dg for a differential 2-form y = Pdy A dz + + Qdz A dx + Rdx A dy, even if P, Q, R were merely contin uous but not necessarily differentiable; one would define dw = Sdx A dy A dz if S was a continuous function such that —————
% *
% ( )It is interesting to remark that in this paper du Bois Reymond uses (probably for the first time) what we now call "test functions", i.e. C functions with compact support.

% 224 CHAPTER VIII
% (6) S dxdydz = (fi (Pdy A dz + Qdz A dx + Rdx A dy)

% V
% 7/

% he gave the form for which . P = 3U3%’ Q _= 3yv R _3u = 37
% for any open set V with smooth boundary Y. As an example,
% where
% U dis the potential of a density p which is only supposed to be continuous; then P, Q, R need not be differentiable, but nevertheless S = =Ump satisfies (6).
% . . . (%) is to be found in a paper of J. Leray in 1934 [141] e In
% The first systematic introduction of such "generalized" operators, for P = o (under the name of "quasi-dérivées") X
% addition, Leray also introduces the process of regularization of a locally integrable function f by a sequence (pn) of ¢” functions with compact support tending to O, such that op 2 O and j’pn dx = 1: he shows that if f is continuous, pn*f is a €~ function which converges uniformly to f in
% every compact subset, and if h 1is continuous and is the "generalized derivative" of f, then pn*h is the (usual) derivative of pn*f (**).
% With our present knowledge, we realize that this notion of
% "generalized derivative" was a natural consequence of the use of the Lebesgue integral. Progressively, analysts had become
% (*) Leray's results were rediscovered independently by K. Friedrichs in 1939 [76].
% (**)The study of integrals p_*f for various types of sequen ces of functions (pn) was a favorite subject of analysts from Weierstrass to Lebesgue, under the name "singular integrals", For continuous functions pn with compact support shrinking to a point, it had been systematically used by H. Weyl on Lie
% groups, as a substitute for the missing unit element in LTM (G) ((227], vol.III, p.73).
% LOCALLY CONVEX SPACES AND THE THEORY OF DISTRIBUTIONS 225
% familiar with the idea that two measurable functions which co incided except in a set of measure O were not to be distin guished from one another in most operations of Analysis. Not so, however, for differentiation: il f is a ¢ function in [R and ¢ is the characteristic function of the set of rationals, f + ¢ 1is almost everywhere equal to f, but f+p has no derivative at any point, in the usual sense! Never
% theless it has of course a "generalized derivative" equal to f’, and this could throw doubts on the adequacy of the '"na tural" definition of a derivative in Analysis!
% It is easy to see that if a function f in R has a "ge neralized derivative" h which is locally integrable, then f dis almost everywhere equal to ( h(t)dt + ¢, where c
% X
% is a constant., This shows that a cgntinuous function may have almost everywhere a derivative in the usual sense, with out having a "generalized derivative", for instance an in creasing function f which is not absolutely continuous, another example of the inadequacy of the concept of derivati ve in the classical sense,
% A fortiori, this also shows that a function which is dis continuous at a point of R cannot have a "generalized deri vative", Nevertheless, following Dirac, theoretical physi
% cists did not hesitate to consider that the Heaviside function Y, equal to O for x< O and to 1 for x = O, had a "generalized derivative", the so-called "Dirac function" , which would have been equal to O for x # 0, but such that
% f.é(x)dx = 13 and they even introduced successive "deriva tives" ’,8” , \ldots of that "function", writing "equations" sSuch as

% 226 CHAPTER VIII
% (7) f g(x)s (") (x-a)ax TM)(a)
% (8) f~6’<a-x>a(n)(x-b>dx = 5"+1) (aop) [ 56]. for a Cn function g, or
% For some time, mathematicians were puzzled by such manipu
% lations, which eventually led to correct statements on genui ne functions. The decisive step was taken in 1936 by S. So bolev EZOO]: the outcome of these jugglings with non-existent
% "functions" was finally to define perfectly decent linear forms such as f +— f(n)(a) on the vector space (Q) of all ¢® functions with compact support defined in an open set N C RN; Sobolev's idea was therefore to deal directly with
% such linear forms, provided one could characterize them by properties involving only genuine mathematics. As he was led to this idea by a very concrete question, the solution of Cauchy's problem for second order hyperbolic equations with general boundary conditions (see chap.IX, 5), he could see
% what kind of properties he needed, and give a general charac terization of what he called "functionals" on ¥®(Q), which we now call (after L. Schwartz) distributions on Q: for each compact subset K€ (3 one considers the subspace 8(Q;K) of 9(0) consisting of all ¢” functions with support in K, ° (9) pm,K(f) = sup |Daf(x) b
% and this is a Fréchet space for the semi-norms
% |g|5m,xEK
% distributions are then the linear forms T on 8(Q), the restriction of which to each subspace 8(Q3;K) is continuous

% LOCALLY CONVEX SPACES AND THE THEORY OF DISTRIBUTIONS 227 for the preceding topology(*). Any locally integrable func tion F in ( defined a distribution fkfii( F(x)f(x)dx, two almost everywhere functions giving rise to thz same distribu tion (which of course were the measures on () having a den
% sity with respect to Lebesgue measure), so that the space LiOC(Q) of classes of locally integrable functions was iden tified with a subspace of the space 8’ () of all distribu tionsj; but more generally all Radon measures on () were par
% ticular distributions, and in particular one could define cor rectly the so-called "Dirac function" x+ = (x-a) as the mea sure e_: f+>f(a) defined by the mass +1 at the point a. Sobolev pointed out that one can multiply a distribution T by any ¢® function g 1in Q, by defining g+T as the dis tribution f+— T(gf); more important still, one may define the derivatives QI— of any distribution T as the distri- QX. bution fF*--T(%é} . Finally, he considered on the space 8’ () the weak topology o(®'(Q),8(Q)), and showed that the
% regularization process could also be applied to distributions: pn*T is defined as the distribution fh»'T(gn*f), which turns out to be the-class of a C function, and pn*T con=- verges weakly to T when mn tends to +o3; the fact that
% ©
% distributions are thus limits (for the weak topology) of C
% functions has led some mathematicians to call them "generalized
% (*)Sobolev does not speak of topology, but defines convergent sequences in #(N) which correspond to these topologies on the spaces ®(Q;K). Another way of expressing the definition is to
% consider the "direct limit" of the topologies of the spaces 9(Q3;K); distribtuions are then the elements of the dual of @(Q) when 8(Q) is given that topology.

% 228 CHAPTER VIIX
% functions" [ 86].
% During the same period, the need to "enlarge" in some way
% the domain of definition of operators other than differential operators was also felt in different parts of Analysis(*), and particularly in classical harmonic Analysis. The definition of the Fourier transform Jf of a function f defined in R" only makes sense when f € Ll(Rn); however, as soon as 1910, the Plancherel theorem (chap.VII, 6) showed that it is pos
% sible to define the operator f+>3Ff as an isometry of the Hilbert space LZ(RH) onto itself, by extending it by contin uity from its original domain of definition Ll N L2; in other words, for a function f € L2 which did not belong to Ll, the Fourier transform Ff could still be defined, but only by a limit process, Later, efforts were made to define similarly a Fourier transform Ff for functions f{ belong ing to other spaces Lp; in his discussion of that problem, A, Weil observed in 1940 [226, p.118] that if A is the space of functions f € Ll n L such that df also belongs to (10) Jm)-é‘f“(x‘)dx - jmx)-f‘(?)dx
% 11 n 1°, then if two functions &, o are such that
% for all functions f € A, it is legitimate to consider that ¢ dis the Fourier transform of (.
% (*)For instance, in the Calculus of variations, one may con sider that a smooth p-dimensional variety V in an Rn defines a linear "functional" g w 1in the vector space of dif ferential p-forms on Rn. This leads to the idea of "gene ralized varieties" [231] and of "currents" [49],

% LOCALLY CONVEX SPACES AND THE THEORY OF DISTRIBUTIONS 229
% Much earlier, in 1911, H. Weyl, in relation with his work on second order linear differential equations (chap.VII, 6), had observed that if f is such that f(x)/(1+]|x|) 4is inte
% grable in R (for instance, if xf(x) is bounded) and sa tisfies an additional regularity condition, then it is pos sible to write for f +the Fourier inversion formula, provid ed one replaces Jf by a Stieltjes measure ([227],vol.I, p. 359=360); +this amounts to defining the Fourier transform of a bounded Stieltjes measure, a definition which was expli citly given by P, Daniell in 1920 [46], and which became la ter a favorite tool of probabilists. Weyl's idea was devel
% oped by Hahn [ 96] and N, Wiener [ 229], and then extended by S. Bochner to functions such that f(x)/(l+|x|k) is integrable, where k dis an arbitrary integer [24]. Using the fact that
% by Fourier transformation derivation becomes multiplication by x (up to a constant), Bochner proceeds as Riemann had done for trigonometric series [182, p.245]: in order to obtain an integrable function, he substracts from e-ng a m-1 function Lk(xg) equal to the first k terms z '%T -ingn of the power series expansion of e-'lxg in a compact neigh borhood of O, and to O outside, and writes
% e 1xE (11) E(E,k) =-; f(x) - Le(x8) ax T

% e
% s v )k (=ix)

% This is of course only defined up to a polynomial in  of degree <k-13; Bochner's idea would be to take a "derivative" in some sense of E(E,k) as "Fourier transform" of f, and indeed he writes "symbolicallyTM"

% 230 CHPATER VIII
% what would be the "inversion formula"; but as he has no defi
% nition of such a "k-th derivative" at his disposal, he is compelled to work only with the functions E(,k) in the applications he gives to difference equations.
% It was one of the main contributions of L. Schwartz that he saw, in 1945 [194], that the concept of distribution introduced by Sobolev (which he had rediscovered independently) could
% give a satisfactory generalization of the Fourier transform including all the preceding ones. Instead of considering the space A dintroduced by A, Weil, which is not easy to describe
% explicitly, he had the idea to take as "test functions" the ¢® functions f in Rn which are such that f and all its
% derivatives are "rapidly decreasing at infinity", i.e. such that their product with any polynomial is integrable. The essential property of that space S(Rn) of "declining func tions" is that the Fourier operator f+—JFf 1is a bijection of 8(R") with the Fréchet topology defined by the semi-norms (12) a, (£) = sup (1+]x])TM |DF£(x) ? lo |5 s, x€RD It is easy to see that for each compact subset K of Rn, the space S(R";K) is contained in 8 (R") and the injection @(Rn;K) -» S(Rp) is continuous; furthermore the union ®(R") of all the Q(Rn;K) is dense in S(Rn). The continuous 1i near forms on S(Rn) can thus be considered as special dis
% tributions, which Schwartz calls tempered distributions, The Fourier transform T ++—3FT 1in the space S'(Rn) of tem pered distributions (dual of (R")) is then defined (by a generalization of equation (10)) as the transposed automor-

% LOCALLY CONVEX SPACES AND THE THEORY OF DISTRIBUTIONS 231
% phism of the Fourier transform in (R"), in other words the Fourier transform T of a tempered distribution T is defined by the relation
% (13) (FT,f) = (T,5f)y for all F£ € (R").
% To the credit of L. Schwartz must be added his persistent efforts to weld all the previous ideas into a unified and complete theory, which he enriched by many definitions and results (such as those concerning the tensor product and the convolution of distributions) in his now classical treatise [194] By his own research and those of his numerous students, he began to explore the potentialities of distributions, and gradually succeeded in convincing the world of analysts that this new concept should become central in all linear problems off Analysis, due to the greater freedom and generality it al
% lowed in the fundamental operations of Calculus, doing away with a great many unnecessary restrictions and pathology (*). One should reserve a particular mention to what is probably
% the most original of his contributions, the "kernel theorem" [195]. Ever since.Hilbert's and F, Riesz's work, it had been
% *
% ( )The role of Schwartz in the theory of distributions 1is very similar to the one played by Newton and Leibniz in the history of Calculus: contrary to popular belief, they of course did not invent it, for derivation and integration were practiced by men such as Cavalieri, Fermat and Roberval when Newton and Leibniz were mere schoolboys. But they were able
% to systematize the algorithms and notations of Calculus in such a way that it became the versatile and powerful tool which we know, whereas before them it could only be handled via complicated arguments and diagrams (see [28]).

% 232 CHAPTER VIIT
% realized that invegral operators f+—K f defined by a "kernel function" K(x,y), as (K-f)(x) = /,K(x,y)f(y)dy,
% were very far from exhausting the general concept of linear operator, since not even the identity could be expressed in that manner! It is therefore very remarkable that il one replaces "kernel functions" by "kernel distributions" in that
% definition, one practically obtains all linear operators which one meets in problems of Analysis. More precisely, if X C PW and Yc R are open sets, any linear mapping K of @(X) into ®°(Y), which is only supposed to yield partial contin uous mappings 8(X;L) » 8’ (Y) for any compact subset L c X (when 9(X;K) is given its Fréchet topology and 8’ (Y) the weak topology), can be defined by a uniquely determined "ker=- nel distribution" K € 8’ (XxY), in such a way that for any u € 9(X), the distribution K °*u satisfies
% (14) ' (K+u,v) = (K,u® v)
% for any function v € 9(Y), The great interest of this result lies in the fact that most spaces E of functions defined in X are such that 8(X) c Ec 8’(X), the injections 9(X;L) =+ + E and E = 9’ (X) being continuous; if A (resp. B) is
% such a space of functions defined in X, and U: A + B a continuous linear mapping, the composed map 9 (X;L) = A%B o + 9’ (X) 4is continuous, hence is defined by a "kernel distri bution". For instance, the identity map A44 A is defined by the distribution I which is a measure carried by the diagonal A in XXX and is such that X
% ( w(x,y)dI(x,y) = jr w(x,x)dx.,
% XXX X

\chapter{Applications of functional analysis to differential and partial differential equations}
\label{ch:9}
\setcounter{equation}{0}

% I will not try to enumerate all the applications which have been made of Functional Analysis in the last 50 years, and which have amply Jjustified the creators of that discipline. But as we have seen in the first 4 chapters how most notions and problems of Functional Analysis had their origin in ques tions relative to ordinary or partial differential equations, I think it is worthwhile to give a sketchy description of a few of the most conspicuous progress in those questions which
% have been made by an imaginative use of the new tools provided by Functional Analysis, mostly spectral theory and the theory of distributions.

\section{Fixed point theorems}
\label{sec:9.1}

% In the first applications which we shall mention, however, little more is used of Banach spaces beyond their definition, and the results primarily concern non linear equations. The main idea is similar to the application of the contraction principle (chap.VI, 3) to the local existence theorems for differential equations, by writing them in the form 2z = F(z) for =z 4in some Banach space E of functionsy; if B is the closure of an open bounded convex set in E and F is a con
% 233

% 234 CHAPTER IX
% traction mapping B dinto itself, then the contraction prin ciple says there exists a unique solution of 2z = F(z) in B, what one calls a fixed point for F., But after the (irst years of the XXth century, new possibilities of obtaining "fixed point theorems" appeared with the first results of a new branch of mathematics, Algebraic Topology, created by H. Poincaré in 1895-1900: wusing the concepts of that theory, L.E.J.Brouwer could show in 1910 that if B is homeomorphic to a closed ball in some finite dimensional vector space E, and F is any continuous map of B into itself (which is not supposed any more to be a contraction), then F has at least one fixed point in B.
% The problem was to find a similar theorem applicable to in finite dimensional Banach spaces K. The first result in that
% direction was obtained in 1922 by G.D. Birkhoflf and O. Kellogg, who considered the case in which E = C(I) or E = LZ, and
% showed that Brouwer's theorem could be extended, provided one took for B a compact convex set [22]; their fundamental de vice consists in using the compactness of B to "approximate"
% it by a finite dimensional compact convex set Bn’ and to si milarly "approximate" F by a continuous mapping Fn of Bn into itself, to which Brouwer's theorem may be applied. This method was taken up and greatly expanded by J. Schauder, who showed that it could be applied to any Banach space E, and also, for separable Banach spaces, that one could replace com pactness of B and continuity of F by weak compactness and weak continuity ([186], [187]). This enabled him to prove, for instance, existence of a solution of the equation

% APPLICATIONS OF FUNCTIONAL ANALYSIS 235 aZ aZ) (l) Az = f(x9y,zy 3%’ 3y
% in a domain C R? with smooth boundary (no connected com ponent of which is reduced to a point), vanishing at the boundary, under the only assumption that f is bounded and continuous for bounded wvalues of the 5 variables on which it depends; the method consists in transforming (1) into an in
% tegro-differential equation
% oz (2) Z(X’Y) = G(X,Y,E,n)f(g,fl,Z(g,n), gg: ai)dg dn 0
% where G dis the Green function for Q.
% A little later, a much more sophisticated approach enabled
% Schauder to solve Cauchy's problem locally for quasi-linear hyperbolic equations( )
% *
% 0z 9z d =z 2 A. (X 9 o 00 X Z '-—,ooo,---—) =
% 2
% i,k ik 71 *n? ’axl axn axiaxk
% (3)4 4

% —_ A(X l,
% Z ’axl’  \ldots"—). axn X n’

% o o o ’
% His method consists, for a given function z(xl, \ldots,xn), to solve the Cauchy problem for the linear hyperbolic equation in the unknown function Z
% () 3z 3z
% 7z d2z 3 Z
% 2
% ik ik 7 1? *n ’axl X axiaxk
% 2 A. (X e e e 49X s Z __,ooo,_—‘)—""‘“'—"‘" =
% 1 n
% = A(Xl, \ldots,xn,z,-a-;(—, \ldots,—a?).
% (*)
% The equation is supposed to be '"normal", which means that
% the left hand side is such that the quadratic form i?k Aikgigk has signature (1,n-1).

% 236 CHAPTER IX
% When 2z 1is in a suitable set K, the problem has a unique solution Z(z), and existence of a solution 2z of (3) in K will be obtained if one shows that the equation Z(z) = z has a solution in XK. The problem consists in choosing K such that the extension of Brouwer's fixed point theorem is appli cable; the main point is to obtain a priori inequalities for solutions of "mormal" linear hyperbolic equations
% L(u) = = Ak(xl’-°°’xn)mf‘+Bj(xl’“" )oX o i,k
% (5)
% + C(Xl,ooo,xn)u = F(Xl, \ldots,Xn)
% defined in a truncated pyramid P having its larger base B in Rn-l. Using a method first introduced in 1926 by
% Friedrichs and H. Lewy, which consists in transforming the integral S‘ g; L(u)dy by integration by parts and Stokes' P n
% formula, Schauder obtains (for suitable restrictions on P

% and the coefficients Aik’
% Bj and C in (5)) an inequality

% ((%EI)Z toeot (%%—)Z)dm <
% < M( (u2 + (gzz) tee ot (ax ) )do + {/ P dw)
% (6)2
% ws) ‘P
% with a constant M dindependent of u; by differentiation he gets similar inequalities for all derivatives (of any order)
% of wu., The space E is then defined by the norm sup ( % |Daf(x)|) for a suitable value of r, but the b (S |a|sr
% set K 4is a ball in E for another norm, namely (( ( z |Do‘f(x)|2)dx)l/2 for another value of s3; his
% APPLICATIONS OF FUNCTIONAL ANALYSIS 237 . (*) : : is compact in E and z+=2Z(z) continuous in K [190].
% a priori inequalities enable then Schauder to show that K . o o - L4 n
% Another of the famous theorems proved by Brouwer was the invariance of domain: if () is an open subset of R , and F an injective continuous map of () dinto Rn, then the image F(Q) is again an open subset. In 1929, Schauder
% showed that the theorem was still true in some types of Banach spaces, for maps F of the type Xha-X+H(X), where H is
% completely continuous in the sense of F. Riesz, but not neces
% sarily linear. From this he deduced for instance that if one (7) AZ-f(X,y,Z,g—E-, %“}E,) = W(X,Y)
% knows that the equation
% has at most a solution taking given values @(s) at the boundary of () C Rz, then, if for given functions P wo
% there exists such a solution, the same is true for functions ®,  sufficiently close to o_, ¥ [189].
% O
% But the most sophisticated application of Algebraic Topology to functional equations was made in the famous 1934 paper by Je Leray and J. Schauder [142]. If U is an open set in Rn, such that U is compact, and f is a continuous mapping of U in Rn, then, for each point 2z € R" which does not be long to the image by f of the boundary of U, Brouwer had shown that one may attach an integer d(f,U,z) which only depends on the connected component of R" - f(Fr(U)) to which z belongs, and varies continuously with XA when f(x) = F(x,}\)
% (*) This is probably the first time the norm of the space " makes its appearance.

% 238 CHAPTER IX
% where F 1s continuous and ) a real parameter; furthermore, when d(f,U,z) #£ O, the inverse image f-l(z) is not empty.
% Leray and Schauder were able, by approximating compact sub sets of a Banach space by finite dimensional subsets, to prove, by application of this theorem of Brouwer, the following exis tence theorem, Let E Dbe a Banach space, IC R a compact inverval, () € EXI a bounded open set in ExI, F: Q-+ E a
% mapping which is completely continuous, and in addition uni formly continuous. One assumes that for every )\ € I, there are no solution of the equation x-F(x,)) = O in the boundary of 2, and that, for one value xo € I, the eqgquation x-F(x,ko) = 0 has exactly one solution in Q3 then, for every A € I, there exists at least one solution of x-F(x,))=
% The application of that theorem to partial differential equations usually necessitates subtle a priori inequalities which guarantee that all the assumptions of the theorem are satisfied.

\section{Carleman operators and generalized eigenvectors}
\label{sec:9.2}

% For applications to partial differential equations, it is
% necessary to generalize the notion of Carleman operators de fined in chap.V, 3. Let X ©be a locally compact metrizable and separable space, | a positive measure on X and H a separable Hilbert space (finite dimensional or not); let (an) be a Hilbert basis of H, where J is a finite or ncJ
% denumerable set., One defines a new Hilbert space L;(X,u) as the space of vector valued functions f: x+— I f_(x)a_, ncJg M n

% APPLICATIONS OF FUNCTIONAIL ANALYSIS 239 *( = £ |, L.“(X,u) with complex values, and that |[f is 2 2
% mappingsof X idinto H such that each fn is a function of

% (8) (flg) = I ( £ 8 du .
% ncJ
% n

% u-integrable; the scalar product in that Hilbert space is then defined by
% ncd /x
% Let now Y be another locally compact metrizable and sepa . . . J
% rable space, V a positive measure on Y, A Carleman kernel on XXY (for the measure |y @ vV and the Hilbert space H) is then a mapping K: (x,y)fifi>(Kn(x,y)) : of XxY into € ncJ
% such that:
% 12 each complex function Kn is (u®v)-measurable; 2¢ +there is a null set N c Y such that, for each y & N, the function XF*-Kn(X,Y) is yy=-measurable and the function x— £ |K (x,y)|2 is y-integrable.

% ncJ
% n

% If f£f = ¥ f_a is a function of LZ(X,u), the function ne J n n H x+— ¥ K (x,y)f (x) 4is y=-integrable for all y £ N, and
% ne J n n
% the function
% (9) e(y) = | T K (x,7)f_ (x)du(x) neJ
% defined in Y-N, is v-measurable., One writes K-+f = g, and K, defined in LH(X,u) is called the Carleman operator de
% 2
% fined by the Carleman kernel K.
% In 1952, F. Mautner discovered that Carleman operators can be characterized by properties which are independent of the definition by a kernel [157]: suppose that there is a null Set Nc Y and, for each 7y € Y-=N, a continuous linear map

% 240 CHAPTER IX
% P Lfi(x,u) 4 ¢ such that for any function f € Lfi(X,u), the map yke»Fy(f) is v-measurable, Then there is a Carleman kernel K = (K) and a null set N'D N such that, for any y ¢ N/, Fy(f) = (K+f)(y) for all functions f € Lfi(X,u).
% We have seen (chap.VII, 5) that a continuous normal opera
% tor has a "diagonalization" which transforms it into multi plication by the function "identity" (+= (. More gene rally, one defines a diagonalization of a continuous normal operatorB ina Hilbert space E for a function (¢ +— 3(() as in chap.VII, 5, by replacing Sp(B) by a separable lo cally compact space Y, the function (+~=( Dbeing replaced by a mapping (+= 8(¢) of Y into (.
% With the help of his characterization of Carleman operators, Mautner was able to get a much more precise description of such a diagonalization when the operator B is defined in a Hilbert space E = L7(X,u), and is a Carleman operator cor responding to a Carleman kernel (x,y) K(x,y) defined in XXX, wrelative to the measure Q®y; 1in addition one assumes that for the isometry T = (T.) defining the diagonali- J7 1< j<uw zation for the function &, one has &(() £ O almost every where in Y (for the measure V), which is equivalent to assuming that B and B* are injective, It is then possible
% to describe the isometries
% T: E, + F, = LZ(Y,mS V)
% J
% performing that diagonalization, in the following way: 12 For each index j such that 1 < j< w, there is a (U®V) -measurable function (x,() ej(x,g) such that

% APPLTICATIONS OF FUNCTIONAL ANALYSIS 241
% ej(x,g) =0 for (¢ £ S5 and that for almost all x € X,
% the function
% c—13(0)1° = e (x0)|
% 1< j<w
% is v=integrable,
% 2¢ Let r(x) = ( { |K(x,y)|2 du(y))l/z, which is gy-measu rable and almost ever?where finite; then, for every function (10) (T,-£)(C) = ( f£(x)e (x,0)du(x)
% f € E; such that I/ r(x)|£f(x)]| du(x) < +o one has
% *
% X
% X
% for almost every ( € Y.
% 32 There is a null set N C X having the following pro perty: for every jJ such that 1 s j< W, let uj be a function belonging to Fj’ and suppose that the function -2 2
% ¢ = |2 (¢)] z Jus(e)]
% 1< 3<w
% is v-integrable., Then, for every x & N, the function (— Z ej(x,g)uj(g) is veintegrable, and
% 1< j<w
% (11) (7))(%) = f (T e (x0)uy(0))av(c).
% J 1< j<w
% Y
% The nature of this result is better understood when one
% sSpecializes it to a situation stemming from harmonic Analysis. Take X = G, a separable commutative locally compact group;
% let |4 be a Haar measure on G, and consider a complex func tion b € Ll(G) N LZ(G); then B: fi+» bxf 1is a continuous normal operator in LZ(G), such that B¥.f = B*f; further
% <
% more, from the definition

% 242 CHAPTER IX
% ,
% (bxf)(x) = b(x-y)f(y)du(y)
% G
% it follows that B 4is a Carleman operator corresponding to the Carleman kernel K(x,y) = b(x-y). The Plancherel theorem
% and the multiplicative property of the Fourier transform (chap.VII, formula (44)) show that the isometry T: f+— JFf defines a diagonalization of the operator B, with Y = é, w =2, &(C) = 3b(g), and Vv a Haar measure on G. Formulas G G
% (10) and (11) then boil down to
% (12) (T-£)(¢)= | £(x)e(x,0)au(x), (T eu)(x) =fe@@fifl3®© with e(x,¢) = (x,() for x€ G and ¢ € G, i.e. the defi
% nitions of the: Fourier transform and of its inversej; these formulas are only valid for f € Ll(G) N L°(G) and
% 2
% u € Ll(a) N Lz(a), which shows that the restrictions imposed on the functions f and wu, in (10) and (11) cannot be com pletely suppressed. Finally, for every ( € é, one has
% (13) bxe(+,0) = 8(C)e(*,C)
% and although the functions e(*,() do not belong to LZ(G) in general, they are in some sense "generalized eigenvectors" for the operator B, exhibiting the same phenomenon already observed by F., Riesz (chap.VII, 2).
% In 1953, it was simultaneously observed by L. Garding [ 81] and F, Browder [34] that this phenomenon of "generalized ei genvectors" occurs for all self-adjoint operators stemming
% from formally self-adjoint elliptic differential operators (see 5). It is assumed that such an operator P or order m . 2
% in L%(X) (where X is an open bounded subset of R") pos-
% APPLICATIONS OF FUNCTIONAL ANALYSIS 243
% sesses a self-adjoint extension AP; then [ = AP+ il 2 -
% (I didentity) is a normal unbounded operator, which is a bi jection of dom(AP) onto L7(X); the inverse [ 1 s 2
% therefore a continuous normal operator in L (X), and the same is true of course of its iterates B = L-q. It follows from the existence of a parametrix of P (see 5) that for gm > n, B 1is a Carleman operator; it then also follows from the hypoellipticity of P (see 5) and from Mautner's theorem that there is a diagonalization of A, with Y = Sp(AZ)) and (¢) = ¢, for which (with the preceding notations) each function ej(°,g), for ¢ ¢ Sp( P)’ is a €~ function (ge nerally not in LZ(X)) solution of the partial differential (14) (P'ej(O,g))(x) = er(x9€)°
% equation

\section{Boundary problems for ordinary differential equations}
\label{sec:9.3}

% The results of H. Weyl on the spectral theory of second or der linear differential equations (chap.VII, 3) naturally raised the question of their generalization to linear differ
% ential equations of arbitrary order, but that problem was only attacked by K., Kodaira in 1949 [126]. Surprizingly enough, although Stone had shown in his book [207] how von Neumann's spectral theory could be applied to yield H. Weyl's rcsults, Kodaira elected to follow Weyl's method, suitably extended. Simultaneous work by Glazman and Neumark, and later papers by many authors completed Kodaira's results and also inserted them within von Neumann's theory; we refer the reader to
% 244 CHAPTER IX
% [62, vol.ITI, p.1588-1592] for more historical details, and we will only describe the main features of the theory.
% One considers a differential operator of even order 2r -1 -1 (15) L: u—D"(pD'u) + D" " (p,D "u) + \ldots+ D(p__,Du) + p_u 1 where po,pl, \ldots,pr are real COo functions in an open inter val J = Ja,8[ (bounded ornot) of R, and po(t) #Z 0 for all te€ J; it is formally self-adjoint, i.e. for any two functions u, v of 8(J) (space of ¢~ functions in J (16) (Leu|v) = (u|L+v)
% with compact support)
% the scalar product being taken in L7 (J). We write TL the
% 2
% operator L , considered as a hermitian (unbounded) operator in LZ(J) with dom(TL) = 9(J). Then the adjoint T;' is densely definedj; more precisely, dom(T;) = HL is the space of all functions u of class C~'~1 in J such that the distribution [ +u is a function of L-(J), with T°u = + -
% = L eu., The von Neumann spectral theory (chap.VII, 4) shows + and E- . . * ¥
% that HL is the direct sum of dom(TL ), E. X where EL is the subspace of functions u of class CZr 1 which are solutions of L eu = £ iu and in addition are square integrable in J; 1in fact, they are of class ¢”. Due to the fact that the pj are real functions, both spaces +
% EL and EE have the same dimension p < 2r. The dual M +
% of ELD EE can be identified with the space of linear forms on H._ which vanish on dom(Tz*); it is the direct sum mg<9<m8, where ‘W} (resp.?mB) is the subspace of M con sisting of all forms 6 such that 0(u) = O for all func-

% APPLICATIONS OF FUNCTIONAL ANALYSIS 245
% tions u € HL which vanish in a neighborhood of a (resp. B). If one writes the Lagrange "adjunction" formula (chap.I, 1, formula (5))
% v(zeu) - u(Lev) = 5 (C(u,v))
% for functions u, v which are c® in J, then it can be shown that any linear form 8 € md (resp.me) can be written t-0 t
% 6 (u) = 1im C(u,w)(t) (resp. 6(u) = 1im C(u,w)(t))
% for a C° function w in J for which the limit exists for every u &€ HL, and conversely any such function defines a linear form in md (resp.‘mB). Due to this result, one says that for any 8 € M, the relation #(u) = 0 is a boundary condition for the differential operator L.
% As the defects of ff;*' are both equal to p, there exist self-ad joint extensions AL of T (infinitely many if p > 0); L ( for each of them, dom(AL) is a subspace of HL of codimen
% sion p, defined by p independent "limit conditions" Gj(u) = O with ej €M (the linear forms ej are not arbi trary, since one must have (T;ou|v) = (ulTZov) for u and v in dom(AL)). For a given self-adjoint extension AL, let S be its spectrum, which is a closed subset of R (it
% may be the whole line R, and it is always infinite and un bounded). For any ( € S, (AL-QI)-l is a continuous normal operator in LZ(J); one shows that it is a Carleman operator, with kernel (s,t)~— G((,s,t), called the Green function of (16) G(¢,t,s) = G(Chs,t).
% AL- Cr 3 one has

% 246 CHAPTER IX
% For each ( ¢ S and t € J, the function s> G({,t,s) is 2r-1 C function, and in each interval Ja,t[, Jt,8[, it is a € solution of the equation Leu - Cu = 0, satisfying the boundary conditions which define 4 , and in addition Ar=1 L
% the function si— é_f?:f G(C,t,s) has limits on the left and S
% on the right at the point s = t, such that
% a2r-l a2r-l
% 9SS oS
% (17) —57o7 6(C,t,t-) - =555 G(C,t,t+4) = 1/p_(t).
% These conditions obviously generalize those seen in chap.VII, 3 for second order equations; they completely determine G once a fundamental system of 2r solutions tr~>vj(g,t) of L*u = (u =0 is known, and it is easily seen that one can
% write in matrix notation
% [G(g,s,t) 3(5,8)* w-(g)g(g,t) for t < s

% (18) <
% G(C,s,t) 3(5,5)* W+(€)3(C,t) for t > s

% where ,3(g,t) is the one column matrix (v .(C,t)) )
% J 1< j<2r
% W-(Q) and w*(g) are two square matrices of order 2r which only depend on (. Furthermore, if the Vj have been chosen so that for each t € J, the functions QF+-vj(g,t) are holomorphic in an open subset H C C (l < J s Zr), then W and Wt are holomorphic in H N (C—S).
% As the operator[ is elliptic and formally self-adjoint, one can apply to it the Garding-Browder theorem (2). It can be shown that, with the notations introduced in 2, one has w < 2r+l, 1in other words, AI} has at most multiplicity 2r in its spectrum S; for convenience, if W < 2r+l, one de fines the function ej(t,g) to be identically O for
% APPLICATIONS OF FUNCTIONAL ANALYSIS 247
% m< j< 2r, t € J and & € S; for the other wvalues of j, (t,g)F-ej(t,) is a (A®v )-measurable function in JxS, such that for each & € S, tka—ej(t,g) is a solution of the equation Leu - Eu = 0 and for almost all t € J, the func tion Fa-ej(t,) is square integrable (for v) in each compact subset of S3; in addition, one has ej(t,g) = O for £ o S, For  € SC R, onemay write (with 2(t,g) = (19) 3(t,8) = @(8)V(g,t)
% = (ej(tag)) 1< icop? OR€ column matrix)
% < JSITr
% and the elements of the matrix @ are -measurable and (20) P(8) = a(8)" a®)
% square integrable in every compact subset of S, Let
% which is a positive hermitian matrix for all & € S. The spectral decomposition of the operator (AL-QI)-l can then be written explicitlyin the following way: for any function fe 9(J), write

% (21) (Ue£)(c) = (| £(t)v,(c,Dar)
% (one column

% J
% then, for f, g in 8(J), one has
% 1< j<2r
% matrix);

% (22) (fle) = ff ((u-r)(E)N)TM PE)((U-e) (B))dv(E) S
% and
% (23) ((4, -¢cr) TM +fle)= | G-0) 7 ((U-£)())*PE) ((v-&)ENAV (3)
% S
% The set Sj-S.+ is then the set of & € S such that the J+1 matrix P(£) has rank equal to j, and the (vector) measure Pey can be recovered from the knowledge of the matrix W

% 248 CHPATER IX
% (formula (18)) by the relation
% g P(8)av(g) - 3 (P(a)v({a}) + P(b)v({b})) =

% [a,b]
% (24)
% b
% - gi— 1im [r (W (c+ie) - W (o-ie))do.

% TT
% e=+0 Ja

% These results had previously been obtained for second order operators by Titchmarsh [212],
% Much work has been done to determine the spectrum S, the various subsets Sj’ and the measure Py under various hypotheses on the operator L ([62], [166]). It should how ever be stressed that the behavior of the measure Pv on R is essentially arbitrary: 1in a remarkable paper, Gelfand and Levitan have shown in 1951 [ 84] that, given on an arbitrary compact subset H of R an arbitrary measure o, it is always possible to find a second order operator [ with c”
% coefficients such that P() has the form and
% p,,(8) O
% O )
% the restriction of the measure pll y to H dis the given measure P

\section[Sobolev spaces and a priori inequalities]{Sobolev spaces and \emph{a priori} inequalities}
\label{sec:9.4}

% Until 1940, there was no general theory of lincear partial differential equations(or systems of such equations) of arbi trary order. With the exception of a few special types of
% equations with constant coefficients (such as the "biharmo nic" equation A%u = 0), the bulk of papers were concerned
% with second order equations in any number of variables, to which must be added a much smaller number of results on
% APPLICATIONS OF FUNCTIONAL ANALYSIS 249
% equations of arbitrary order in two independent variables. When mathematicians began to be interested in "weak" solu tions, and later with the arrival of the theory of distribu
% tions (chap. VIII), the scope of the theory of linear partial (25) u+—= Peu = z a D*u
% differential equations was greatly widened; if
% lo|gm &
% . (*) . n . .
% is a linear differential operator with complex C° coeffi- cients in an open subset (2 of R, then for any distri bution T € 8’ (), each product %ngT is defined, hence
% also P*T, and it makes sense to ask for solutions T of (26) P*T = S
% the equation
% where S is any given distribution in ’(Q). In particular, one may take for S a ¢” function in 2, and then one asks if this imposes conditions on the distributions T solutions of (26). 1In some cases, solutions of P+T = O may be distri
% butions of arbitrary order (i.e. as "irregular" as possible); this happens for instance for P = where not only ar
% 22
% 3xdy ’
% bitrary locally integrable functions A(x) + B(y) are solu tions, but also arbitrary distributiomsof type A®1 + 18B, where A and B are arbitrary distributions of 8’ (R). On the other hand, it may happen that for all ¢” functions f
% (*)The interest shown to equations with ¢® coefficients (or ¢’ coefficients generally) is chiefly due to the pioneering efforts of Hadamard, who repeatedly emphasized that for appli cations to Physics it was unreasonable to study exclusively equation with analytic coefficients [93].

% 250 CHAPTER IX
% in Q, all solutions of P+T = f are necessarily ¢” func
% tionsj; such operators P are now called hypoelliptic. This is the case, for instance, when n =1 and P = P 4 ale-l + + oo et ap is any linear differential operator with leading coefficient 1, an elementary result which follows by inductim from the case p = 1, which is du Bois-Reymond's lemma (chap. VITI, 3). 1In 1927, S. Zaremba [ 233] proved a result which, in modern language, means that the laplacian A is hypo elliptic, and H. Weyl in 1940 gave another proof of that result ([227], vol.III, p.758-791).
% After 1950, such questions, as well as extensions of the classical boundary problems, began to be studied for operators (25) of arbitrary order, heralding a period of unprecedented expansion in tlie theory of partial differential equations. Among the many methods developed during that period, we shall postpone to 5 those linked to the concepts of elementary so lution and parametrix, and consider here the applications of the "a priori inequalities" which were made possible by the appearance of new tools linked to the theory of distributions, the Sobolev spaces and their generalizations.
% We have already seen (1) that Schauder had considered the space of functions f of class Cp in such that all de rivatives D*f for |a| < p belong to LZ(Q), and had used
% the norm |f||_ = (I’( X IDbg'f(x)lz)dx)l/2 on that space, b
% 0 1&|=p
% which unfortunately was not complete for that norm, 1In 1936, Sobolev had the idea of considering the functions f € 1?(0) which have weak (= distributional) derivatives D"f belong
% &
% ing also to LZ(Q) for |a| < p, and this time this space

% APPLICATIONS OF ‘'FUNCTIONAL ANALYSIS 251
% (with the same norm) HP(Q) is complete (i.e. a Hilbert space) moreover, Sobolev observed that the larger the number p, the more regular are the functions of Hp(Q); for p > Eg] + 1, they are functions of class C with r = p - [g] - 1, hence the intersection of all Hp(Q) consists of the functions of class C~ such that all their derivatives are in LZ(Q) [201]. Later, it was realized that the H°(R") could be defined
% using the Fourier transform, as the space of distributions T € S'(Rn) such that the Fourier transform JT dis a locally
% integrable function for which the function
% g (1 + ||2)S |3T()|2 is integrable. It is then clear that the definition may be extended to all real numbers s,
% These properties were the source of what one may call the "bootstrap" method to prove that a function is c’: it is enough to show that if it belongs to some Sobolev space 1 (q), it also belongs to H*+1(Q). The first idea of this method is apparently due to K. Friedrichs [77], who applied it to prove that elliptic operators (25) (see 5) are hypoelliptic, a question which we postpone until 5; his tool is a new
% type of a priori inequality, which was the starting point of a very large number of similar results, for elliptic(*) and other types of operators. We shall only mention here one of the most refined ones [117, p.207]; it concerns what are cal led "principally normal" operators P, which we shall not attempt to describe more precisely here, but which include operators with real coefficients and operators with constant
% *
% ( )For a description of the various methods based on a priori inequalities in 1956, see [168].

% 252 CHAPTER IX
% coefficients; under assumptions on ()} 1in relation with the characteristic hyperplanes of the operator P, too compli=- cated to reproduce here, the fundamental result is that if u is a distribution on ()} having a compact support K, and such that Pe+*u Dbelongs to some HS(Q), then u necessarily belongs to Hs+m-l(Q), and there is a constant CS,K inde pendent of wu and such that
% (27) ”u”s+m—l = Cs,K(”P.u”s * ”u”s+m-2)° The "bootstrap" method shows that if Peu = f where f is a C° function, then u itself is a €~ function. But from inequality (27) one may derive much more information: for instance the space of solutions of Peu = O having sup=- port in K dis finite dimensional and consists of ¢” func tionsy if f is a ¢” function such that f f(x)u(x)dx = O for all these functions u, then there is a QCoo function v in 2 solution of the adjoint equation tP-V = f din K (ibid., p.210)., Similar uses of such inequalities have been successful in proving existence and uniqueness of Cauchy problems for hyperbolic (see 5) equations of arbitrary order ([62], vol.II, p.1748-1766).

\section{Elementary solutions, parametrices and pseudo-differential operators}
\label{sec:9.5}

% We have seen (chap.II, 3) that from the beginning the idea
% of "mewtonian" potential was closely related to the laplacian operator; if one considers in R the integral operator U
% 3

% APPLTCATIONS OF FUNCTIONAL ANALYSIS 253 defined by the kernel (where |x| is the eucli
% 1
% L | x=vy|
% dean norm), then Poisson's equation (chap.II, formula (12)) can be written A+(U+p) =p for p € @(RB), and one also may write U+(A+f) = f if £ is the potential having density p, so that this equation is also valid for all f € @(RB). Later,
% when the existence of the Green function G was proved for a domain in RB, one had similarly the relations As(U«f) = = f and Us(A*f) = £ for all f € 8(Q?), where now U is the integral operator defined by the kernel fi%—G(x,y). In both cases, YP*’T;%;T and y+— G(x,y) are solutions of the Laplace equation Au = 0, but with a singular point for y=x.
% In 1860 Riemann proposed to solve Cauchy's problem for 1i
% near hyperbolic equations of second order in 2 wvariables Peu = 0O by using a particular solution y+— R(x,y) of the adjoint equation tPov = O depending on the point x, in
% such a way that the integral operator U defined by the kernel R satisfies again the relation Pe(Usf) = £ for fE€ @(RZ), the function R playing for P a role similar to the Green
% funcfiion for the laplacian. In that case R is continuous; but when Volterra and Hadamard undertook to extend the method to second order hyperbolic equations in n = 3 +variables, they were beset by difficulties stemming from the fact that the function corresponding to R would now have, not only (as the Green function) a singularity at one point x, but singularities along lines or surfaces,
% We cannot here describe the details of these researches, for which we refer to [93]. What we want to emphasize is that, by the end of the x1x? century, one had the (rather Vague) idea that, for a second order linear differential ope-

% 25l CHAPTER IX
% rator P, one should look for a solution y +~—s R(x,y) of the ad joint equation tP'V = O having a suitable singularity at the point x, and that the integral operator U defined by the kernel R would be such that P*(Uf) = £ for fec 8(Q); such a function R was called an elementary (or fundamental) solution for tP. This worked, not only for A,

% 3
% 2

% but also for instance for the heat operator 3% —-;—E—, one 1 -x* /it 0x has R(t,x) = ———e for t > 0, R(t,x) = 0 for 2 it
% t < 0 (R(O,x) 4dis undefined); for hyperbolic equations, things were not so simple, for one had to apply the integral operator defined by the kernel R, mnot to f but to some derivatives of f [122],
% After 1900, mathematicians began to investigate the possi bility of extending the notion of elementary solution and its applications to equations of higher order. To an operator (25), one associates the polynomial in £ = (gl,gz, \ldots,gn)
% (28) op(x,8) = T a (x)(2nig)" | [sm
% (later called the "symbol" of P) and the homogeneous poly nomial in  consisting in the terms of highest degree m (29) oo(x,8) = T a (x)(2mig)%. P o |=m &
% (the "principal symbol" of P)
% The operator P is called elliptic if G;(x,g) £ 0 for all Xx€ 2 and all £ £ O, In his thesis, Fredholm considered, for n = 3, elliptic operators with constant coefficients, and proved the existence of an elementary solution by writing it explicitly as an abelian integral ([74], p.17-57); this was

% APPLTICATIONS OF FUNCTIONAL ANALYSIS 255
% later generalized to elliptic operators with constant coeffi cients in any number of wvariables (Holmgren, Herglotz E109]). In 1907, E,E. Levi considered elliptic operators with varia ble coefficients, and either n = 2 variables, or operators or order 2 in any number of variables; in both cases, using the fact that for constant coefficients elementary solutions were explicitly known, he showed how one could prove the exis tence of elementary solutions by showing that their determi nation could be reduced to the solution of a Fredholm integral equation ([145], vol.II, p.28-84).
% For operators with constant coefficients, the theory of dis
% tributions completely clarified the concept of "elementary solution" [194]; such an operator may be written u +— Axu, where A = g aaDgeo, linear combination of derivatives of the Dirac measure ¢ at the origin of Rn. An elementary solu o tion is then, by definition, a distribution E on R such (30) AxE = eO s
% that
% it follows at once from that definition that, for any distri bution T with compact support (and not only for a function), (31) Ax (E*T) = Ex(A*T) = T,
% one has
% In 1954, Ehrenpreis [63] and Malgrange [ 156] independently proved that any operator P of form (25) with constant coef ficients has an elementary solution E, Of course, such a
% solution is only determined up to addition of any distribu tion S solution of the homogeneous equation PSS = 0, It

% 256 CHAPTER IX
% is not obvious that among these distributions there would exist tempered ones; as &A is the polynomial GP(g) such (32) 0p(8)3E = 1;
% an elementary solution should be such that
% it is only in 1958 that independently H8rmander [118] and Yojasiewicz [153] showed that it is always possible, for any polynomial Q, to find a tempered distribution T such that QT = 1.
% FElementary solutions proved to be useful to show that an operator P is hypoelliptic [117, p.100], or to prove uni queness of the Cauchy problem for some operators [117,p.1l41]. But gradually it was realized that instead of looking for a "right inverse" to a differential operator, it could be much simpler to obtain an "approximate right inverse" which would be put to the same uses.,
% Such an idea was first introduced by Hilbert in 1907 under the name of parametrix, in a particularly simple context, the study of an elliptic operator P or order 2 on the sphere SZ; he proves that there is an integral operator ¢, defined by a kernel having a singularity similar to the logarithmic sin gularity of the Green function, and such that QP= I + R, where PR is an integral operator; a solution of P ¢u = f is therefore a solution of the integral equation u + Reu = = Q+f [112, p.233-242]; @ could therefore be considered as an "approximate left inverse" of P,
% Two years later, E.,E, Levi independently introduced a simi lar method in a much more general and difficult question, the generalization of the Dirichlet problem for elliptic operators

% APPLICATIONS OF FUNCTIONAL ANALYSTS 257
% of arbitrary even order, completely unexplored until then except for a few special operators such as the iterated la placian., On these special examples it transpired that what should correspond to the Dirichlet problem for an operator of even order 2m was the boundary condition consisting in fix ing the values of the solution and its first m-1 normal de rivatives on the boundary I of a bounded open set (O C Rn. Levi is only concerned with the case of n = 2 +variables and first shows that the problem (for smooth boundary I') may be reduced to the case in which one has to find a solution of Peu = f such that u and its m=1 first normal derivatives take the value O on I's. His idea is then to determine two functions, ¢(x,y) and K(x,y,xl,yl) defined respectively in 0 and Qx0, and such that: 1¢ for each point (x,,7,) € € 0, the function (x,y)F¢~K(X,y,xl,yl) and its m=1 first normal derivatives vanish on T3 292 +the function
% 70
% satisfies the equation Peu= f, 1In order to obtain that
% result, he chooses K 1in such a way that
% (34) (P°u) (X9y) = CD(X9Y) + (Kl(X9Y9X19yl)CP(xlyyl)dxldyl Q
% where Kl is a kernel to which Fredholm's theory is appli
% cable, and he has thus reduced the problem to a Fredholm in tegral equation., If one writes ¢ o the right hand side of (33), one may say that the operator  is such that PQ= I+R where £ is an integral operator; this time ¢ is an "appro ximate right inverse" of P, The determination of the func-

% 258 CHAPTER IX
% tion K satisfying these conditions is a difficult problem, and it is not surprising that after Levi not much work was done in that direction until around 1960 ([ 145], vol.II, p. 207-343).
% At that time progress came from a completely different di rection, In his work on integrals of complex functions along paths in €, Cauchy, in 1814, had observed that if f is a ¢l function in an interval [-a,a] of R, the function f(x)/x is not integrable if f(0) £ O, but the sum
% - a
% ( 3 z dx + (, £LE%QE , when ¢ tends to O, has a limit
% -3 €
% which he called the "principal value" of the integral. Simi (35) (Hef)(x) = lim flg_)_i_z
% larly, if L is a Cl curve in C, the limit
% € -0 1
% where x € L and Lb is the part of L for which the arc of L Jjoining x and 2z has length >¢, exists for each Cl function f defined on L, and is written Vp{' ELElflE_; if L dis a simple closed curve, the boundary 7 =X L
% o i is defined for x¢ L, and is in ) a holo
% of a bounded open set (, the usual line integral 1 f(z)dz
% 7 ==X
% L
% morphic function F (x), and in the exterior € - () another holomorphic function F (x); when x +tends to a point t€L,
% these functions have limits respectively equal to 1 1 £ d
% F+(t) =5 f(t) + E-TT—i-VP (zztz ’
% L
% - 1 1 f d
% L

% APPLICATIONS OF FUNCTIONAL ANALYSIS 259
% In his third paper on integral equations, Hilbert, using
% these formulas, showed that one could find two holomorphic functions F ' (x) 4in Q, F (x) in € -Q, such that for t € L, the limits F'(t) and F (t) of these functions - . 1 .
% exist when x tends to t, and satisfy a relation F'(t) = = g(t)F (t), where g dis a C~ function on L [112, p. 81-108]3 this led to calling the function H+f defined by (35) the Hilbert transform of £,
% Between 1910 and 1955, many mathematicians studied wvarious generalizations of this operator to functions of any number of variables, and applied them to various problems of Analysis; we cannot describe this evolution in any detail, and refer the reader to [197]. The most general of these "singular in tegral operators" (or "Calderon=-Zygmund operators" as they were also called) are defined in the following way: Q is an open subset of R, (x,6)—K(x,6) a locally integrable map ping of (Q X (mn-{o}) into €, which is positively homogen eous of degree -n in  for every x € ()3 4in addition it is assumed that for any x € Q, g’ K(x,£)do() = 0 (o be

% N
% n-1

% ing the invariant measure on (n-1)' Then, for any function u€ 9(Q), the limit
% (36) (Peu)(x) = 1im K(x,v-x)u(y)dy e -0 |y -x|=¢
% exists for every x and defines a singular integral operator P.
% Around 1960, it was realized that the use of Fourier trans form (generalized to distributions) enabled one to define a class of linear operators which contained at the same time

% 260 CHAPTER IX
% differential operators of type (25), singular integral opera tors and some ordinary integral operators (with locally in tegrable kernels); several mathematicians independently con tributed to this new theory, but again we cannot go into any historical detail, and we shall merely give a short descrip tion of its present status (for more references, see (D], [119] and [68]). The inversion formula for Fourier trans forms shows that the operator (25) can be written
% (37) (Peu)(x) =| exp(2mi(x|g))a(x,58)Fu(E)dg for ue 8(Q) [Rn
% where a(x,8) = 0p(x,8) (formula (28)). The generalization consists in replacing in (37) the polynomial (in ) op(x,8) by a more general ¢® function defined in men and which is only submitted to conditions concerning its growth as || tends to +o: for a polynomial GP, DiGP has the same be havior for || + o as Op itself, whereas ngp is a poly o . n .
% nomial in & of degree m-IB + One defines then a symbol as a C mapping (x,8)+= a(x,£) of QxR into €, such that (38) p%0f a(x,8)| s o, (1+z])"18
% one has
% for all multiindices @, B and all compact subsets LC (2, where x € L and  € R" are arbitrary, and C is inde- aBL pendent of x € L and & ¢ Rn; the main difference is that here m (the order of a, or of P) is an arbitrary real number. The corresponding operator P defined by (37) for u ¢ Q(Q) is called a pseudo=-differential operator defined by the symbol a; differential operators therefore correspond to symbols of order m equal to an integer m 2 1, the new-
% APPLICATIONS OF FUNCTIONAL ANALYSIS 261
% tonian potential to m = =2, and the singular integral ope rators (36) (for K of class Cw) to m = O, The most in teresting case is the one in which a = ao+al, where a is po sitively homogeneous of degree m in , and al is a symbol of order < mj; one then says that a is the princi pal symbol of the pseudo-differential operator (37), and one writes a = GP.
% The main properties of pseudo-differential operators are
% the following ones:
% I) P maps the space 9(Q) into the space &(Q) of all c” (39) (Peulv) = (u|p®.v)
% complex functions in (3 it has an adjoint P*, satisfying
% for all u, v in 9(Q) (scalar product of LZ(Q)), which
% is a pseudo-differential operator of same order my; if P has a principal symbol, P* has a principal symbol such that O 0
% ITI) One says P is of proper type if bothP and P* apply 9(Q) idinto itself; for any pseudo-differential operator @ of order r, the compositions &P and Pg are then defined and are pseudo=-differential operators of order m+r; 4if P and €@ have principal symbols, so have P and QP and
% o 0 0_oO (41) Oy, = Opp = OpOy
% IITI) If m< -1, P is an integral operator, having a kernel which is locally interable in Qx?, but has singular ities for x = yjy 4if m< -n-k, the kernel is of class Ck in the whole of (QXxQ). One says that the symbol a (and the
% 262 CHAPTER IX
% corresponding pseudo-differential operator Z’) are of order -0 1if a satisfies inequalities (38) for every real number m; P 4is then an integral operator with a kernel which is of class Cw, and conversely any such operator is a pseudo-dif ferential operator of order -o, and its principal symbol is O,
% IV) Any pseudo-differential operator P may be extended by continuity (for the weak topology) from 98(Q) to the space e’ (1) of all distributions on with compact support. The operators P of order -o are characterized by the property that for any distribution T € &’(Q), P+*T idis a C~ function on (}; one says that these operators are smoothing operators. Any pseudo-differential operator is the sum of a pseudo-dif ferential operator of proper type and of a smoothing operator. WhenK is a smoothing operator, so are the products KP and PK for any pseudo-differential operator P, if one of the two operators K , P is of proper type.
% One writes P~ @ 4if P - @ dis a smoothing operator, V) The most remarkable feature of pseudo-differential opera tors is the possibility of defining a symbol by an asymptotic
% expansion, Suppose given an infinite sequence ao,al, \ldots,ak,. eey Of symbols, having orders mO > M, Seee> M >ee6 with 1 k lim m, = -3 then there exists a symbol a of order m k=
% such that, for any k, a - (ao+al+ \ldots+ak) has order < m,; this is expressed by writing
% a ~ aO + al + o0 0o+ ak + o00
% and saying that the right hand side is an asymptotic expansion of a, If PO, \ldots,P k,ooo are the pseudo-differential opera-
% APPLICATIONS OF FUNCTIONAL ANALYSIS 263
% tors defined by the symbols A seeesByyee, and P the pseudo-differential operator defined by a, one also writes Ol k+ ® o O
% P ~ P +P +ooo+P
% VI) One says a pseudo-differential operator P having a principal symbol of order m is elliptic if G;(x,g) £ 0 for x€ Q and £ #£ 03 for differential operators, this coincides with the previous definition., It is equivalent to
% say that there exists a pseudo-differential operator  of order -m and of proper type such that QP =I+R andpg=I+R’where R and R’ are smoothing operators; in other words, P has a (left and right) parametrix in a very strong sense, The proof
% is very simple; the necessity follows from the fact that if QP~ I , one must have G; = (02)-1 by (41). Conversely,
% if P dis elliptic, there is a pseudo-differential operator Ql of proper type defined by the symbol (G;)-l; one has then QlP =71 =P l’ where Pl has order < -1, and one is reduced to finding a pseudo=-differential operator Q2 such that QZ(I-Pl) = I+R , where R 1is a smoothing operator; but it is enough to take @, ~ I+ P. 4+ P_ 4,404 P? +eee t+tOo oObtain 2 1 1
% 2
% that result!
% VII) An immediate consequence of the existence of a para metrix €@ for an elliptic differential operator P is that P is hypoelliptic, for if T is a distribution such that PeTr = f € (), one has &@°f =T + R*T, and as R is a smoothing operator, Re*T and @ *f are both ¢~ functions, hence also T. Another easy consequence of the use of pseudo differential operators is that for each point X € 1, there is a small neighborhood U c (? of X such that the equatim
% 264 CHAPTER IX
% Peu = f has solutions in U (in other words, the H. Lewy phenomenon (chap.II, 2) cannot occur for an elliptic differ ential operator P); one should note, however, that there are examples of elliptic operators P defined in Rn, and such that there are equations Peu = f which have no solution in a large ball containing the support of f € Q(Rn) [176].
% VIII) When QQ C RTM is bounded, pseudo-differential opera tors of proper type in ()} have simple continuity properties with respect to the Sobolev spaces: 4if P 1is such an opera tor of order 1r, defined in a neighborhood of 5, and s is any real number, there is a constant C depending only on P (42) [2eu]l, < cllull. _
% and s, such that
% for any u € 9(Q), the norms being those of HS(Rn) and g **(®")., If r >0 and P is elliptic, applying this result to a parametrix of P immediately yields an a priori (43) lull . < c(leeal, + [l ).
% inequality of Friedrichs type
% Suppose in addition that P = P* is a differential opera tor such that op(x,g) > O for large lgl; then an easy in ductive argument determines (by an asymptotic expansion of its symbol) a pseudo-differential operator S of proper type and of order r/2 such that P = S5 + R, where R is a smoothing operator. Applying (42) and (43) to S, one obtains the existence of constants a > 0, b > 0, ¢ > O such that, for u and v in 8(Q), one has
% (4) |(P’u|V)O| < C”u”r/z HVHr/Z

% APPLICATIONS OF FUNCTIONAL ANALYSIS 265 (45) (Prufu)g = allull?, - olull]
% The second one (for an even integer r) was first proved by Garding in 1953 [80]. It enabled him to apply the von Neumann
% spectral theory to the hermitian operator TP in the Hilbert space LZ(Q), with dom(T;) = 8(Q). In general, the defects of @;* are both infinite, and one can define a particular
% self-adjoint extension AP of Ib by the following process: dom(AP) is the dense subspace of LZ(Q) consisting of func tions u such that the distribution Peu is ggain in LTM (Q)
% 2
% and then Ap*u = Peu; furthermore, dom(AP) is contained in the space Hz/Z(Q), the closure in Hr/Z(Rn) of ®(Q). The
% spectrum of A.P is reduced to the point spectrum, consisting of an increasing sequence (xn) off real eigenvalues of fini te multiplicity, tending to +o3; the corresponding eigen functions (suitably normalized) form a Hilbert basis of LZ(Q) and are of class Cw; for any ( € € distinct from the Kn’ GQ = @%,-QI)-l is a compact operator, which one may call the
% Green operator of P- (T. It is easy to see that the res triction of GQ to 9(Q) is a pseudo-differential operator of order -r, which in general is not of proper type; how ever, for every distribution T with compact support in 2, one has (P-QI)'(GQ°T) = Gg°((P-gI)'T) = T and in particular, for any point x € (2,
% (46) (P-QI)°(G€-eX) = GQ'((P-QI)'GX) =€ so that one may say that the distribution @ 'ex is an ele-
% g
% mentary solution of P -(I at the point x.

% 266 CHAPTER IX
% IX) The results of VIII) apply in particular to a differen tial operator of even order 2p = 2
% (47) (Peu)(x) == D* (a (x)DBu(x)) lo[<p, |8 <P o where the anB are bounded €~ functions in a neighborhood of the bounded set 5, such that:
% 19 (-l)lBla = (-1)|@|5f“ , which guarantee that P* = P; B&. aB 22 there is a constant C > O such that, for every x € ( and every family (g@) of complex numbers, one has o |=p —_— 2
% (48) z (—l)p a (x)z.z_ = c¢( I Iz I ).
% la |=|g |=p aB @ B lo|=p
% The Green operator GQ is then (for ( € Sp(AP)) an inte gral operator, with kernel (x,y)r> G((,x,y) which is local 1y integrable in 0Qx0), C_ outside of the diagonal and such that &(¢,x,v) = G(¢,y,x) (the Green function of P); from (46) it follows that y~— G(C,x,y) is a solution of P°u = = Cu in the cofiplement Q-{x} of the point x.
% One may always take ( = =b for a sufficiently large num ber b > 0, and for every function f € €(Q) N 1?(0), there is therefore a unique solution of the equation Peu + bu = f belonging to the space Hg(Q) and of class C° in Q.
% These results were obtained (of course without the theory of pseudo-differential operators) by Garding and Vi¥ik (inde pendently) in 1953 ([80], [218]); they may be considered as
% a "weak" solution of the generalization of Dirichlett's problem considered by E,E, Levi: no assumption is made on the boundary I’ of Q, but all which is required of the solution is that it should be arbitrarily close, for the topology of

% APPLICATIONS OF FUNCTIONAL ANALYSIS 267
% Hp(Rn), of € functions vanishing in a neighborhood of I"s but it (or its derivatives) may have a very pathological behaviour at points of T 4if I is not smooth, (Peu)(x) = z p*(a_(x)DPu(x)) + £ DY(a (x)DVu(x)) la|=p, |8 |=p ap |v]<p v )
% If one makes the additional assumption that
% where (--l)l\)l av(x) > 0 for all |v]| < p, then one may even take b = O in the Garding—Vigik theorem (this is the case in particular for (-A)P).
% X) If E and F are two complex vector bundles over a compact differentiable manifold X, and T(E), I'(F) are the vector spaces of ¢” sections of these bundles over X, one can define pseudo-differential operators P: T'(E) » I'(F), which become matrices of ordinary pseudo-differential opera tors when expressed in local coordinates., It is then possible to define intrinsically a principal symbol oO: for each P x € X and every tangent covector  to X at the point x, g;(x,g) is a homomorphism E_ 4 F_ of the vector spaces, fibres of E and F at xj; idin local coordinates, g;(x,g)
% is the matrix of the principal symbols of the elements of the matrix equal to P. It is possible to define on T (E) and I'(F) structures of prehilbert spaces, and to attach to any pseudo-differential operator P i1its adjoint p”s T'(F) » T(E) such that (Peu|v) = (u|p¥*+v); properties (40) and (41) stinl
% hold.
% A pseudo-differential operator P: T(E) = F(E) is then called elliptic if for every x € X and every & # O, g;(x,g) is a bijection of E_ onto itself, and the existence

% 268 CHAPTER IX
% of a parametrix for such an operator can then be proved as in VI). For elliptic operators such that P* = p, the applica tion of spectral theory to the hermitian operator TP (in the Hilbert space, completion of T(E)) is here much simpler than in VIII) due to the absen;e of "boundary conditions": i
% there is a Hilbert basis (u,) of T(E) such that Pe.u_= = My Uy where |, 1s real and I“kl tends to +o with Kkj for every f € I'(E), one has f = i (fluk)uk, the series being convergent for the topology of the Fréchet space P(E), and P ef = E uk(fluk)uk with the same convergence, 1In par ticular, Ker(pP) is the finite dimensional subspace having as a basis the u, for which My = 0, and Im(P) dis closed
% and is a topological supplement of Ker( P).
% If now P is any elliptic operator T(E) » I'(E), P*P and PP* are both elliptic and equal to their adjoints; the study
% of these operators enable one to evaluate the difference dim(Ker(E’)) - dim(Ker( P*)), the index of P , and to express it by a formula in terms of the principal symbol of P and of the cohomology of X3 +this is the famous Atiyah-Singer for mula, a fundamental result which has many applications and has spurred research in many directions (ElO], [31], [198], [199]). It was in fact due to the development of the neces
% sary tools for the proof of that formula that the theory of pseudo-differential operators got started.
% XI) The Garding-Vigik theorem of IX) leaves unanswered two questions: 1¢ Why dis it that, in the Dirichlet problem and its generalizations, half of the Cauchy data on the boundary are enough to determine the solution? 2¢ What can be said of
% APPLICATIONS OF FUNCTIONAL ANALYSIS 269
% the behavior of the unique solution belonging to Hg(Q) in the vicinity of a point of the boundary I' where T' is smooth?
% To answer these questions, one starts by investigating the Cauchy problem for an elliptic operator and seeing why in
% general it has no solution, Suppose that the bounded open set € R has a smooth boundary I, and (for simplicity's sake) that P is a differential operator defined in a neigh borhood Qo of 5, has even order 2p = 2, and possesses in QO a parametrix ¢ of order =2p such that Q -@“T) = = P+(@+T) for any distribution T € 8'(00) (this is the case for the operator P - (I in VIII), but here we do not suppose that P " = P). For any function u € 8(00), we note Dch(u) the function (go’gl""’gZp—l) defined on the boun dary T’ and with values in (€ p’ where gj is the normal
% 2
% derivative at a point of T« The starting point is an
% Ju
% an?o) idea due to Sobolev ([200], p.63): 1let u be the discontin unous function equal to u in 5 but to O in the complement QO - Q. Then Peu® is well defined as a distribution on Qo’ (49) Peu® = (Peu)® + NeDch(u)
% and it is easy to check that one can write
% where N is a linear operator (independent of the function u) which to every € function in (G(P))Zp associates a distribution with support in [ (what one now calls a multi layer on I'). As both sides of (49) are distributions on Qo with compact support, the operator ¢ may be applied to them, and yvields the relation
% 270 CHAPTER IX
% (50) u® = @ «((P-u)®) + Q- (N:Deh(u)).
% This is the general form of Green's formula (17) of chap.II, 3;’ for any function f ¢ G(QO), the distribution Q+f° has a restriction to O which is a €© function such that all its derivatives have limits at every point of I3 one says that it is the Q@ =potential of the mass distribution of den sity £ on . Similarly, for any vector function gE(&(T»Zp, the distribution Q°(N°g) has the same properties, and one says that its restriction to ()} is the -potential of the multilayer N +*g; these properties obviously generalize the classical properties of the newtonian potentials of a mass
% distribution, of a single layer and of a double layer (chap. IX, 3) (of course the restriction of Q «(N.g) to Q, - Q also has similar properties, but the limits at a point of T
% differ in general from the limits at the same point of the restriction of Qo(Nog) to Q). Therefore, equation (50) shows that if there is an u € €(Q) such that Peu = f and Dch(u) = g are given functions, this function u (restric tion of u°) is unique, which corresponds to what one may expect of the Cauchy problem, But in addition one must have Dch(u®) = g, which gives the necessary condition
% (51) g = DCh(Q'fO) + Dch(@+«(N-g))
% between f and g. One proves that C: g+ Dch(Q+*(N*g)) is a pseudo-differential operator of (G(P))Zp into itself,
% which is called the Calderon operator corresponding to the parametrix g .
% A more detailed study shows that (51) is equivalent to p

% APPLICATIONS OF FUNCTIONAL ANALYSIS 271
% linear relations between the 2p functions go""’gZp 1 and du 82p-l 2p functions u, In ' 5p-1 on T, but only p of
% their derivatives; this explains why one cannot prescribe the
% them. More generally, one may consider a differential opera torB of (8(?))2p into (S(F))p, and instead of the Cauchy problem, consider the boundary conditions B +(Dch(u)) = g for a given vector function g ¢ (S(P))p. It is then possible to describe explicitly a set of sufficient conditions (called the Lopatinski conditions) linking B and ( and implying that the problem can be reduced to Fredholm integral equations on T3 more precisely, these conditions imply that the map (52) u**—(P-u,B°(DCh(u)))
% ping
% of €(0) (space of the restrictions to 0 of functions of S(QO)) into 8(5) X (8(?))p has a finite dimensional kernel
% and a closed image of finite codimension.
% In particular, one checks that the Lopatinski conditions are
% always satisfied if one takes for B.g a consecutive sequence (gq,g R =3 ) of p of the functions gj, and the q+p-1
% q+1l
% corresponding problem for q = O dis Jjust the Dirichlet prob lem as posed by E.E. Levi,
% At this point, one might think, from the example (47), that except for a denumerable set of values of ( € C, the mapping ui—~ ((P-¢I)*u, B+(Dch(u))) would in fact be bijective. How ever, this is not always the case, and there are examples for which that mapping is injective for no ( € C.
% Vv For operators (47) to which the Garding-Visik theorem applies, to prove that the preceding mapping is bijective for QGSp(AP)

% 272 CHAPTER IX
% (with Beg = (go, \ldots,gp_l)), it is enough to show that, when " 1is smooth, the unique solution u € HE(Q), and all its de rivatives, can be extended by continuity to O =0UT (the second problem mentioned above). Actually, even if T is not smooth everywhere, the existence of limits for these functions is guaranteed at each point where [T is smoothj; this was first proved by L, Nirenberg in 1955 [167], and has been
% proved by Peetre in 1961 using a different method, which how ever still relies on a priori inequalities [171]. These re
% sults may be extended to other types of boundary conditions B+(Dch(u)) = g satisfying the Lopatinski conditions, the so called coercitive problems for elliptic operators P for which
% p¥ =p.
% Further generaliizations. Formula (37) defining a pseudo-dif ferential operator can also be written, replacing ¥u by its
% definition
% (P°u)(X) = l exp(2fl1(x-y|))a(x,b)u(y)dydg
% JOxRTM
% where the integral is not any more a Lebesgue integral, but an "improper" (or "oscillating") one, obtained by passage to
% the 1limit from the integral of the same function multiplied by a function h(g/q), where h e 8(R") is equal to 1 in a neighborhood of 0O, and q tends to +o. It turns out that
% one can define similar integrals when one replaces exp(2mi(x-y|)) by "phase functions" ¢(x,v,f) positively homogeneous in £, and a(x,f) by more general "symbols"
% a(x,y,8).
% I) Such operators naturally occur in the theory of strictly

% APPLICATIONS OF FUNCTIONAL ANALYSIS 2773
% hyperbolic operators, of which the simplest is the wave opera=- (53) Ou = i—% - (a ; Foaud 2 ;) .
% tor (or dalembertian)2 2
% ot axl aXn
% The Cauchy problem for that operator, consisting in finding a solution of [Ju = O such that u(0,x) = go(x) and g%(o,x) = gl(x) are given functions, had already been solved
% by Cauchy; the explicit formula he gave for the solution can be written u(t,x) = u+(t,x) + u_(t,x), where
% (5%) u, (tx) = %flexp(Zfli((x-fl% )+[5 160 (g, (v) + 55 —I—]—)dydé
% the integrals being "improper" in a sense easy to describe. In general, one considers an operator of order m in n+l variables t, xl, \ldots,kn
% m m -J
% 3t =1 |a|<3 2TMd
% and one assumes that its principal symbol OP(T,g,t,x) can (56) 0;(799t9x) = ]—L (T‘qj(t’xsg))
% be written
% J=
% where the qj are real functions of class Cm in IxQx(Rn-{O}) (I open subset of R, (2 open subset of Rn), positively homogeneous or degree 1 in £, and such that for j # k, qj(t,x,g) # qk(t,x,g) everywhere, The Cauchy problem to be solved is to find a function v(t,x) such that Pev= £ and J
% o . v(t_,x) = g.(x) (0< j <m-1) fora t_ € I, in a con venient neighborhood of (to,xo) € Ix2, f and 8 being C° functions. Taking (54) as a model, one introduces m2 opera=-

% 274 CHAPTER IX
% t in I, such that, if E_(s) = I E J.h(s) for O €< h < m-1, o h 0
% tors E&h(s) (O < j,h £ m-1) for s in a neighborhood of
% one has (locally)
% 3 (K (57) PEh(s) = Rh(s), (ét) Eh(s) = Ath for t=s and Ogk<m-1,
% where the Rh(s) are smoothing operators. If one writes t
% (L-u)(t,x) = ( (Em_l(s)'u(s,‘))(t,x)ds
% 't)
% one has (%¥) (L-u)(to,x) =0 for O< k £ m-1, and
% k
% Pe(L*u) = u- Veu, where V is a Volterra integral operator (58) (V.u)(tQX) = ds K(t9S9X9Y)u(89Y)dY
% t
% t U
% U Dbeing a neighborhood of X in Q and K a C function. It is easy to see that I +V 4is inverted by I +W, where W is a similar Volterra operator. The Cauchy problem is then
% solved by taking in a sufficiently small neighborhood of (to’xo)m=1 m-1
% J-—O J:O
% (59) = 2 E. ‘g . * * - . g . . v= T Bt )e; +L(Iew)e(f - Z Ri(t)):g;))
% The construction of the Eh follows an idea introduced by P. Lax in 1957, and patterned after the known behavior of the solutions of the wave equation, which "propagate'" along "rays". For operators (25) with analytic coefficients, it follows from the Cauchy=-Kowalewska theorem that the Cauchy problem for data
% given on a hypersurface will fail to have a unique solution if the hypersurface is given locally by an equation z(xl,“”xn)z = const., where =z 1is a solution of the partial differential equation of order 1

% APPLTICATIONS OF FUNCTIONAL ANALYSIS 275 ’_—, \ldots, n axl axn (60) @(xl, \ldots,x
% 3z 3z )
% in which & is obtained from the principal symbol Og(x,g) by replacing the vector & Dby s such hyper
% (az oz )
% 39X, X
% -~ 9 ® 0 e g —
% surfaces are called characteristic for the operator P. For strictly hyperbolic operators (55), it follows from (56 that the equation of characteristic hypersurfaces (60) splits into m equations
% (61) %%~- qj(t,x,gradxz) = 0, 0 < j< m=1 . 3z 3z (with gradxz = (SEE, \ldots,ggg)). For the wave operator (53) the equations (61) are
% (674 é—g- = fl:|gradxz|
% with solutions
% z = 2m((x|g) £ [E]¢t)
% which reduce to 2m(x|) for t = O; they are precisely the "phase functions" which enter in Cauchy's formula (54). For
% general strictly hyperbolic operators (54), one therefore in troduces the m- operators th(s) defined by
% (th(s)°u)(t,x) =

% (62) y
% l( exp(i(wj(t,s,x,g)-2fl(y|)))ajh(t,s,x,y,g)u(y)dydg UxR

% where wj(t,s,x,g) is the unique solution of (61) satisfying the initial condition
% and ajh is a symbol of order =~h (in the sense defined for

% 276 CHAPTER IX
% pseudo-differential operators). The goal is to determine the m-1
% 2 i in such a way that, if one writes Fh(s) = jO E}h(s),
% the following conditions are satisfied:
% 19 Qh(s) = Pth(s) is a smoothing operator;
% 20 for each g € 9(U), +the restriction to the hyperplane t = s of the function -%;E (Fh(s)°g) has the form (s,x)+ > 6hkg(x) + (<Qhk(s)-g)(x) where Qhk(s) is a smoothing operator, for O < h,k € m-1,
% The conditions (57) are then met by taking
% and Eh(s) = F (s) - R, (s).
% k=0
% To achieve that goal, one defines the a. by asymptotic jh (00) () (64) 2y ~ LEO a sy
% expansions
% where ajh() is a symbol or order -h = 4 ; the agfi) are determined by induction on 1, in such a way that if one writes| (thN(S)'U)(t,X) =
% X . (1) = X Jf{ exp(l(wj(t,s,x,)-2fl(y|)»ajh (t,s,x,y,g)u(y)dydg =07 Juxr"TM
% then:
% 12 Each operator Pon (s) is defined by a symbol of 0 F _ . . —
% order m-h-N-=2.
% m-1
% 20 TIf hN(s) = jo E}hN(s), the restriction to t = s K 9=
% of the function'gzi(FhN(s)'g) has the form (s,x)r*-bhkg(x) +

% APPLICATIONS OF FUNCTIONAL ANALYSTS 277
% + (QhkN(s)-g)(x), where QhkN(S) is a pseudo-differential operator of order k-h-N-1l, for O < h,k £ m=1.
% It is in this inductive process that the analogs of the "rays" enter. The classical Cauchy method for integration of partial differential equations of order 1 consists, f(or each equation (61), in considering, in the space IXQan+ )
% 1
% the "characteristic" curves
% t (t9xl(t)’°°°9xn(t)9po(t)9pl(t)9°°°apn(t))
% solutions of the system of ordinary differential equations rdxk qu
% & = 7 aE, (Brareees Ty Preeeopy)
% (65)  l < k £ n dpk oq .

% \
% k

% and verifying the condition P, = qj(t’xl’""Xn’pl""’pn); one says that their projections th*(t,xl(t), \ldots,xn(t)) on I x Q constitute the j-th family of bicharacteristic curves
% for the operator P. For the wave operator, the bicharacter istic curves which are such that xk(s) = ¥y for 1 <k < n
% are in fact the classical "rays"
% t t— yk:l:—l-gfr(t-s) (1 <k < n).
% g
% In the general theory, each agg)(t,s,x,g) is taken inde pendent of vy, and is obtained by integrating, along each
% bicharacteristic curve, an ordinary linear differential equa tion of the first order (in the variable t), whose coeffi cients are determined when the agfi) are known for {f < N-1;
% finally the induction starts with the values of the

% 278 CHAPTER IX
% agg)(s,s,x,g), which are given by the linear system
% where qg(s,x,g) = qj(s,x,gradij(s,s,x,g)); the determinant of that system is #£0 ©because the a; have been supposed to be distinct.,
% One of the consequences of this remarkable construction is that, in the explicit formula (59) for f = O, one may re place the "initial conditions" gj by arbitrary distributions Sj on )3 Vv 1is then replaced by a distribution T solution of P.T = O, For these equations, the "trace" Tt of such a distribution on the hyperplane {t} x Q2 may be defined (although T is not a function) as a distribution on Q, varying with t, and which may be said to "propagate" with
% the "time" t, starting from the "initial" distribution Tt = O
% = SO. It is then possible to show that the singular support of Tt is contained in a set Mt obtained in the following way: one considers the union MO of the singular supports of the distributions Sj’ and all the bicharacteristics issued from points of MO; Mt is the set of all points on these bicharacteristics at time t. This gives a precise meaning to a phenomenon which had been well known for second order strictly hyperbolic equations, and particular types of "initial values": the singularities propagate along the bi
% characteristics,
% Under additional assumptions, it is possible to extend these results when in the decomposition (56), some of the qj are equal [L41].
% II) Another important application of operators generalizing the pseudo-differential operators is the problem of local e
% xistence of solutions for a partial differential equation Peu = {, which has stimulated much research after H. Lewy's discovery (chap.II, 2). Over a period of more than 15 years,

% the combined efforts of HOrmander, Nirenberg ceeded in formulating a system of conditions ly proved to be necessary and sufficient for by Beals and Fefferman [ 20], using new types We cannot here do more than refer the reader
% and Tréves suc which were final local existence of operators [19]. to these papers.

\nocite{*}
\printbibliography[title=References]
\markboth{References}{References}
\fancyhead[C]{\textsc{\leftmark}}

\end{document}
